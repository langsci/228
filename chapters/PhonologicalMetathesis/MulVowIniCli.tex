\section{Multiple enclitics}\label{sec:ConInsConIns}
Amarasi allows sequences of enclitics to occur.
When the second enclitic is vowel-initial,
it usually triggers the normal processes of metathesis
and consonant insertion on the previous enclitic,
though there are some exceptions.

A number of examples of \ve{=een} `{\een}'
attached to \ve{=ee} `{\ee}/{\eeV}'
are given in \qf{ex:=ee+en->=eej=een} below.
All these examples show expected insertion of /\j/
before \ve{=een} as conditioned by the final vowel of \ve{=ee}.\footnote{
		There is one example in my corpus in which /ɡw/
		is irregularly inserted after \ve{=ee}. This is \ve{hai mi-ʔuab=eegw=een}
		`we've already spoken about it' ({\hai} \mi-speak={\ee=\een}).}

\begin{exe}
	\ex{\ve{=ee} + \ve{=een} {\ra} \ve{=ee\j=een} \label{ex:=ee+en->=eej=een}}
		\sn{\stl{0.35em}\gw\begin{tabular}{llll}
			\ve{na-sopu}		&+ \ve{=ee} + \ve{=een} \ra& \ve{na-soopgw=ee\j=een}	& `finished it' \\
			\ve{buku}				&+ \ve{=ee} + \ve{=een} \ra& \ve{buukgw=ee\j=een}		& `the book (already)' \\
			\ve{mepu}				&+ \ve{=ee} + \ve{=een} \ra& \ve{meepgw=ee\j=een}		& `the work (already)' \\
			\ve{na-kratiʔ}	&+ \ve{=ee} + \ve{=een} \ra& \ve{na-kraitʔ=ee\j=een} & `destroyed it' \\
			\ve{n-porin} 		&+ \ve{=ee} + \ve{=een} \ra& \ve{n-poirn=ee\j=een} 		& `threw it' \\
			\ve{n-isa} 			&+ \ve{=ee} + \ve{=een} \ra& \ve{n-iis=ee\j=een} 			& `defeated him' \\
			\ve{ʔsobeʔ} 		&+ \ve{=ee} + \ve{=een} \ra& \ve{ʔsoebʔ=ee\j=een} & `the hat (already)' \\
		\end{tabular}}
\end{exe}

In such examples the metathesis of the penultimate clitic
is not detectable as the first clitic has a sequence of two identical vowels.
Examples of vowel-initial enclitics attached
to pronominal enclitics showing vowel assimilation are given in
\qf{ex:130825-7, 0.11} and \qf{ex:130825-6, 19.24} below.

\begin{exe}
	\ex{\glll	n-aan=\tbr{kaagw}=\tbr{ii} onai, \sf{pak} \\
						n-ana=\tbr{kau}=\tbr{ii} onai \sf{pak} \\
						\n-get={\kau}={\ii} like.this dad \\
			\glt	`S/he got me like this, dad.'
						\txrf{130825-7, 0.11}{\emb{130825-7-00-11.mp3}{\spk{}}{\apl}}}\label{ex:130825-7, 0.11}
	\ex{\glll a|n-\sf{koleŋ}=\tbr{kaa\j}=\tbr{ena} =m he t-pasan reʔ ʔfutuʔ \\
						{\a}n-\sf{koleŋ}=\tbr{kai}=\tbr{ena} =ma he t-pasan reʔ ʔfutuʔ \\
						{\a\n}-call={\kai=\een} =and {\he} \t-tie {\reqt} belt \\
			\glt \lh{a|}`They started calling (to) us to tie the seatbelt.'
						\txrf{130825-6, 19.24}{\emb{130825-6-19-24.mp3}{\spk{}}{\apl}}}\label{ex:130825-6, 19.24}
\end{exe}

Under the analysis of vowel assimilation given in \srf{sec:VowAss},
the assimilation of the final vowels of \ve{=ka\tbr{u}} `{\kau}' {\ra} \ve{=ka\tbr{a}gw}
and \ve{=ka\tbr{i}} {\ra} \ve{=ka\tbr{a}\j} `{\kai}' in such examples
is due to metathesis of the medial empty C-slot.
Similarly, as discussed in \srf{sec:ConIns}, the insertion of the consonant
in all such examples occurs to provide the following foot with an onset consonant.

That consonant insertion and metathesis affect
enclitics when an additional enclitic is added
has exactly the same explanation as that for every other clitic host.
Phrases such as \ve{n-aan=kaagw=ii} `got me'
or \ve{meepgw=ee\j=een} `the work (already)'
both have two internal prosodic words,
with a crisp edge required after each.
The phonological and morphological structures
of \ve{meepgw=ee\j=een} `the work (already)' are shown in \qf{as:meepgw=eej=een}
below with the crisp edges indicated.\footnote{
		The use of the M\=/form for the enclitic \ve{=ena/=een} `{\een}' in \qf{as:meepgw=eej=een}
		is due to discourse structures, as discussed in Chapter \ref{ch:DisMet}}

\begin{exe}
	\exa{\xy
		<2.5em,5cm>*\as{\hp{\sub{1}}PrWd\sub{1}}="PrWd1",<4.5em,6cm>*\as{\hp{\sub{2}}PrWd\sub{2}}="PrWd2",<6em,7cm>*\as{\hp{\sub{3}}PrWd\sub{3}}="PrWd3",
		<2.5em,4cm>*\as{\hp{\sub{\tsc{m}}}Ft\sub{\tsc{m}}}="ft1",
		<6.5em,4cm>*\as{\hp{\sub{\tsc{m}}}Ft\sub{\tsc{m}}}="ft2",<10.5em,4cm>*\as{\hp{\sub{\tsc{m}}}Ft\sub{\tsc{m}}}="ft3",
		<1.5em,3cm>*\as{\hp{\sub{1}}σ\sub{1}}="s1",<3.5em,3cm>*\as{\hp{\sub{2}}σ\sub{2}}="s2",
		<5.5em,3cm>*\as{\hp{\sub{3}}σ\sub{3}}="s3",<7.5em,3cm>*\as{\hp{\sub{4}}σ\sub{4}}="s4",
		<9.5em,3cm>*\as{\hp{\sub{5}}σ\sub{5}}="s5",<11.5em,3cm>*\as{\hp{\sub{6}}σ\sub{6}}="s6",
		<1em,2cm>*\as{C}="CV1",<2em,2cm>*\as{V}="CV2",<3em,2cm>*\as{V}="CV3",<4em,2cm>*\as{C}="CV4",<5em,2cm>*\as{C}="CV5",
		<6em,2cm>*\as{V}="CV6",<7em,2cm>*\as{V}="CV7",<8em,2cm>*\as{C}="CV8",
		<9em,2cm>*\as{C}="CV9",<10em,2cm>*\as{V}="CV10",<11em,2cm>*\as{V}="CV11",<12em,2cm>*\as{C}="CV12",
		<1em,1cm>*\as{m}="cv1",<2.5em,1cm>*\as{e}="cv2",<3em,1cm>*\as{ }="cv3",<4em,1cm>*\as{p}="cv4",<5em,1cm>*\as{gw}="cv5",
		<6.5em,1cm>*\as{e}="cv6",<7em,1cm>*\as{}="cv7",<8em,1cm>*\as{}="cv8",<9em,1cm>*\as{\j}="cv9",
		<10em,1cm>*\as{e}="cv10",<11em,1cm>*\as{e}="cv11",<12em,1cm>*\as{n}="cv12",
		<2.5em,0cm>*\as{\hp{\sub{1}}M\sub{1}}="m1",<6.5em,0cm>*\as{\hp{\sub{2}}M\sub{2}}="m2",<11em,0cm>*\as{\hp{\sub{3}}M\sub{3}}="m3",
		<5em,0cm>*\as{=}="=1",<9em,0cm>*\as{=}="=2",
		"m1"+U;"cv1"+D**\dir{-};"m1"+U;"cv2"+D**\dir{-};"m1"+U;"cv4"+D**\dir{-};"m2"+U;"cv6"+D**\dir{-};
		"m3"+U;"cv10"+D**\dir{-};"m3"+U;"cv11"+D**\dir{-};"m3"+U;"cv12"+D**\dir{-};
		"cv1"+U;"CV1"+D**\dir{-};"cv2"+U;"CV2"+D**\dir{-};"cv2"+U;"CV3"+D**\dir{-};"cv4"+U;"CV4"+D**\dir{-};"cv5"+U;"CV5"+D**\dir{-};
		"cv6"+U;"CV6"+D**\dir{-};"cv6"+U;"CV7"+D**\dir{-};"cv9"+U;"CV9"+D**\dir{-};
		"cv10"+U;"CV10"+D**\dir{-};"cv11"+U;"CV11"+D**\dir{-};"cv12"+U;"CV12"+D**\dir{-};
		"CV1"+U;"s1"+D**\dir{-};"CV2"+U;"s1"+D**\dir{-};"CV3"+U;"s2"+D**\dir{-};"CV4"+U;"s2"+D**\dir{-};
		"CV5"+U;"s3"+D**\dir{-};"CV6"+U;"s3"+D**\dir{-};"CV7"+U;"s4"+D**\dir{-};"CV8"+U;"s4"+D**\dir{-};
		"CV9"+U;"s5"+D**\dir{-};"CV10"+U;"s5"+D**\dir{-};"CV11"+U;"s6"+D**\dir{-};"CV12"+U;"s6"+D**\dir{-};
		"s1"+U;"ft1"+D**\dir{-};"s2"+U;"ft1"+D**\dir{-};"s3"+U;"ft2"+D**\dir{-};"s4"+U;"ft2"+D**\dir{-};"s5"+U;"ft3"+D**\dir{-};"s6"+U;"ft3"+D**\dir{-};
		"ft1"+U;"PrWd1"+D**\dir{-};"PrWd1"+U;"PrWd2"+D**\dir{-};"ft2"+U;"PrWd2"+D**\dir{-};"PrWd2"+U;"PrWd3"+D**\dir{-};"ft3"+U;"PrWd3"+D**\dir{-};
		<8.5em,3.5cm>*\as{\tikz[red,thick,dashed,baseline=0.9ex]\draw (0,0) -- (0,4.5cm);}="line",
		<4.5em,3.5cm>*\as{\tikz[red,thick,dashed,baseline=0.9ex]\draw (0,0) -- (0,4.5cm);}="line",
	\endxy}\label{as:meepgw=eej=een}
\end{exe}

While the normal process occurs in most cases
when a second vowel-initial enclitic is added,
there are a small number of exceptions.
The first exception is insertion of /ɡw/
when an enclitic attaches to an enclitic which has
already triggered insertion of /\j/.

Examples are scarce.
I have located only four in \citet{or16c} and nine
in the Amarasi Bible translation,
yielding the five unique examples given in \qf{ex:j=ee+en->=ee=gwen} below.
Nonetheless, native speakers reject forms with two insertions of /\j/ as ungrammatical, thus
\ve{\tcb{*}oo\j=ee\j=een} `the water already' or \ve{\tcb{*}n-raar\j=ee\j=een}.
There is, however, a single example in my corpus: \ve{n-heek\j=ee\j=een} `caught it already'.

\begin{exe}
	\ex{\ve{\j=ee} + \ve{=een} {\ra} \ve{\j=eegw=een} \label{ex:j=ee+en->=ee=gwen}}
		\sn{\stl{0.4em}\gw\begin{tabular}{llll}
			\ve{oe}			&+ \ve{=ee} + \ve{=een} \ra& \ve{oo\tbr{\j}=ee\tbr{gw}=een}			& `the water (already)' \\
			\ve{ʔ-piri}	&+ \ve{=ee} + \ve{=een} \ra& \ve{ʔ-piir\tbr{\j}=ee\tbr{gw}=een}	& `(I've) chosen him' \\
			\ve{n-moʔe}	&+ \ve{=ee} + \ve{=een} \ra& \ve{n-mooʔ\tbr{\j}=ee\tbr{gw}=een}	& `(s/he's) made it' \\
			\ve{ʔ-eki}	&+ \ve{=ee} + \ve{=een} \ra& \ve{ʔ-eek\tbr{\j}=ee\tbr{gw}=een}	& `(I've) brought him' \\
			\ve{n-rari}	&+ \ve{=ee} + \ve{=een} \ra& \ve{n-raar\tbr{\j}=ee\tbr{gw}=een}	& `finished it' \\
		\end{tabular}}
\end{exe}

Insertion of /ɡw/ after insertion of /\j/ is probably a kind of dissimilation.
After /\j/ has been inserted, insertion of a second /\j/ is blocked.
Thus, the default medial consonant /ɡw/ is inserted at the second clitic boundary.

Another possible case of dissimilatory consonant insertion occurs
when the inceptive enclitic \ve{=een} occurs attached
to the Indonesian loanword \it{estiga} `PhD, doctoral degree'.\footnote{
		The phrase \it{estiga} is borrowed from Indonesian S3,
		an abbreviation of \it{sarjana tiga} `third bachelors/scholar'.}
In this case the consonant /\j/ is inserted,
perhaps due to the presence of [ɡ] in the clitic host
and/or the penultimate vowel being /i/.
I heard this phrase not infrequently during my fieldwork
after explaining I was learning Amarasi for my PhD.
It is given in \qf{ex:estiga=jen} below.

\begin{exe}
	\ex{\gll	\sf{estiga}\j=een\\
						PhD={\een}\\
			\glt	`(So, you're) now doing a PhD?' \txrf{observation}}\label{ex:estiga=jen}
\end{exe}

\largerpage
The second exception involving multiple
enclitics is when an enclitic attaches to the plural enclitic \ve{=eni/=ein},
or the pronominal enclitics
\ve{=kiti/=kiit} `{\kiit}', or \ve{=sini/=siin} `{\siin}'.
These enclitics occur in the M\=/form
before anther enclitic, but consonant insertion
does not usually occur.
Examples are given in \qf{ex:NoIns} below.

\begin{exe}
	\ex{No consonant insertion after \ve{=ein}, \ve{=kiit} and \ve{=siin}\label{ex:NoIns}}
		\sn{\stl{0.3em}\gw\begin{tabular}{rclclcll}
			\ve{anah}		&+&\ve{=ein}&+&\ve{=aa}	&\ra&\ve{aanh=ein=aa}		&`the children'\\
			\ve{bareʔ}	&+&\ve{=ein}&+&\ve{=ee}	&\ra&\ve{baerʔ=ein=ee}	&`the stuff'\\
			\ve{upu-ʔ}	&+&\ve{=ein}&+&\ve{=ee}	&\ra&\ve{uup-ʔ=ein=ee}	&`the grandchildren'\\
			%\ve{bae-f}	&+&\ve{=ein}&+&\ve{=ee}	&\ra&\ve{bae-f=ein-=ee}	&`brothers-in-law/mates'\\
			\ve{papaʔ}	&+&\ve{=ein}&+&\ve{=ii}	&\ra&\ve{paapʔ=ein=ii}	&`the wounds'\\
			\ve{neka-m}	&+&\ve{=ein}&+&\ve{=ii}	&\ra&\ve{neek-m=ein=ii}	&`your feelings'\\
			\ve{tua-k}	&+&\ve{=ein}&+&\ve{=ii}	&\ra&\ve{tua-k=ein=ii}	&`their-selves'\\
			\ve{n-saen=n}	&+&\ve{=kiit}&+&\ve{=een}&\ra&\ve{n-sae=n=kiit=een}&`loaded on us'\\
			\ve{na-pein}	&+&\ve{=siin}&+&\ve{=een}&\ra&\ve{n-pein=siin=een}&`has got them'\\
	%		\ve{=n}&+&\ve{=ein}&+&\ve{=}&\ra&\ve{=n=gwen}&`'\\
		\end{tabular}}
\end{exe}

As discussed in Chapter \ref{ch:DisMet}, the M\=/form
of these enclitics (and a number of other word classes) is the default form.
Thus, we can propose that subsequent enclitics attach to the
(consonant-final) default form of these enclitics.

Verbs also have a default M\=/form,
but consonant insertion \emph{does} occur after verbs.
The difference between verbs and enclitics is probably
due to the productivity of the U\=/form/M\=/form alternation.
For verbs this alternation is completely productive
while for enclitics it is less productive
and the U\=/forms are rarely used.
Thus for verbs, the U\=/form is still the underlying
morphological form, while for enclitics the M\=/form
may be the underlying morphological form.

A second reason that consonant insertion does not
occur after the pronominal enclitics \ve{=kiit} `{\kiit}' and \ve{=siin} `{\siin}',
is because these have alternate U\=/forms
with a final /a/: \ve{=kita} and \ve{=sina} (\srf{sec:IrrMfor}).
These forms are more conservative than the U\=/forms \ve{=kiti} and \ve{=sini},
as can be seen by comparing them with their Proto-Malayo-Polynesian etyma *kita and *sida.
Before the innovation of the U\=/forms \ve{=kiti} and \ve{=sini},
a second enclitic would have attached to /a/ final forms
after which consonant insertion does not occur (\srf{sec:CliHosFinA}).
This older pattern has been retained after
the innovation of \ve{=kiti} and \ve{=sini}.

The usual pattern after \ve{=ein} `{\ein}',
\ve{=kiit} `{\kiit}' and \ve{=siin} `{\siin}' is
for no consonant to be inserted after the attachment of a vowel-initial enclitic.
However, there are sporadic examples in which /ɡw/ is inserted after \ve{=ein}.
I have so far found three examples, one in my corpus
and two in the Amarasi Bible translation,
all given in \qf{ex:=ein=gwX} below.
In this case the unexpected insertion
of /ɡw/ may be by analogy with insertion of /ɡw/
after \ve{=n} and \ve{=nu} (\srf{sec:ConInsPluEnc}).\footnote{
		There is also one example of insertion of /ɡw/ in my corpus
		after the alternate plural enclitic \ve{=enu/=uun}:
		\ve{oemetan} `dirty' + \ve{=uun} `{\ein}' + \ve{ii} `{\ii}' {\ra}
		\ve{oemeetn=uuŋg=ii} `the dirty one'}

\begin{exe}
	\ex{=ein + =V {\ra} =eiŋgwV\label{ex:=ein=gwX}}
		\sn{\stl{0.15em}\gw\begin{tabular}{rclclcll}
%			\ve{oemetan}	&+&\ve{=uun}&+&\ve{=ii}	&{\ra}&\ve{oemeetn=uu{\ng}gw=ii}&`the dirty (ones)'\\
			\ve{skora-m}	&+&\ve{=ein}&+&\ve{=ii}	&{\ra}&\ve{skoor-m=ei{\ng}gw=ii}&`your schooling'\\
			\ve{anah}			&+&\ve{=ein}&+&\ve{=aa}	&{\ra}&\ve{aanh=ei{\ng}gw=aa}		&`the children'\\
			%\ve{a-toup} 	& &					& &					&			&\ve{a-toup}							&								\\ \hhline{~}
			%\ve{noniʔ}		&+&\ve{=ein}&+&\ve{=aa}	&{\ra}&\ve{noinʔ=ei{\ng}gw=aa}	&`the disciples'\\
			\ve{a-toup noniʔ}		&+&\ve{=ein}&+&\ve{=aa}	&{\ra}&\ve{a-toup noinʔ=ei{\ng}gw=aa}	&`the disciples'\\
		\end{tabular}}
\end{exe}


\section{Historical development}\label{sec:HisDev}
In this section I present some comparative data from
other varieties of Meto which indicates
that Amarasi consonant insertion before vowel-initial enclitics
arose through fortition of an earlier glide.
This comparative data also provides the crucial
data which leads me to analyse vowel-initial enclitics
as containing at least two vowels.

In Amanuban the attachment of a vowel-initial
enclitic to a VV{\#} final host triggers insertion
of the glide /w/ after back vowels and /j/ after front vowels
Examples are given in \qf{ex:AmanuGLiIns} below.
Data comes from a speaker from Niki Niki
(central Amanuban) and Noemuke (south Amanuban).

\begin{exe}
	\ex{Amanuban glide insertion}\label{ex:AmanuGLiIns}
	\sn{\gw\begin{tabular}{rlllll}
		\ve{ai} 	&+&\ve{=ees}&{\ra}&\ve{ai\tbr{j}ees}		& `one fire' \\
		\ve{tei} 	&+&\ve{=ees}&{\ra}&\ve{tei\tbr{j}ees}	& `one (pile of) dung' \\
		\ve{oe} 	&+&\ve{=ees}&{\ra}&\ve{oe\tbr{j}ees}		& `one (body of) water' \\
		\ve{fee} 	&+&\ve{=ees}&{\ra}&\ve{fee\tbr{j}ees}	& `one wife' \\
		\ve{ao} 	&+&\ve{=ees}&{\ra}&\ve{ao\tbr{w}ees}		& `one (container of) slaked lime' \\
		\ve{too} 	&+&\ve{=ees}&{\ra}&\ve{too\tbr{w}ees}	& `one population' \\
		\ve{hau} 	&+&\ve{=ees}&{\ra}&\ve{hau\tbr{w}ees}	& `one tree/piece of wood' \\
		\ve{kiu} 	&+&\ve{=ees}&{\ra}&\ve{kiu\tbr{w}ees}	& `one tamarind tree' \\
		%\ve{} 	&+&\ve{=ees}&{\ra}&\ve{\tbr{j}=ee}	& `one' \\
	\end{tabular}}
\end{exe}

Consonant insertion in Amarasi after VV{\#} sequences
is probably a development from an Amanuban-like system,
with the addition of vowel assimilation 
and insertion of voiced obstruents rather than glides.

Regarding the voiced obstruents, Amarasi /\j/ and /ɡw/
result from fortition of *j > \ve{\j} and *w > \ve{gw}.
Other Amanuban/Amarasi cognates showing such glide fortition
include Amanuban \ve{ai\tbr{j}oo}, Amarasi \ve{ai\tbr{\j}oʔo} `casuarina tree'
and Amanuban \ve{nai\tbr{j}eeʔ} Amarasi \ve{nai\tbr{\j}eer} `ginger'.

In Amanuban when a vowel-initial enclitic is attached
to a CV{\#} final word the first vowel of the enclitic assimilates to the quality
of the final vowel of the host, the host undergoes metathesis,
and the final vowel of the host assimilates to the quality of the previous vowel.
Examples are given in \qf{ex:AmaVowAss} below.
Vowel assimilation does not otherwise affect nouns
after metathesis in Amanuban, thus \ve{fa\tbr{fi}} + \ve{anaʔ}
`small, baby' {\ra} \ve{fa\tbr{if} anaʔ} `piglet'.\footnote{
		The way the process exemplified in \qf{ex:AmaVowAss}
		may be connected with the metathesis with
		final non-syllabic glides attested in
		some varieties of Amanuban (see \srf{sec:OriMetAma})
		deserves further investigation.}

\begin{exe}
	\ex{Amanuban vowel assimilation}\label{ex:AmaVowAss}
	\sn{\gw\begin{tabular}{rlllll}
		\ve{faf\tbr{i}}	&+&\ve{=\tbr{e}es}&{\ra}&\ve{fa\tbr{afi}es}	& `one pig' \\
		\ve{uk\tbr{i}}	&+&\ve{=\tbr{e}es}&{\ra}&\ve{u\tbr{uki}es}		& `one banana tree' \\
		\ve{bes\tbr{i}}	&+&\ve{=\tbr{e}es}&{\ra}&\ve{be\tbr{esi}es}	& `one knife' \\
		\ve{mon\tbr{e}}	&+&\ve{=\tbr{e}es}&{\ra}&\ve{mo\tbr{one}es}	& `one husband' \\
		\ve{um\tbr{e}}	&+&\ve{=\tbr{e}es}&{\ra}&\ve{u\tbr{ume}es}		& `one house' \\
		\ve{nen\tbr{o}}	&+&\ve{=\tbr{e}es}&{\ra}&\ve{ne\tbr{eno}es}	& `one day' \\
		\ve{kol\tbr{o}}	&+&\ve{=\tbr{e}es}&{\ra}&\ve{ko\tbr{olo}es}	& `one bird' \\
		%\ve{non\tbr{o}}	&+&\ve{=\tbr{e}es}&{\ra}&\ve{no\tbr{ono}es}	& `one stem/stick' \\
		\ve{as\tbr{u}} 	&+&\ve{=\tbr{e}es}&{\ra}&\ve{a\tbr{asu}es}	& `one dog' \\
		\ve{han\tbr{u}} &+&\ve{=\tbr{e}es}&{\ra}&\ve{ha\tbr{anu}es}	& `one mortar' \\
		\ve{tef\tbr{u}}	&+&\ve{=\tbr{e}es}&{\ra}&\ve{te\tbr{efu}es}	& `one sugar-cane stalk' \\
		%\ve{} 	&+&\ve{=ees}&{\ra}&\ve{}e}es}	& `one' \\
	\end{tabular}}
\end{exe}

The first vowel of the enclitic in such examples
is often quite short, and in the case of /i/ sometimes reduced to a glide.
Thus, \ve{uki} `banana' + \ve{=ees} `one' {\ra} \ve{uukies} {\ra} [ˈʔʊːkĭɛs] {\tl} [ˈʔʊːkjɛs].

Regarding stems ending in /u/, there seems to be variation
between metathesis and vowel assimilation, as illustrated
above, and simple attachment of the enclitic with no further changes.
An example of the latter in my data
is \ve{fatu} + \ve{=ees} {\ra} \ve{fatu=ees} [ˈfatʊɛs]{\tl}[ˈfatʊwɛs] `one stone/rock'.

The vowel assimilations which take place in Amanuban
when a vowel-initial enclitic is attached to a CV{\#}
stem could also represent a precursor to the process
of consonant insertion and vowel assimilation in Amarasi.

A system intermediate between that of Amanuban,
with assimilation of the first vowel of the enclitic,
and that of Kotos Amarasi, with consonant insertion,
is attested in Ro{\Q}is Amarasi from Buraen.

In Buraen Ro{\Q}is Amarasi /b/ is inserted after back vowels
and /\j/ after front vowels.
The first vowel of the enclitic also assimilates
as conditioned by the quality of the vowels of the enclitic.
If the enclitic is \ve{=aa} `{\aa}' the first vowel undergoes
complete assimilation, while if the enclitic is \ve{=ii} `{\ii}'
\ve{=ee} `{\ee}/{\eeV}' or \ve{=ees} `one' the first vowel
assimilates to the backness, but not the height of the final vowel of the clitic host.
Examples are given in \qf{ex:Buraen} below.
I do not have sufficient data on the behaviour of other
enclitics in Buraen Ro{\Q}is after vowel-final hosts to state their behaviour.\footnote{
		There are also two examples in my Buraen Ro{\Q}is data from a single speaker
		in which consonant insertion does not take place
		after the phrase \ve{aan feto} `daughter' + \ve{=ee}
		{\ra} \ve{aan feet=oe} `the daughter' and + \ve{=ees}
		{\ra} \ve{aan feet=oes} `one daughter'.
		These examples probably represent a more conservative pattern.
		The same speaker has consonant insertion in other situations,
		e.g. \ve{noo tenu} + \ve{=ii} {\ra} \ve{noo teen\tbr{b}ui} `the third time'.}

\begin{exe}
	\ex{Buraen Ro{\Q}is consonant insertion and vowel assimilation}\label{ex:Buraen}
		\sn{\stl{0.4em}\gw\begin{tabular}{rlllll}
		\ve{kor\tbr{o}} 	&+&\ve{=\tbr{a}a}&{\ra}&\ve{ko\tbr{orbo}a}		& `the bird' \\
		\ve{nen\tbr{o}} 	&+&\ve{=\tbr{e}e}&{\ra}&\ve{ne\tbr{enbo}e}		& `the sky' \\
		\ve{n-top\tbr{u}} &+&\ve{=\tbr{e}e}&{\ra}&\ve{n-to\tbr{opbo}e}	& `receives it' \\
		\ve{nif\tbr{u}} 	&+&\ve{=\tbr{e}es}&{\ra}&\ve{ni\tbr{ifbo}es}	& `one thousand' \\
		\ve{aan fet\tbr{o}} 	&+&\ve{=\tbr{i}i}&{\ra}&\ve{aan fe\tbr{etbu}i}		& `the daughter' \\
		\ve{ten\tbr{u}} 	&+&\ve{=\tbr{i}i}&{\ra}&\ve{te\tbr{enbu}i}		& `the third' \\
		\ve{braf\tbr{i}} 	&+&\ve{=\tbr{a}a}&{\ra}&\ve{bra\tbr{af{\j}i}a}	& `sea cucumber' \\
		\ve{te\tbr{i}} 		&+&\ve{=\tbr{a}a}&{\ra}&\ve{te\tbr{e{\j}i}a}		& `the faeces' \\
		\ve{meʔ\tbr{e}} 	&+&\ve{=\tbr{a}a}&{\ra}&\ve{me\tbr{eʔ{\j}e}a}	& `the red ones' \\
		\ve{mon\tbr{e}} 	&+&\ve{=\tbr{a}a}&{\ra}&\ve{mo\tbr{on{\j}e}a}	& `the husband' \\
		\ve{fe\tbr{e}} 		&+&\ve{=\tbr{a}a}&{\ra}&\ve{fe\tbr{e{\j}e}a}		& `the wife' \\
	\end{tabular}}
\end{exe}

As in Amanuban, the first vowel of the enclitic 
is usually extremely short,
and in Buraen Ro{\Q}is when this vowel is back rounded 
it can be realised as a glide [w],
thus \ve{aan feto} `daughter' + \ve{=ii} {\ra}
\ve{aan feetbui} [ˌʔaˑnˈfɛːtbʊ̆i] {\tl} [ˌʔaˑnˈfɛːtbwi]. 

Assimilation of the first vowel of \ve{=aa} after front
vowels does not seem to be obligatory in Buraen Ro{\Q}is.
One example from my data is \ve{umi} `house' + \ve{=aa}
`{\aa}' {\ra} \ve{uum\j=aa} `the house' in my data.

Ro{\Q}is Amarasi from Tunbaun is similar,
though after back rounded vowels the first vowel
of the enclitic does not undergo assimilation,
and /ɡw/ is usually inserted.
Historically, the /ɡw/ at clitic boundaries in 
Tunabun and Kotos Amarasi may come from re-analysis of
[ɡ] and an initial back vowel of the following clitic,
though examples such as Kotos \ve{ai\j oʔo} + \ve{=esa}
`one{\U}' {\ra} \ve{ai\j ooʔgw=esa} `one casuarina tree'
rather than \ve{*ai\j ooʔg=osa} indicate that this
glide can no longer be analysed as an underlying vowel.

A system similar to that of Ro{\Q}is Amarasi
operates in the variety of Meto spoken in Oepaha,
though in this case I only have data for the enclitic \ve{=aa}
and one example of \ve{=ii}.
In Oepaha /b/ is inserted after /o/, /l/ after /e/
and /\j/ is inserted after /i/.
Examples are given in \qf{ex:Oepaha} below.
Assimilation of the first vowel of the enclitic does
not take place in all examples, though data is too
scarce to state any conditions.\footnote{
		Oepaha data is limited, coming from a single wordlist and text.
		Possessed nouns were usually cited with the
		first person inclusive pronoun \ve{hiit} as a default possessor.}

\begin{exe}
	\ex{Oepaha consonant insertion and vowel assimilation}\label{ex:Oepaha}
	\sn{\stl{0.4em}\gw\begin{tabular}{rlllll}
		\ve{kmi\tbr{i}} &+ \ve{=aa} {\ra}&\ve{hiti kmi\tbr{i{\j}i}a}&[hɪt̪ɪkˈmiːʒia]	&\emb{hiti-kmiij-ia.mp3}{\spk{}}{\apl}		& `our urine' \\
		\ve{te\tbr{i}} 	&+ \ve{=aa} {\ra}&\ve{hiit te\tbr{e{\j}i}a}	&[hiˈt̪ːeː{{\j}}ɪa]&\emb{hit-teej-ia.mp3}{\spk{}}{\apl}			& `our faeces' \\
		\ve{uk\tbr{i}} 	&+ \ve{=aa} {\ra}&\ve{u\tbr{uk{\j}i}a}			&[ˈʔʊːkʒɪa]			&\emb{uukj-ia.mp3}{\spk{}}{\apl}					& `banana' \\
		\ve{o\tbr{o}} 	&+ \ve{=aa} {\ra}&\ve{o\tbr{obo}a}					&[ˈʔɔːβwɐ]			&\emb{oob-oa.mp3}{\spk{}}{\apl}						& `bamboo' \\
		\ve{nen\tbr{o}} &+ \ve{=aa} {\ra}&\ve{ne\tbr{enb}aa}				&[ˈnɛːnbɐt̪ʊnːa]	&\emb{neenb-oa-tuunn-aa.mp3}{\spk{}}{\apl}& `sky' \\ \hhline{~}
							&								 			 &\,\ve{tuun-n=aa}					&[ˈnɛːnβɐt̪ʊnːa]	&\emb{neenb-oa-tuunn-aa2.mp3}{\spk{}}{\apl}& \\
		\ve{mon\tbr{e}} &+ \ve{=aa} {\ra}&\ve{hiit mo\tbr{onle}a}		&[hit̪̚ ˈmɔːnlɛa]	&\emb{hit-moonl-ea.mp3}{\spk{}}{\apl}			& `our husband' \\
		\ve{fe\tbr{e}} 	&+ \ve{=aa} {\ra}&\ve{hiit fe\tbr{el}aa}			&[hɪt̪ˈfɛːla]		&\emb{hit-feel-aa.mp3}{\spk{}}{\apl}			& `our wife' \\
		\ve{fe\tbr{e}} 	&+ \ve{=ii} {\ra}&\ve{fe\tbr{el}ii}					&			&& `the wife' \\
		%\ve{\tbr{}} 	&+&\ve{=aa}&{\ra}&\ve{\tbr{e}a}	& `' \\
	\end{tabular}}
\end{exe}

Finally, although slightly orthogonal to the development
of consonant insertion in Kotos Amarasi,
the processes described for Amanuban CV{\#}
stems and vowel-initial enclitics also affect stems
with a final glottal stop CVʔ{\#}.
The first vowel of the enclitic assimilates to the quality
of the final vowel of the host, the host undergoes metathesis,
and the final vowel of the host assimilates to the quality of the previous vowel.
Examples are given in \qf{ex:AmaVowAssGlot} below.\footnote{
		I do not have data from Amanuban for the behaviour
		of words with a final consonant other than the glottal stop.}

\newpage
\begin{exe}
	\ex{Amanuban vowel assimilation with /ʔ/}\label{ex:AmaVowAssGlot}
	\sn{\gw\begin{tabular}{rlllll}
		\ve{as\tbr{i}ʔ} 	&+&\ve{=\tbr{e}es}&{\ra}&\ve{a\tbr{a}sʔ=\tbr{i}es}		& `one flea' \\
		\ve{mas\tbr{i}ʔ} 	&+&\ve{=\tbr{e}es}&{\ra}&\ve{ma\tbr{a}sʔ=\tbr{i}es}		& `one packet of salt' \\
		\ve{sun\tbr{i}ʔ} 	&+&\ve{=\tbr{e}es}&{\ra}&\ve{su\tbr{u}nʔ=\tbr{i}es}		& `one sword' \\
		\ve{kbat\tbr{e}ʔ} &+&\ve{=\tbr{e}es}&{\ra}&\ve{kba\tbr{a}tʔ=\tbr{e}es}	& `one grub' \\
		\ve{ten\tbr{o}ʔ} 	&+&\ve{=\tbr{e}es}&{\ra}&\ve{te\tbr{e}nʔ=\tbr{o}es}		& `one egg' \\
		\ve{en\tbr{o}ʔ} 	&+&\ve{=\tbr{e}es}&{\ra}&\ve{e\tbr{e}nʔ=\tbr{o}es}		& `one door' \\
		\ve{es\tbr{u}ʔ} 	&+&\ve{=\tbr{e}es}&{\ra}&\ve{e\tbr{e}sʔ=\tbr{u}es}		& `one mortar' \\
		\ve{ʔsun\tbr{u}ʔ} &+&\ve{=\tbr{e}es}&{\ra}&\ve{ʔsu\tbr{u}nʔ=\tbr{u}es}	& `one spoon' \\
		%\ve{\tbr{}ʔ} 	&+&\ve{=\tbr{e}es}&{\ra}&\ve{\tbr{}ʔ=\tbr{e}es}	& `one' \\
	\end{tabular}}
\end{exe}

Ro{\Q}is Amarasi (both from Buraen and Tunbaun)
shows a similar process of vowel assimilation
when a vowel initial enclitic attaches to a CVʔ{\#} word.
The process in Ro{\Q}is is different as the first
vowel of the clitic retains its height,
with the exception of \ve{=aa} in which complete assimilation takes place.
Examples are shown in \qf{ex:RoqVowAssGlot} below.\footnote{
		The Ro{\Q}is data for stems whose final vowel is a front vowel
		is somewhat ambiguous as I only have one example
		in which the first vowel of the enclitic has clearly assimilated:
		\ve{atoniʔ} + \ve{=aa} {\ra} \ve{atoonʔ=ia} `the man'.}

\begin{exe}
	\ex{Ro{\Q}is Amarasi vowel assimilation with /ʔ/}\label{ex:RoqVowAssGlot}
	\sn{\gw\begin{tabular}{rlllll}
		\ve{n-sen\tbr{u}ʔ} 		&+&\ve{=\tbr{e}e}&{\ra}&\ve{n-se\tbr{e}nʔ\tbr{o}e}	& `replaces it' \\
		\ve{na-knin\tbr{u}ʔ} 	&+&\ve{=\tbr{e}e}&{\ra}&\ve{na-kni\tbr{i}nʔ\tbr{o}e}& `cleans it' \\
		\ve{na-ser\tbr{o}ʔ} 	&+&\ve{=\tbr{e}e}&{\ra}&\ve{na-se\tbr{e}rʔ\tbr{o}e}	& `mixes it' \\
		\ve{un\tbr{u}ʔ} 			&+&\ve{=\tbr{i}i}&{\ra}&\ve{uunʔ\tbr{u}i}					& `long ago' \\
		\ve{mnan\tbr{u}ʔ} 		&+&\ve{=\tbr{i}i}&{\ra}&\ve{mna\tbr{a}nʔ\tbr{u}i}		& `the length' \\
		\ve{met\tbr{o}ʔ} 			&+&\ve{=\tbr{i}i}&{\ra}&\ve{me\tbr{e}tʔ\tbr{u}i}		& `the dry land' \\
		\ve{mor\tbr{o}ʔ} 			&+&\ve{=\tbr{i}i}&{\ra}&\ve{mo\tbr{o}rʔ\tbr{u}i}		& `the yellow one' \\
		\ve{b{\j}akas\tbr{e}ʔ}&+&\ve{=\tbr{e}es}&{\ra}&\ve{b{\j}aka\tbr{a}sʔ\tbr{e}es}		& `one horse' \\
		\ve{na-suk\tbr{i}ʔ}		&+&\ve{=\tbr{e}e}&{\ra}&\ve{na-su\tbr{u}kʔ\tbr{e}e}	& `supports it' \\
		\ve{aton\tbr{i}ʔ} 		&+&\ve{=\tbr{i}i}&{\ra}&\ve{ato\tbr{o}nʔ\tbr{i}i}		& `the man' \\
		\ve{aton\tbr{i}ʔ} 		&+&\ve{=\tbr{a}a}&{\ra}&\ve{ato\tbr{o}nʔ\tbr{i}a}		& `the man' \\
		%\ve{\tbr{}ʔ} 	&+&\ve{=ees}&{\ra}&\ve{\tbr{}ʔ=\tbr{e}es}	& `one' \\
	\end{tabular}}
\end{exe}


The data in which the first vowel of the enclitic
undergoes assimilation to the final vowel of the host
provides the crucial evidence which has swayed me to analyse
all vowel-initial enclitics as containing two vowels,
rather than a single vowel as I proposed in my PhD thesis \citep{ed16b}.
Under an analysis in which these enclitics contain a double vowel,
this process can be explained as a case of the first
vowel of the enclitic undergoing assimilation.
However, if such enclitics contained only a single vowel,
it is difficult to explain the presence of the additional vowel in these examples.

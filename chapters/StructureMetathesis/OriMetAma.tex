\section{Origins of Amarasi metathesis}\label{sec:OriMetAma}
As discussed in \srf{sec:OriMorMet}, \cite{blga98,blga04} propose
a number of ways in which a language can acquire a synchronic process
of metathesis through a number of phonetically natural steps.
Under their account the kind of metathesis seen
in Amarasi is compensatory metathesis,
which arose originally in certain prosodically conditioned environments:

\begin{quote}
``Compensatory metatheses originate when VCV sequences
are pronounced with extreme coarticulation of one vowel,
resulting in a seepage or shift of that vowel to the other side of the medial consonant.
This extreme form of coarticulation occurs in syllables which are already long due to stress.
The peripheral unstressed vowel, whose cues are now primarily on the opposite side of the consonant,
withers into a reduced form, and is ultimately lost.
The migration of the peripheral vowel across the intervening
consonant into tonic position is complete.'' \hfill\citep[529]{blga98}
\end{quote}

Under this account a noun such as \ve{fatu} `stone' goes 
through a process like that illustrated in \qf{ex:fautu} below.
This process would only occur in certain prosodic environments,
with the end result that the forms \ve{fatu} and \ve{faut}
are found in different phonological environments.

\begin{exe}
	\ex{\ve{fatu} > *fautu > *faut\u{u} > \ve{faut}}\label{ex:fautu}
\end{exe}

In Kotos Amarasi only the first stage (\ve{fatu})
and the final stage (\ve{faut}) are attested.
If this is indeed the process that gave rise to metathesis in Amarasi,
we would expect to find data attesting the hypothesised medial stages.
Indeed, data from other varieties of Meto attests these stages.

\subsection{First intermediate stage}
Intermediate stages with an initial diphthong and final full vowel
(i.e. *fautu) are found in Ro{\Q}is Amarasi.
As discussed in \srf{sec:RoqAmaDip},
in Ro{\Q}is the U\=/form of certain consonant final roots
show spread of the final vowel to the first syllable.
The examples from \prf{tab:RoqStrVSloDip}
are repeated in \trf{tab:RoqStrVSloDip2},
which shows Kotos Amarasi U\=/forms, Ro{\Q}is Amarasi U\=/forms
and M\=/forms before enclitics (Chapter \ref{ch:PhoMet}).
%See \srf{sec:SurVVCVWor} for more discussion of such forms.

\begin{table}[ht]
	\centering\caption{Ro{\Q}is Amarasi diphthongisation}\label{tab:RoqStrVSloDip2}
	\begin{threeparttable}[b]
		\begin{tabular}{llll}\lsptoprule
				Kotos 								& Ro{\Q}is									&	Kotos/Ro{\Q}is				&		\\
				U\=/form								&	U\=/form										& M̿-form								&	gloss	\\ \midrule
				\ve{t\tbr{e}fis}			&	\ve{t\tbrtb{e}{\i}fik}		&	\ve{t\tbr{ei}fs/k=}		&	`roof'	\\
				\ve{m\tbr{a}sik}			&	\ve{m\tbrtb{a}{\i}sik}		&	\ve{m\tbr{ai}sk=}			&	`salt'	\\
				\ve{t\tbr{o}ʔis}			&	\ve{t\tbrtb{o}{\i}ʔis}		&	\ve{t\tbr{oi}ʔs=}			&	`trumpet'	\\
				\ve{h\tbr{u}nik}			&	\ve{h\tbrtb{u}{\i}nik}		&	\ve{h\tbr{ui}nk=}			&	`turmeric'	\\
				\ve{\tbr{a}net}				&	\ve{\tbrtb{a}{e}net}			&	\ve{\tbr{ae}nt=}			&	`needle'	\\
				\ve{r\tbr{o}ne-f}			&	\ve{r\tbrtb{o}{e}ne-f}		&	\ve{r\tbr{oe}n-f=}		&	`brain'	\\
				\ve{niis \tbr{e}no-f}	&	\ve{niis \tbrtb{e}{o}no-f}&	\ve{niis \tbr{eo}n-f=}&	`incisors'	\\
				\ve{n-ʔ\tbr{a}tor}		&	\ve{n-ʔ\tbrtb{a}{o}tor}		&	\ve{n-ʔ\tbr{ao}tr=}		&	`arrange'	\\
				\ve{s\tbr{i}ʔu-f}			&	\ve{s\tbrtb{\i}{u}ʔu-f}		&	\ve{s\tbr{iu}ʔ-f=}		&	`elbow'	\\
				\ve{\tbr{e}suk}				&	\ve{\tbrtb{e}{u}suk}			&	\ve{\tbr{eu}sk=}			&	`mortar'	\\
				\ve{m\tbr{a}nus}			&	\ve{m\tbrtb{a}{u}nus}			&	\ve{m\tbr{au}ns=}			&	`beetle vine'	\\
				\ve{p\tbr{o}nu-f}			&	\ve{p\tbrtb{o}{u}nu-f}		&	\ve{p\tbr{ou}n-f=}		&	`moustache'\su{†}	\\
			\lspbottomrule
		\end{tabular}
			\begin{tablenotes}
				\item [†] Kotos \ve{ponu-f} is `moustache'
									and Ro{\Q}is \ve{po͡unu-f} is `body hair'.
			\end{tablenotes}
		\end{threeparttable}
\end{table}

This diphthongisation is productive and, as discussed in \srf{sec:RoqAnaCCIniMod},
either this diphthongisation or complete metathesis
occurs in Ro{\Q}is Amarasi before modifiers which
begin with a consonant cluster.
Thus, the complete process of compensatory metathesis as
hypothesised by \citet{blga98} is attested
in a phrase such as \ve{umi} `house' + \ve{kbubuʔ} `round'.
Kotos Amarasi \ve{umi kbubuʔ} attests the first
stage, while Ro{\Q}is Amarasi \ve{u͡{\i}mi kbubuʔ} and \ve{uim kbubuʔ}
attest the intermediate and final stages.

Additionally, there is one instance in which diphthongisation
is attested before a modifier with a single consonant in my
corpus of Ro{\Q}is Amarasi texts.
This is the phrase \ve{rasi} `matter' + \ve{matsao-s} `marriage'
which occurs as \ve{ra͡{\i}si matsaos} `marriage arrangements' four
times in one of my Ro{\Q}is texts.
One of these examples is given in \qf{ex:RO-170830-1-06-54} below.

\begin{exe}
	\ex{\glll	\sf{{\j}adi} na-ʔuabaʔ r\tbrtb{a}{\i}\tbr{si} matsao-s=ii\j=ii\\
						\sf{{\j}adi} na-ʔuabaʔ r\tbr{asi} matsao-s=ii=ii\\
						so \na-speak matter marry-{\at}={\ii}={\reqt}\\
			\glt	`So they talk about the marriage arrangements'\txrf{RO-170830-1, 6.54}
						{\emb{RO-170830-1-06-54.mp3}{\spk{}}{\apl}}}\label{ex:RO-170830-1-06-54}
\end{exe}

Ro{\Q}is Amarasi attests spread of the final
vowel to the penultimate position creating a medial
diphthong. This is the first intermediate
stage which can give rise to synchronic metathesis.

\subsection{Second intermediate stage}
The second intermediate stage showing forms with
a reduced final vowel (i.e. *faut\u{u}),
is found in some varieties of South Amanuban,
Timaus, Fatule{\Q}u, Amfo{\Q}an and Kopas.
My discussion here focusses on the variety of
South Amanuban spoken in Se{\Q}i village.\footnote{
		In known varieties of Fatule{\Q}u and Amfo{\Q}an
		with such vowel reduction, it only affects words with a final
		back vowel /o/ or /u/. This is also the case
		for Timaus spoken in Sanenu.}

In Se{\Q}i Amanuban verbs and numerals are usually
cited in a form with a medial double vowel
corresponding to the penultimate vowel of the root
and a final non-syllabic or voiceless vowel
corresponding to the root final vowel.
A simple example is \ve{tenu} {\ra} \ve{teenu̯} `three'
More examples are given in \trf{tab:SouAmaMetFor} on the next page.
Such words also have metathesised forms,
such as \ve{tenu} {\ra} \ve{teun} `three'.
The only words in my Se{\Q}i Amanuban data which do not have
forms with a final non-syllabic vowel are those with final /a/
and a penultimate vowel other than /a/ such as \ve{nima} {\ra} \ve{niim} `five'.

\begin{table}[ht]
	\caption[Se{\Q}i Amanuban Citation Forms]
					{Se{\Q}i Amanuban Citation Forms\su{†}}\label{tab:SouAmaMetFor}
	\centering
		\begin{threeparttable}
			\begin{tabular}{llllll}	\lsptoprule
					&Se{\Q}i 	&							&							&Amarasi	&\\
		Root	&citation	&Phonetic			&							&M\=/form		&gloss\\ \midrule
	\ve{{\rt}mani}	&\ve{n-maani̯}		&[ˈn͡maˑnj]		&\emb{NB-Sei-nmaani.mp3}{\spk{}}{\apl}		&\ve{n-main}	&`laugh'\\
%	&&[naˑˈmnaˑsj̊]&\emb{NB-Sei-namnaasi.mp3}{\spk{}}{\apl}&&\\
%	\multirow{-2}{*}{\ve{{\rt}mnasi}}	&\multirow{-2}{*}{\ve{na-mnaasi̯}}	&[naˈmna̟ˑçː]
%	&\emb{NB-Sei-namnaasi2.mp3}{\spk{}}{\apl}	&\multirow{-2}{*}{\ve{na-mnais}}&\multirow{-2}{*}{`old'}\\
	\ve{{\rt}mnasi}	&\ve{na-mnaasi̯}	&[naˑˈmnaˑsj̊]	&\emb{NB-Sei-namnaasi.mp3}{\spk{}}{\apl}	&\ve{na-mnais}&`old'\\
									&								&[naˈmna̟ˑçː]	&\emb{NB-Sei-namnaasi2.mp3}{\spk{}}{\apl}	&&\\
	\ve{{\rt}honi}	&\ve{na-hooni̯}	&[naˈhoˑnj]		&\emb{NB-Sei-nahooni.mp3}{\spk{}}{\apl}		&\ve{na-hoin}	&`be born'\\
	\ve{{\rt}luli}	&\ve{n-luuli̯}		&[ˈnlʊːlʝ̥]		&\emb{NB-Sei-nluuli.mp3}{\spk{}}{\apl}		&\ve{}				&`burn'\\
	\ve{{\rt}hake}	&\ve{n-haake̯}		&[ˈnhaˑkɜ̥̆] 		&\emb{NB-Sei-nhaake.mp3}{\spk{}}{\apl}		&\ve{n-haek}	&`stand'\\
	\ve{{\rt}mate}	&\ve{n-maate̯}		&[ˈn͡maːt̪ə̥]		&\emb{NB-Sei-nmaate.mp3}{\spk{}}{\apl}		&\ve{n-maet}	&`die'\\
	\ve{{\rt}lole}	&\ve{na-loole̯}	&[naˈlɔˑlɜ̯̆]		&\emb{NB-Sei-naloole.mp3}{\spk{}}{\apl}		&\ve{}				&`far'\\
	\ve{{\rt}loʔe}	&\ve{t-looʔe̯}		&[ˈt̪lɔˑʔɛ̥̆]		&\emb{NB-Sei-tlooqe.mp3}{\spk{}}{\apl}		&\ve{}				&`swim'\\
	\ve{{\rt}paumaka}&\ve{paumaaka̯}	&[ˌpɐwˈmaˑkɐ̥̆]	&\emb{NB-Sei-paumaaka.mp3}{\spk{}}{\apl}	&\ve{n-paumaak}	&`near'\\
	\ve{{\rt}mahata}&\ve{n-mahaata̯}	&[ˈn͡maˑhaːt̪ə̥]	&\emb{NB-Sei-nmahaata.mp3}{\spk{}}{\apl}	&\ve{n-mahaat}&`itchy'\\
	\ve{{\rt}kiso}	&\ve{n-kiiso̯}		&[ˈnkiˑsw̥]		&\emb{NB-Sei-nkiiso.mp3}{\spk{}}{\apl}		&\ve{n-kius}	&`see'\\
	\ve{{\rt}tselo}	&\ve{na-tseelo̯}	&[naːˈt̪sɛlɔ̆]	&\emb{NB-Sei-natseelo.mp3}{\spk{}}{\apl}	&\ve{}				&`fall'\\
	\ve{{\rt}meno}	&\ve{n-meeno̯}		&[ˈn͡mɛːnɔ̯]		&\emb{NB-Sei-nmeeno.mp3}{\spk{}}{\apl}		&\ve{n-meon}	&`thirsty'\\
	\ve{{\rt}meto}	&\ve{n-meeto̯}		&[ˈnmɛˑt̪ɔ̯̊]		&\emb{NB-Sei-nmeeto.mp3}{\spk{}}{\apl}		&\ve{n-meot}	&`be dry'\\
	\ve{{\rt}nano}	&\ve{na-naano̯}	&[nɐˈnaˑnɔ̯]		&\emb{NB-Sei-nanaano.mp3}{\spk{}}{\apl}		&\ve{na-kaon}	&`braid'\\
	\ve{{\rt}toko}	&\ve{t-tooko̯}		&[ˈt̪ɔˑkw̥]			&\emb{NB-Sei-ttooko.mp3}{\spk{}}{\apl}		&\ve{t-took}	&`sit'\\
	\ve{{\rt}hitu}	&\ve{hiitu̯}			&[ˈhɪːt̪w̥]			&\emb{NB-Sei-hiitu.mp3}{\spk{}}{\apl}			&\ve{hiut}		&`seven'\\
	\ve{{\rt}inu}		&\ve{t-iinu̯}		&[ˈt̪ɪˑnw]			&\emb{NB-Sei-tiinu.mp3}{\spk{}}{\apl}			&\ve{t-iun}		&`drink'\\
	\ve{{\rt}matleʔu}	&\ve{n-matleeʔu̯}	&[ˈn͡mat̪l̥eːʔw]	&\emb{NB-Sei-nmatleequ.mp3}{\spk{}}{\apl}	&\ve{}		&`dream'\\
	\ve{{\rt}tenu}	&\ve{teenu̯}			&[ˈt̪eːnw]			&\emb{NB-Sei-teenu.mp3}{\spk{}}{\apl}			&\ve{teun}		&`three'\\
	\ve{{\rt}fanu}	&\ve{faanu̯}			&[ˈfaːnw]			&\emb{NB-Sei-faanu.mp3}{\spk{}}{\apl}			&\ve{faun}		&`eight'\\
	\ve{{\rt}ʔapu}	&\ve{na-ʔaapu̯}	&[naˈʔaːpw̥]		&\emb{NB-Sei-naqaapu.mp3}{\spk{}}{\apl}		&							&`pregnant'\\
	\ve{{\rt}mofu}	&\ve{n-moofu̯}		&[ˈn͡moˑfw]		&\emb{NB-Sei-nmoofu.mp3}{\spk{}}{\apl}		&\ve{n-mouf}	&`fall'\\
%	\ve{{\rt}}	&\ve{}	&[ˈ]	&\emb{NB-Sei-.mp3}{\spk{}}{\apl}		&\ve{}	&`'\\
					\lspbottomrule
				\end{tabular}
			\begin{tablenotes}
				\item [†] Words were elicited from a group of three speakers
									and several sound files have multiple speakers
									giving the word at the same time.
			\end{tablenotes}
		\end{threeparttable}
\end{table}

Phonetically, such final non-syllabic vowels are usually realised by the organs
of the mouth taking the position for the articulation of the root final vowel,
but without any subsequent vibration of the vocal cords.
When the final consonant is a voiceless plosive there is also a subsequent puff of air.
After other consonants there is not usually any additional sound or air expelled.
%In the case of final /o/ or /u/, it is visually quite clear that speakers round their lips
%after the root final consonant.

Final non-syllabic vowels were judged by my Se{\Q}i consultants
to be different from normal vowels, and forms
with a syllabic vowel were interpreted as U\=/forms.
In the case of \ve{na-naano̯} `braid' [nɐˈnaˑnɔ̯]
\emb{NB-Sei-nanaano.mp3}{\spk{}}{\apl} one of my consultants stated,
``There's clearly an \emph{o} but it doesn't leave [the mouth].''
\it{(kentara \emph{o}, tapi tidak keluar)}
and regarding \ve{na-maani̯} `laugh' [ˈn͡maˑnj] \emb{NB-Sei-nmaani.mp3}{\spk{}}{\apl}
they stated ``It's like there is an \emph{i}
at the end, but the \emph{i} is lost.'' \it{(ke ada \emph{i}
di belakang, tapi \emph{i}-nya hilang})

Based on current textual data it appears that the
Se{\Q}i Amanuban forms are an additional M\=/form
which might only be used phrase finally while normal M\=/forms with
metathesis are used phrase medially.\footnote{
		This is not to say that the use of each
		form in Se{\Q}i Amanuban is purely conditioned
		by phrase position. Instead, when an M\=/form is
		grammatically appropriate the selection of M\=/form
		might be determined by phrase position.}
However, a more comprehensive investigation of Se{\Q}i Amanuban
is needed to properly determine how U\=/forms and
different M\=/forms are used in this variety of Meto.
An example of each kind of M\=/form is given in
\qf{ex:NB-171026-4, 0.56} which shows a
normal M\=/form of \ve{{\rt}honi} `give birth'
medially and an M\=/form with non-syllabic vowel
phrase finally.

\begin{exe}
	%\ex{\glll	amaʔ =ma au enaʔ na-ho\tbr{in} =kau\\
	%					amaʔ =ma au enaʔ na-ho\tbr{ni} =kau\\
	%					father =and {\au} mother \na-born =kau\\
	%		\glt	`My father and mother gave birth to me' \txrf{NB-171026-4, 0.28}
	%					{\emb{NB-171026-4-00-28.mp3}{\spk{}}{\apl}}}\label{ex:NB-171026-4, 0.28}
	\ex{\glll	n-ak: ``ena hoo mu-ho\tbr{in}'' n-ak: ``au u-ho\tbr{oni̯}''\\
						n-ak \hp{``}ena hoo mu-ho\tbr{ni} n-ak \hp{``}au u-ho\tbr{ni}\\
						\n-ay \hp{``}mother {\hoo} \muu-born{\tbrM} \n-say \hp{``}{\au} \qu-born{\tbrM}\\
			\glt	`He said ``Mother, have you given birth?'', she said ``I've given birth''.'
						\\ \txrf{NB-171026-4, 0.56}
						{\emb{NB-171026-4-00-56.mp3}{\spk{}}{\apl}}}\label{ex:NB-171026-4, 0.56}
\end{exe}

In Timaus from Sanenu verbs and numerals with final /o/
and /u/ also have M\=/forms with a final non-syllabic vowel.
In Timaus these are the M\=/forms used in all phrase positions
and words which take such M\=/forms have
not been attested with a normal M\=/form derived by simple metathesis.
Three examples of such Timaus M\=/forms are
given in \qf{ex:FGT-171013-1, 0.31} and
\qf{ex:FGT-171016-2, 2.05} below.
The second instance in \qf{ex:FGT-171013-1, 0.31} is phrase
final while the other two instances in \qf{ex:FGT-171013-1, 0.31}
and \qf{ex:FGT-171016-2, 2.05} are phrase medial.

\begin{exe}
	\ex{\glll	atoniʔ te\tbr{enu̯}, bifee-l te\tbr{enu̯}\\
						atoniʔ te\tbr{nu} bifee-l te\tbr{nu}\\
						man three{\tbrM} woman-\tsc{u} three{\tbrM}\\
			\glt	`Three men and three women.' \txrf{FGT-171013-1, 0.31}
						{\emb{FGT-171013-1-00-31.mp3}{\spk{}}{\apl}}}\label{ex:FGT-171013-1, 0.31}
	\ex{\glll	hai m-e\tbr{eku̯} kotugw leʔ iin\\
						hai m-e\tbr{ku} koto-gw leʔ iin\\
						{\hai} \m-eat{\tbrM} hyacinth.bean-\tsc{u} {\req} {\ia}\\
			\glt	`We ate these hyacinth beans.' \txrf{FGT-171016-2, 2.05}
						{\emb{FGT-171016-2-02-05.mp3}{\spk{}}{\apl}}}\label{ex:FGT-171016-2, 2.05}
\end{exe}

Final non-syllabic vowels in varieties of Meto such as Se{\Q}i Amanuban
are intermediate between fully unmetathesised and fully metathesised forms.
However, while I have attested forms with
an intermediate sequence of two identical vowels and
final non-syllabic vowel such as \ve{tenu} {\ra}
\ve{teenu̯} `three', I do not yet have any clear examples
of forms with a final non-syllabic vowel and intermediate
sequence of two different vowels such as *teunu̯.
% In wordlist Se'i speakers were quite insistent and took great delight in the fact that their logat was like that,
% laughed at the fact that Om Nus's was different

\subsection{Loss of final consonants}\label{sec:LosFinCon}
The final process which needs to be accounted
for in the derivation of M\=/forms in Amarasi
is deletion of final consonants of nominals.
This is seen in the formation of M\=/forms of CVC{\#} final words
such as \ve{muʔit} {\ra} \ve{muiʔ} `animal' (\srf{sec:MetConDel}),
as well as VVC{\#} final words such as
\ve{kaut} {\ra} \ve{kau} `papaya' (\srf{sec:ConDel}).
There are several pieces of evidence
indicating that, diachronically, final consonant
deletion preceded metathesis.

\largerpage[-2]
Firstly, in some varieties of Meto
certain nominals derive their M\=/form
only by consonant deletion.
This is the case in Naitbelak and Nai{\Q}bais Amfo{\Q}an
in which all VVC{\#} final nominals, as well as CVC{\#}
final nominals whose final vowel is not /a/ mark attributive
modification simply through consonant deletion.
\mbox{Examples}{\pagebreak} are given in \trf{tab:AmfConDel}.
This system may attest an older system, with Amarasi then
applying metathesis to (newly) CV{\#} final words.\footnote{
		Naitbeak/Nai{\Q}bais Amfo{\Q}an attributive modification
		for vowel-final words is marked by a lack of consonant insertion,
		an example is \ve{fafi-\j} `pig' + \ve{anaʔ} `small, baby' {\ra} \ve{fafi anaʔ} `piglet'.
		CVC{\#} final words with final /a/ mark modification through
		consonant deletion and metathesis. An example is
		\ve{ekam} `pandanus' + \ve{neno-g} `day/sky' {\ra} \ve{eek neno-g} `wild pandanus'.
		See \cite{cu18} for more details.}


\begin{table}[ht]
	\caption{Naitbelak/Nai{\Q}bais Amfo{\Q}an consonant deletion}\label{tab:AmfConDel}
	\centering
		%\begin{tabular}{r@{\hspace{0.4em}}c@{\hspace{0.4em}}llr@{\hspace{0.4em}}c@{\hspace{0.4em}}l} \lsptoprule
		\begin{tabular}{rclllcl} \lsptoprule
			N\sub{1}							&+&N\sub{2}			&Phrase									&N\sub{1}			&+&N\sub{2}		\\\midrule
			\ve{muke\tbr{ʔ}}			&+&\ve{kase-l}	&\ve{muke kase-l}				&`citrus'			&+&`foreign'	\\
			\ve{muʔi\tbr{t}}			&+&\ve{fui-\j}	&\ve{muʔi fui-\j}				&`animal'			&+&`wild'			\\
			\ve{manu\tbr{s}}			&+&\ve{noo-f}		&\ve{manu noo-f}				&`betel vine'	&+&`leaves'		\\
			\ve{fee mnasi\tbr{ʔ}}	&+&\ve{amenat}	&\ve{fee mnasi amenat}	&`old woman'	&+&`sick'			\\
			\ve{kua\tbr{n}}				&+&\ve{tuaf}		&\ve{kua tuaf}					&`village'		&+&`person'		\\
			\ve{kau\tbr{t}}				&+&\ve{noo-f}		&\ve{kau noo-f}					&`papaya'			&+&`leaves'		\\
			%\ve{\tbr{}}			&+&\ve{}		&\ve{}			&`'	&+&`'			&`'\\
		\lspbottomrule
		\end{tabular}
\end{table}

Secondly, only nouns used attributively have final
consonant deletion. As discussed in
\srf{sec:ConDel (Higher Level)} and further exemplified in \srf{sec:ConFinVer},
verbs with a final consonant do not usually have distinct
M\=/forms in Amarasi; the presence of a final consonant blocks verbal metathesis.
This indicates that consonant deletion
is a necessary precondition for metathesis
to apply to consonant-final roots.

Thirdly, before CC-initial modifiers final consonant
deletion is the only marker of the M\=/form,
with a word like \ve{muʔit} `animal' taking
the {\MC}-form \ve{muʔi}. This consonant deletion is
discussed further in \srf{sec:CCIniMod} below.

Fourthly, despite the fact that Ro{\Q}is Amarasi
permits clusters of three consonants (\srf{sec:RoqAnaCCIniMod}),
the final consonant of VVC{\#} words is still deleted
before CC-initial modifiers, i.e. \ve{kniit} `crab' + \ve{snaen} `sand'
{\ra} \ve{knii{\gap}snaen} `horned ghost crab'.

These facts indicate that at an earlier
stage of Amarasi final consonants of nouns were
deleted before attributive modifiers;
in the same way as they still are before CC-initial modifiers,
or as is still found with the surface M\=/form of VVC{\#} final words.

While deletion of final consonants
can be analysed synchronically as a result of a prohibition against
final consonant clusters created after metathesis,
from a diachronic perspective deletion
of final consonants probably occurred first, with
this consonant deletion then opening the way for
metathesis to apply to consonant-final roots.

\subsection{Morphologisation of metathesis}
The Ro{\Q}is Amarasi data with stressed vowel diphthongisation
and varieties of Meto with M\=/forms with final non-syllabic vowels
attest intermediate stages between fully unmetathesised
forms (e.g. \ve{fatu}) and fully metathesised forms
(e.g. \ve{faut}) which are consistent
with the development of compensatory metathesis
as predicted by \citet{blga98}.

The final stage in the development of Amarasi metathesis was
for the prosodic environments in which each form occurred
to be reinterpreted as different morphological environments
(Chapters \ref{ch:SynMet} and \ref{ch:DisMet}).
This creation of a paradigm of morphological metathesis
probably partly led to the imposition of the CVCVC template to all words of the language
in order to provide the necessary machinery for consonant-vowel metathesis to operate
and thereby allow each word to fill both cells of the morphological paradigm.

Finally, recall from Chapter \ref{ch:SynchMet} that final CV {\ra} VC
metathesis occurs in several languages of Timor including Helong and Mambae.
The presence of metathesis in these three languages
is almost certainly due to historic contact.
In the case of Helong and Meto, such contact is still ongoing,
while Mambae and Meto are no longer in contact with one another.
Thus, while final CV {\ra} VC metathesis may have first arisen
according to the process of compensatory metathesis as described above,
it is unlikely that these processes occurred independently
in each of Meto, Helong, and Mambae. Instead they likely occurred
in one of these languages from which they then diffused into the others.
Which of these languages first acquired metathesis remains to be investigated.

\subsection{Consonant deletion}\label{sec:ConDel (Higher Level)}
Word-final consonants of nominals are deleted in the formation of the M\=/form.
This process is unique to the derivation of nominal M\=/forms
and does not affect other word classes, of which the largest is verbs.
This means that consonant-final verbs, such as \ve{na-tuin} `follow'
or \ve{n-boʔis} `praise' do not have basic M\=/forms.

\subsubsection{Metathesis and consonant deletion}\label{sec:MetConDel}
Words with a final consonant (CVC{\#}) derive their M\=/form through metathesis
of the penultimate consonant with the final vowel
and deletion of the final consonant.
The surface relationship between
\ve{muʔit} [ˈmʊʔit̪] {\ra} \ve{muiʔ} [ˈmʊ.iʔ] `animal'
is shown in \qf{as:muqit/muiq} below,
with more examples given in \qf{ex:VCVC->VVC}.

\begin{exe}
	\exa{\xy
		<0em,2.5cm>*\as{`animal'}="gloss",
		<2.5em,2cm>*\as{m}="u1",<3.5em,2cm>*\as{u}="u2",<4.5em,2cm>*\as{ʔ}="u3",<5.5em,2cm>*\as{i}="u4",<6.5em,2cm>*\as{t}="u5",<0em,2cm>*\as{U\=/form:}="u",
		<2.5em,1.5cm>*\as{C}="uC1",<3.5em,1.5cm>*\as{V}="uC2",<4.5em,1.5cm>*\as{C}="uC3",<5.5em,1.5cm>*\as{V}="uC4",<6.5em,1.5cm>*\as{C}="uC5",
		<2.5em,0.5cm>*\as{C}="mC1",<3.5em,0.5cm>*\as{V}="mC2",<4.5em,0.5cm>*\as{V}="mC4",<5.5em,0.5cm>*\as{C}="mC3",
		<2.5em,0cm>*\as{m}="m1",<3.5em,0cm>*\as{u}="m2",<4.5em,0cm>*\as{i}="m4",<5.5em,0cm>*\as{ʔ}="m3",<0em,0cm>*\as{M\=/form:}="m",
		{\ar@{->} "uC1"+D;"mC1"+U};{\ar@{->} "uC2"+D;"mC2"+U};{\ar@{->} "uC3"+D;"mC3"+U};{\ar@{->} "uC4"+D;"mC4"+U};
	\endxy}\label{as:muqit/muiq}
\end{exe}
%\newpage
\begin{exe}
	\ex{{\ldots}V\sub{1}C\sub{1}V\sub{2}C\sub{2}{\#} {\ra} {\ldots}V\sub{1}V\sub{2}C\sub{1}{\#}}\label{ex:VCVC->VVC}
	\sn{\stl{0.45em}\gw\begin{tabular}{rcll|rcll}
		 U\=/form					&			&\mc{2}{l}{M\=/form}					&U\=/form						&			&\mc{2}{l}{M\=/form}\\
		\ve{mu\tbr{ʔit}}&{\ra}&\ve{mu\tbr{iʔ}}&`animal'		&\ve{po\tbr{ʔon}}	&{\ra}&\ve{po\tbr{oʔ}}&`orchard'	\\
		\ve{te\tbr{nuk}}&{\ra}&\ve{te\tbr{un}}&`umbrella'	&\ve{o\tbr{ʔof}}	&{\ra}&\ve{o\tbr{oʔ}}&`pen, corral'	\\
		\ve{te\tbr{noʔ}}&{\ra}&\ve{te\tbr{on}}&`egg'			&\ve{ma\tbr{nus}}	&{\ra}&\ve{ma\tbr{un}}&`betel vine'	\\
		\ve{uk\tbr{um}}	&{\ra}&\ve{u\tbr{uk}}	&`cuscus'		&\ve{a\tbr{nah}}	&{\ra}&\ve{a\tbr{an}}	&`child'	\\
		\end{tabular}}
\end{exe}

Word-final consonant clusters are not permitted in Amarasi.
The consonant deletion in the M\=/form of VCVC{\#} final words
can be accounted for by language specific phonotactic constraints.
Metathesis occurs, resulting in a disallowed final consonant cluster
which is resolved by deletion of the final consonant.

\begin{table}[p]
	\centering\caption[Ro{\Q}is Final Consonant Clusters]
	{Ro{\Q}is Final Consonant Clusters\su{†}}\label{tab:RoqFinConClu}
		\begin{threeparttable}
			\begin{tabular}{lllll}\lsptoprule
				Kotos	&	Ro{\Q}is	&	Ro{\Q}is CC{\#}	&	Kotos/Ro{\Q}is	&		\\
				U\=/form	&	U\=/form	&	M\=/form	&	M\=/form	&	gloss	\\ \midrule
					\ve{batan}	&	\ve{batan}	&	\ve{baa\tbr{tn}}	&	\ve{baat}	&	`generation'	\\
					\ve{funan}	&	\ve{funun}	&	\ve{fuu\tbr{nn}}	&	\ve{fuun}	&	`month'	\\
					\ve{knapan}	&	\ve{knapan}	&	\ve{knaa\tbr{pn}}	&	\ve{knaap}	&	`butterfly'	\\
					\ve{manas}	&	\ve{manas}	&	\ve{maa\tbr{ns}}	&	\ve{maan}	&	`sun'	\\
					\ve{metan}	&	\ve{meten}	&	\ve{mee\tbr{tn}}	&	\ve{meet}	&	`black'	\\
					\ve{prenat}	&	\ve{prenet}	&	\ve{pree\tbr{nt}}	&	\ve{preen}	&	`government'	\\
					\ve{ranan}	&	\ve{ranan}	&	\ve{raa\tbr{nn}}	&	\ve{raan}	&	`road'	\\
					\ve{surat}	&	\ve{surut}	&	\ve{suu\tbr{rt}}	&	\ve{suur}	&	`paper'	\\
					\ve{uran}	&	\ve{urun}	&	\ve{uu\tbr{rn}}	&	\ve{uur}	&	`rain'	\\
					\ve{amfoʔan}	&	\ve{amfoʔan}	&	\ve{amfoa\tbr{ʔn}}	&	\ve{}	&	`Amfo{\Q}an'	\\
					\ve{benas}	&	\ve{fenes/fenas}	&	\ve{fee\tbr{ns}}	&	\ve{}	&	`machete'	\\
					\ve{bonak}	&	\ve{bonak}	&	\ve{boo\tbr{nk}}	&	\mc{2}{r}{`fragrant pandanus'} 	\\
					\ve{ekam}	&	\ve{erem/eram}	&	\ve{ee\tbr{rm}}	&	\ve{}	&	`wild pandanus'	\\
					\ve{koor{\gap}kapiten}	&	\ve{koor{\gap}kapitin}	&	\ve{koor{\gap}kapii\tbr{tn}}	&	\ve{}	&	`swiftlet'	\\
					\ve{kopan}	&	\ve{kopon/kopan}	&	\ve{koo\tbr{pn}}	&	\ve{}	&	`Kupang'	\\
					\ve{ksamun}	&	\ve{ksa͡unum}\su{‡}	&	\ve{ksau\tbr{nm}}	&	\ve{}	&	`starling'	\\
					\ve{oras}	&	\ve{oros}	&	\ve{oo\tbr{rs}}	&	\ve{}	&	`time'	\\
					\ve{ruman}	&	\ve{rumun}	&	\ve{ruu\tbr{mn}}	&	\ve{}	&	`empty'	\\
					\ve{ukum}	&	\ve{urum}	&	\ve{uu\tbr{rm}}	&	\ve{}	&	`cuscus'	\\
					\ve{ʔhenes}	&	\ve{henes}	&	\ve{hee\tbr{ns}}	&	\ve{}	&	`winter melon'	\\
					\ve{anin}	&	\ve{}	&	\ve{ai\tbr{nn}}	&	\ve{ain}	&	`wind'	\\
					\ve{menas}	&	\ve{}	&	\ve{mee\tbr{ns}}	&	\ve{meen}	&	`sickness'	\\
					\ve{krisan}	&	\ve{}	&	\ve{krii\tbr{sn}}	&	\mc{2}{r}{`red-cheeked parrot'} 	\\		
					\ve{meisʔokan}	&	\ve{}	&	\ve{meisiʔnoo\tbr{rn}}	&	\ve{}	&	`dark(ness)'	\\
					\ve{nini{\gap}tboran}	&	\ve{}	&	\ve{niin{\gap}tboo\tbr{rn}}	&	\ve{}	&	`dollarbird'	\\
					\ve{ninik}	&	\ve{}	&	\ve{nii\tbr{nk}}	&	\ve{}	&	`wax'	\\
					\ve{onen}	&	\ve{}	&	\ve{oe\tbr{nn}}	&	\ve{}	&	`prayer'	\\
					\ve{paah{\gap}pinan}	&	\ve{}	&	\ve{paah{\gap}pii\tbr{nn}}	&	\ve{}	&	`earth, world'	\\
					\ve{pinis}	&	\ve{}	&	\ve{pii\tbr{ns}}	&	\ve{}	&	`dew'	\\
				\lspbottomrule
			\end{tabular}%}
			\begin{tablenotes}
				\item [†]	Empty cells indicate forms which are
									currently unattested in my data.
				\item [‡]	One of Kotos \it{ksamun} or Ro{\Q}is \it{ksa͡unum}
									`startling' has undergone historical metathesis of the
									penultimate and final consonants.
									Ro{\Q}is \it{ksa͡unum} further has diphthongisation (\srf{sec:RoqAmaDip}).
			\end{tablenotes}
		\end{threeparttable}
\end{table}

In Ro{\Q}is Amarasi some
consonant-final words with certain phonological properties
(see \srf{sec:MforFinConClu}) have two M\=/forms:
an M\=/form derived in the same way as Kotos Amarasi
by metathesis and deletion of the final consonant,
and an M\=/form derived by metathesis but with preservation
of the final consonant cluster.
Examples of Ro{\Q}is nouns which have been attested with
this second CC-final M\=/form are given in \trf{tab:RoqFinConClu}
alongside Kotos U\=/forms and Kotos/Ro{\Q}is basic M\=/forms.

The Ro{\Q}is M\=/forms with final consonant deletion
are used to mark attributive modification
in the same way as their Kotos equivalents.
Ro{\Q}is M\=/forms with a final cluster are used
phrase finally with a discourse function like
Kotos verbs, where U\=/forms mark lack of resolution and
M\=/forms resolution (Chapter \ref{ch:DisMet}).

The data from Ro{\Q}is Amarasi in which certain
word-final consonant clusters are permitted phrase finally
indicates that the deletion of final consonants in the basic M\=/form
is not simply due to a general prohibition
against word-final clusters but rather against
word-final consonant clusters in medial members of the noun phrase.
Additional evidence for this comes from Amfo{\Q}an,
in which certain CVC{\#} final words delete their
final consonant without metathesis when modified.
This Amfo{\Q}an data is discussed in \srf{sec:LosFinCon}.

\subsubsection{Consonant deletion}\label{sec:ConDel}
Nominals which end in VVC{\#} in the U\=/form derive their
M\=/form by deletion of the final consonant.
The surface relationship between the segments of
\ve{kaut} [ˈkə.ʊt̪] {\ra} \ve{kau} [ˈkə.ʊ] `papaya' is shown in \qf{as:kaut/kau},
with more examples in \qf{ex:VVC->VV}.
Assimilation of /a/ does not occur in such M\=/forms.
In \srf{sec:AssOfA} I analyse this as being due to
a final empty C-slot in the M\=/form of these words.

\begin{exe}
	\exa{\xy
		<0em,2.5cm>*\as{`papaya'}="gloss",
		<2.5em,2cm>*\as{k}="u1",<3.5em,2cm>*\as{a}="u2",<4.5em,2cm>*\as{u}="u3",<5.5em,2cm>*\as{t}="u4",<0em,2cm>*\as{U\=/form:}="u",
		<2.5em,1.5cm>*\as{C}="uC1",<3.5em,1.5cm>*\as{V}="uC2",<4.5em,1.5cm>*\as{V}="uC3",<5.5em,1.5cm>*\as{C}="uC4",
		<2.5em,0.5cm>*\as{C}="mC1",<3.5em,0.5cm>*\as{V}="mC2",<4.5em,0.5cm>*\as{V}="mC4",
		<2.5em,0cm>*\as{k}="m1",<3.5em,0cm>*\as{a}="m2",<4.5em,0cm>*\as{u}="m4",<0em,0cm>*\as{M\=/form:}="m",
		{\ar@{->} "uC1"+D;"mC1"+U};{\ar@{->} "uC2"+D;"mC2"+U};{\ar@{->} "uC3"+D;"mC4"+U};
	\endxy}\label{as:kaut/kau} 
	\ex{{\ldots}VVC{\#} {\ra} {\ldots}VV{\#}}\label{ex:VVC->VV}
	\stl{0.4em}\gw\sn{\begin{tabular}{rcll|rcll}
		 U\=/form					&		&\mc{2}{l|}{M\=/form}		&U\=/form						&		&\mc{2}{l}{M\=/form}\\
		\ve{kau\tbr{t}}	&\ra&\ve{kau}		&`papaya'	&\ve{kua\tbr{n}}	&\ra&\ve{kua}		&`village'	\\
		\ve{bruu\tbr{k}}&\ra&\ve{bruu}	&`pants'	&\ve{ʔnae\tbr{f}}	&\ra&\ve{ʔnae}	&`old man'	\\
		\ve{knaa\tbr{ʔ}}&\ra&\ve{knaa}	&`beans'	&\ve{poe\tbr{s}}	&\ra&\ve{poe}		&`prawn/shrimp'	\\
		\ve{heu\tbr{m}}	&\ra&\ve{heu}		&`mango'	&\ve{noa\tbr{h}}	&\ra&\ve{noa}		&`coconut'	\\
		\end{tabular}}
\end{exe}

Unlike the consonant deletion seen for VCVC{\#} words (\srf{sec:MetConDel}),
this consonant deletion cannot be accounted for by surface phonotactic constraints of the language.
By positing medial empty C-slots this consonant deletion
can be analysed as an automatic result of metathesis
and a prohibition against word-final consonant clusters,
including clusters involving empty C-slots.

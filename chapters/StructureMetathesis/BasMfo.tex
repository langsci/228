\section{Basic M\=/form}\label{sec:BasMfo}
In this section I describe the structure of the basic M\=/form.
This is the form taken by nouns when modified by a nominal
which does not begin with a consonant cluster
and is the default form for vowel-final verbs.
While the functions of these M\=/forms are described
in full detail in chapters \ref{ch:SynMet} and \ref{ch:DisMet},
I provide here a brief overview as context for the following discussion.

Within the noun phrase M\=/forms are a construct form (\srf{sec:ConFor})
used when an attributive modifier occurs within the noun phrase.
Compare examples \qf{ex:StoAreBig} and \qf{ex:BigSto}.
Each phrase consists of the noun \ve{fatu} {\tl} \ve{faut} `stone'
followed by the modifier \ve{koʔu} `big, great'.
In \qf{ex:StoAreBig} \ve{fatu} `stone' is in the U\=/form
and the modifier has a predicative reading.
In \qf{ex:BigSto} \ve{faut} `stone' is in the M\=/form,
and the modifier has an attributive meaning.
The syntactic structures of each of these phrases
are represented with trees in \qf{tr:StoAreBig} and \qf{tr:BigSto} respectively.

\begin{multicols}{2}
	\begin{exe}
		\ex{\gll \brac{NP} fa\tbr{tu} \bracr{} \brac{NP} koʔu \bracr{}\\
						%	\hp{\brac{NP}} fatu {} {} koʔu {}\\
							{} stone {} {} big {}\\
				\glt \lh{\brac{NP} }`Stones are big.'}\label{ex:StoAreBig}
		\ex{\gll \brac{NP} fa\tbr{ut} koʔu \bracr{}\\
						%	\hp{\brac{NP}} fatu koʔu {}\\
							{} stone big {}\\
				\glt \lh{\brac{NP} }`(a) big stone'}\label{ex:BigSto}
	\end{exe}
\end{multicols}

\begin{multicols}{2}
	\begin{exe}
		\ex{\begin{forest} %where n children=0{tier=word}{}
			[S,[NP,[N,[\ve{fa\tbr{tu}}\\stlone{\tbrU}]]][NP,[N,[\ve{koʔu}\\big{\U}]]]]
		\end{forest}}\label{tr:StoAreBig}
		\ex{\begin{forest} %where n children=0{tier=word}{}
			[S,[\vp{NP}{\ldots},[,phantom]][NP,[N,[\ve{fa\tbr{ut}}\\stlone{\tbrM}]][N,[\ve{koʔu}\\big{\U}]]]]
		\end{forest}}\label{tr:BigSto}
	\end{exe}
\end{multicols}

A number of different phonological processes occur in
the formation of the basic M\=/form according to the shape of the U\=/form stem.
These processes include metathesis (\srf{sec:BasPro}),
consonant deletion (\srf{sec:MetConDel}, \srf{sec:ConDel}),
two kinds of vowel assimilation (\srf{sec:VowAss}), and vowel deletion (\srf{sec:VowDel}).

\subsection{Metathesis}\label{sec:BasPro}
When a root ends in VCV{\#},
the M\=/form is formed by metathesis of the final consonant-vowel sequence.
The surface relationship between the segments of
\ve{fatu} [ˈfat̪ʊ] {\ra} \ve{faut} [ˈfa.ʊt̪] `stone' is shown in \qf{as:fatu/faut1},
with more examples in \qf{ex:V1CV2->V1V2C}.

\begin{exe}
	\exa{\xy
		<0em,2.5cm>*\as{`stone'}="gloss",
		<2.5em,2cm>*\as{f}="u1",<3.5em,2cm>*\as{a}="u2",<4.5em,2cm>*\as{t}="u3",<5.5em,2cm>*\as{u}="u4",<0em,2cm>*\as{U\=/form:}="u",
		<2.5em,1.5cm>*\as{C}="uC1",<3.5em,1.5cm>*\as{V}="uC2",<4.5em,1.5cm>*\as{C}="uC3",<5.5em,1.5cm>*\as{V}="uC4",
		<2.5em,0.5cm>*\as{C}="mC1",<3.5em,0.5cm>*\as{V}="mC2",<4.5em,0.5cm>*\as{V}="mC4",<5.5em,0.5cm>*\as{C}="mC3",
		<2.5em,0cm>*\as{f}="m1",<3.5em,0cm>*\as{a}="m2",<4.5em,0cm>*\as{u}="m4",<5.5em,0cm>*\as{t}="m3",<0em,0cm>*\as{M\=/form:}="m",
		{\ar@{->} "uC1"+D;"mC1"+U};{\ar@{->} "uC2"+D;"mC2"+U};{\ar@{->} "uC3"+D;"mC3"+U};{\ar@{->} "uC4"+D;"mC4"+U};
	\endxy}\label{as:fatu/faut1}
\end{exe}

\begin{exe}
\ex{{\ldots}V\sub{1}CV\sub{2}{\#} {\ra} {\ldots}V\sub{1}V\sub{2}C{\#}}\label{ex:V1CV2->V1V2C}
	\sn{\gw\begin{tabular}{rcll|rcll}
		 U\=/form					&			&\mc{2}{l}{M\=/form}					&U\=/form						&			&\mc{2}{l}{M\=/form}	\\
		\ve{fi\tbr{ni}}	&{\ra}&\ve{fi\tbr{in}}	&`seed'		&\ve{ne\tbr{no}}	&{\ra}&\ve{ne\tbr{on}}	&`day; sky'\\
		\ve{be\tbr{si}}	&{\ra}&\ve{be\tbr{is}}	&`knife'	&\ve{kna\tbr{fo}}	&{\ra}&\ve{kna\tbr{of}}	&`mouse'\\
		\ve{fa\tbr{fi}}	&{\ra}&\ve{fa\tbr{if}}	&`pig'		&\ve{ko\tbr{ro}}	&{\ra}&\ve{ko\tbr{or}}	&`bird'\\
		\ve{o\tbr{ni}}	&{\ra}&\ve{o\tbr{in}}		&`bee'		&\ve{hi\tbr{tu}}	&{\ra}&\ve{hi\tbr{ut}}	&`seven'\\
		\ve{u\tbr{ki}}	&{\ra}&\ve{u\tbr{ik}}		&`banana'	&\ve{te\tbr{nu}}	&{\ra}&\ve{te\tbr{un}}	&`three'	\\
		\ve{re\tbr{ne}}	&{\ra}&\ve{re\tbr{en}}	&`field'	&\ve{fa\tbr{tu}}	&{\ra}&\ve{fa\tbr{ut}}	&`stone'\\
		\ve{ba\tbr{re}}	&{\ra}&\ve{ba\tbr{er}}	&`place'	&\ve{no\tbr{pu}}	&{\ra}&\ve{no\tbr{up}}	&`hole'\\
		\ve{no\tbr{pe}}	&{\ra}&\ve{no\tbr{ep}}	&`cloud'	&\ve{hu\tbr{tu}}	&{\ra}&\ve{hu\tbr{ut}}	&`louse'\\
		\end{tabular}}
\end{exe}

It is worth emphasising that in most cases the order
of the final consonant and vowel of the word is the only phonetic difference
between the U\=/form and the M\=/form of VCV{\#} final roots.
Metathesis is not accompanied by any reduction in the number of syllables
nor by any change in the placement of stress.\footnote{
		The only exceptions are words with identical penultimate
		and final vowels such as \ve{fini} [ˈfini] {\ra} \ve{fiin} [ˈfiːn] `seed',
		in which case there is a reduction in the number of phonetic syllables
		and thus arguably also in the placement of stress.
		As discussed in \srf{sec:DouVow}
		there is no basis for analysing sequence of two identical vowels
		differently from sequences of different vowels.}

Such metathesis applies to all VCV{\#} final roots,
with the exception of roots in which the final vowel is /a/ (\srf{sec:AssOfA})
or when the penultimate vowel is high and the final vowel is mid (\srf{sec:MidVowAss}).
Such roots undergo metathesis followed by vowel assimilation.

\section{Discourse-driven metathesis in Ro{\Q}is Amarasi}\label{sec:DisDriMetRoqAma}
There are a number of differences between Ro{\Q}is Amarasi
and Kotos Amarasi regarding discourse metathesis.
A full discussion of Ro{\Q}is Amarasi is beyond the scope of this book
and I provide here only a description and 
overview of some of the most salient points of difference,
without a consistent attempt to analyse or explain these differences.
Additionally, the observations in this section must be taken
as preliminary as I have spent much less time in Ro{\Q}is
speaking areas compared with Kotos speaking areas.
The main differences I have observed include the presence of M\=/forms with final
consonant clusters (\srf{sec:MforFinConClu}), the use of M\=/forms before
connectors (\srf{sec:MfoDepCoo}) and the use of U\=/forms in negation (\srf{sec:RoqNeg}).

\subsection{M\=/forms with final consonant clusters}\label{sec:MforFinConClu}
Certain consonant-final words are eligible
to undergo metathesis with a resulting
final consonant cluster in Ro{\Q}is Amarasi.
As a result such consonant-final words can have discourse
driven U\=/forms and M\=/forms in Ro{\Q}is Amarasi,
with U\=/forms marking lack of resolution
in a way which appears to be very similar to that of Kotos Amarasi
as described in the previous sections.

In texts from my oldest speaker from Batuna
-- Tonci Niti who is about 90 years old\footnote{
		Tonci Niti learnt both some Dutch and some
		Japanese at school, indicating that he was attending
		school prior to and during the second world war period,
		thus verifying his claim to be about 90 years old.} --
stems with a final consonant do not take U\=/forms
to mark lack of resolution as consistently
as in texts from younger speakers.
Thus, the use of U\=/forms for certain
consonant-final stems as described in this section 
may be a recent development in Ro{\Q}is Amarasi.

\subsubsection{Verbs}
For verbs, only stems with a word-final /n/ (i.e. CVn{\#})
are attested with CC-final M\=/forms in my Ro{\Q}is corpus.
CC-final M\=/forms are further only permitted for
/n/ final verbs which also fulfil at least one of the following
criteria: (1) the final vowel is /a/,
(2) the final vowel is identical to the penultimate vowel,
or (3) the penultimate consonant is also /n/.
When these phonotactic criteria are fulfilled,
CVn{\#} final verbs have U\=/forms
with discourse functions which appear to be similar
to those described for Kotos Amarasi in this chapter. 

A number of examples are given below.
Example \qf{ex:RO-170830-4, 4.03} is a kind of dependent coordination (\srf{sec:DepCoo})
in which the U\=/form \ve{n-kono=n} `continue' encodes an activity which
is resolved by that encoded by the M\=/form \ve{n-tuup=n} `sleep'.
The final consonant of the CC-final M\=/form is the plural enclitic
\ve{=n} and this is common in my Ro{\Q}is database. 

\begin{exe}
	\ex{\gll	{oka =te} ahh nu n-nao=n n-ko\tbr{no=n} en S{\i}͡uʔuf, \hspace{25mm} nu n-tu\tbr{up=n} ek nae =te, n-tuup=n ek aa\j=ee n-peen\\
						after.that {} {\he} \n-go={\einV} \n-keep{\tbrU}={\einV} {\on} Si{\Q}uf {} {\he} \n-sleep{\tbrM} {\ek} {\nee} ={\te} \n-sleep\M={\einV} {\ek} {\ia}={\ee} \n-not.want\\
			\glt	`After that they wanted to continue on to Si{\Q}uf, they wanted to sleep there, they didn’t want to sleep here.'
						\txrf{RO-170830-4, 4.03}{\emb{RO-170830-4-04-03.mp3}{\spk{}}{\apl}}}\label{ex:RO-170830-4, 4.03}
\end{exe}

In \qf{ex:RO-170829-1, 12.18} below, the U\=/form
\ve{na-peʔan} `raise, bring up, cultivate'
introduces two verbs that specify this concept further
and is resolved by the M\=/forms
of the parallel pair \ve{na-riikn na-peaʔn}.

\begin{exe}
	\ex{\glll	uis aafgw=ii, hiin naiʔ na-pe\tbr{ʔan}, na-mo͡{\i}ni-b, na-toro-b, na-ri\tbr{ikn}, na-pe\tbr{aʔn}.\\
					usif afu=ii hini naiʔ na-peʔan, na-moni-b, na-toro-b, na-rikin, na-peaʔn.\\
					king ground={\ii} {\iin} then \na-raise{\tbrU} \na-live-{\b} \na-sprout-{\b} \na-raise{\tbrM} \na-raise{\tbrM}\\
		\glt	`The lord of earth (God), he raised (life), he gave life [doublet], and raised (it up) [doublet].'
					\txrf{RO-170829-1, 12.18}{\emb{RO-170829-1-12-18.mp3}{\spk{}}{\apl}}}\label{ex:RO-170829-1, 12.18}
\end{exe}

An example of the parallel pair \ve{na-rikin na-peʔan} `raise'
with an initial U\=/form and CC-final M\=/form
is given in \qf{ex:RO-170829-1, 5.34} below.
This is an example of U\=/forms and M\=/forms
being used to express poetic parallelism (\srf{sec:PoePar}).

\begin{exe}
	\ex{\glll	\sf{karna} Uisneengw=ii, eseʔ {} na-ri\tbr{kin} na-pe\tbr{aʔn}\\
						\sf{karna} Uisneno=ii ees heʔ na-rikin na-peʔan\\
						because God={\ii} one {\req} \na-raise{\tbrU} \na-raise{\tbrM}\\
			\glt	`Because God is the one who raised up (life).'
						\txrf{RO-170829-1, 5.34}{\emb{RO-20170829-1-05-34.mp3}{\spk{}}{\apl}}}\label{ex:RO-170829-1, 5.34}
\end{exe}

\subsubsection{Nouns}
As discussed in \srf{sec:MetConDel} certain Ro{\Q}is
Amarasi nouns have two M\=/forms:
one formed by metathesis with deletion of the final
consonant and one with retention of the final consonant.
An example is \ve{ranan} `road', with the M\=/form \ve{raan}
and CC-final M\=/form \ve{raann}.

Nouns which are attested with such CC-final M\=/forms
in my data fulfil either set of criteria given in
\qf{ex:ResRoqCCMforOpt1} or \qf{ex:ResRoqCCMforOpt2}.
More data on Ro{\Q}is Amarasi will probably 
lead to changes and refinements of these criteria.\footnote{
		Every Ro{\Q}is noun which has been attested with a CC-final M-form in
		my current data is given in \trf{tab:RoqFinConClu}
		on page \pageref{tab:RoqFinConClu},
		though this table does not include forms which are consonant
		final due to attachment of a nasal suffix.}

\begin{exe}
	\ex{Nouns with CC-final M\=/forms in Ro{\Q}is Amarasi}\label{ex:ResRoqCCMfor}
	\begin{xlist}
		\ex{Option 1:}\label{ex:ResRoqCCMforOpt1}
			\begin{xlist}
				\ex{The penultimate and final consonants are both nasals}
			\end{xlist}
		\ex{Option 2:}\label{ex:ResRoqCCMforOpt2}
			\begin{xlist}
				\ex{The penultimate consonant is /n/ or /r/}
				\ex{The final consonant is not /ʔ/ or /h/}
				\ex{The final consonant is not a non-nasal suffix}
				\ex{The final vowel is either /a/ or identical to the penultimate vowel}
			\end{xlist}
	\end{xlist}
\end{exe}

For nouns which fit either set of criteria in \qf{ex:ResRoqCCMfor},
the normal M\=/form (with deletion of the final consonant)
is used as a construct form in the same way as in Kotos Amarasi,
as discussed in Chapter \ref{ch:SynMet},
while the other forms appear to be used to mark discourse structures.
That is, in Ro{\Q}is Amarasi a U\=/form of a noun which
fits the criteria in \qf{ex:ResRoqCCMfor} marks an unresolved
situation while a CC-final M\=/form marks a resolved situation.

This also means that the CC-final M\=/forms of such nouns
are usually the default forms which are given as the citation
forms and in simple declarative clauses.
Two examples of nouns with CC-final M\=/forms in simple declarative clauses
are given in \qf{ex:RO-170830-5, 2.24} and \qf{ex:RO-170830-1, 5.16} below.

\begin{exe}
	\ex{\glll	ees n-reek nu n-nao n-koaʔ=siin nu na-nena-ʔ=siin pre\tbr{ent}.\\
						esa n-reka nu n-nao n-koaʔ=sini nu na-nena-ʔ=sini prenat\\
						{\es} \n-order {\he} \n-go \n-call={\siin} {\he} {\nat-listen-\qV=\siin} instruction\\
			\glt	`he ordered (him) to go call them to have them listen to instructions.'
						\txrf{RO-170830-5, 2.24}{\emb{RO-170830-5-02-24.mp3}{\spk{}}{\apl}}}\label{ex:RO-170830-5, 2.24}
	\ex{\glll	hiin u\tbr{ab-n} \sf{se{\tS}ara} \sf{umum} neme n-tua k heʔ ai\\
						hini uaba-n \sf{se{\tS}ara} \sf{umum} nema n-tua ek heʔ ai\\
						{\iin} speech{\tbrM-\N} manner general {\nema} \n-finish {\ek} {\req} {\ia}\\
			\glt	`Its story finishes in a general manner here.'
						\txrf{RO-170830-1, 5.16}{\emb{RO-170830-1-05-16.mp3}{\spk{}}{\apl}}}\label{ex:RO-170830-1, 5.16}
\end{exe}

An example of the U\=/form of such a noun marking lack of
resolution is given in \qf{ex:RO-170827-3, 3.14-3.20} below.
This is an example of tail-head linkage (\srf{sec:TaiHeaLin}).
The U\=/form tail \ve{ranan} `road'  in \qf{ex:RO-170827-3, 3.14}
is resolved by the M\=/form in \qf{ex:RO-170827-3, 3.18} which
then introduces an unexpected event in \qf{ex:RO-170827-3, 3.20}.

\begin{exe}
	\ex{A female leader supervises road construction:
			\txrf{RO-170827-3}{\emb{RO-170827-3-03-14-03-20.mp3}{\spk{}}{\apl}}}\label{ex:RO-170827-3, 3.14-3.20}
	\begin{xlist}
		\ex{\gll	n-nao en preent=ee nu na-taah mepu ra\tbr{nan}.\\
							%n-nao en prenet=ee nu na-taha mepu ranan\\
							\n-go {\on} government={\ee} {\he} \na-answer work road{\tbrU}\\
				\glt	`(She should) go to the government to report on the road work,' \txrf{3.14}}\label{ex:RO-170827-3, 3.14}
		\ex{\glll	n-nao n-toup mepu ra\tbr{ann}.\\
							n-nao n-topu mepu ranan\\
							\n-go \n-receive work road{\tbrM}\\
				\glt	`(She should) go and get road work.' \txrf{3.18}}\label{ex:RO-170827-3, 3.18}
		\ex{\gll {nai =te}, hiin moon\j=ii ees ka-nao-t=ii\\
							and.then {\iin} husband={\ii} {\esc} {\at}-go-{\at=\ii}\\
				\glt `And then her husband was the one who went (\emph{lit.} the goer)'
							\txrf{3.20}}\label{ex:RO-170827-3, 3.20}
	\end{xlist}
\end{exe}

Two examples of CC-final nominal M\=/forms in parallelism are given in
\qf{ex:RO-170829-1, 14.18} and \qf{ex:RO-170827-1, 3.09} below.
Each of these examples consists of two juxtaposed
noun phrases each of which encodes part of a single
entity construed as a complete whole.
In each case the initial U\=/form is resolved by
the following CC-final M\=/form.

\begin{exe}
	\ex{\glll	naiʔ na-snasa-b u\tbr{run} a\tbr{inn} =ama,\\
						naiʔ na-snasa-b uran anin =ma,\\
						then \na-stop-{\b} rain{\tbrU} wind{\tbrM} =and\\
			\glt	`Then (he) stopped the rain and the wind{\ldots}'
						\txrf{RO-170829-1, 14.18}{\emb{RO-170829-1-14-18.mp3}{\spk{}}{\apl}}}\label{ex:RO-170829-1, 14.18}
	\ex{\glll	naiʔ ka-moeʔ ahh neno tu\tbr{nu}-\tbr{n} paah pi\tbr{in}-\tbr{n}\\
						naiʔ ka-moʔe-t {} neno tuna-n paha pina-n\\
						then {\at}-make {} sky top{\tbrU}-{\N} world below{\tbrM}-{\N}\\
			\glt	`He (God) was the maker of heaven and earth.'
						\txrf{RO-170827-1, 3.09}{\emb{RO-170827-1-03-09.mp3}{\spk{}}{\apl}}}\label{ex:RO-170827-1, 3.09}			
\end{exe}

\subsection{Dependent coordination}\label{sec:MfoDepCoo}
U\=/forms are not obligatory in Ro{\Q}is Amarasi before the connectors \ve{=ma} and \ve{=te}
when the event preceding the connector is dependent on the next event for its resolution.
An example each of an M\=/form before \ve{=te} and \ve{=ma}
are given in \qf{ex:RO-170824-1, 1.46} and \qf{ex:RO-170820-1, 3.38} respectively below.

\begin{exe}
	\ex{\glll	\sf{tapi} karu n-mama =maeʔ, ahh n-me\tbr{up} =te, en na-peeh.\\
						\sf{tapi} karu n-mama =maeʔ {} n-mepe =te en na-pehe\\
						but if \n-chew ={\maeq} {} \n-work{\tbrM} ={\te} like \na-lazy\\
			\glt	`But if they don't chew betel nut, then when they work it's like they're lazy'
						\txrf{RO-170824-1, 1.46}{\emb{RO-170824-1-01-46.mp3}{\spk{}}{\apl}}}\label{ex:RO-170824-1, 1.46}
	\ex{\glll	\sf{lalu}, siin n-fe\tbr{en} =ma n-nao =heen\\
						\sf{lalu} sini n-fena =ma n-nao =hena\\
						then {\siin} \n-get.up ={\ma} \n-go ={\een}\\
			\glt	`Then they got up and left.'
						\txrf{RO-170820-1, 3.38}{\emb{RO-170820-1-03-38.mp3}{\spk{}}{\apl}}}\label{ex:RO-170820-1, 3.38}
\end{exe}

In fact, U\=/forms are extremely rare before the connectors
\ve{=ma} and \ve{=te} in my Ro{\Q}is Amarasi corpus,
with only about a dozen examples out of more than 500 examples
of \ve{=ma} and \ve{=te}.

When a CVC{\#} final stem occurs before a connector,
the connector usually takes its vowel-initial
allomorph \ve{=ama} or \ve{=ate} in Ro{\Q}is Amarasi (\srf{sec:SenEnc}),
except when the final consonant of the host is a glottal stop /ʔ/.
While in Kotos Amarasi these vowel-initial allomorphs optionally
trigger phonologically conditioned metathesis (Chapter \ref{ch:PhoMet})
in Ro{\Q}is Amarasi these vowel-initial allomorphs \emph{obligatorily} trigger metathesis.
Examples of CVC{\#} final stems before connectors in Ro{\Q}is Amarasi
are given in \qf{ex:RO-170830-4, 7.15}--\qf{ex:RO-170829-1, 13.10} below.

\begin{exe}
	\ex{\glll	ees {oka =te}, n-ma-ro\tbr{im}=\tbr{n} =ate n-matsao=n \sf{karna},\\
						esa {oka =te}, n-ma-romi=n =te n-matsao=n \sf{karna},\\
						one then \n-\mak-like{\tbrMv}={\einV} ={\te} \n-marry={\einV} because\\
			\glt	`then if they like one another, they get married because{\ldots}'
						\txrf{RO-170830-4, 7.15}{\emb{RO-170830-4-07-15.mp3}{\spk{}}{\apl}}}\label{ex:RO-170830-4, 7.15}
	\ex{\glll	{oke =te}, a|n-ho\tbr{ok}=\tbr{n} =ama n-tofo=n\\
						{oke =te} {\a}n-hoka=n =ma n-tofa=n\\
						then {\a\n}-call.up{\tbrMv}={\einV} =and \n-weed={\einV}\\
			\glt	`Then (people) were called up and (they) weeded.'
						\txrf{RO-170902-1, 2.04}{\emb{RO-170902-1-02-04.mp3}{\spk{}}{\apl}}}\label{ex:RO-170902-1, 2.04}
	\ex{\glll	Uisneno hiin pre\tbr{ent} =am hiin ka͡{\i}bin\\
						Uisneno hini prenat =ma hini kabin\\
						God {\iin} instruction{\tbrMv} =and {\iin} word\\
			\glt	`God's words and instructions.'
						\txrf{RO-170829-1, 13.10}{\emb{RO-170829-1-13-10.mp3}{\spk{}}{\apl}}}\label{ex:RO-170829-1, 13.10}
\end{exe}

\subsection{Negation}\label{sec:RoqNeg}
The normal negation strategy in Ro{\Q}is Amarasi
is with \ve{=maeʔ}, which occurs after the negated predicate.\footnote{
		The proclitic \ve{ka=} also occasionally occurs as a negator
		in Ro{\Q}is Amarasi. I have nine examples of \ve{ka=}
		in my Ro{\Q}is corpus against 99 examples of \ve{=maeʔ}.}
Vowel final stems directly followed by \ve{=maeʔ}
are only attested in the U\=/form in my Ro{\Q}is Amarasi data.
Examples are given in \qf{ex:RO-170820-2, 1.12}--\qf{ex:RO-170821-1, 14.32} below.

\begin{exe}
	\ex{\gll	n-tui na-hi\tbr{ni} =maeʔ, n-rees na-hi\tbr{ni} =maeʔ.\\
						\n-write \na-know{\tbrU} ={\maeq} \n-read \na-know{\tbrU} ={\maeq}\\
			\glt	`He didn't know how to read or write.'
						\txrf{RO-170820-2, 1.12}{\emb{RO-170820-2-01-12.mp3}{\spk{}}{\apl}}}\label{ex:RO-170820-2, 1.12}
	\ex{\gll	meseʔ hiin na-to\tbr{na} =maeʔ\\
						but {\iin} \na-tell{\tbrU} ={\maeq}\\
			\glt	`But he didn't tell (us).'
						\txrf{RO-170820-1, 8.24}{\emb{RO-170820-1-08.24.mp3}{\spk{}}{\apl}}}\label{ex:RO-170820-1, 8.24}
	\ex{\gll	au ku-sboo =t, \sf{berarti} au bisa ku-\tbr{ha} maeʔ. \\
						{\au} \qu-smoke {=\te} meaning {\au} can \qu-eat{\tbrU} ={\maeq} \\
			\glt	`If I smoke, that means I can't (afford to) eat.'
						\txrf{RO-170821-1, 14.32}{\emb{RO-170821-1-14-32.mp3}{\spk{}}{\apl}}}\label{ex:RO-170821-1, 14.32}
\end{exe}

However, when a stem with a final consonant other than
a glottal stop /ʔ/ is followed by \ve{=maeʔ},
the negator takes a vowel-initial form \ve{=amaeʔ}, thus triggering
automatic metathesis on the preceding word (Chapter \ref{ch:PhoMet}).
Two examples are given in \qf{ex:RO-170917-1, 3.22} and \qf{ex:RO-170917-1, 8.16} below.

\begin{exe}
	\ex{\glll	ma heʔ uunʔ=ui t-iit so\tbr{iʔ}-\tbr{t} =amaeʔ\\
						ma heʔ unuʔ=ii t-ita soʔi-t =maeʔ\\
						and {\req} long.ago={\ii} \tg-exist count{\tbrMv}-{\at} ={\maeq}\\
			\glt	`But long ago there wasn't any counting.'
						\txrf{RO-170917-1, 3.22}{\emb{RO-170917-1-03-22.mp3}{\spk{}}{\apl}}}\label{ex:RO-170917-1, 3.22}
	\ex{\glll	na-baar=n =am ne\tbr{em}=\tbr{n} =amaeʔ\\
						na-bara=n =ma nema=n =maeʔ\\
						\na-stay={\einV} =and {\nema\tbrMv=\einV} ={\maeq}\\
			\glt	`(they) stayed and didn't come back.'
						\txrf{RO-170917-1, 8-16}{\emb{RO-170917-1-08-16.mp3}{\spk{}}{\apl}}}\label{ex:RO-170917-1, 8.16}
\end{exe}

\section{Interactional metathesis alternations}\label{sec:IntUnm}
Another use of discourse U\=/forms is in conversation
to maintain interaction between speakers.
By using a U\=/form in conversation, the speaker flags
that s/he considers the communicative act unresolved.
This provides an opportunity for other participants
to make their own contribution and resolve the U\=/form.
In my corpus there are 53 instances of U\=/forms
which are intended to elicit a response from the addressee.
This frequency is discussed in more detail in \srf{sec:FreUfoCon}.

\subsection{Question and answer}\label{sec:Q/A}
The clearest example of U\=/forms being used in
interactions between speakers is in question-answer pairs.
U\=/forms are used to ask questions and M\=/forms
are used to answer such questions.
The normal structure of an Amarasi question-answer pair is given in \qf{ex:Q/A} below.
The question and answer usually contain identical verbs,
with the question U\=/form being resolved by an M\=/form answer.
A typical example is given in \qf{ex:02/08/13, p.20}.
Here, and throughout this section, Greek letters
are used to mark different speakers.

\begin{exe}
	\ex{\xytext{Speaker\sub{1}:&\fbox{U\=/form}\xybarconnect[2][-](D,D){2}\xybarconnect[2][->]{2}&Speaker\sub{2}:&\fbox{M\=/form}}}\label{ex:Q/A}
\end{exe}

\begin{multicols}{2}
	\begin{exe}
		\ex{Climbing a steep hill:}\label{ex:02/08/13, p.20}
			\begin{xlist}
				\ex[α:]{\gll	hoo mu-be\tbr{ʔi}?\\
									%	hoo mu-beʔi\\
									{\hoo} \muu-capable{\tbrU}\\
						\glt	`Can you do it?'}
			\end{xlist}
		\sn{ \txrf{observation 02/08/13, p.20}}
			\begin{xlist}
				\exi{b.}[β:]{\gll	au u-be\tbr{iʔ}!\\
									%	au u-beʔi\\
									{\au} \qu-capable{\tbrM}\\
						\glt	`Yes, I can!'}
			\end{xlist}
	\end{exe}
\end{multicols}

Such question-answer pairs are similar to tail-head linkage (\srf{sec:TaiHeaLin})
or poetic parallelism (\srf{sec:PoePar}), with the difference
that the U\=/form/M\=/form doublet is constructed by multiple speakers.
A U\=/form question must be complemented by an M\=/form answer
and a U\=/form answer is judged as infelicitous.
This is shown in (\ref{ex:02/08/13, p.20}′) below,
which can be compared with grammatical \qf{ex:02/08/13, p.20} above.

\begin{multicols}{2}
	\begin{exe}
		\exp{ex:02/08/13, p.20}{Elicitation:\label{ex:03/10/14 p.112}}
			\begin{xlist}
				\ex{\gll	hoo mu-beʔi?\\
									%	hoo mu-beʔi\\
									{\hoo} \muu-capable{\U}\\
						\glt	`Can you do it?'}
			\end{xlist}
		\sn{\txrf{elicit. 03/10/14 p.112}}
			\begin{xlist}
				\exi{b.}[{\#}]{\gll	au u-beʔi\\
									%	au u-beʔi\\
									{\au} \qu-capable{\U}\\
						\glt	`Yes, I can!'}
			\end{xlist}
	\end{exe}
\end{multicols}

Two examples of question{\U}-answer{\M} pairs from recorded
conversations are given in \qf{ex:130914-1, 0.20-0.21}
and \qf{ex:130909-6, 3.23-3.25} below.
In each example a question posed in the U\=/form is answered
by another speaker with an M\=/form version of the same verb.

\begin{exe}
	\ex{Weaving cloth: \txrf{130914-1} {\emb{130914-1-00-20-00-21.mp3}{\spk{}}{\apl}}}\label{ex:130914-1, 0.20-0.21}
		\begin{xlist}
			\ex[α:]{\glll	he t-fu\tbr{tu}?\\
										he t-futu\\
										{\he} \t-bind{\tbrU}\\
							\glt	`Should we tie it?' \txrf{0.20}}
			\ex[β:]{\glll	t-fu\tbr{ut}, reʔ mutiʔ reʔ ia.\\
										t-futu reʔ mutiʔ reʔ ia\\
										\t-bind{\tbrM} {\req} white {\req} {\ia}\\
							\glt	`(Yes,) we tie it. The white one which is here.' \txrf{0.21}}
		\end{xlist}
	\ex{Inquiring about family: \txrf{130909-6} {\emb{130909-6-03-23-03-25.mp3}{\spk{}}{\apl}}}\label{ex:130909-6, 3.23-3.25}
		\begin{xlist}
		\ex[α:]{\glll	ehh, n-fain=n=ena? naiʔ Rius iin\j=aan n-fa\tbr{ni}?\\
									{} n-fani=n=ena naiʔ Rius ini=ana n-fani\\
									{} \n-back{\Mv}=\einV={\een} {\naiq} Lius {\iin}={\aan} \a\n-back{\tbrU}\\
						\glt	`Ahh, they've come back, right? Lius's (child) has come back?' \txrf{3.23}}
		\ex[β:]{\glll	iin n-fa\tbr{in}, tua.\\
									ini n-fani tua\\
									{\iin} \n-back{\tbrM} {\tua}\\
						\glt	`He's come back.' \txrf{3.25}}
		\end{xlist}
\end{exe}

However, it is not a rule of Amarasi grammar that
questions \it{must} be posed in the U\=/form.
Two examples of questions posed in the M\=/form
are given in \qf{ex:130825-1.28-1.31} and \qf{ex:130913-1, 0.57-0.59 ch:DisMet} below.
In each example the M\=/form question also
elicits a response (partially) in the M\=/form.

\begin{exe}
	\ex{Going to Jakarta: \txrf{130825-7} {\emb{130825-7-01-28-01-31.mp3}{\spk{}}{\apl}}}\label{ex:130825-1.28-1.31}
		\begin{xlist}
			\ex[α:]{\glll	n-moaʔ on mee =m esa n-heek n-aan =koo n-ok bifee? \sf{atau,} aiʔ hoo m-mo\tbr{uf}?\\
										n-moʔe on mee =ma esa n-heke n-ana =koo n-oka bifee \sf{atau} aiʔ hoo m-mofu\\
										\n-do{\M} like how =and {\es} \n-catch{\M} \n-{\ana\M} ={\koo} \n-{\ok\M} woman or or {\hoo} \m-fall{\tbrM}\\
							\glt	`How did it happen, that is did they catch you with the woman? Or did you fall (morally)?' \txrf{1.28}}
			\ex[β:]{\glll	ka= n-he\tbr{ek} =kau =f!\\
										ka= n-heke =kau =fa\\
										{\ka}= \n-catch{\tbrM} ={\kau} ={\fa}\\
							\glt	`They didn't catch me!' \txrf{1.31}}
		\end{xlist}
	\ex{A man who's already made preparations for his funeral:
				\txrf{130913-1} {\emb{130913-1-00-57-00-59.mp3}{\spk{}}{\apl}}}\label{ex:130913-1, 0.57-0.59 ch:DisMet}
			\begin{xlist}
				\ex[α:]{\glll	m-ak iin n-hain n-me\tbr{es}?\\
											m-ak ini n-hani n-mese\\
											\m-say {\iin} \n-dig{\M} \n-alone{\tbrM}\\
								\glt	`Do you think he dug it alone?' \txrf{0.57}}
				\ex[β:]{\glll	iin ofa n-hani n-me\tbr{es}.\\
											ini ofa n-hani n-mese\\
											{\iin} sure \n-dig{\Uc} \n-alone{\tbrM}\\
								\glt	`He must've dug it himself.' \txrf{0.59}}
			\end{xlist}
\end{exe}

A useful tool for analysing U\=/form/M\=/form question-answer pairs in Amarasi
is provided by the notion of an \emph{adjacency pair},
a concept developed by \citet[295]{scsa77} within the field of conversation analysis.
An adjacency pair has the following properties:

\begin{exe}
	\ex{An adjacency pair: \txrf{}}\label{ex:AnAdjPai}
	\begin{xlist}
		\ex{consists of two conversational turns:}
			\begin{xlist}
				\ex{which are by different speakers}\label{ex:AnAdjPai-1}
				\ex{which are placed next to one another}\label{ex:AnAdjPai-2}
				\ex{which are ordered}\label{ex:AnAdjPai-3}
				\ex{which are differentiated into pair types}\label{ex:AnAdjPai-4}
			\end{xlist}
	\end{xlist}
\end{exe}

The first part of an adjacency pair
is known as the \emph{first pair part}
and the second part is called the \emph{second pair part}.
Property \qf{ex:AnAdjPai-3} refers to the fact
that these two pairs come in a set order:
i.e. a question (first pair part) precedes an answer (second pair part).
Property \qf{ex:AnAdjPai-4} refers to
the fact that which second pair part is allowed
is constrained by the first pair part.
An acceptable second pair part for a greeting
is another greeting, while an acceptable
second pair part for a question is an answer.

The Amarasi examples seen so far in this section
are question-answer adjacency pairs.
U\=/forms occur as first pair parts (questions)
and M\=/forms occur in second pair parts (answers).

A first pair part projects the relevant second pair part.
If the relevant second pair part is lacking,
the conversation is viewed as problematic or incomplete.
Thus, for instance when a speaker asks a question,
they expect to receive an answer.
This is illustrated with the English example in \qf{ex:ACon} below,
taken from \cite{li07} with the transcription adapted to the same
transcription conventions used in this book.

\begin{exe}
	\ex{A conversation: \hfill\cite[108]{li07}}\label{ex:ACon}
	\begin{xlist}
		\ex[α:]{\it{Did you speak to Mary today?}}\label{ex:ACon-1}
		\ex[\hp{α:}]{{[}0.2 seconds of silence{]}}\label{ex:ACon-2}
		\ex[α:]{\it{Did you speak to Mary?}}\label{ex:ACon-3}
		\ex[β:]{\it{Oh, yeah I saw her at lunch.}}\label{ex:ACon-4}
	\end{xlist}
\end{exe}

In \qf{ex:ACon-1} speaker-α's question is followed by 0.2 seconds of silence,
which is interpreted by speaker-α as the answer being absent,
as a result s/he repeats the question in \qf{ex:ACon-3}
which induces the required answer in \qf{ex:ACon-4}.

Within the terminology of conversation analysis,
a U\=/form in Amarasi explicitly flags a turn as
the first pair part of a question-answer adjacency pair.\footnote{
		An interesting topic to pursue for further work
		would be whether questions posed in the M\=/form
		do not require an answer in the same way as those posed in the M\=/form.
		It may be the case that U\=/forms questions are first pair parts
		while M\=/form questions are not.}
This thus projects forward an answer as the second pair part.
Within more general terminology,
U\=/forms are one way of marking a question
which expects an explicit answer.
Such U\=/form questions are complemented and completed by an M\=/form answer.

\subsection{Maintaining interaction}\label{sec:MaiInt}
U\=/forms are not only used in questions.
but are used more broadly to maintain
ongoing interaction and conversation between speakers.
One example is given in \qf{ex:19/09/14 p.97} below.
In this example speaker-α wants to interact with speaker-β.
Speaker-α initiates a conversation in \qf{ex:19/09/14 p.97-1} and
speaker-β responds with the M\=/form \ve{ʔ-took} `sit' in \qf{ex:19/09/14 p.97-2}.
Speaker-α then repeats this answer with a U\=/form \ve{m-toko} `sit' in \qf{ex:19/09/14 p.97-3}.
By using a U\=/form in \qf{ex:19/09/14 p.97-3} speaker-α
signals that the interaction is not yet socially complete.
When speaker-β fails to resolve the U\=/form,
speaker-α does so himself by offering betel nut
in \qf{ex:19/09/14 p.97-4}, the chewing of which is
a core Timorese social activity.

\begin{exe}
	\ex{Speaker-α approaches speaker-β and friends: \txrf{observation 19/09/14 p.97}}\label{ex:19/09/14 p.97}
		\begin{xlist}
			\ex[α:]{\glll	hoo mu-nsaaʔ? \\
										hoo mu-nsaaʔ \\
								 {\hoo} \muu-do.what\\
					\glt `What are you doing?'}\label{ex:19/09/14 p.97-1}
			\ex[β:]{\glll	au ʔ-to\tbr{ok}. \\
										au ʔ-toko \\
										{\au} {\q}-sit{\tbrM} \\
							\glt `I'm sitting.'}\label{ex:19/09/14 p.97-2}
			\ex[α:]{\gll	hoo m-to\tbr{ko}? \\
								 {\hoo} {\m}-sit{\tbrU}\\
					\glt `So, you're sitting, are you?'}\label{ex:19/09/14 p.97-3}
			\ex[α:]{\emph{[approaches group and offers betel nut]}}\label{ex:19/09/14 p.97-4}
		\end{xlist}
\end{exe}

A similar example is given in \qf{ex2:19/09/14 p.97} below.
In \qf{ex2:19/09/14 p.97-1} speaker-β
invites speaker-α to go first at a buffet.
This invitation is accepted by speaker-α
in \qf{ex2:19/09/14 p.97-2} with the U\=/form \ve{u-hunu} `first';
this is a kind of rhetorical question casting.
This U\=/form is then resolved by speaker-β
nodding that this is indeed the intended desire.
%Both examples \qf{ex:19/09/2014 p.97} and \qf{ex2:19/09/2014 p.97}
%exhibit an M\=/form followed by a U\=/form,
%the same pattern seen in head-tail linkage (\srf{sec:HeaTaiLin}).

\begin{exe}
	\ex{Lining up at a buffet to get food: \txrf{observation 19/09/14 p.97}}\label{ex2:19/09/14 p.97}
	\begin{xlist}
		\ex[β:]{\glll	hoo mu-hu\tbr{un}. \\
									hoo mu-hunu \\
							 {\hoo} \muu-first{\tbrM}\\
				\glt `You go first.' [simultaneously gestures with hands]}\label{ex2:19/09/14 p.97-1}
		\ex[α:]{\gll	au u-hu\tbr{nu}.\\
									{\au} {\qu}-first{\tbrU}\\
						\glt `I'll go first, then?'}\label{ex2:19/09/14 p.97-2}
		\ex[β:]{[nods once and gestures]}
		\ex[α:]{[starts serving food]}
	\end{xlist}
\end{exe}

A number of more complex interactional U\=/forms are given
in \qf{ex:130825-7, 3.30-3.35}--\qf{ex:130909-6, 1.12-1.26}.
In \qf{ex:130825-7, 3.30-3.35} a group of speakers are discussing what
to do about the presence of a voice recorder.
Speaker-β announces in \qf{ex:130825-7, 3.33} his intention
with a U\=/form \ve{ʔ-nene} `press'.
This verb is then repeated in the M\=/form by speaker-α who
points out that speaker-β is not achieving his goal.
The U\=/form is not resolved by the action,
but it is resolved by the interaction.

\begin{exe}
	\ex{Turning a voice recorder off: \txrf{130825-7} {\emb{130825-7-03-30-03-40.mp3}{\spk{}}{\apl}}}\label{ex:130825-7, 3.30-3.35}
	\begin{xlist}
		\ex[α:]{\gll	t-\sf{sambuŋ} peo-t=ee, he bisa bees\j=ee na-taah =kiit.\\
									\t-continue talk-\at={\ee} {\he} can machine={\ee} {\na}-answer ={\kiit}\\
						\glt	`(If) we keep talking the machine will be able to answer us.' \txrf{3.30}}
		\ex[β:]{\glll	maut he au ʔ-ne\tbr{ne}!\\
									maut he au ʔ-nene\\
									patient {\he} {\au} \q-press{\tbrU}\\
						\glt	`Hold on, I'll press (the buttons)!' \txrf{3.33}}\label{ex:130825-7, 3.33}
		\ex[α:]{\ve{Aeʔ!} \\
								`Hey!' \txrf{3.34}}
		\ex[β:]{\emph{[laughs]}}
		\ex[α:]{\glll	maan m-ne\tbr{en} mu-tafiʔ bees\j=ee naan, kamaʔ. Hoe! \sf{Australi} \sf{puɲa} \sf{ini.}\\
									maan m-nene mu-tafiʔ besi naan kamaʔ hoe \sf{Australi} \sf{puɲa} \sf{ini}\\
									like.that \m-press{\tbrM} \muu-random{\Uc} machine{\U} {\naan} what's-it hey Australia have this\\
						\glt	`You're randomly pressing the machine there, (that) what's-it. Hey! This belongs to Australia.' \txrf{3.35}}
		\ex[α:]{\gll	hoo m-ak besi kraufn=ees, ees reʔ naan on reʔ \sf{hapei}=ein reʔ a-taf{\tl}taifʔ=ein.\\
									{\hoo} \m-say machine useless={\es} one {\req} {\naan} like {\reqt} mobile.phone={\ein} {\req} {\at}-{\prd}agape={\ein}\\
						\glt	`You think it's a useless machine, the one there, like those mobile phones which leave you agape (with confusion).' \txrf{3.40}}
	\end{xlist}
\end{exe}

Example \qf{ex:130911-2, 0.30-0.39} below involves a number of U\=/forms.
None of these \mbox{U\=/forms} is repeated by another speaker,
but in each instance a U\=/form is followed by the contribution of another speaker.
By using U\=/forms, the speakers indicate that they
do not consider the communicative act resolved
and thereby open the floor up for 
contributions from other speakers.
The only change of speaker in \qf{ex:130911-2, 0.30-0.39} which does
not involve a U\=/form is that after \qf{ex:130911-2, 0.30},
in which \qf{ex:130911-2, 0.32} is an interruption 
cutting off the first speaker mid-sentence.

\begin{exe}
	\ex{A conversation about a car which came off the road:
			\txrf{130911-2} {\emb{130911-2-00-31-00-40.mp3}{\spk{}}{\apl}}}\label{ex:130911-2, 0.30-0.39}
	\begin{xlist}
		\ex[α:]{\gll	iin na-reen=oo-n =ma n-\sf{ʔantareek} a|n-bi n-- \\
									{\iin} \na-force{\Mv}={\oo=\N} =and \n-backing \a\n-{\bi} {}\\
						\glt `He forced himself, and went back into it, was in{\ldots}' \txrf{0.31}}\label{ex:130911-2, 0.30}
		\ex[β:]{\gll	na-ba\tbr{ra} maʔbakeʔ mhh.\\
									 {\na}-forever{\tbrU} narrow {}\\
						\glt `He was stuck in the narrow (place)' \txrf{0.32}}\label{ex:130911-2, 0.32}
		\ex[γ:]{\gll	iin he n-bi\tbr{bi}. \\
									 {\iin} {\he} {\n}-shrink{\tbrU}\\
						\glt `He would've wanted to shrink (the car).' \txrf{0.34}}
		\ex[δ:]{\gll n-ak, ootgw=ii, na-snii m-ak, =am, na-kamaf =am,\\
									\n-say car={\ii} \na-slope \m-say and \na-what's.it{\Uc} =and\\
						\glt `he said, the car was sloping, you think, and what's it and' \txrf{0.35}}
		\ex[]{\glll na-snii n-taikobi n-koon, naʔ na-te\tbr{tu}\\
									na-snii n-taikobi n-kono naʔ na-tetu\\
									\na-slope \n-fall{\Uc} \n-keep.on{\M} then \nat-upright{\tbrU}\\
						\glt `it was sloping, fell over, kept on, and only then he got the car upright' \txrf{0.38}}
		\ex[β:]{\gll	{onai =ma}  srutun reʔ ia, iin n-moofgw=een. \\
									and.so suddenly {\req} {\ia} {\iin} \n-fall{\Mv}={\een} \\
				\glt `and suddenly now it fell down' \txrf{0.40}}
	\end{xlist}
\end{exe}

A similar example of U\=/forms initiating a change of speakers
is given in \qf{ex:130909-6, 1.12-1.26},
which only involves two speakers.
In this example the change of speakers
after each of the following clauses is initiated by a U\=/form:
\qf{ex:130909-6, 1.15}, \qf{ex:130909-6, 1.16}, and \qf{ex:130909-6, 1.22}.
This conversation also involves a large amount of repetition,
a discourse structure already noted in \srf{sec:DisStrAma}
as a feature of Amarasi monologues.

\begin{exe}
	\ex{Preparing a field for planting: \txrf{130909-6} {\emb{130909-6-01-12-01-26.mp3}{\spk{}}{\apl}}}\label{ex:130909-6, 1.12-1.26}
	\begin{xlist}
		\ex[α:]{\gll	mu-boor=een, ta-boor n-ok fuun ne\<ʔ\>e.\\
									\muu-make.hole={\een} \ta-make.hole \n-{\ok\M} moon six\<\qnum\>\\
						\glt	`You dug a hole. We dig holes (for planting) in June.' \txrf{1.12}}
		\ex[β:]{\glll	heʔ, t-ka\tbr{nu} =t, naʔ fuun ne\<ʔ\>e.\\
									{} t-kanu =te naʔ funan ne\<ʔ\>e\\
									hey \t-cut.field{\tbrU} ={\te} then moon six\<\qnum\>\\
						\glt	`What? We cut open a new field, only then is it June.' \txrf{1.15}}\label{ex:130909-6, 1.15}
		\ex[α:]{\glll	ehh,  t-ka\tbr{nu} =t, naʔ fuun ne\<ʔ\>e.\\
									{} t-kanu =te naʔ funan ne\<ʔ\>e\\
									oh \t-cut.field{\tbrU} ={\te} then moon six\<\qnum\>\\
						\glt	`Ohh{\ldots}, we cut open a new field, only then is it June.' \txrf{1.16}}\label{ex:130909-6, 1.16}
		\ex[β:]{\glll	t-tofa ʔteets=ii.\\
									t-tofa ʔtetas=ii\\
									\t-weed{\Uc} old.field={\ii}\\
						\glt	`We weed the old field..' \txrf{1.18}}\label{ex:130909-6, 1.18}
		\ex[α:]{\gll	t-toof n-ok fuun se\<ʔ\>o.\\
									\t-weed{\M} \n-{\ok\M} month{\M} nine\<{\qnum}\>\\
						\glt	`We weed (the field) in September.' \txrf{1.20}}
		\ex[β:]{\glll	hau, t-toof nai he n-me\tbr{to}, oo.\\
									hau t-tofa nai he n-meto oo\\
									yes \t-weed{\M} already he \n-dry{\tbrU} {\aaQ}\\
						\glt	`Yes, we weed (the field) after it's dried out, as you know.' \txrf{1.22}}\label{ex:130909-6, 1.22}
		\ex[α:]{\gll	nejaa, nean fauk=ii na-ʔuur?\\
								%	{} neno fauk=ii na-ʔura\\
									yeah day how.many={\ii} \na-rain{\M}\\
						\glt	`Yeah{\ldots} which day did it rain?' \txrf{1.26}}
	\end{xlist}
\end{exe}

U\=/forms can be used in conversation to maintain interaction
between speakers and to motivate a change of speaker.
By using a U\=/form a speaker signals a lack of resolution,
while other features such as intonation
and silence indicate that the speaker will not resolve the U\=/form.
It thus becomes incumbent on the addressee or audience
to provide a resolution to the U\=/form.

\subsection{Frequency of U\=/forms in conversation}\label{sec:FreUfoCon}
Discourse U\=/forms are nearly twice as frequent in conversations as in monologues in my corpus.
My Kotos Amarasi text collection consists of 182.49 minutes (three hours two minutes)
of recorded, transcribed, and glossed texts.
Of this, 154.17 minutes (two hours thirty-four minutes)
are monologues: texts which consist mainly of a single speaker,
and 28.32 minutes (nearly half an hour) are conversations:
texts in which more than one person regularly speaks.

Of the 423 U\=/forms in my corpus which
cannot be explained by phonotactic constraints (\srf{sec:PhoCon}),
312 occur in monologues and 111 occur in conversations.
This gives a frequency of 2.02 discourse U\=/forms per minute in monologues
and 3.92 discourse U\=/forms per minute in conversations.
These figures are summarised in \trf{tab:VerUfoConMon}.

\begin{table}[h]
	\caption{Discourse U-forms in monologues and conversations}\label{tab:VerUfoConMon}
	\centering
		\begin{tabular}{rlll} \lsptoprule
															& Mon.	& Conv. & all			\\ \midrule
			total length (minutes)	& 154.17& 28.32	& 182.49	\\
			discourse U\=/forms				& 312		& 111		& 423			\\
			U\=/forms per minute			&	2.02	&	3.92	&	2.32		\\
			\lspbottomrule
		\end{tabular}
\end{table}

That discourse-driven U\=/forms are nearly twice as frequent in conversations
as in monologues lends quantitative support to an analysis of U\=/forms
as being used interactionally by speakers in conversations
to motivate turn taking and change of speaker.

\subsection{Other interactional resources}\label{sec:OthIntRes}
U\=/forms are only one of several resources in Amarasi
available to speakers to maintain interaction with other speakers.
In this section I discuss the way a number of discourse particles
interact with discourse-driven U\=/forms.

The addressee particle \ve{tua} is a polite way
in which a speaker can mark their contribution to the discourse as complete.
Thus, it cannot co-occur with U\=/forms, which explicitly flag a lack of resolution.
On the other hand the question particles \ve{oo} and \ve{kaah}
require a response from the addressee.
Thus, they combine naturally with U\=/forms in direct questions.

\subsubsection{Addressee particle \it{tua}}\label{sec:AddParTua}
The addressee particle \ve{tua} cannot co-occur with interactional U\=/forms.
This is because such a U\=/form is unresolved or incomplete and
places an obligation on the addressee to respond to the speaker,
while \ve{tua} signals that the speaker considers their contribution complete.

The particle \ve{tua} is translated by native speakers as `yes' or `Sir/maam',
and they explain that this word makes one's speech \it{halus};
Indonesian for `smooth, refined, polite'.
The particle \ve{tua} is a distinctive feature
of Amarasi and nearby varieties of Meto.
The different functions of the particle \ve{tua}
found in my corpus are summarised in \qf{ex:UseTua} below,
with discussion and exemplification following.
It almost always occurs phrase finally.

\begin{exe}
	\ex{Uses of \ve{tua}:}\label{ex:UseTua}
		\begin{xlist}
			\ex{addressing the deceased}
			\ex{acknowledging one is listening to someone else}
			\ex{acknowledging instructions to begin a monologue}
			\ex{ending a monologue}
			\ex{indicating the end of a turn in a conversation}
			\ex{taking leave of someone}
		\end{xlist}
\end{exe}

The particle \ve{tua} has two functions: to politely address
another person and for the speaker to signal
that their contribution to the discourse is potentially complete.
The first part of this use, to address someone,
is seen clearly in one particular text;
a woman mourning for her recently deceased grandmother.
After a death in Amarasi society, the body of the deceased is washed, clothed,
prepared for burial, and then laid in an open casket overnight while the family stays awake.
When a family member wishes to express their grief,
they can do so by addressing the deceased, whose body is present in the room.
Two examples of \ve{tua}, addressing the deceased, from this text
are given in \qf{ex:130823-8, 4.44} and \qf{ex:130823-8, 5.22} below.

\begin{exe}
	\ex{\glll	airoo! kasian! ma bait hoo saaʔ naa na-mena =te, hoo mu-toon =kai he hai mi-hiin \tbr{tua}, nene!\\
						airoo kasian ma baiti hoo saaʔ naa na-mena =te hoo mu-tona =kai he hai mi-hini tua nene\\
						oh! pity! and actually {\hoo} something {\naa} \na-sick{\U} ={\te} {\hoo} \muu-tell{\M} ={\kai}
						{\he} {\hai} \mi-know{\M} {\tua} grandma\\
			\glt	`Oh! Pity! And you had something that was sick and you told us so we knew. Oh, Grandma!'
						\txrf{130823-8, 4.44}}\label{ex:130823-8, 4.44}
	\ex{\glll	airoo! benuʔ! ma t-beeʔ =te {okeʔ =te} ʔ-reun \hspace{5mm} =koo =fa he m-tupa =te, ka= m-roim =fa, \tbr{tua}!\\
						airoo benuʔ ma t-beʔe =te {okeʔ =te} ʔ-renu {} =koo =fa he m-tupa =te ka= m-romi =fa tua\\
						oh! goodness! and \t-stay.awake{\M} ={\te} after.that \q-order{\M}
						{} ={\koo} ={\fa} {\he} \m-sleep{\U} ={\te} {\ka}= \m-like{\M} ={\fa} {\tua}\\
			\glt	`Oh! Goodness! And when we stayed up I then told you to sleep, but you didn't want to!'
							\txrf{130823-8, 5.22}}\label{ex:130823-8, 5.22}
\end{exe}

The word \ve{tua} alone constitutes an acceptable conversational turn,
in which case it merely indicates that the speaker is listening,
or to give an affirmative answer to a question.
In such situations it can often be glossed `yes' or `OK'.
Two examples are given in \qf{ex:120923-1, 8.51-8.59} below.

\begin{exe}
	\ex{Asking about the \it{biku} curse: \txrf{120923-1} {\emb{120923-1-08-51-08-59.mp3}{\spk{}}{\apl}}}\label{ex:120923-1, 8.51-8.59}
		\begin{xlist}
			\ex[α:]{\emph{Dad, the person who casts the \emph{biku} curse.} \txrf{8.37}}
			\ex[]{\emph{Does s/he proclaim it to  spirits in the land? Or what does s/he proclaim it to?} \txrf{8.44}}
			\ex[]{\emph{So that the \emph{biku} curse takes effect?} \txrf{8.48}}
			\ex[β:]{\gll	m-ak nehh, on karu he on moa-- mu-taan =kau n-ok reʔ biku, \sf{ʧara} biikgw=ii?\\
										\m-say {} {\on} if {\he} {\on} {} \muu-ask ={\kau} {\n-\ok} {\reqt} curse method curse={\ii}\\
							\glt	`So you're asking me about the \it{biku} curse, the method by which the \it{biku} curse is cast?' \txrf{8.51}}
			\ex[α:]{	\ve{\tbr{tua}}.\\
								`Yes.' \txrf{8.55}}
			\ex[β:]{\gll	biku \sf{bukan} na-tona=n paah=ii.\\
										curse \tsc{neg} \na-tell={\einV} country={\ii}\\
							\glt	`A \it{biku} curse is not proclaimed to the (spirits in the) land.' \txrf{8.56}}
			\ex[α:]{	\ve{\tbr{tua}}.\\
								`OK' \txrf{8.58}}
			\ex[β:]{\gll	a|n-mooʔ\j=ee n-ok hau, papa!\\
										\hp{\a}\n-do={\eeV} \n-{\ok} spell dad\\
							\glt	{\leavevmode\hp{a|}}`It's done with a spell, dad!' \txrf{8.59}}
		\end{xlist}
\end{exe}

If someone else has asked the narrator to tell a particular story,
\ve{tua} can be used by the narrator at the very beginning of the
story to acknowledge the other speaker's instruction and begin their monologue.
Two examples are given in \qf{ex:120715-2, 0.01} and \qf{ex:120715-3, 0.07} below.
In each example the narrator has been instructed
by someone else to begin.

\begin{exe}
	\ex{\glll	au kaan-k=ii bi Oma, \tbr{tua}.\\
						au kana-k=ii bi Oma tua\\
						{\au} name-\k={\ii} {\BI} Oma {\tua}\\
			\glt	`My name is Oma.' \txrf{120715-2, 0.01} {\emb{120715-2-00-01.mp3}{\spk{}}{\apl}}}\label{ex:120715-2, 0.01}
	\ex{\glll	reʔ ahh uab uunʔ=ein nai \tbr{tua}.\\
						reʔ {} uaba unuʔ=ein nai tua. \\
						{\req} {} speech past={\ein} already {\tua}. \\
			\glt	`So (I'll tell) some old stories.'
						\txrf{120715-3, 0.07} {\emb{120715-3-00-07.mp3}{\spk{}}{\apl}}}\label{ex:120715-3, 0.07}
\end{exe}

In monologues \ve{tua} commonly occurs at the end
of a story or speech to indicate that the monologue is over.
Two examples are given in \qf{ex:130825-3, 2.35} and \qf{ex:120715-2, 1.31} below.
Example \qf{ex:120715-2, 1.31} is a typical high-level discourse closure.

\begin{exe}
		\ex{\glll	hai aaʔ-t=ii na-m-soup =ma n-heun-ʔ=oo-n, \hspace{5mm} on naan nai, \tbr{tua}.\\
							hai aʔa-t=ii na-m-sopu =ma n-henu-ʔ=oo-n, {} on naan nai, tua\\
							{\hai} poetry-{\at}={\ii} \na-\mv-finish{\M} =and \n-fill{\Mv}-{\qV}={\oo}-{\N} {} {\on} {\naan} already {\tua}\\
				\glt	`Our poetry is now finished and complete like that.'
							\txrf{130825-3, 2.35} {\emb{130825-3-02-35.mp3}{\spk{}}{\apl}}}\label{ex:130825-3, 2.35}
	\ex{\gll	on reʔ naan, \tbr{tua}.\\
						like {\reqt} {\naan} {\tua} \\
			\glt	`That's how it is.' \txrf{120715-1, 1.31} {\emb{120715-1-01-31.mp3}{\spk{}}{\apl}}}\label{ex:120715-2, 1.31}
\end{exe}

The use of \ve{tua} at the end of monologues
is a part of the more general function of this
particle to mark the end of a conversational turn,
after which others are free to contribute to the conversation.
Two examples are given in \qf{ex:130909-6, 2.42-2.45}
and \qf{ex:120923-1, 0.01-0.04} below.
In \qf{ex:130909-6, 2.42-2.45} below speaker-α and speaker-β
are the main participants in the conversation.
In \qf{ex:130909-6, 2.42} speaker-α makes a statement.
Speaker-β then expresses his interest in this statement with an exclamation in \qf{ex:130909-6, 2.44}.
However, speaker-γ interjects but ends his statement with \ve{tua},
thus indicating that speaker-α and speaker-β are free to resume their conversation.

\begin{exe}
	\ex{Talking about farming: \txrf{130909-6} {\emb{130909-6-02-42-02-45.mp3}{\spk{}}{\apl}}}\label{ex:130909-6, 2.42-2.45}
	\begin{xlist}
		\ex[α:]{\glll	n-hetu uutn=ii =t, ees ka= bisa =fa.\\
									n-hetu utan=ii =te esa ka= bisa =fa\\
									\n-pick{\U} vegetables={\ii} ={\te} {\esc} {\ka}= can ={\fa}\\
						\glt	`Picking vegetables, (he) can't even do that.' \txrf{2.42}}\label{ex:130909-6, 2.42}
		\ex[β:]{\ve{Hau} \ve{bah!}\\
						`Yes, indeed!' \txrf{2.44}}\label{ex:130909-6, 2.44}
		\ex[γ:]{\glll	n-pea =t na-ʔkoroʔ bian, \tbr{tua}.\\
									n-peo =te na-ʔkoroʔ bian tua\\
									\n-talk ={\te} \na-hide{\Uc} other {\tua}\\
						\glt	`(He) talked (about it) and hid others.' \txrf{2.44}}
		\ex[β:]{\glll	ahh baiʔ Tobias n-ak, na-ʔkoroʔ bian, haa!\\
									{} baʔi Tobias n-ak na-ʔkoroʔ bian {}\\
									{} PF Tobias \n-say \na-hide{\Uc} other {}\\
						\glt	\lh{ahh }`Grandfather Tobias said he hid others.' \txrf{2.45}}\label{ex:130909-6, 2.45}
	\end{xlist}
\end{exe}

In example \qf{ex:120923-1, 0.01-0.04} below speaker-α is collecting metadata.
This metadata consists of two questions: the narrator's name and where he comes from.
In \qf{ex:120923-1, 0.01 ch:DisMet} speaker-α asks the first question and
also addresses the narrator as \ve{papa} `dad' to express politeness.
In \qf{ex:120923-1, 0.04} speaker-α ends the second question with \ve{tua},
indicating that he does not intend to ask more questions.
The collection of metadata is over and speaker-β can begin his story.

\begin{exe}
	\ex{Collecting metadata: \txrf{120923-1} {\emb{120923-1-00-01-00-08.mp3}{\spk{}}{\apl}}}\label{ex:120923-1, 0.01-0.04}
	\begin{xlist}
		\ex[α:]{\glll	papa, hoo kaan-m=ii sekau, papa?\\
							papa hoo kana-m=ii sekau papa\\
							dad {\hoo} name-{\mg}={\ii} who dad\\
				\glt	`Dad, what's your name, dad?' \txrf{0.01}}\label{ex:120923-1, 0.01 ch:DisMet}
		\ex[β:]{\glll	au kaan-k=ii Melkias Mnaʔo.\\
							au kana-k=ii Melkias Mnaʔo\\
							{\au} name-\k={\ii} Melchias Mna{\Q}o\\
				\glt	`My name is Melchias Mna{\Q}o.' \txrf{0.03}}
		\ex[α:]{\gll	hoo mu-ʔko mee, \tbr{tua}.\\
							{\hoo} \muu-{\qko} where {\tua}\\
				\glt	`Where are you from?' \txrf{0.04}}\label{ex:120923-1, 0.04}
		\ex[β:]{\gll	au u-ʔko Binoni Aufmeʔe, \sf{desa} \sf{dua}.\\
							au \qu-{\qko} Binoni Aufme{\Q}e village two\\
				\glt	`I'm from Binoni Aufme{\Q}e, village number two.' \txrf{0.08}}
	\end{xlist}
\end{exe}

The particle \ve{tua} is also used to take leave of someone.
In Amarasi culture it is rude to pass by someone and not speak to them.
Silence towards another person is interpreted as a sign
of a damaged relationship or anger, which is considered dangerous.
As a result, people coming across one another during everyday activities
are socially obliged to make small talk.
Such small talk typically involves asking questions such
as where the other person is going or where they are coming from.
Two typical small talk questions and possible answers are given in \qf{ex:hoo mnao on mee?}
and \qf{ex:hoo mo'ka mee?} below.

\begin{exe}
	\ex{\begin{xlist}
		\ex{\gll	hoo m-nao on mee?\\
							{\hoo} \m-go {\on} where\\
				\glt	`Where are you going?' \txrf{}}
		\ex{\gll	(au) ʔ-nao on rene.\\
							\hp{(}{\au} \q-nao {\on} field\\
					\glt	\lh{(}`I'm going to my field.' \txrf{}}
	\end{xlist}}\label{ex:hoo mnao on mee?}
	\ex{\begin{xlist}
		\ex{\gll	hoo m-oʔka mee?\\
							{\hoo} \m-{\qko} where\\
					\glt	`Where have you come from?' \txrf{}}
		\ex{\gll	(au) ʔ-oʔka ata nee.\\
							\hp{(}{\au} \q-{\qko} up {\nee}\\
					\glt	`(I've come) from up there.' \txrf{}}
	\end{xlist}}\label{ex:hoo mo'ka mee?}
\end{exe}

In Amarasi society the cultural imperative to interact in this way
is so strong that speakers will yell out to one another across
valleys or through the bush if they are aware that someone else is present.
If the bush is so thick, or the distance so great that the location
of the other person cannot be pinpointed exactly,
speakers will call out (Amarasi \ve{n-koaʔ} `whoop, yell a sound (without words)').
Similarly, when going past someone at speed on a motorbike or in a car,
honking the horn is sufficient social interaction,
though a comment is considered even more polite.

Interactions such as those in \qf{ex:hoo mnao on mee?}
and \qf{ex:hoo mo'ka mee?} do not occur on their own.
Once someone has made small talk,
they need strategies for ending the interaction 
to carry on whatever activity they were doing
or to continue on their way.

It is in this context that the particle \ve{tua}
most often occurs in day-to-day use.
By using the particle \ve{tua} the speaker
politely acknowledges that they have interacted
and that this interaction is potentially complete.

A sample of the most common leave-taking phrases
are given in \qf{ex:au 'koon goen tua}--\qf{ex:hai mihuun tua} below.
The usual -- and sufficient -- response to all such 
leave-taking phrases is the word \ve{tua} by itself.
Any of these phrases constitutes a sufficient social interaction on its own.

\begin{multicols}{2}
	\begin{exe}
		\ex{Passing a stationary person:}\label{ex:au 'koon goen tua}
			\sn{\glll	au ʔ-kooŋgw=een, \tbr{tua}.\\
								au ʔ-kono=ena tua\\
								{\au} \q-pass{\Mv}={\een} {\tua}\\
					\glt	`I'll keep going now.' \txrf{}}
		\ex{Returning home:}\label{ex:au 'fain jeen tua}
			\sn{\glll	au ʔ-faan\j=een, \tbr{tua}.\\
								au ʔ-fani=ena tua\\
								{\au} \q-back{\Mv}={\een} {\tua}\\
					\glt	`I'm going to head back now.' \txrf{}}
	\end{exe}
\end{multicols}

\begin{multicols}{2}
	\begin{exe}
		\ex{Continuing after conversation:}\label{ex:au 'nao goen tua}
			\sn{\glll	au ʔ-naagw=een, \tbr{tua}.\\
								au ʔ-nao=ena tua\\
								{\au} \q-go{\Mv}={\een} {\tua}\\
					\glt	`I'll get going again.' \txrf{}}
		\ex{Overtaking (e.g. on motorbike):}\label{ex:hai mihuun tua}
			\sn{\glll	hai mi-huun, \tbr{tua}.\\
								hai mi-hunu tua\\
								{\hai} \mi-first{\M} {\tua}\\
					\glt	`We're going on ahead.' \txrf{}}
	\end{exe}
\end{multicols}

The particle \ve{tua} does not co-occur with discourse U\=/forms.
Not only is \ve{tua} unattested with discourse U\=/forms, it is infelicitous with them.
Every possible way of saying `I don't know' in Kotos Amarasi
with each combination of ±metathesis and ±\ve{tua}
is given in \qf{ex:I don't know} below.

Of these, native speakers consider \qf{ex:Au ka u-hiin fa} and \qf{ex:Au ka u-hini f}
normal, with \qf{ex:Au ka u-hiin fa} being more polite.
Native speakers judge example \qf{ex:Au ka u-hiin fa tua}
to be even more respectful or polite while \qf{ex:Au ka u-hini f tua}
-- with both a U\=/form and \ve{tua} -- is considered funny.

\begin{exe}\judgewidth{?}
	\ex{\label{ex:I don't know}
		\begin{xlist}
			\ex[]{\glll	au ka= u-hiin =fa.\\
									au ka= u-hini =fa\\
									{\au} {\ka}= \qu-know{\M} ={\fa}\\}\label{ex:Au ka u-hiin fa}
			\ex[]{\glll	au ka= u-hini =f.\\
									au ka= u-hini =fa\\
									{\au} {\ka}= \qu-know{\U} ={\fa}\\}\label{ex:Au ka u-hini f}
			\ex[]{\glll	au ka= u-hiin =fa, tua.\\
									au ka= u-hini =fa tua\\
									{\au} {\ka}= \qu-know{\M} ={\fa} {\tua}\\}\label{ex:Au ka u-hiin fa tua}
			\ex[{\#}]{\glll	au ka= u-hini =f, tua.\\
										au ka= u-hini =fa tua\\
										{\au} {\ka}= \qu-know{\U} ={\fa} {\tua}\\
							\glt `I don't know.' }\label{ex:Au ka u-hini f tua}
		\end{xlist}}
\end{exe}

The inability of \ve{tua} to co-occur felicitously with U\=/forms
is explained by a clash in the functions of these two discourse resources.
One part of the function of \ve{tua} is to politely acknowledge
the addressee as an interlocutor, while the other part of its
function is to indicate that the interaction is potentially complete.
A U\=/form, on the other hand, explicitly marks a lack of resolution
and the lack of completion in a conversation.
The combination of potential completion (\ve{tua})
and explicit lack of resolution/completion (U\=/form)
cannot be sensibly combined in Amarasi.

\subsubsection{Question particles}\label{sec:QuePar}
There are three common tag question particles in Amarasi,
given in \trf{tab:AmAQuePar} below.
The tag question particles \ve{oo} and \ve{kaah}
invite a response and combine naturally with discourse U\=/forms
which signal lack of resolution.

\begin{table}[h]
	\caption{Amarasi tag question particles}\label{tab:AmAQuePar}
	\centering
		\begin{tabular}{ll} \lsptoprule
			Particle		& Usage \\ \midrule
			\ve{aa}			& `I think this' \\
			\ve{oo}			& `You should do this.' \\
			\ve{kaah}		&	`I think this, what	\\
									&	\hp{`}do you think?' \\ \lspbottomrule
		\end{tabular}
\end{table}

The particle \ve{oo} is often used as the language
of power to obligate the addressee to respond and
confirm or comply with the expectation of the speaker,
thus resolving any U\=/form with which it occurred.

The particle \ve{kaah} marks that the speaker is not sure
of the content of their question and invites the
addressee to correct, confirm, or deny the assumption,
and thus resolve any U\=/form with which \ve{kaah} occurred.
The particle \ve{aa} is often used in rhetorical questions
to which the addressee is not expected to respond.
An example is given in \qf{ex:130914-3, 1.03 ch:DisMet}.

\begin{exe}
	\ex{\gll	mama, au huutgw=ii maʔtaneʔ aa?\\
						mum {\au} louse={\ii} excessive {\aaQ}\\
			\glt	`Mum, I've got a lot of lice, haven't I?'
						\txrf{130914-3, 1.03}{\emb{130914-3-01-03.mp3}{\spk{}}{\apl}}}\label{ex:130914-3, 1.03 ch:DisMet}
\end{exe}

The particle \ve{aa} cannot be felicitously combined with a direct question
when the speaker is genuinely unsure about the answer.
This is the case no matter whether a U\=/form or M\=/form is used.
This is shown in \qf{ex:DrinkAa} below.

\newpage
\begin{exe}
	\ex{Asking if someone drinks alcohol: \txrf{elicit. 13/06/16 p.15}}\label{ex:DrinkAa}
		\begin{xlist}
		\ex[\#]{\glll	hoo m-inu, aa?\\
									hoo m-inu aa\\
									{\hoo} \m-drink{\U} {\aaQ}\\
						\glt	`You'll drink, right?'
									\txrf{}}\label{ex:DrinkAa1}
		\ex[\#]{\glll	hoo m-iun, aa?\\
									hoo m-inu aa\\
									{\hoo} \m-drink{\M} {\aaQ}\\
						\glt	`You'll drink, right?'
									\txrf{}}\label{ex:DrinkAa2}
		\end{xlist}
\end{exe}

When used in direct questions where the speaker genuinely wants an answer,
the tag question particles \ve{oo} and \ve{kaah} combine naturally
with a U\=/form, as shown in (\ref{ex:el. 13/06/16 p.15}a)
and (\ref{ex:el. 13/06/16 p.15}c) below,
but are not natural with an M\=/form, as shown in (\ref{ex:el. 13/06/16 p.15}b)
and (\ref{ex:el. 13/06/16 p.15}d) below.\footnote{
		The form \ve{kaah} is also a negator.
		Its use as a tag question can be compared with English examples
		such as \emph{`You drink, do\tbf{n't} you?'}}

\begin{exe}
	\ex{Asking if someone drinks alcohol: \txrf{elicit. 13/06/16 p.15}}\label{ex:el. 13/06/16 p.15}
		\begin{xlist}
			\ex[]{\glll	hoo m-i\tbr{nu}, oo?\\
									hoo m-i\tbr{nu} oo\\
									{\hoo} \m-drink{\tbrU} {\aaQ}\\
						\glt	`You'll drink, won't you?' \txrf{}}
			\ex[\#]{\glll	hoo m-i\tbr{un}, oo?\\
										hoo m-inu oo\\
										{\hoo} \m-drink{\tbrM} {\aaQ}\\
							\glt	`You'll drink, won't you?' \txrf{}}
			\ex[]{\glll	hoo m-i\tbr{nu}, kaah?\\
									hoo m-i\tbr{nu} kaah\\
									{\hoo} \m-drink{\tbrU} {\kaah}\\
						\glt	`You'll drink, won't you?' \txrf{}}
			\ex[\#]{\glll	hoo m-i\tbr{un}, kaah?\\
										hoo m-inu kaah\\
										{\hoo} \m-drink{\tbrM} {\kaah}\\
							\glt	`You'll drink, won't you?' \txrf{}}
	\end{xlist}
\end{exe}

A discourse U\=/form combines naturally with
the tag question particles \ve{oo} and \ve{kaah}.
This is because a U\=/form signals lack of resolution,
the particle \ve{oo} places an obligation on the addressee
to respond and thus resolve the U\=/form,
and the particle \ve{kaah} invites the addressee to answer
and thus resolve the U\=/form.

\subsection{Summary}
U\=/forms are used in conversation to maintain interaction between speakers.
A speaker can use a U\=/form to signal a lack of resolution.
If the same speaker does not provide a resolution,
it becomes incumbent upon the addressee to provide a resolution.
Question/answer pairs are one subtype of this function,
with questions being posed in the U\=/form and answered in the M\=/form.

Discourse U\=/forms combine naturally with the question particles \ve{oo} and \ve{kaah},
which both require a response from the addressee,
but these particles do not combine naturally with M\=/forms in direct questions.
Because U\=/forms mark a lack of resolution,
they do not combine with the particle \ve{tua},
which indicates that the interaction is potentially complete.
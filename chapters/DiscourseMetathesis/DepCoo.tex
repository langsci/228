\section{Dependent coordination}\label{sec:DepCoo}
The most common use of U\=/forms in discourse is to mark one event/situation
as dependent on another event/situation.
When the U\=/form word encodes an event or state,
this signals a temporal relation between two events
with the U\=/form event beginning prior to and leading into the next event.
The typical structure of dependent coordination is given in \qf{ex:ChrCoo} below.

\begin{exe}
	\ex{event\sub{1}{\U}
	$\left(\left\{\hspace{-1.7mm} \begin{array}{l} 
						\ve{=ma} \\
						\ve{=te} \\ \end{array} \hspace{-1.7mm}\right\}\right)$
	event\sub{2}({\M})}\label{ex:ChrCoo}
\end{exe}

More than half (261/423) of all discourse-driven U\=/forms in my corpus
are instances of dependent coordination.
Most examples of dependent coordination involve either of
the connectors \ve{=ma} `and' or \ve{=te} \tsc{set} `when, as'.
One of these connectors occurs in 86{\%} (225/261) of all examples in my corpus.
I discuss each in turn, followed in \srf{sec:CooCon}
by dependent coordination without any connector.

Each of the connectors \ve{=ma} and \ve{=te} has four allomorphs.
Firstly, after consonants these connectors usually
(though not obligatorily) take an initial /a/, thus \ve{=ama} and \ve{=ate}.
As discussed in \srf{sec:SenEnc} and \srf{sec:Int ch:PhoMet}
the allomorphs of these enclitics with initial /a/ are optionally
treated as vowel-initial and thus trigger metathesis on consonant-final hosts.

Secondly, it is common for the final vowel of these connectors
to be deleted, thus \ve{=m} and \ve{=t}, or after consonants \ve{=am} and \ve{=at}.
The allomorphy of these connectors is summarised in \qf{ex:ConAll} below.
(See \srf{sec:SenEnc} for more details.)

\begin{exe}\let\eachwordone=\textnormal
	\ex{Connector allomorphy}\label{ex:ConAll}
	\sn{\gw\begin{tabular}{llllll}
			\ve{=te}	&{\ra}&\ve{=te}	&{\tl}&\ve{=t}	&/V{\#}{\gap} \\
								&{\ra}&\ve{=ate}&{\tl}&\ve{=at}&/C{\#}{\gap} \\
			\ve{=ma}	&{\ra}&\ve{=ma}	&{\tl}&\ve{=m}	&/V{\#}{\gap} \\
								&{\ra}&\ve{=ama}&{\tl}&\ve{=am}&/C{\#}{\gap} \\
	\end{tabular}}
\end{exe}

\subsection{Dependent coordination with \it{=ma} `and'}\label{sec:Coo=Ma}
When the connector \ve{=ma} `and' occurs after a U\=/form,
it signals that this event precedes the next event.
This often also implies that the first event caused the second event.
The event encoded by the U\=/form is resolved by the following event.
This is illustrated in \qf{ex:=ma} below.
There are 118 examples of dependent coordination
with the connector \ve{=ma} in my corpus.

\begin{exe}\let\eachwordone=\textnormal
	\ex{\rule{0pt}{0pt}{} \\[-2.5ex]
	\begin{tabular}{ll}
		$\xrightarrow{\textnormal{\normalsize event\sub{1}{\U} \ve{=ma} \hspace{5mm}}}$ & $\xrightarrow{\textnormal{\normalsize event\sub{2}(\M) \hspace{20mm}}}$\\
	\end{tabular}}\label{ex:=ma}
\end{exe}

A U\=/form followed by \ve{=ma} `and'
is viewed as a separate event discrete from the next event
rather than both events being viewed as a single complex whole.
This contrasts with M\=/forms followed by \ve{=ma} `and',
in which the events encoded by each verb are identical.
Four examples of a U\=/form and the connector \ve{=ma}
are given in \qf{ex:130928-1, 1.54}--\qf{ex:120715-4, 0.45 ch:DisMet} below.
In each example the U\=/form describes an event which
preceded and led to the event encoded by the verb following \ve{=ma}.
The resolving event is that following the U\=/form.

\begin{exe}
	\ex{\glll	iin aam-f=ii esa \bxA{n-re\tbr{nu}} =\tbr{ma} \bxB{n-ha\tbr{in}} reʔ nopu. {\lk}\\
						ini ama-f=ii esa {\gp}n-renu =ma {\gp}n-hani reʔ nopu\\
						{\iin} father-{\F}={\ii} {\esc} \gp\n-order{\tbrU} =\tbr{and} \gp\n-dig{\M} {\reqt} hole\\
			\glt	`It was his\sub{\it{i}} father who gave the order and (he\sub{\it{i}}) dug the grave.'
						\txrf{130928-1, 1.54} {\emb{130928-1-01-54.mp3}{\spk{}}{\apl}}}\label{ex:130928-1, 1.54}
	\ex{\glll	m-ak hai nua =kai \bxA{m-taiko\tbr{bi}} =m hai \bxB{m-ma\tbr{et}} okeʔ {\lk}\\
						m-ak hai nua =kai {\gp}m-taikobi =ma hai {\gp}m-mate okeʔ \\
						\m-say {\hai} two {\kai} \gp\m-fall{\tbrU} =and {\hai} \gp\m-die{\tbrM} all \\
			\glt	`So we two will fall down and (then) both die.'
						\txrf{130909-6, 0.39} {\emb{130909-6-00-39.mp3}{\spk{}}{\apl}}\vspace{4pt}}\label{ex:130909-6, 0.39 ch:DisMet}
	\ex{\glll	au maeb=ees=ii, \bxA{ʔ-to\tbr{ko}} =\tbr{ma} \bxB{ʔ-tui} sina =m au ʔ-kaububuʔ siin eta=n neʔ suurt=ee =m {\lk}\\
						au mabe-ʔ=esa=ii {\gp}ʔ-toko =ma {\gp}ʔ-tui sina =m au ʔ-kaububuʔ sina eta=n neʔ suurt=ee =ma\\
						{\au} time={\es=\ii} \gp\q-sit{\tbrU} =\tbr{and} \gp\q-write {\siin} =and {\au} \q-gather{\Uc} {\siin} {\et}={\einV} {\reqt} paper={\ee} =and \\
			\glt	`A few nights ago I sat down and (then) wrote them down and collected them in the book and {\ldots}'
						\txrf{130909-5, 0.28} {\emb{130909-5-00-28.mp3}{\spk{}}{\apl}}}\vspace{4pt}\label{ex:130909-5, 0.28}
	\ex{\glll	 \bxA{a|n-mo\tbr{ʔe}} =ma \bxB{n-poo\j=ena} n-bi metoʔ. {\lk}\\
						{\gp\a}n-moʔe =ma {\gp}n-poi=ena n-bi metoʔ\\
						{\gp\a\n}-make{\tbrU} =\tbr{and} \gp\n-exit{\Mv}={\een} {\n}-{\bi} dry\\
			\glt	\lh{a|\hspace{1.2mm}}`he created and (then) went out onto dry land.'
						\txrf{120715-4, 0.45} {\emb{120715-4-00-45.mp3}{\spk{}}{\apl}}}\label{ex:120715-4, 0.45 ch:DisMet}
\end{exe}

When the event followed by \ve{=ma} temporally precedes the next event,
it is not grammatical for the first event to be in the M\=/form.
This is shown in (\ref{ex:130928-1, 1.54}′) and (\ref{ex:120715-4, 0.45 ch:DisMet}′) below,
each of which is a manipulated version of the equivalent examples (without primes)
above with the only difference being the use of an M\=/form verb instead of a U\=/form.

\begin{exe}
	\exp{ex:130928-1, 1.54}[*]{\glll	iin aam-f=ii esa n-re\tbr{un} =\tbr{ma} n-hain reʔ nopu\\
						ini ama-f=ii esa n-renu =ma n-hani reʔ nopu\\
						{\iin} father-{\F}={\ii} {\esc} \n-order{\tbrM} =\tbr{and} \n-dig{\M} {\reqt} hole\\
			\glt	`(It was his\sub{\it{i}} father who gave the order and (he\sub{\it{i}}) dug the grave.)' \txrf{elicit. 09/02/16 p.9}}
	\exp{ex:120715-4, 0.45 ch:DisMet}[*]{\glll	a|n-mo\tbr{eʔ} =\tbr{ma} n-poo\j=ena n-bi metoʔ\\
						{\a}n-moʔe =ma n-poi=ena n-bi metoʔ\\
						{\a\n}-make{\tbrM} =\tbr{and} \n-exit{\Mv}={\een} \n-{\bi} dry\\
			\glt	\lh{a|}`(he made and (then) went out onto dry land.)' \txrf{elicit. 13/02/16 p.16}}
\end{exe}

It \emph{is} possible for an M\=/form to occur before \ve{=ma}.
When this is the case, the words connected by \ve{=ma}
encode the same event, as discussed in \srf{sec:Mfo=Ma} below.
The ungrammaticality of examples (\ref{ex:130928-1, 1.54}′)
and (\ref{ex:120715-4, 0.45 ch:DisMet}′) is thus explained by the impossibility
of each of the verbs encoding an identical event.

\subsubsection{M\=/forms before \it{=ma} `and'}\label{sec:Mfo=Ma}
Examples (\ref{ex:130928-1, 1.54}′) and (\ref{ex:120715-4, 0.45 ch:DisMet}′)
can be contrasted with examples in which an M\=/form verb occurs before \ve{=ma}
and both verbs describe the same event, as illustrated in \qf{ex:=ma2} below.

%\begin{multicols}{2}
	\begin{exe}\let\eachwordone=\textnormal
		\ex{\rule{0pt}{0pt}{} \\[-2.5ex]
		\begin{tabular}{ll}
			$\xrightarrow{\textnormal{\normalsize event {[VERB{\M} \ve{=ma} VERB]} \hspace{2mm}}}$ & \\
		\end{tabular}}\label{ex:=ma2}
	\end{exe}

An example of two verbs connected by \ve{=ma} describing a single event
is given in \qf{ex:120715-3, 0.14} below.
In this example the event encoded by the verb following \ve{=ma}
anaphorically refers to the same event encoded by the verb preceding \ve{=ma}.
%An example of two temporally independent verbs
%in which the first is not encoded as occurring directly before
%the second is given in \qf{ex:120715-3, 1.09} below.

\begin{exe}
	\ex{\glll	fee{\gap}mnaisʔ=ee na-su\tbr{un} =\tbr{ma} n-moaʔ on reʔ ia.\\
						fee{\gap}mnasiʔ=ee na-suna =ma n-moʔe on reʔ ia\\
						wife{\gap}old={\ee} \na-spin.thread{\tbrM} =\tbr{and} \n-do like {\reqt} {\ia}\\
			\glt	`The old woman spun thread doing it like this.'
						\txrf{120715-3, 0.14} {\emb{120715-3-00-14.mp3}{\spk{}}{\apl}}}\label{ex:120715-3, 0.14}
\end{exe}

This pattern is particularly common in poetic parallelism,
in which two semantically parallel verbs are used to describe a single event.
An example is given in \qf{ex:160326, 1.50} below,
in which the verbs on either side of the connector \ve{=ma}
are near-synonyms used to describe a single event.
Poetic parallelism is discussed in more detail in \srf{sec:PoePar}.

\begin{exe}
	\ex{\glll	mu-he\tbr{un} =\tbr{ma} mu-tiis paah{\gap}pina-n\\
						mu-henu =ma mu-tisi paha{\gap}pina-n\\
						{\muu}-fill{\tbrM} =\tbr{and} {\mut}-pour country{\gap}below-{\N}\\
			\glt	`Fill [doublet] the earth.' \txrf{160326, 1.50} {\emb{160326-01-50.mp3}{\spk{}}{\apl}}}\label{ex:160326, 1.50}
\end{exe}

\subsubsection{Large numerals}\label{sec:LarNum}
One specific kind of dependent coordination with \ve{=ma} `and' involves large numbers.
In this case numerals before the connector \ve{=ma} obligatorily occur in the U\=/form.
Three examples are given in \qf{ex:83}--\qf{ex:130823-5, 0.42} below.

\begin{exe}
	\ex{\glll	boʔ \bxA{fa\tbr{nu}} =\tbr{m} \bxB{teun} {\lk}\\
						boʔ {\gp}fanu =ma {\gp}tenu\\
						ten {\gp}eight{\tbrU} =\tbr{and} {\gp}three{\M}\\
			\glt `eighty-three' (83)  \hfill{\emb{boq-fanum-teun.mp3}{\spk{}}{\apl}}\vspace{4pt}}\label{ex:83}
	\ex{\glll	nifun \bxA{ni\tbr{ma}} =\tbr{m} natun \bxB{hi\tbr{tu}} =\tbr{m} boʔ \rnode{C}{\psframebox[linewidth=0.4pt]{nee}} =m faun
						\ncbar[offsetB=-4pt,arm=5pt,angle=90,linewidth=0.4pt]{->}{A}{B}
						\ncbar[offsetA=-4pt,arm=5pt,angle=90,linewidth=0.4pt]{->}{B}{C} \\
						nifun {\gp}nima =ma natun {\gp}hitu =ma boʔ {\gp}nee =ma fanu\\
						thousand {\gp}five{\tbrU} =\tbr{and} hundred {\gp}seven{\tbrU} =\tbr{and} ten {\gp}six and eight{\M}\\
			\glt `five thousand seven hundred and sixty-eight' (5,768)
						 \hfill{\emb{nifun-nimam-natun-hitum-boq-neem-faun.mp3}{\spk{}}{\apl}}\vspace{4pt}}
	\ex{\glll	nifun boʔ \bxA{hi\tbr{tu}} =\tbr{m} \bxB{niim} {\lk} \\
						nifun boʔ {\gp}hitu =ma {\gp}nima \\
						thousand ten {\gp}seven{\tbrU} =\tbr{and} {\gp}five{\M}\\
			\glt	`Seventy-five thousand' (75,000)
						\txrf{130823-5, 0.42} {\emb{130823-5-00-42.mp3}{\spk{}}{\apl}}}\label{ex:130823-5, 0.42}
\end{exe}

In such instances the U\=/form numeral is not an event which
occurs chronologically prior to the following numerals,
but instead the U\=/form signals that the numeral is not complete.
The final numeral -- an M\=/form in each of the examples above --
resolves all previous U\=/forms and signals completion of the numeral.
%This is in contrast to the use of U\=/forms in the noun phrase
%in which a nominal in the M\=/form signals that the noun phrase
%is not yet complete with a U\=/form signalling completion of the phrase.
%See Chapter \ref{ch:SynMet}, especially \srf{sec:AttMod}.

\subsection{Dependent coordination with \it{=te}}\label{sec:Coo=Te}
The connector \ve{=te} marks a background event
which sets the scene for the following event.
The clause preceded by \ve{=te} is the stage
on which the following event takes place.
The event or situation followed by \ve{=te} begins before the second event
and is usually ongoing when the second event begins.
This is illustrated in \qf{ex:=te} below
in which the arrows represent the temporal duration of an event.
There are 105 examples of dependent coordination with
a U\=/form and the connector \ve{=te} in my corpus.

\begin{exe}\let\eachwordone=\textnormal
	\ex{\rule{0pt}{0pt}{} \\[-2.5ex]
	\begin{tabular}{ll}
		\hspace{15mm}$\xrightarrow{\textnormal{\normalsize event\sub{2}({\M}) \hspace{12mm}}}$ &\\
		$\xrightarrow{\textnormal{\normalsize event\sub{1}{\U} \ve{=te} \hspace{5mm}}}$ & \\
	\end{tabular}}\label{ex:=te}
\end{exe}

Two examples are given in \qf{ex:130913-1, 0.00} and \qf{ex:130913-1, 2.43} below.
In example \qf{ex:130913-1, 0.00} the U\=/form verb \ve{n-mate} `dies, is dead'
encodes a state which must happen before the M\=/form verb
\ve{t-suub} `bury' can be carried out.
Likewise, in example \qf{ex:120923-2, 5.25} the U\=/form \ve{mu-hini} `know'
encodes a state which must hold if the event encoded by the M\=/form final
serial verb construction \ve{m-suir m-aan} `heal' is to occur.

\begin{exe}
	\ex{\glll	nehh, \sf{{\j}adi} iin \bxA{n-ma\tbr{te}} =\tbr{te} \bxB{t-suub=ee} \hspace{30mm} on pani-n neefgw=ee? {\lk}\\
						{} \sf{{\j}adi} ini {\gp}n-mate =te {\gp}t-suba=ee {} on pani-n nefo=ee\\
						{} so {\iin} \gp\n-die{\tbrU} {=\tbr{\te}} \gp\t-bury{\Mv}={\eeV} {} {\on} across-{\N} lake={\ee}\\
			\glt	`So, when he's dead we bury him beside the lake?'
						\txrf{130913-1, 0.00} {\emb{130913-1-00-00.mp3}{\spk{}}{\apl}}\vspace{4pt}}\label{ex:130913-1, 0.00}
	\ex{\glll	reko papa =m hoo \bxA{mu-hi\tbr{ni}} =\tbr{t} a|m-turan he \bxB{m-suir} m-aan =kau hee. {\lk}\\
						reko papa =ma hoo {\gp}mu-hini =te {\a}m-turan he {\gp}m-suri m-ana =kau hee\\
						good dad =and {\hoo} \gp\muu-know{\tbrU} {=\tbr{\te}} \a\m-help {\he} \gp\m-heal{\M} \m-\ana{\M} ={\kau} hey\\
			\glt	`It's good, dad, if you know how to help heal me.'
						\txrf{120923-2, 5.25} {\emb{120923-2-05-25.mp3}{\spk{}}{\apl}}}\label{ex:120923-2, 5.25}
\end{exe}

Another two examples are given in \qf{ex:120923-2, 5.25} and \qf{ex:130825-6, 21.34} below.
In example \qf{ex:130825-6, 21.34} the U\=/form verb \ve{ʔ-toko} `sit'
describes a state which held when the M\=/form verb \ve{n-aun} `disturb' occurred.
Similarly, in example \qf{ex:130913-1, 2.43} the U\=/form verb \ve{n-toko} `sits'
encodes an event which will be ongoing at the time of the next event.

\begin{exe}
\vspace{4pt}
	\ex{\glll	\bxA{a|ʔ-tok{\tl}to\tbr{ko}} =\tbr{t} n-eu, kmii\j=ii \bxB{n-aun} =kaagw=een. {\lk}\\
						{\gp\a}ʔ-tok{\tl}toko =te n-eu kmii=ii {\gp}n-anu =kau=ena\\
						\gp\a\q-{\prd}sit{\tbrU} =\tbr{\te} {\n-\eu} urine={\ii} \gp\n-disturb{\M} ={\kau}={\een}\\
			\glt	\lh{\a}`I was sitting there and needed to relieve myself.'\\
						\lh{\a}(\emph{lit.} `While sitting, the urine disturbed me.')
						\txrf{130825-6, 21.34} {\emb{130825-6-21-34.mp3}{\spk{}}{\apl}}\vspace{4pt}}\label{ex:130825-6, 21.34}
	\ex{\glll	iin \bxA{n-to\tbr{ko}} =t, iin ofa n-reis \bxB{n-ain} areʔ haef=ein msaʔ.	{\lk}\\
						ini {\gp}n-toko =te ini ofa n-resi {\gp}n-ani areʔ haef=eni msaʔ \\
						{\iin} \gp\n-sit{\tbrU} ={\te} {\iin} sure \n-plan \gp\n-first{\M} each messenger={\ein} even \\
			\glt	`He'll sit and surely even plan the messengers beforehand.'
						\txrf{130913-1, 2.43} {\emb{130913-1-02-43.mp3}{\spk{}}{\apl}}}\label{ex:130913-1, 2.43}
\end{exe}

There is some overlap in the use of U\=/forms before the connectors \ve{=te} `\tsc{set}' and \ve{=ma} `and'.
For instance, example \qf{ex:130909-5, 0.28} on \prf{ex:130909-5, 0.28}
has the verb \ve{{\rt}toko} `sit' as the U\=/form before \ve{=ma},
much like examples \qf{ex:130825-6, 21.34} and \qf{ex:130913-1, 2.43} above
in which \ve{{\rt}toko} `sit' precedes \ve{=te}.
While all three examples encode an event which happened while sitting,
in \qf{ex:130909-5, 0.28} with \ve{=ma} there is more emphasis
on the initial action of the subject assuming a sitting position.
In examples \qf{ex:130825-6, 21.34} and \qf{ex:130913-1, 2.43} on the other hand,
the initial action of sitting down is less relevant
and the emphasis is on the sitting as an ongoing state.

It is not uncommon for the event/state preceded by \ve{=te} to refer to a specific time.
Two examples are given in \qf{ex:130825-6, 4.51} and \qf{ex:130825-6, 7.28} below.
In each of these examples the U\=/form verb encodes the time of 
day at which the event encoded by the next verb takes place.

\begin{exe}
		\ex{\glll	Mere, airoo, Mere, maans=ee \bxA{n-ma\tbr{be}} =\tbr{t} hoo \bxB{mu-kpesaʔ} {\lk}\\
							Mere airoo Mere manas=ee {\gp}n-mabe =te hoo {\gp}mu-kpesaʔ\\
							Mary oh Mary sun={\ee} \gp\n-afternoon{\tbrU} =\tbr{\te} {\hoo} \gp\muu-sift{\Uc}\\
				\glt	`Mary, oh Mary, it's afternoon while you're sifting.'
							\txrf{130825-6, 4.51} {\emb{130825-6-04-51.mp3}{\spk{}}{\apl}}}\label{ex:130825-6, 4.51}
		\ex{\glll	{nmeu{\gap}\bxA{n-fi\tbr{ni}}} =\tbr{t}, \bxB{n-aena} n-bi aat \sf{dees}=ii, =m n-ak {\lk}\\
							{nmeu{\gap}n-fini} =te {\gp}n-aena n-bi ata \sf{desa}=ii =ma n-ak\\
							early.morning{\tbrU} =\tbr{\te} \gp\n-run{\Uc} \n-{\bi} up{\M} village={\ii} =and \n-say\\
				\glt	`Early in the morning he ran up to the village (head) and said' \txrf{130825-6, 7.28} {\emb{130825-6-07-28.mp3}{\spk{}}{\apl}}}\label{ex:130825-6, 7.28}
\end{exe}

Another two examples are given in \qf{ex:130902-1, 4.32 ch:DisMet}
and \qf{ex:130920-1, 0.51 ch:DisMet} below.
In each of these examples the U\=/form before \ve{=te}
is a cardinal numeral (\srf{sec:NumPhr}) and each
describes the exact day or time at which the next event occurs.

\begin{exe}
		\ex{\glll	neno \bxA{ni\tbr{ma}} =\tbr{te} hai \bxB{m-piir} \sf{bupati}. {\lk}\\
							neno {\gp}nima =\tbr{te} hai {\gp}m-piri \sf{bupati}\\
							day {\gp}five{\tbrU} {=\tbr{\te}} {\hai} \gp\m-choose{\M} regent\\
				\glt	`In five days we'll elect a (new) regent.'
							\txrf{130902-1, 4.32} {\emb{130902-1-04-32.mp3}{\spk{}}{\apl}}\vspace{4pt}}\label{ex:130902-1, 4.32 ch:DisMet}
		\ex{\glll	n-reuk \bxA{fa\tbr{nu}} =\tbr{te}, \sf{paʔ} \hp{Charl} Charles, \sf{paʔ} Graims \hspace{30mm} a|n-koen=oo-n \bxB{neem.} {\lk}\\
							n-reku fanu =te \sf{paʔ} {} Charles \sf{paʔ} Graims {} {\a}n-koen=oo-n nema\\
							\n-hit eight{\tbrU} {=\tbr{\te}} Mr. {} Charles Mr. Grimes {} \a\n-depart={\oo-\N} {\nema\M}\\
				\glt	`As it struck 8:00 Mr. Charles, Mr. Grimes came.'
							\txrf{130920-1, 0.51} {\emb{130920-1-00-51.mp3}{\spk{}}{\apl}}}\label{ex:130920-1, 0.51 ch:DisMet}
\end{exe}

The connector \ve{=te} almost always occurs after U\=/forms
and it is usually ungrammatical for \ve{=te} or its allomorph \ve{=ate} (used after consonants)
to occur after a word in the M\=/form.
This ungrammaticality is explained by the fact that \ve{=te}
explicitly marks an event as only relevant in the context of another event.
Thus, it must co-occur with a U\=/form which
marks an event as resolved by a following event.
Two examples are given in (\ref{ex:130902-1, 4.32 ch:DisMet}′)
and (\ref{ex:130920-1, 0.51 ch:DisMet}′) below.

\begin{exe}
	\exp{ex:130902-1, 4.32 ch:DisMet}[*]{\glll
								neno ni\tbr{im} =\tbr{te} hai m-piir \sf{bupati}\\
								neno nima =\tbr{te} hai m-piri \sf{bupati}\\
								day five{\tbrM} {=\tbr{\te}} {\hai} \m-choose{\M} regent\\
					\glt	`(In five days we'll elect a (new) regent.)' \txrf{elicit. 22/02/16 p.21}}
	\exp{ex:130920-1, 0.51 ch:DisMet}[*]{\glll
								n-reuk fa\tbr{un} =\tbr{ate} paʔ Charles paʔ Graims \hspace{34mm} a|n-koen=oo-n neem\\
								n-reku fanu =te paʔ Charles paʔ Graims {} {\a}n-koen=oo-n nema\\
								\n-hit eight{\tbrM} {=\tbr{\te}} Mr. Charles Mr. Grimes {} \a\n-depart={\oo-\N} {\nema}\\
					\glt	`(As it struck 8:00 Mr. Charles, Mr. Grimes came.)' \txrf{elicit. 13/02/16 p.15}}
\end{exe}

While it would be possible to analyse this as a case of
morphemically conditioned metathesis (\srf{sec:MorpheConMet}),
this analysis would ignore the generalisation that
U\=/forms are used mark events resolved by a following event.
The inability of \ve{=te} to occur with an M\=/form is due to \ve{=te}
requiring another event for which it sets the stage.\footnote{
		Some evidence in favour of analysing this as morphemically conditioned
		metathesis may come from the universal occurrence in my corpus of verbal U\=/forms
		before the enclitic \ve{=ha} `just, only'.
		(Though this has not yet been tested under elicitation.)
		While morphemically conditioned metathesis may be able
		to account for the use of U\=/forms before \ve{=te} `{\te}'
		and \ve{=ha} `just, only', it cannot account for the use of
		both U\=/forms and M\=/forms before \ve{=ma} (\srf{sec:Coo=Ma}) or
		examples in which no connector occurs (\srf{sec:CooCon}).
		It also cannot account for the use of U\=/forms in conversation (\srf{sec:IntUnm}).}

\subsubsection{\it{rari =te} `after that'}
One verb which frequently occurs with \ve{=te}
in dependent coordination is \ve{rari} `finish'.
Such instances of \ve{rari =te} are examples of a reduced adverbial clause \citep[211]{le88}.
Two examples are given in \qf{ex:130902-1, 0.39-0.51} and \qf{ex:130825-6, 10.05-10.41} below.
In each example the event preceding \ve{rari =te} was completed before
the beginning of the event following \ve{rari =te}.

\begin{exe}
	\ex{Organising a wedding reception: \txrf{130902-1} {\emb{130902-1-00-39-00-51.mp3}{\spk{}}{\apl}}}\label{ex:130902-1, 0.39-0.51}
	\begin{xlist}
		\ex{\glll	{okeʔ =te}, hai m-ʔator, \sf{aʧara,} n-eu reʔ, ahh, \hspace{31mm} oras toup \sf{tamu}, \sf{resepsi}\\
							{okeʔ =te} hai m-ʔator \sf{aʧara} n-eu reʔ {} {} oras topu \sf{tamu} \sf{resepsi}\\
							after.that {\hai} \m-arrange event {\n-\eu} {\reqt} {} {} time receive guest reception\\
				\glt	`After that we arranged an event, a time to receive guests, a reception.' \txrf{0.39}}
		\ex{\gll	hai mi-\tbr{rari} =\tbr{te},\\
							{\hai} \mi-finish{\tbrU} {=\tbr{\te}}\\
				\glt	`When we finished that,' \txrf{0.48}}\label{ex:130902-1, 0.48}
		\ex{\gll	hai m--, m-fee mainuan n-eu anaʔapreent =ama areʔ saksii mahonit he n-fee, ahh, fainekat. \\
							{\hai} {} \m-give opportunity {\n}-{\eu} official =and every witness elder {\he} {\n}-give {} advice\\
				\glt `We gave an opportunity to the government officials
							and each of the witnesses and clan elders to give advice.' \txrf{0.51}}
	\end{xlist}
	\ex{Organising clothes to go to a wedding: \txrf{130825-6} {\emb{130825-6-10-16-10-18.mp3}{\spk{}}{\apl}}}\label{ex:130825-6, 10.05-10.41}
		\begin{xlist}
			\ex{\glll	ʔ-\sf{istarika} =m,\\
								ʔ-\sf{istarika} =ma \\
								\q-iron{\U} =and\\
					\glt	`I ironed (my pants) and,' \txrf{10.16}}\label{ex:130825-6, 10.16}
			\ex{\glll	u-\tbr{rari} =\tbr{te}, ʔ-aena ʔ-bi nahen Jes Oraʔ nee,\\
								u-rari =te ʔ-aena ʔ-bi nahen Jes Oraʔ nee\\
								\qu-finish{\tbrU} {=\tbr{\te}} \q-run{\Uc} \q-{\bi} down Jes Ora{\Q} {\nee}\\
					\glt	`having finished I ran down there to Jes Ora{\Q}.' \txrf{10.18}}\label{ex:130825-6, 10.18}
		\end{xlist}
\end{exe}

When \ve{rari} co-occurs with \ve{=te},
it does not have to take agreement prefixes.
Such instances of \ve{rari =te}
are often best translated as `after that'.
There are three such examples in my corpus.
Two of these are given in \qf{ex:130902-1, 3.23-3.28} and \qf{ex:130921-1, 0.43-0.50} below.

In example \qf{ex:130902-1, 3.23-3.28} \ve{rari =te} `finish'
serves to transition between two episodes of the story.
It marks that the penultimate event of the wedding reception
had finished (\ve{na-prir{\tl}riraʔ} `dance')
before the final event took place (\ve{n-ma-taeb} `shake hands'),
and the main characters of the story left the wedding reception.

\begin{exe}
	\ex{Attending a wedding reception: \txrf{130902-1} {\emb{130902-1-03-23-03-34.mp3}{\spk{}}{\apl}}}\label{ex:130902-1, 3.23-3.28}
		\begin{xlist}
			\ex{\glll	naiʔ Owen a|msaʔ n-ok na-bsooʔ na-priraʔ kuu-n.\\
								naiʔ Owen {\a}msaʔ n-oka na-bsoʔo na-priraʔ kuu-n\\
								{\naiq} Owen {\a}also \n-{\ok\M} {\na}-dance{\M} {\na}-dance{\Uc} self-{\N}\\
					\glt	`Owen also joined in the dancing by himself (i.e. without me).' \txrf{3.23}}
			\ex{\gll	na-prir{\tl}riraʔ mhh.\\
								\na-{\prd}dance.with.arms{\Uc} {}\\
					\glt	`He danced and danced.' \txrf{3.26}}
			\ex{\glll	ahh, \tbr{rari} =\tbr{te}, n-ma-taeb n-ok ahh baroit=n=eni =ma hai m-tebi m-fain iim.\\
								{} rari =te n-ma-tabe n-oka {} baroti=n=eni =ma hai m-tebi m-fani ima\\
								{} finish{\tbrU} {=\tbr{\te}} \n-{\mak}-greet{\M} \n-{\ok\M} {} bride/groom={\ein}={\ein} =and {\hai} \m-turn{\Uc} \m-return{\M} {\ima\M}\\
					\glt	`After that he shook hands with each of the bride and groom and we turned and came back.' \txrf{3.34}}
		\end{xlist}
\end{exe}

Example \qf{ex:130921-1, 0.43-0.50} below shows that such uses of \ve{rari =te}
have become semantically bleached, with the meaning `finish'
giving way to a more general `after that'.
In example \qf{ex:130921-1, 0.43-0.50} the event preceding
the reduced adverbial clause is \ve{mi-sopu m-rair} `finished completing',
in which the last verb of the serial verb construction
has the same root as that of \ve{rari =te}.

\begin{exe}
	\ex{Reading books of the Bible: \txrf{130921-1} {\emb{130921-1-00-43-00-50.mp3}{\spk{}}{\apl}}}\label{ex:130921-1, 0.43-0.50}
		\begin{xlist}
			\ex{\gll	hai mi-sopu m-rair Roma, ees nean haa-ʔ=ii\\
							%	hai mi-sopu m-rari Roma esa neno haa-ʔ=ii\\
								{\hai} \mi-complete \m-finish Romans {\esc} day four-{\qnum}={\ii}\\
					\glt	`We completed Romans, it was on Thursday.' \txrf{0.43}}
			\ex{\glll	\tbr{rari} =\tbr{t}, nean niim-ʔ=ii,\\
								rari =te neno nima-ʔ=ii \\
								finish{\tbrU} {=\tbr{\te}} day five-{\qnum}={\ii} \\
					\glt	`After that, on Friday,' \txrf{0.47}}
			\ex{\glll	 hai mi-koon-b=ee n-ok naiʔ Yohanis iin surat reʔ, a-hunu-t\\
								hai mi-kono-b=ee n-oka naiʔ Yohanis ini surat reʔ a-hunu-t\\
								{\hai} \mi-keep.on{\Mv}-\b={\eeV} \n-{\ok\M} {\naiq} John {\iin} paper {\req} {\at}-first-{\at}\\
					\glt	`we kept going with John's first book (1 John).' \txrf{0.50}}
		\end{xlist}
\end{exe}

There are only three examples of \ve{rari =te}
without an agreement prefix found in my corpus.
However, a search of the Amarasi Bible yielded 2,733
instances of \ve{rari} without an agreement prefix preceding \ve{=te}.
All but two of these are orthographic \ve{<rarit>}
or \ve{<Rarit>} with \ve{te} reduced to a single consonant,
as in \qf{ex:130921-1, 0.43-0.50} above.
The Amarasi Bible contains 27 instances of \ve{<-rari>}
with an agreement prefix followed by full \ve{<te>}.

\subsection{Dependent coordination with no connector}\label{sec:CooCon}
There are also a handful of examples in my corpus
of dependent coordination in which neither of the connectors
\ve{=ma} `and' or \ve{=te} `\tsc{set}' occur.
In examples \qf{ex:130907-3, 5.13} and \qf{ex:120923-1, 4.18}
the event encoded by the U\=/form chronologically precedes the next event.
%In \qf{ex:120923-1, 4.18} the connector \ve{mes} occurs.

\begin{exe}
	\ex{\gll	usi n-ro\tbr{mi} uma ʔ-nao.\\
						king \n-like{\tbrU} \uma{\Uc} \q-go\\
			\glt	`The king liked (that), so I came back.'
						\txrf{130907-3, 5.13} {\emb{130907-3-05-13.mp3}{\spk{}}{\apl}}}\label{ex:130907-3, 5.13}
	\ex{\glll	n-aka n-mani\tbr{ni} mes na-see\j=oo-n \hspace{40mm} reʔ ia ro n-tuup=een.\\
						n-aka n-manini mes na-see=oo-n {} reʔ ia ro n-tupa=ena\\
						\n-say {\n}-fever{\tbrU} but \na-excuse{\Mv}={\oo}-{\N} {} {\req} {\ia} must \n-sleep{\Mv}={\een}\\
			\glt	`He said he had fever but excused himself to sleep.'
						\xrf{120923-1, 4.18} {\emb{120923-1-04-18.mp3}{\spk{}}{\apl}}}\label{ex:120923-1, 4.18}
\end{exe}

In example \qf{ex:160326, 5.37} below the serial
verb construction \ve{ta-hiin t-ana} `figure out, get to know'
introduces a list of different information which resolves this U\=/form.
This usage is not dissimilar to the
use of U\=/forms in large numerals (\srf{sec:LarNum}).

\begin{exe}
	\ex{The settling of Koro{\Q}oto hamlet: \txrf{160326} {\emb{160326-05-37-05-45.mp3}{\spk{}}{\apl}}}\label{ex:160326, 5.37-5.45}
		\begin{xlist}
		\ex{\glll	siin neem na-tua Koorʔoot ees reʔ oras mee \hspace{25mm} ka= ta-hiin t-a\tbr{na} =f.\\
							sini neem na-tua Koorʔoto esa reʔ oras mee {} ka= ta-hini t-ana =f.\\
							{\siin} {\nema} \na-settle Koro{\Q}oto{\M} {\esc} {\req} time where {} {\ka}= {\tg}-know {\tg}-{\ana\tbrU} ={\fa}\\
				\glt	`They came and settled in Koro{\Q}oto, it was at a time which hasn't been figured out.' \txrf{5.37}}\label{ex:160326, 5.37}
		\ex{\gll	bian n-ak, of fuunn=ees reʔ \sf{kira-kira} \sf{abat} \sf{ke-ʔempat} \sf{blas.}\\
							some \n-say sure month={\es} {\req} around century \tsc{ord}-four ten\\
				\glt	`Some say/think it was a month in the fourteenth century.' \txrf{5.45}}\label{ex:160326, 5.45}
		\ex{\gll	bian n-ak, ma-tu\<ʔ\>i n-bi \sf{balai} \sf{desa} =te n-ak, \sf{kira-kira} \sf{abat} \sf{ke-delapan} \sf{blas.}\\
							some \n-say {\ma}-write\<\ma\> \n-{\bi} office village ={\te} \n-say around century \tsc{ord}-eight ten\\
				\glt	`Some say/think, (as) is written in the village office, that it was around the eighteenth century.' \txrf{5.45}}\label{ex:6.01}
	\end{xlist}
\end{exe}

The introduction of a list is particularly common with
the U\=/form of the plural enclitic \ve{=eni} (\srf{sec:PluEnc}).
In such cases, \ve{=eni} occurs attached to a nominal
and the list enumerates the members of that nominal.
%The U\=/form \ve{=eni} is resolved by the list.
Such examples represent just under half (7/18) of all U\=/forms
of the plural enclitic \ve{=eni} in my corpus.

Three examples are given in \qf{ex:130920-1, 1.32-1.36}--\qf{ex:160326, 18.26} below.
In each case the contents of the list resolve the U\=/form.
In example \qf{ex:130920-1, 1.32-1.36} the form \ve{=eni} is attached to \ve{a-resa-t}
`reader' and introduces a list of proper names:
the people who were the readers.

\begin{exe}
	\ex{Reading books of the Bible: \txrf{130920-1} {\emb{130920-1-01-32-01-36.mp3}{\spk{}}{\apl}}}\label{ex:130920-1, 1.32-1.36}
	\begin{xlist}
%		\ex{\glll	hai m-rees, {okeʔ =te} hai m-mak-tana=n, aiʔ na-taan kai,\\
%							hai m-resa {okeʔ =te} hai m-mak-tana=n  aiʔ na-tana kai\\
%							{\hai} \m-read{\M} then {\hai} \m-\mak-ask={\einV} or \na-ask{\M} {\kai}\\
%				\glt	`We read, then we asked one another or we were asked,' \txrf{1.26}}
		\ex{\glll	aiʔ na-taan a-rees-t=e\tbr{ni}, ahh\\
							aiʔ na-tana a-resa-t=eni\\
							or \na-ask{\M} {\at}-read-{\at}={\ein\tbrU}\\
				\glt	`or the readers were asked' \txrf{1.32}}
		\ex{\gll	bi Yane, ain Lince, aam Ferdi\\
							{\BI} Yane mother Lince father Ferdi\\
				\glt	`Yane, Lince (and) Ferdi (were the readers).' \txrf{1.36}}
	\end{xlist}
\end{exe}

Similarly, in \qf{ex:130928-1, 2.05-2.09} the form \ve{=eni}
introduces a list of people who correspond to the head nominal
\ve{nuuk tuaf} `people in grief'.
In this example only the main member of this group (\ve{Fanu})
is introduced with a proper name while the other members
are mentioned by their relationship to him.

\begin{exe}
	\ex{The death of Nahor Bani: \txrf{130928-1} {\emb{130928-1-02-05-02-09.mp3}{\spk{}}{\apl}}}\label{ex:130928-1, 2.05-2.09}
	\begin{xlist}
		\ex{\glll	nuuk tuaf=e\tbr{ni} naiʔ Fanu n-ok areʔ iin tata-f,\\
							nuka tuaf=eni naiʔ Fanu n-oka areʔ ini tata-f\\
							grief person={\ein\tbrU} {\naiq} Fanu \n-{\ok\M} each {\iin} eSi-{\F}\\
				\glt	`The ones in grief, Fanu and each of his older siblings,' \txrf{2.05}}
		\ex{\gll	es{\tl}esa =t n-ok iin fee iin mone	\\
							{\prd}{\es}{\U} ={\te} \n-{\ok\M} {\iin} wife {\iin} husband\\
				\glt	`each with their wife or husband.' \txrf{2.09}}
	\end{xlist}
\end{exe}

In example \qf{ex:160326, 18.26} \ve{=eni} introduces a list of (two) names
but in this instance these names are not people but rather members
of the group \ve{kaan aku-f} `special name'.

\newpage
\begin{exe}
	\ex{\glll	siin naiʔ Bain mone \sf{kusus}, \hspace{60mm} siin kaan auk-k=e\tbr{ni} bisa, Mea aiʔ Tutun.\\
						sini naiʔ Bani mone \sf{kusus} {} sini kana aku-k=eni bisa Mea aiʔ Tutun\\
						{\siin} {\naiq} Bani{\M} male exclusive {} {\siin} name{\M} praise.name-{\k=\ein\tbrU} can Mea or Tutun\\
			\glt	`Members of the Bani clan classified as male can
						exclusively have the praise names Mea or Tutun.'\footnote{
								In Amarasi society the classification of households as \ve{mone} `male'
								or \ve{feto} `female' refers to their social relationship to one another
								rather than biological gender.
								%Specifically, those
								%who have received a woman from another household as a wife is classified as \ve{feto} `female'
								%in relation to the wife-givers,
								%while the wife-givers who have received a husband
								%are classified as \ve{mone} `male' \cite[411f]{scno71}.
								See \srf{sec:MetPar} for discussion of the complementary pair \ve{feto-mone} `female-male'
								as well as the connection between metathesis and the
								Amarasi division of the world into parallel and complementary pairs.}
						\txrf{160326, 18.26} {\emb{160326-18-26.mp3}{\spk{}}{\apl}}}\label{ex:160326, 18.26}
\end{exe}

In summary, dependent coordination can also occur when
neither of the connectors \ve{=ma} or \ve{=te} occur.
One specific kind of dependent coordination without a connector
is the use of the U\=/form \ve{=eni} `{\ein}' to introduce a list.
In such instances the list resolves the plural marker.

\subsection{Place names}
Native place names participate in discourse-driven metathesis.
As with verbs, the default form of vowel-final place names in Amarasi is the M\=/form.
This includes certain locational nouns
such as \ve{pina} {\ra} \ve{piin} `below',
and \ve{fafo} {\ra} \ve{faof} `above'.
Consonant-final place names, such as \ve{Kopan} `Kupang' and
\ve{Kuarenoʔ} (see \srf{sec:DisStrAma})
occur in the U\=/form (glossed {\Uc}) except before determiners.
However, place names which are vowel final occur by default in the M\=/form.

Three textual examples of a simple declarative
clause with a place name with a vowel-final root are given in
\qf{ex:130825-8, 1.00}--\qf{ex:160326, 17.41} below.
In each example the place name occurs in the M\=/form.

\begin{exe}
	\ex{\begin{xlist}
		\ex[α:]{\glll	Bein Masnenoʔ umi mee?\\
							Beni Masnenoʔ umi mee\\
							Benny Masneno{\Q} house where?\\
				\glt	`Where is Benny Masneno{\Q}'s house?' \txrf{130825-8, 1.00} {\emb{130825-8-01-00.mp3}{\spk{}}{\apl}}}
		\ex[β:]{\glll	Sonra\tbr{en}.\\
							Sonrane\\
							Sonraen{\tbrM}\\
				\glt	`Sonraen.' \txrf{}}
	\end{xlist}}\label{ex:130825-8, 1.00}
	\ex{\glll	\sf{paʔ} ʔnaak-- Inabui ʔnaak aanʔ=ii n-bi Oekbi\tbr{it}.\\
						\sf{paʔ} {} Inabuy ʔnaka anaʔ=ii n-bi Oekbiti\\
						Mr. {} Inabuy head small={\ii} \n-{\bi} Oekbiti{\tbrM}\\
			\glt	`Mr. Inabuy was the deputy leader in Oekbiti.'
						\txrf{130907-3, 5.31} {\emb{130907-3-05-31.mp3}{\spk{}}{\apl}}}\label{ex:130907-3, 5.31}
	\ex{\glll	ees reʔ Koorʔo\tbr{ot} na-heun bare{\tl}bare bian.\\
						esa reʔ Koorʔoto na-henu bare{\tl}bare bian\\
						{\esc} {\req} Koro{\Q}oto{\tbrM} \na-fill {\frd}place other\\
			\glt	`Koro{\Q}oto was the one which filled other places.'
						\txrf{160326, 17.41} {\emb{160326-17-41.mp3}{\spk{}}{\apl}}}\label{ex:160326, 17.41}
\end{exe}

The only environment in which place names are attested in the U\=/form in my corpus
is before either of the connectors \ve{=ma} or \ve{=te}
in a dependent coordination construction.
In such examples it is not the place name itself which is unresolved,
but rather the entire clause within which the place name occurs,
with the following clause adding additional information.
Three examples are given in \qf{ex:130914-3, 0.23}--\qf{ex:160326, 16.14} below.
While \ve{Sonraen} occurs in the M\=/form in \qf{ex:130825-8, 1.00},
when before the connector \ve{=te} in \qf{ex:130914-3, 0.23}
below it occurs in the U\=/form.

\begin{exe}
	\ex{\glll	iin n-tee Sonra\tbr{ne} =t, maans=ee n-peeʔ.\\
						ini n-tea Sonrane =te manas=ee n-peʔe\\
					{\iin} \n-arrive Sonraen{\tbrU} {=\tbr{\te}} sun={\ee} \n-break{\M}\\
			\glt	`When s/he arrived at Sonraen, it was sunrise'
						\txrf{130914-3, 0.23} {\emb{130914-3-00-23.mp3}{\spk{}}{\apl}}}\label{ex:130914-3, 0.23}
\end{exe}

Similarly, in \qf{ex:130907-3, 5.31} above \ve{Oekbiit} occurs in the M\=/form,
while in \qf{ex:130907-3, 4.41-4.45} below it is before \ve{=ma} and occurs in the U\=/form.

\begin{exe}
		\ex{\glll	nhh ʔ-nao =ma ʔ-nao ʔ-bi kantoor \sf{na} Oekbi\tbr{ti} =\tbr{ma},\\
							{} ʔ-nao =ma ʔ-nao ʔ-bi kantoor \sf{na} Oekbiti =ma,\\
							{} \q-go =and \q-go \q-{\bi} office well Oekbiti{\tbrU} =\tbr{and}\\
				\glt	`And so I went to the office (of), well, Oekbiti and {\ldots}'
							\txrf{130907-3, 4.41} {\emb{130907-3-04-41.mp3}{\spk{}}{\apl}}}\label{ex:130907-3, 4.41-4.45}
\end{exe}

Likewise, the name \ve{Koorʔoot} is in the M\=/form in \qf{ex:160326, 17.41} above,
but before the connector \ve{=te} in \qf{ex:160326, 16.14 a} below
it occurs in the U\=/form \ve{Koor{ʔ}oto}.
Example \qf{ex:160326, 16.17} also has an M\=/form of this place name.

\begin{exe}
	\ex{Praying for rain: \txrf{160326, 16.14} {\emb{160326-16-14.mp3}{\spk{}}{\apl}}}\label{ex:160326, 16.14}
	\begin{xlist}
		\ex{\glll	karu n-boefanu =m n-ak uurn=ii n-mouf n-eu =ha, n-eu =ha reʔ Koorʔo\tbr{to} =te, \\
							karu n-boefanu =ma n-ak uran=ii n-mofu n-eu =ha n-eu =ha reʔ Koorʔoto =te\\
							if \n-pray.fervently{\U} =and \n-say rain={\ii} \n-fall{\M} \n-{\eu} =only \n-{\eu} =only {\reqt} Koro{\Q}oto{\tbrU} ={\te}\\
			\glt	`If they prayed fervently for the rain to fall just on Koro{\Q}oto,' \txrf{}}\label{ex:160326, 16.14 a}
		\ex{\glll	uurn=ii n-eu =ha reʔ Koorʔo\tbr{ot}. kuan bian ka= na-peni=f.\\
							uran=ii n-eu =ha reʔ Koorʔoto kuan bian ka= na-peni=fa\\
							rain={\ii} \n-{\eu} =only {\reqt} Koro{\Q}oto{\M} village other {\ka}= \na-get={\fa}\\
				\glt	`the rain (fell) only on Koro{\Q}oto. Other villages wouldn't get any.' \txrf{}}\label{ex:160326, 16.17}
	\end{xlist}
\end{exe}

U\=/forms of place names likely occur in other environments
where discourse-driven U\=/forms occur,
such as tail-head linkage (\srf{sec:TaiHeaLin})
and question-answer pairs (\srf{sec:IntUnm}).
However, in my current corpus I only have clear
examples in dependent coordination.

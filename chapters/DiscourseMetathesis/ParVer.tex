\subsection{Semantically parallel verbs}\label{sec:ParVer}
Although the normal pattern in tail-head linkage is for tail and head to be encoded by identical verbs,
it is quite frequent for the two words to
be semantically parallel but not identical.
Of the 72 instances of tail-head linkage in my corpus,
fifteen involve parallel word pairs (21\%).

One example is given in \qf{ex:120715-4, 0.25-0.27} below,
in which the first clause consists of the serial verb construction
\ve{na-skeke n-fena n-hake} `sudden rise stand' with a final U\=/form \ve{n-hake}.\footnote{
		The only U\=/form of this serial verb construction which is in the U\=/form
		for discourse reasons alone is the final verb.
		The other U\=/forms are all followed by a consonant cluster,
		an environment which discourages M\=/forms (\srf{sec:CCIniMod}).}
This U\=/form is resolved in the third clause by the elaboration
introduced by the M\=/form \ve{n-feen}.

\newpage
\begin{exe}
	\ex{How the snake \it{Moo{\Q}hitu{\Q}} separated the sky from the land:
				\txrf{120715-4} {\emb{120715-4-00-25-00-30.mp3}{\spk{}}{\apl}}}\label{ex:120715-4, 0.25-0.27}
	\sn{\xytext{\fbox{stand{\U}\vp{\Uc}}\xybarconnect[2][-](D,D){2}\xybarconnect[2][->]{4}&and&\fbox{rise{\M}\vp{|\Uc}}&when&\fbox{sky spread{\Uc}\vp{|}}}}
	\begin{xlist}
		\ex{\gll	iin na-skeke n-fena n-ha\tbr{ke} =ma,\\
							{\iin} {\na}-sudden{\Uc} {\n}-rise{\Uc} {\n}-stand{\tbrU} =and\\
				\glt	`He suddenly stood up and'
							\txrf{0.25}}\label{ex:120715-4, 0.25}
		\ex{\glll	iin, iin n-fe\tbr{en} es mee =t \\
							ini ini n-fena es mee =te\\
							{\iin} {\iin} {\n}-rise{\tbrM} {\et} where ={\te}\\
				\glt	`he, as he rose up to somewhere,' \txrf{}}\label{ex:120715-4, 0.27a}
		\ex{\glll	neeŋgw=ii na-tsiriʔ, na-tsiriʔ \sf{sampe} iin n-tea reʔ aat neno nee msaʔ iin na-tuin=ee =ma\\
							neno=ii na-tsiriʔ na-tsiriʔ \sf{sampe} ini n-tea reʔ ata neno nee msaʔ ini na-tuin=ee =ma \\
							sky={\ii} {\na}-spread{\Uc} {\na}-spread{\Uc} until {\iin} {\n}-arrive {\req} up sky {\nee} also {\iin} {\na}-follow={\eeV} =and\\
				\glt	`the sky spread (and) spread until when he arrived at (the place) where the top of the sky also is, he followed it and' \txrf{0.30}}
	\end{xlist}
\end{exe}

Another example is given in \qf{ex:130825-6, 9.06-9.14} below.
In this example the serial verb construction \ve{ʔ-foro ʔ-mate}
`dead (completely) blind' with a final U\=/form in \qf{ex:130825-6, 9.12} is the tail.
This U\=/form is resolved by the M\=/form head
\ve{ka= ʔ-iit} `not see' in \qf{ex:130825-6, 9.14a},
which introduces the elaboration.

\begin{exe}
	\ex{Receiving a text message that can't be read: \txrf{130825-6} {\emb{130825-6-09-12-09-19.mp3}{\spk{}}{\apl}}}\label{ex:130825-6, 9.06-9.14}
	\sn{\xytext{\fbox{blind{\U}\vp{|}}\xybarconnect[2][-](D,D){2}\xybarconnect[2][->]{3}&and look down&\fbox{not see{\M}\vp{|}}&\fbox{put back{\M}\vp{|}}}}
	\begin{xlist}
		\ex{\emph{`I was bathing and this SMS made a noise in the mobile phone.'} \txrf{9.06}}
		\ex{\emph{`I took it and looked at it but'} \txrf{9.10}}
		\ex{\glll	hoo m-bi reʔ nahen pooʔn=ee =te, \hspace{25mm} ʔ-\tbr{foro} ʔ-\tbr{mate} =m\\
							hoo m-bi reʔ nahen poʔon=ee =te {} \q-foro \q-mate =ma\\
							{\hoo} {\m-\bi} {\reqt} down orchard={\ee} ={\te} {} \q-blind{\tbrUc} \q-die{\tbrU} =and\\
				\glt	`When you were down in the orchard, I was dead (completely) blind and' \txrf{9.12}}\label{ex:130825-6, 9.12}
		\ex{\glll	ʔ-tae, ka= ʔ-\tbr{iit}, u-tunuʔ u-fain.\\
							ʔ-tae ka= ʔ-ita u-tunuʔ u-fani\\
							\q-look.down {\ka}= \q-see{\tbrM} \qu-put{\Uc} \qu-back{\M}\\
				\glt	`I looked down at (it), couldn't see (it), (so I) put it back.' \txrf{9.14}}\label{ex:130825-6, 9.14a}
%		\ex[]{\glll	maut uma ʔ-tee kuan a|ʔ-paek \sf{kacamata} henaʔ,\\
%							maut uma ʔ-tea kuan {\a}ʔ-pake \sf{kacamata} henaʔ\\
%							let come{\Uc} \q-arrive village \a\q-use{\M} spectacles {\he}\\
%				\glt	`I should (wait) until I get to the village and use spectacles to {\ldots}' \txrf{9.14}}
%		\ex[β:]{\glll m-rees \\
%									m-resa \\
%									{\m}-read{\M} \\
%						\glt `read (it).' \txrf{9.19}}
	\end{xlist}
\end{exe}

The tail and head of a tail-head linkage construction
can either be identical verbs or semantically parallel verbs.
The use of U\=/forms and M\=/forms with parallel verbs is discussed
in more detail in \srf{sec:PoePar} on parallelism in Amarasi poetry.
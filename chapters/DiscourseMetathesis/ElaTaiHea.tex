\subsection{Elaboration between tail and head}\label{sec:ElaTaiHea}
While the usual pattern in tail-head linkage
is for the elaboration to follow the head,
the elaboration can also occur between the tail and the head.
There are fourteen such examples in my corpus.
An English example of tail-head linkage with elaboration
between the tail and head is given in \qf{ex:EngTaiHeaMed} below.

\begin{exe}\let\eachwordone=\p
	\ex{\begin{xlist}
		\ex{\it{I \tbr{arrived} home.}}
		\ex{\it{I went straight to the fridge when I \tbr{arrived}.}}
	\end{xlist}}\label{ex:EngTaiHeaMed}
\end{exe}

\subsubsection{U\=/form tail {\ldots} M\=/form head}
There are eight examples in my corpus in which the elaboration
occurs between a U\=/form head and an M\=/form tail.
Such examples are a kind of chiasmus,
illustrated in two ways in \qf{ex:ChiTaiHeaLin} below.
The U\=/form tail indicates that more information
is required for the event to be resolved,
with the M\=/form verb indicating that with the
previous information this event is resolved.

\begin{exe}
	\ex{Chiastic tail-head linkage:}\label{ex:ChiTaiHeaLin}
		\begin{xlist}
			\ex{
				\begin{xlist}
					\exi{A.}{event\sub{1}{\U} (tail)}
					\sn{B. elaboration}
					\exi{A.}{event\sub{1}{\M} (head)}
				\end{xlist}}
			\ex{\xytext{\fbox{event\sub{1}{\U}}\xybarconnect[2][-](D,D){2}\xybarconnect[2][->]{1}&
									\fbox{elaboration\vp{|}}&\fbox{event\sub{1}{\M}}}}
		\end{xlist}
\end{exe}

In example \qf{ex:130907-4, 3.13-3.35} below the tail and head
of \ve{topu} `receive, accept' occur on either side of information
explaining the time and manner in which this event occurred.

\begin{exe}
	\ex{A son's education: \txrf{130907-4} {\emb{130907-4-03-26-03-35.mp3}{\spk{}}{\apl}}}\label{ex:130907-4, 3.13-3.35}
	\sn{\xytext{\fbox{not accepted{\U}}\xybarconnect[2][-](D,D){5}\xybarconnect[2][->]{1}\xybarconnect[4][->]{3}&\fbox{study}&and.so&\fbox{arrive\vp{dy}}&when&\fbox{not accepted{\M}}}}
	\begin{xlist}
		\ex{\emph{`I thought about it first, we'll have the next one study government'} \txrf{3.13}}
		\ex{\emph{`So Adi went down there,'} \txrf{3.15}}
		\ex{\emph{`And he came back, he came back and told me'} \txrf{3.21}}
		\ex{\glll	ka= n-to\tbr{pu} =f\\
							ka= n-topu =fa\\
							{\ka}= {\n}-receive{\tbrU} ={\fa}\\
				\glt	`he wasn't accepted' \txrf{3.26}}
		\ex{\glll	bait he aam Adi iin na-skora prenat.\\
							bait he ama Adi ini na-skora prenat\\
							actually {\he} father Adi {\iin} {\na}-study{\Uc} government\\
				\glt	`Actually Adi was going to study government.' \txrf{3.28}}
		\ex{\glll	{onai =m} mes a|n-tee neʔ nahe-n neʔ skoor nahe-n nee =te ka= n-to\tbr{up} =fa. \\
							{onai =ma} mes {\a}n-tea neʔ nahe-n neʔ skoor nahe-n nee =te ka= n-topu =fa\\
							and.so but {\a\n}-arrive {\reqt} down-{\N} {\reqt} school down-{\N} there ={\te} {\ka}= {\n}-receive{\tbrM} ={\fa}\\
				\glt	`But he arrived at the school down there and wasn't accepted.' \txrf{3.35}}
	\end{xlist}
\end{exe}

A more complex example is given in \qf{ex:130913-1, 0.16} below,
in which the U\=/form verb \ve{n-mate} `dies' is the tail.
In this case the elaboration between the tail and head
is itself an instance of dependent coordination (\srf{sec:DepCoo})
which describes the manner in which the death will occur.
This elaboration is followed by the M\=/form head, which signals
that the previous information has resolved the event (\ve{n-mate} `dies')
encoded by the tail and head.

\begin{exe}
	\ex{Someone who is ready for when he dies:
			\txrf{130913-1, 0.16} {\emb{130913-1-00-16.mp3}{\spk{}}{\apl}}}\label{ex:130913-1, 0.16}
	\sn{\xytext{\fbox{die{\U}}\xybarconnect[2][-](D,D){6}\xybarconnect[2][->]{2}\xybarconnect[4][->]{4}&when&\fbox{wear shirt already{\U}}\xybarconnect[2][->](UR,UL){2}&and&\fbox{enter\M}&and then&\fbox{die{\M}}}}
	\begin{xlist}
		\ex{\glll	{ona =t} of iin n-ma\tbr{te} =t, iin na-baur n-ani =m,\\
							{onai =te} of ini n-mate =te ini na-baru n-ani =ma\\
							then later {\iin} {\n}-die{\tbrU} ={\te} {\iin} {\na}-shirt{\M} {\n}-before{\U} =and\\
				\glt	`Then later he'll die while wearing a shirt
							(previously selected for the occasion of his death) and' \txrf{}}
		\ex{\glll	iin n-taam n-eu peet\j=ee =m, naʔ n-ma\tbr{et}.\\
							ini n-tama n-eu peti=ee =ma naʔ n-mate\\
							{\iin} {\n}-enter{\M} {\n-\eu} coffin={\ee} =and then {\n}-die{\tbrM}\\
				\glt	`he'll get into the coffin and only then (will he) die.' \txrf{}}
	\end{xlist}
\end{exe}

\subsubsection{M\=/form tail {\ldots} U\=/form head}
I also have six examples in my corpus in which a tail-head
linkage construction with medial elaboration has an M\=/form tail and U\=/form head.
Such constructions typically have two pieces of elaboration,
one between the tail and head and one after the head,
as illustrated in \qf{ex:ChiTaiHeaLin-2} below.
In such constructions the U\=/form head
signals that the previous information is not the only extra information.

\begin{exe}
	\ex{Chiastic tail-head linkage:}\label{ex:ChiTaiHeaLin-2}
		\begin{xlist}
			\ex{
				\begin{xlist}
					\exi{A.}{event\sub{1}{\M} (tail)}
					\sn{B. elaboration\sub{1}}
					\exi{A.}{event\sub{1}{\U} (head)}
					\sn{B. elaboration\sub{2}}
				\end{xlist}}
			\ex{\xytext{\fbox{event\sub{1}{\M}}\xybarconnect[2][-](D,D){2}&elaboration\sub{1}
										&\fbox{event\sub{1}{\U}}\xybarconnect[2][->]{1}&
										\fbox{elaboration\sub{2}}}}\label{ex:ChiTaiHeaLin-2b}
			\end{xlist}
\end{exe}

One example is given in \qf{ex:160326} below in which the narrator
describes the destruction of various objects associated with traditional
religion after the village of Koro{\Q}oto converted to Christianity.
In \qf{ex:160326, 11.10} the noun \ve{fua-t} `items used in traditional religion'
occurs as the patient of the M\=/form verb \ve{n-out} `burnt'.
After this M\=/form verb, \qf{ex:160326, 11.10} and \qf{ex:160326, 11.13}
contain an elaboration of the kinds of items destroyed.
This elaboration closes with the U\=/form verb \ve{n-otu} `burnt',
which introduces \qf{ex:160326, 11.18}, an explanation on the method of destruction.

\begin{exe}
	\ex{Converting to Christianity: \txrf{160326} {\emb{160326-11-06-11-18.mp3}{\spk{}}{\apl}}}\label{ex:160326}
	\sn{\xytext{\fbox{burn{\M}\vp{p}}\xybarconnect[2][-](D,D){3}&all weapons&dismantle before{\Mv}&\fbox{burn{\U}\vp{p}}\xybarconnect[2][->]{1}&\fbox{use{\M} method\vp{p}}}}
	\begin{xlist}
		\ex{\it{`They worshipped all kinds of things.'} \txrf{11.00}}
		\ex{\it{`Too many things.'}}
		\ex{\it{`When the Church came, it said ``Stop those things.'} \txrf{11.02}}
		\ex{\glll	areʔ siin baer fua-t=eni n-nonaʔ =sini =ma n-o\tbr{ut} =siin.\\
							areʔ sini bareʔ fua-t=eni n-nonaʔ =sini =ma n-otu =sini\\
							every {\siin} thing traditional.religion-{\at}={\ein\Uc} \n-hand{\Uc} ={\siin\U} =and \n-burn{\tbrM} ={\siin}\\
				\glt	`All their items of traditional religion were handed over and burnt.' \txrf{11.06}}\label{ex:160326, 11.06}
		\ex{\gll	areʔ suniʔ areʔ kenat, \\
							every sword every weapon\\
				\glt	`All (their) swords, all (their) weapons.' \txrf{11.10}}\label{ex:160326, 11.10}
		\ex{\glll	areʔ uim fua-t, uim reeʔgw=ee msaʔ \hspace{5mm} a|n-pukai n-aan\j=ee, n-o\tbr{tu}\\
							areʔ umi fua-t umi reʔu=ee msaʔ {} {\a}n-pukai n-ani=ee n-otu.\\
							every house traditional.religion-{\at} house sacred={\ee} also {} {\a\n}-dismantle \n-before={\eeV} \n-burn{\tbrU}\\
				\glt	`every house of traditional religion, even the sacred house was pulled down and then burnt.'
							\txrf{11.13}}\label{ex:160326, 11.13}
		\ex{\gll	henatiʔ, n-paek reʔ \sf{{\tS}ara} reʔ ia,\\
							{\he} \n-use{\M} {\reqt} method {\req} {\ia}\\
				\glt	`This was the method they would use,' \txrf{11.18}}\label{ex:160326, 11.18}
		\ex{\it{`(they did it) this way so that people forgot the kinds of things they used to do in past days.'} \txrf{11.25}}
	\end{xlist}
\end{exe}

A second example is given in \qf{ex:130907-3, 8.31-8.40} below.
In this example the M\=/form verb \ve{n-ok} `with' in \qf{ex:130907-3, 8.31}
precedes a description of the kinds of things the narrator and his companion
did together, while the U\=/form verb in \qf{ex:130907-3, 8.36} both follows
and precedes something that the narrator did alone.

\begin{exe}
	\ex{Attending church meetings: \txrf{130907-3} {\emb{130907-3-08-31-08-40.mp3}{\spk{}}{\apl}}}\label{ex:130907-3, 8.31-8.40}
	\sn{\xytext{\fbox{with{\M}\vp{|}}\xybarconnect[2][-](D,D){3}&work{\M}&meetings&\fbox{not with{\U}\vp{|}}\xybarconnect[2][->]{1}&\fbox{meetings\vp{|}}&}}
		\begin{xlist}
			\ex{\it{`And so when he arrived here I hadn't stopped (working) yet.'}\txrf{8.27}}
			\ex{\glll	n-\tbr{ok} =kau =m hai nua =kai m-meup {onai =m}\\
								n-oka =kau =ma hai nua =kai m-mepu {onai =ma}\\
								\n-{\ok\tbrM} ={\kau} and {\hai} two {\kai} \m-work{\M} and.so\\
					\glt	`(He) joined with me and both of us worked and so' \txrf{8.31}}\label{ex:130907-3, 8.31}
			\ex{\glll	karu \sf{si}-- \sf{sidaŋ}, \sf{sidaŋ} \sf{klasis} =ate, iin ka= n-\tbr{oka} =f.\\
								karu {} \sf{sidaŋ}, \sf{sidaŋ} \sf{klasis} =te ini ka= n-oka =fa\\
								if {} meeting meeting presbytery ={\te} {\iin} {\ka}= \n-{\ok\tbrU} ={\fa}\\
					\glt	`If it was a meeting, a presbytery meeting, he didn't join.' \txrf{8.36}}\label{ex:130907-3, 8.36}
		%\end{xlist}
%\end{exe}
%\newpage
%\begin{exe}
	%\sn{\begin{xlist}
		%\exi{d.}{
		\ex{
							\glll	au ees a-na{\tl}nao-t. au ees a-tok{\tl}took \sf{sidaŋ}.\\
							au esa a-na{\tl}nao-t au esa a-tok{\tl}toko-s \sf{sidaŋ}\\
							{\au} {\esc} {\at}-{\prd}go-{\at} {\au} {\esc} {\at}-{\prd}sit meeting\\
				\glt	`I was the one who went (\emph{lit.} goer). I was the one who attended
							the meetings (\emph{lit.} meeting sitter).' \txrf{8.40}}
	\end{xlist}
		%}
\end{exe}

Example \qf{ex:130907-3, 8.31-8.40} has two interlocking chiastic structures.
One is the tail-head linkage construction composed
of the M\=/form and U\=/form of \ve{n-oka} `with',
the other is the repetition of attendance at meetings which occurs
on either side of the head of this tail-head linkage construction.
The structure of \qf{ex:130907-3, 8.31-8.40} is given in \qf{ex:DouChi} below.

\begin{exe}
	\ex{Double chiasmus in \qf{ex:130907-3, 8.31-8.40}:}\label{ex:DouChi}
		\begin{xlist}
			\exi{A.}{\it{join with}{\M}}
			\sn{B. \it{work}}
			\sn{\lh{B. }C. \it{attend meetings}}
			\exi{A.}{\it{didn't join with}{\U}}
			\sn{\lh{B. }C. \it{attend meetings}}
		\end{xlist}
\end{exe}

A tail-head linkage construction can also have a piece of
elaboration between the tail and the head.
When this is the case U\=/form tails are resolved by the intermediate
piece of elaboration, as illustrated in \qf{ex:ChiTaiHeaLin-1b-2} below.
When the head is in the U\=/form, it introduces another piece of elaboration
in addition to that which occurs between the tail and head,
as illustrated in \qf{ex:ChiTaiHeaLin-2b-2} below.

\begin{exe}
	\ex{\xytext{\fbox{event\sub{1}{\U}}\xybarconnect[2][-](D,D){2}\xybarconnect[2][->]{1}&\fbox{elaboration\vp{|}}
								&\fbox{event\sub{1}{\M}}}}\label{ex:ChiTaiHeaLin-1b-2}
	\ex{\xytext{\fbox{event\sub{1}{\M}}\xybarconnect[2][-](D,D){2}&elaboration\sub{1}
								&\fbox{event\sub{1}{\U}}\xybarconnect[2][->]{1}&\fbox{elaboration\sub{2}}}}\label{ex:ChiTaiHeaLin-2b-2}
\end{exe}
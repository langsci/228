\section{The death of Nahor Bani}\label{sec:DeaNahBan}
\subsection{Metadata}

\wg\begin{tabular}{lL{.67\textwidth}}
File-name:					& aaz20130928{\_}01\\
Archive-link:				& \url{http://catalog.paradisec.org.au/collections/OE1/items/aaz20130928_01}\\
Original Name:			& aaz-20130928-1-HeronimusBani- CeritaNahorBaniMati\\
Language:						& Amarasi [aaz] \\
Dialect:						& Kotos; Koro{\Q}oto hamlet \\
Location:						& Nekmese{\Q}, Amarasi Selatan, Kupang \\
Date:								&	28/09/2013\\
Speaker(s):					& Heronimus Bani\\
Recorded by: 				& Heronimus Bani, Owen Edwards\\
Transcribed by:			& Heronimus Bani\\
Interlinear by:			& Owen Edwards \\
Indonesian/Kupang:	& Heronimus Bani\\
Free English by:		& Charles E. Grimes\\
Genre:							& narrative\\
Summary:						& Roni relates a disagreement over where recently
										deceased Nahor Bani should be buried\\
%Length:						& 2.51\\
%Notes:							& \\
\end{tabular}


\subsection{Notes}
Heronimus Bani relates about a disagreement over where recently deceased
Nahor Bani should be buried.
Most families want to bury their loved
ones in their yard so they can care for the grave.
The government has been pushing for everyone
to be buried in designated community graveyards
(Indonesian \it{T.P.U.} = \it{tempat pemakaman umum})
``for public health reasons''.
Culturally in Timor, the \ve{nitu} `spirit of the dead' can disturb,
disrupt, cause sickness, crop failure, etc. to the living if angry or neglected.
Monitoring and taking good care of the grave
is one way to show respect and prevent bad
things happening to good people.\footnote{
		Thanks go to Charles Grimes for the cultural background to this text.}

\newpage
\subsection{The text}
\begin{exe}

\ex{\ve{Neno ia aam Nahor Bani in nmaet.}}
\exi{0.02}{\glll	neno ia aam Nahor Bani iin n-maet\\
									neno ia ama Nahor Bani ini n-mate \\
									day {\ia} father{\M} Nahor Bani {\iin} \n-die \\
						\glt	`Today father Nahor Bani died.'}

\ex{\ve{Oras in nmate te,}}
\exi{0.06}{\glll	oras iin n-mate =te\\
									oras ini n-mate =te\\
									time {\iin} \n-die{\U} ={\te}\\
						\glt	`When he died, ' }

\ex{\ve{in aan moen jes, kaan ee nai{\Q} Fanu,}}
\exi{0.11}{\glll	iin aan moon\j=ees, kaan-n=ee naiʔ Fanu\\
									ini anah mone=esa kana-n=ee naiʔ Fanu \\
									{\iin} child{\M} male=one name-{\N}={\ee} {\naiq} Fanu\\
						\glt	`one of his sons named Fanu' }

\ex{\ve{anhain nain nopu.}}
\exi{0.17}{\glll	a|n-hain n-ain nopu\\
									{\a}n-hani n-ani nopu \\
									\a\n-dig \n-before hole \\
						\glt	\lh{\a}`had dug the grave beforehand.' }

\ex{\ve{Re{\Q} uaba ma too mfaun re{\Q} kuan ii naan ii nak am: nopu mnanun.}}
\exi{0.20}{\begin{xlist}
	\ex{\glll	reʔ uaba =m too mfaun	\\
						reʔ uaba =ma too mfaun	\\
						{\req} speech =ma populace many \\}

	\ex{\glll	reʔ kuan=ii naan-n=ii n-aka =m ehh nopu mnanun\\
						reʔ kuan=ii nana-n=ii n-ak =ma {} nopu mnanun \\
						{\req} village={\ii} inside-{\N}={\ii} \n-say =and {} hole deep\\
			\glt	`of which it was said, (by) many people who (are) in this village, (they) said the grave was deep.}
	\end{xlist}}

\newpage
\ex{\ve{In ka nhain je ruum aah fa te, nhani nrair je te, nrame.}}
\exi{0.30}{\begin{xlist}
	\ex{\glll	iin ka= n-haan\j=ee ruum=aah =fa =te\\
						ini ka= n-hani=ee ruum=aha =fa =te\\
						{\iin} {\ka}= \n-dig={\eeV} plain=just ={\fa} ={\te} \n-dig \n-finish={\eeV} ={\te} \n-plaster{\U}\\
			\glt `He did not just dig it plainly (i.e. with plain dirt walls),'}

	\ex{\glll	n-hani n-raar\j=ee =t n-rame\\
						n-hani n-rari=ee =te n-rame\\
						\n-dig \n-finish={\eeV} ={\te} \n-plaster{\U}\\
			\glt	`(when) he finished digging it, he plastered it (with concrete).' }
\end{xlist}}

\ex{\ve{Nraem je reko-reko.}}
\exi{0.33}{\glll	n-raam\j=ee reko{\tl}reko\\
									n-rame=ee reko{\tl}reko \\
									\n-plaster={\eeV} {\frd}good \\
						\glt `He plastered it properly.' }

\ex{\ve{Onaim re{\Q} natfeek onai te, are{\Q} amahonit, ana{\Q}a prenat, too mfaun ein, neem nabuan am,}}
\exi{0.36}{\begin{xlist}
	\ex{\glll	{onai =m}, reʔ na-tfeek {onai =te} areʔ amahonit, anaʔaprenat\\
						{onai =ma} reʔ na-tfeka {onai =te} areʔ a-ma-honi-t a-naʔaprenat\\
						and.so {\req} \na-stop and.then every parent official\\}
	
	\ex{\glll	too mfaun=ein, neem na-bua=n =am\\
						too mfaun=eni nema na-bua=n =ma\\
						populace many={\ein} {\nema} \na-gather={\einV} =and\\
			\glt `So, when (the deceased) stopped [= died] then all the parents/clan elders,
						(local) government officials, and many of the populace, came and gathered'\footnote{
								The form \ve{amahonit} `parent' (with variant \ve{mahonit}) is a lexicalised nominalisation
								from \ve{a-ma-honi-t} {\at}-{\ma}-born-{\at}.
								The phrase \ve{anaʔaprenat} `official' is a lexicalised historic nominalisation 
								from \ve{a-naʔa-t} {\at}-hold-{\at} + \ve{prenat} `govern'.}}
\end{xlist}}

\ex{\ve{he na{\Q}uab ein neu re{\Q} he tpafa{\Q} ai{\Q} t-suba ma, on re{\Q} mee?}}
\exi{0.48}{\begin{xlist}
	\ex{\glll	he na-ʔuab=ein n-eu reʔ he t-pafaʔ aiʔ t-suba =ma\\
						he na-ʔuaba=eni n-eu reʔ he t-pafaʔ aiʔ t-suba =ma\\
						{\he} \na-speak={\einV} \n-{\eu} {\req} {\he} \t-protect or \t-bury{\U} =and\\}

	\ex{\glll on reʔ mee\\
						on reʔ mee\\
						like {\reqt} how\\
			\glt `to discuss about how we are going protect or bury (the body) and how (are we going to go about this)?' }
\end{xlist}}

\ex{\ve{Oat hau gui on re{\Q} mee, ai{\Q} noup paarn ii on re{\Q} mee?}}
\exi{0.57}{\glll	oat haagw=ii on reʔ mee, aiʔ noup paarn=ii, on reʔ mee\\
									ote hau=ii on reʔ mee aiʔ nopu paran=ii on reʔ mee\\
									cut{\M} wood={\ii} like {\reqt} how or hole{\M} short={\ii} like {\reqt} how\\
						\glt `How should the cutting of the wood (for the casket) be?
									Or, how long should the hole (for the grave) be? ' }

\ex{\ve{Ma nopu mnaun{\Q} ii te, on re{\Q} mee?}}
\exi{1.02}{\glll	ma nopu mnaunʔ=ii =t on reʔ mee\\
									ma nopu mnanuʔ=ii =te on reʔ mee\\
									and hole deep={\ii} ={\te} like {\reqt} how\\
						\glt `and how deep should the hole be?' }

\ex{\ve{Onai te, re{\Q} nai{\Q} Faun gui, fee{\Q}n ii uab ii,}}
\exi{1.06}{\glll	{onai =te} reʔ naiʔ Faaŋgw=ii, feeʔn=ii uab=ii, \\
									{onai =te} reʔ naiʔ Fanu=ii feʔen=ii uaba=ii \\
									then {\reqt} {\naiq} Fanu={\ii} earlier={\ii} speech={\ii} \\
						\glt `So then this Fanu (that I) mentioned earlier,' }

\ex{\ve{nak on in ka natonan fa ana{\Q}a preent ein ii, ai{\Q} mahoint ein ii, neu re{\Q} in nhain nain nopu.}}
\exi{1.10}{\glll	n-ak on iin ka= na-tona=n =fa anaʔapreent=ein=ii aiʔ mahoint=ein=ii n-eu reʔ iin n-hain n-ain nopu\\
									n-ak on ini ka= na-tona=n =fa anaʔaprenat=eni=ii aiʔ amahonit=eni=ii n-eu reʔ ini n-hani n-ani nopu \\
									\n-say {\on} {\iin} {\ka}= \n-tell={\ein}={\ii} ={\fa} official={\ein}={\ii} or parents={\ein}={\ii} \n-{\eu} {\reqt} {\iin} \n-dig \n-before hole \\
						\glt	`(it was) said that he had not told the government officials,
									or the clan leaders that he had dug the grave beforehand,' }

\ex{\ve{Ai{\Q} in nmesel anrari, nrame nrari.}}
\exi{1.18}{\glll	aiʔ iin n-\sf{mesel} a|n-rari, n-rame n-rari\\
									aiʔ ini n-\sf{mesel} {\a}n-rari n-rame n-rari\\
									or {\iin} \n-grave.cover \a\n-finish{\U} \n-plaster \n-finish{\U}\\
						\glt `or that he had built the grave cover and had plastered it with cement.' }

\newpage
\ex{\ve{In nmeerk on.}}
\exi{1.22}{\glll	iin n-meerk=oo-n\\
									ini n-merak=oo-n \\
									{\iin} \n-quiet=\oo-{\N}\\
						\glt	`He kept himself quiet.' }

\ex{\ve{Onaim ana{\Q}a preent ein nok nai{\Q} Fanu in taatf eni,}}
\exi{1.22}{\glll	{onai =m} anaʔapreent=ein n-ok naiʔ Fanu iin taat-f=eni\\
									{onai =ma} anaʔaprenat=eni n-oka naiʔ Fanu ini tata-f=eni\\
									and.so official={\ein} {\n-\ok} {\naiq} Fanu {\iin} eSi-{\f}={\ein\U}\\
						\glt	`So the government officials and Fanu’s elder siblings' }

\ex{\ve{aam Simson nok aam Ayub, nema ntean onai te,}}
\exi{1.28}{\glll	aam Simson n-ok aam Ayup nema n-tea=n {onai =t}\\
									ama Simson n-oka ama Ayup nema n-tea=n {onai =te}\\
									father{\M} Simson {\n-\ok} father{\M} Ayub {\nema} \n-until={\einV} and.then\\
						\glt	`Mr. Simson (Samson) and Mr. Ayub (Job) came' }

\ex{\ve{nak on na{\Q}uab ein am ma,}}
\exi{1.32}{\glll	n-ak on na-ʔuab=ein =ama\\
									n-ak on na-ʔuaba=eni =ma\\
									\n-say like \na-speak={\einV} =and\\
						\glt	`thinking like they were going to discuss, and' }

\ex{\ve{sin he nnaon nsuban on bare {\Q}bua{\Q} re{\Q} nteek ee nak T.P.U.}}
\exi{1.34}{\begin{xlist}
	\ex{\glll	sin he n-nao=n n-suba=n on, bare ʔ-bua-ʔ \\
						sin he n-nao=n n-suba=n on bare ʔ-bua-ʔ\\
						{\siin} {\he} \n-go={\einV} \n-bury={\einV} {\on} place {\qq}-gather-{\qq}\\
			\glt `they were going to go bury him at the place (where graves are) gathered'}

	\ex{\glll	reʔ n-teek=ee n-ak, \sf{tee-pee-ʔuu}\\
						reʔ n-teka=ee n-ak \sf{tee-pee-ʔuu}\\
						{\req} \n-call={\eeV} {\n-\ak} T.P.U.\\
			\glt `which is called T.P.U. (\it{tempat pemakaman umum} = public burial place)'}
\end{xlist}}

\ex{\ve{Hei, maans ee nmaeb ia te,}}
\exi{1.41}{\glll	heeʔ maans=ee n-maeb ia =te\\
									heeʔ manas=ee n-mabe ia =te\\
									hey sun={\ee} \n-afternoon {\ia} ={\te}\\
						\glt	`Well, late this afternoon' }

\ex{\ve{uab ii nfain suir jeen, nasurin. }}
\exi{1.43}{\glll	uab=ii n-fain suur\j=een, na-suri=n.\\
									uaba=ii n-fani suri=ena na-suri=n\\
									speech={\ii} \n-turn collide={\een} \na-collide={\einV}\\
						\glt	`the discussion had turned into a clash, they were at cross purposes.'}

\ex{\ve{Nasurin neu re{\Q} aam Fanu in neekn ii he nsuub nabaar re{\Q} kintal natuin}}
\exi{1.47}{\begin{xlist}
\ex{\glll	na-suri=n n-eu reʔ aam Fanu iin neek-n=ii \\
					na-suri=n n-eu reʔ ama Fanu ini neka-n=ii \\
					\na-collide={\einV} \n-{\eu} {\reqt} father{\M} Fanu {\iin} feelings-{\N}={\ii}\\}

\ex{\glll he n-suub na-baar reʔ \sf{kintal} na-tuin\\
					he n-suba  na-bara reʔ \sf{kintal} na-tuin\\
					{\he} \n-bury \na-forever {\reqt} yard \na-because\\
		\glt `They were at odds over father Fanu's desire to bury (him), permanently in the yard because' }
\end{xlist}}

\ex{\ve{in aamf ii es anrenu ma nhain re{\Q} nopu,}}
\exi{1.54}{\glll	iin aam-f=ii esa n-renu =ma n-hain reʔ nopu\\
									ini ama-f=ii esa n-renu =ma n-hani reʔ nopu \\
									{\iin} father-{\F}={\ii} one \n-order{\U} =and \n-dig {\reqt} hole \\
						\glt	`his father was the one who ordered (him), and he had dug the hole,' }

\ex{\ve{anraem je nok.}}
\exi{1.58}{\glll	a|n-raam\j=ee n-ok\\
									{\a}n-rame=ee n-oka\\
									\a\n-plaster={\eeV} {\n-\ok}\\
						\glt	\lh{\a}`and had even plastered it' }

\ex{\ve{Onaim ana{\Q}a preent ein nmatoof ein et re{\Q} nee nok are{\Q} mahoint eni ma}}
\exi{2.01}{\glll	{onai =m} anaʔapreent=ein n-ma-toof=ein et reʔ nee n-ok areʔ mahoint=eni =m\\
									{onai =ma} anaʔaprenat=eni n-ma-tofa=eni et reʔ nee n-oka areʔ mahonit=eni =ma\\
									and.so official={\ein} \n-\mak-quarrel={\ein} {\et} {\req} {\nee} {\n-\ok} every parents={\ein\U} =and\\
						\glt	`So (consequently) the government officials, they argued there with all the clan elders, and…' }

\ex{\ve{nuuk tuaf eni, nai{\Q} Fanu nok are{\Q} in tataf}}
\exi{2.05}{\glll	nuuk tua-f=eni, naiʔ Fanu n-ok areʔ iin tata-f\\
									nuka tua-f=eni naiʔ Fanu n-oka areʔ ini tata-f\\
									grief{\M} person-{\f}={\ein\U} {\naiq} Fanu {\n-\ok} every {\iin} eSi-{\f}\\
						\glt	`the bereaved, Mr. Fanu and with all his elder siblings, ' }

\ex{\ve{es-es ate nok in fee in mone,}}
\exi{2.09}{\glll	es{\tl}esa =te n-ok iin fee iin mone\\
									es{\tl}esa =te n-oka ini fee ini mone \\
									{\prd}one ={\te} {\n-\ok} {\iin} wife {\iin} husband \\
						\glt	`each one (of them) with his wife or her husband,' }

\ex{\ve{nasurin am ka tahiin he suubt ii on re{\Q} mee.}}
\exi{2.11}{\glll	na-suri=n =am ka= ta-hiin he suub-t=ii on reʔ mee\\
									na-suri=n =ma ka= ta-hini he suba-t=ii on reʔ mee\\
									\na-collide={\einV} =and {\ka}= {\ta}-know {\he} bury-{\at}={\ii} like {\reqt} how\\
						\glt	`they were at odds and we didn't know where we would bury him.' }

\ex{\ve{Ana{\Q}a preent ein naiti {\Q}niimk ein am ka tahiin he}}
\exi{2.16}{\glll	anaʔapreent=ein n-aiti ʔniim-k=ein =am ka= ta-hiin he\\
									anaʔaprenat=eni n-aiti ʔnima-k=eni =am ka= ta-hini he\\
									official={\ein} \n-lift hand-{\k}={\ein} and {\ka}= {\ta}-know {\he}\\
						\glt	`The government officials lifted their hands (= didn't want to have anything more to do with it) and we didn't know whether ' }

\ex{\ve{urusan he reek haef, are{\Q} tobiru on tiis raur gui he on re{\Q} mee?}}
\exi{2.28}{\begin{xlist}
	\ex{\glll	\sf{urusan} reʔ he reek hae-f \\
						\sf{urusan} reʔ he rekaʔ hae-f \\
						dealings {\req} {\he} order{\M} messenger-{\f} \\
			\glt `arrangements like sending messengers out (with news of the death),' }

	\ex{\glll areʔ tobiru on ahh tiis raargw=ii he on reʔ mee\\
						areʔ tabiru on {} tisi raru=ii he on reʔ mee\\
						every work {\on} ahh pour{\M} palm.wine={\ii} {\he} like {\reqt} how\\
			\glt `every detail that had to be attended to, like (the ceremonial) pouring palm-wine, was going to happen how?'}
\end{xlist}}

\newpage
\ex{\ve{Maut hena{\Q} taniin sin.}}
\exi{2.33}{\glll	maut henaʔ ta-tniin =siin\\
									maut henaʔ ta-tnina =sini \\
									let {\he} {\ta}-listen {\siin}\\
						\glt `We really should listen to them.' [Deliberately vague.]}

\ex{\ve{Aam Nahor Bani nmaet, in raisn ii {\Q}tet-teta{\Q} kuun.}}
\exi{2.35}{\begin{xlist}
	\ex{\glll	aam Nahor Bani n-maet\\
						ama Nahor Bani n-mate \\
						father{\M} Nahor Bani \n-die \\}

	\ex{\glll	iin rais-n=ii ʔtet{\tl}tetaʔ kuu-n\\
						ini rasi-n=ii ʔtet{\tl}tetaʔ kuu-n\\
						{\iin} issue-{\N}={\ii} {\prd}different alone-{\N}\\
			\glt	`Father Nahor Bani died, and his issue (relating to his death) is entirely different by itself.' }
\end{xlist}}

\end{exe}
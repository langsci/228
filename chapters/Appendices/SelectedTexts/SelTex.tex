\chapter{Selected Amarasi texts}\label{app:SelAmaTex}

\counterwithin{exx}{section}

\section{Preface}
In this appendix I present four Amarasi texts,
three from the Kotos dialect,
which is the focus of this book, and one from Ro{\Q}is.
The Kotos texts include a narrative about a recent event (\S\ref{sec:DeaNahBan}),
a myth about the beginning of the world (\S\ref{sec:MooHit}),
and a conversation (\S\ref{sec:CarCrash}).
The Ro{\Q}is text (\srf{sec:GatBanFam}) is a short life history.

Each sentence/intonation unit is given twice.
The first line is sequentially numbered and gives an orthographic transcription.
The second line is numbered according to its time in the recording.
This line is given in phonemic transcription,
which is followed by the gloss and a free translation.

Glossing follows the same conventions used throughout this book,
with the exception that the third person singular genitive
suffix \ve{-n} is glossed \tsc{3gen} in the Kotos texts
and \tsc{3psr;sg.psm} in the Ro{\Q}is text (\srf{sec:RoqPoss}).

The only instances of metathesis which are indicated in the glosses
are those which have a morphological meaning:
M\=/forms of nouns modified by an attributive modifier (Chapter \ref{ch:SynMet})
and U\=/forms of vowel-final verbs and other word classes
which take discourse-driven U\=/forms (Chapter \ref{ch:DisMet}).
Non-morphological M\=/forms and U\=/forms
can be detected by comparing the phonemic transcription in the top line
with the corresponding underlying forms in the second line.

\renewcommand{\N}{\tsc{3gen}}
\section{The death of Nahor Bani}\label{sec:DeaNahBan}
\subsection{Metadata}

\wg\begin{tabular}{lL{.67\textwidth}}
File-name:					& aaz20130928{\_}01\\
Archive-link:				& \url{http://catalog.paradisec.org.au/collections/OE1/items/aaz20130928_01}\\
Original Name:			& aaz-20130928-1-HeronimusBani- CeritaNahorBaniMati\\
Language:						& Amarasi [aaz] \\
Dialect:						& Kotos; Koro{\Q}oto hamlet \\
Location:						& Nekmese{\Q}, Amarasi Selatan, Kupang \\
Date:								&	28/09/2013\\
Speaker(s):					& Heronimus Bani\\
Recorded by: 				& Heronimus Bani, Owen Edwards\\
Transcribed by:			& Heronimus Bani\\
Interlinear by:			& Owen Edwards \\
Indonesian/Kupang:	& Heronimus Bani\\
Free English by:		& Charles E. Grimes\\
Genre:							& narrative\\
Summary:						& Roni relates a disagreement over where recently
										deceased Nahor Bani should be buried\\
%Length:						& 2.51\\
%Notes:							& \\
\end{tabular}


\subsection{Notes}
Heronimus Bani relates about a disagreement over where recently deceased
Nahor Bani should be buried.
Most families want to bury their loved
ones in their yard so they can care for the grave.
The government has been pushing for everyone
to be buried in designated community graveyards
(Indonesian \it{T.P.U.} = \it{tempat pemakaman umum})
``for public health reasons''.
Culturally in Timor, the \ve{nitu} `spirit of the dead' can disturb,
disrupt, cause sickness, crop failure, etc. to the living if angry or neglected.
Monitoring and taking good care of the grave
is one way to show respect and prevent bad
things happening to good people.\footnote{
		Thanks go to Charles Grimes for the cultural background to this text.}

\newpage
\subsection{The text}
\begin{exe}

\ex{\ve{Neno ia aam Nahor Bani in nmaet.}}
\exi{0.02}{\glll	neno ia aam Nahor Bani iin n-maet\\
									neno ia ama Nahor Bani ini n-mate \\
									day {\ia} father{\M} Nahor Bani {\iin} \n-die \\
						\glt	`Today father Nahor Bani died.'}

\ex{\ve{Oras in nmate te,}}
\exi{0.06}{\glll	oras iin n-mate =te\\
									oras ini n-mate =te\\
									time {\iin} \n-die{\U} ={\te}\\
						\glt	`When he died, ' }

\ex{\ve{in aan moen jes, kaan ee nai{\Q} Fanu,}}
\exi{0.11}{\glll	iin aan moon\j=ees, kaan-n=ee naiʔ Fanu\\
									ini anah mone=esa kana-n=ee naiʔ Fanu \\
									{\iin} child{\M} male=one name-{\N}={\ee} {\naiq} Fanu\\
						\glt	`one of his sons named Fanu' }

\ex{\ve{anhain nain nopu.}}
\exi{0.17}{\glll	a|n-hain n-ain nopu\\
									{\a}n-hani n-ani nopu \\
									\a\n-dig \n-before hole \\
						\glt	\lh{\a}`had dug the grave beforehand.' }

\ex{\ve{Re{\Q} uaba ma too mfaun re{\Q} kuan ii naan ii nak am: nopu mnanun.}}
\exi{0.20}{\begin{xlist}
	\ex{\glll	reʔ uaba =m too mfaun	\\
						reʔ uaba =ma too mfaun	\\
						{\req} speech =ma populace many \\}

	\ex{\glll	reʔ kuan=ii naan-n=ii n-aka =m ehh nopu mnanu-n\\
						reʔ kuan=ii nana-n=ii n-ak =ma {} nopu mnanu-n \\
						{\req} village={\ii} inside-{\N}={\ii} \n-say =and {} hole deep-{\N}\\
			\glt	`of which it was said, (by) many people who (are) in this village, (they) said the grave was deep.}
	\end{xlist}}

\newpage
\ex{\ve{In ka nhain je ruum aah fa te, nhani nrair je te, nrame.}}
\exi{0.30}{\begin{xlist}
	\ex{\glll	iin ka= n-haan\j=ee ruum=aah =fa =te\\
						ini ka= n-hani=ee ruum=aha =fa =te\\
						{\iin} {\ka}= \n-dig={\eeV} plain=just ={\fa} ={\te} \n-dig \n-finish={\eeV} ={\te} \n-plaster{\U}\\
			\glt `He did not just dig it plainly (i.e. with plain dirt walls),'}

	\ex{\glll	n-hani n-raar\j=ee =t n-rame\\
						n-hani n-rari=ee =te n-rame\\
						\n-dig \n-finish={\eeV} ={\te} \n-plaster{\U}\\
			\glt	`(when) he finished digging it, he plastered it (with concrete).' }
\end{xlist}}

\ex{\ve{Nraem je reko-reko.}}
\exi{0.33}{\glll	n-raam\j=ee reko{\tl}reko\\
									n-rame=ee reko{\tl}reko \\
									\n-plaster={\eeV} {\frd}good \\
						\glt `He plastered it properly.' }

\ex{\ve{Onaim re{\Q} natfeek onai te, are{\Q} amahonit, ana{\Q}a prenat, too mfaun ein, neem nabuan am,}}
\exi{0.36}{\begin{xlist}
	\ex{\glll	{onai =m}, reʔ na-tfeek {onai =te} areʔ amahonit, anaʔaprenat\\
						{onai =ma} reʔ na-tfeka {onai =te} areʔ amahonit anaʔaprenat\\
						and.so {\req} \na-cut.off and.then every parent official\\}
	
	\ex{\glll	too mfaun=ein, neem na-bua=n =am\\
						too mfaun=eni nema na-bua=n =ma\\
						populace many={\ein} {\nema} \na-gather={\einV} =and\\
			\glt `So, when (the breath of the deceased) was cut off (= died) then all the parents/clan elders,
						(local) government officials, and many of the populace came and gathered'\footnote{
								The form \ve{amahonit} `parent' (with variant \ve{mahonit}) is a lexicalised nominalisation
								from \ve{a\=/ma\=/honi\=/t} {\at}-{\ma}\=/born-{\at}.
								The phrase \ve{anaʔaprenat} `official' is a lexicalised historic nominalisation 
								from \ve{a-naʔa-t} {\at}-hold-{\at} + \ve{prenat} `govern'.}}
\end{xlist}}

\newpage
\ex{\ve{he na{\Q}uab ein neu re{\Q} he tpafa{\Q} ai{\Q} tsuba ma, on re{\Q} mee?}}
\exi{0.48}{\begin{xlist}
	\ex{\glll	he na-ʔuab=ein n-eu reʔ he a|t-pafaʔ aiʔ t-suba =ma\\
						he na-ʔuaba=eni n-eu reʔ he {\a}t-pafaʔ aiʔ t-suba =ma\\
						{\he} \na-speak={\einV} \n-{\eu} {\req} {\he} \a\t-protect or \t-bury{\U} =and\\}

	\ex{\glll on reʔ mee\\
						on reʔ mee\\
						like {\reqt} how\\
			\glt `to discuss about how we are going protect or bury (the body) and how (are we going to go about this)?' }
\end{xlist}}

\ex{\ve{Oat hau gui on re{\Q} mee, ai{\Q} noup paarn ii on re{\Q} mee?}}
\exi{0.57}{\glll	oat haagw=ii on reʔ mee, aiʔ noup paar-n=ii, on reʔ mee\\
									ote hau=ii on reʔ mee aiʔ nopu para-n=ii on reʔ mee\\
									cut{\M} wood={\ii} like {\reqt} how or hole{\M} short-{\N}={\ii} like {\reqt} how\\
						\glt `How should the cutting of the wood (for the casket) be?
									Or, how short should the hole (for the grave) be? ' }

\ex{\ve{Ma nopu mnaun{\Q} ii te, on re{\Q} mee?}}
\exi{1.02}{\glll	ma nopu mnaunʔ=ii =t on reʔ mee\\
									ma nopu mnanuʔ=ii =te on reʔ mee\\
									and hole deep={\ii} ={\te} like {\reqt} how\\
						\glt `and how deep should the hole be?' }

\ex{\ve{Onai te, re{\Q} nai{\Q} Faun gui, fee{\Q}n ii uab ii,}}
\exi{1.06}{\glll	{onai =te} reʔ naiʔ Faaŋgw=ii, feeʔn=ii uab=ii, \\
									{onai =te} reʔ naiʔ Fanu=ii feʔen=ii uaba=ii \\
									then {\reqt} {\naiq} Fanu={\ii} earlier={\ii} speech={\ii} \\
						\glt `So then this Fanu (that I) mentioned earlier,' }

\newpage
\ex{\ve{nak on in ka natonan fa ana{\Q}a preent ein ii, ai{\Q} mahoint ein ii, neu re{\Q} in nhain nain nopu.}}
\exi{1.10}{\glll	n-ak on iin ka= na-tona=n =fa anaʔapreent=ein=ii aiʔ \hspace{10mm} mahoint=ein=ii n-eu reʔ iin n-hain n-ain nopu\\
									n-ak on ini ka= na-tona=n =fa anaʔaprenat=eni=ii aiʔ {} amahonit=eni=ii n-eu reʔ ini n-hani n-ani nopu \\
									\n-say {\on} {\iin} {\ka}= \n-tell={\ein} ={\fa} official={\ein}={\ii} or {} parents={\ein}={\ii} \n-{\eu} {\reqt} {\iin} \n-dig \n-before hole \\
						\glt	`(it was) said that he had not told the government officials,
									or the clan leaders that he had dug the grave beforehand,' }

\ex{\ve{Ai{\Q} in nmesel anrari, nrame nrari.}}
\exi{1.18}{\glll	aiʔ iin n-\sf{mesel} a|n-rari, n-rame n-rari\\
									aiʔ ini n-\sf{mesel} {\a}n-rari n-rame n-rari\\
									or {\iin} \n-grave.cover \a\n-finish{\U} \n-plaster \n-finish{\U}\\
						\glt `or that he had built the grave cover and had plastered it with cement.' }

\ex{\ve{In nmeerk on.}}
\exi{1.22}{\glll	iin n-meerk=oo-n\\
									ini n-merak=oo-n \\
									{\iin} \n-quiet=\oo-{\N}\\
						\glt	`He kept himself quiet.' }

\ex{\ve{Onaim ana{\Q}a preent ein nok nai{\Q} Fanu in taatf eni,}}
\exi{1.22}{\glll	{onai =m} anaʔapreent=ein n-ok naiʔ Fanu iin taat-f=eni\\
									{onai =ma} anaʔaprenat=eni n-oka naiʔ Fanu ini tata-f=eni\\
									and.so official={\ein} {\n-\ok} {\naiq} Fanu {\iin} eSi-{\F}={\ein\U}\\
						\glt	`So the government officials and Fanu’s elder siblings' }

\ex{\ve{aam Simson nok aam Ayub, nema ntean onai te,}}
\exi{1.28}{\glll	aam Simson n-ok aam Ayup nema n-tea=n {onai =t}\\
									ama Simson n-oka ama Ayup nema n-tea=n {onai =te}\\
									father{\M} Simson {\n-\ok} father{\M} Ayub {\nema} \n-until={\einV} and.then\\
						\glt	`Mr. Simson (Samson) and Mr. Ayub (Job) came' }

\newpage
\ex{\ve{nak on na{\Q}uab ein am ma,}}
\exi{1.32}{\glll	n-ak on na-ʔuab=ein =ama\\
									n-ak on na-ʔuaba=eni =ma\\
									\n-say like \na-speak={\einV} =and\\
						\glt	`thinking like they were going to discuss, and' }

\ex{\ve{sin he nnaon nsuban on bare {\Q}bua{\Q} re{\Q} nteek ee nak T.P.U.}}
\exi{1.34}{\begin{xlist}
	\ex{\glll	siin he n-nao=n n-suba=n on, bare ʔ-bua-ʔ \\
						sini he n-nao=n n-suba=n on bare ʔ-bua-ʔ\\
						{\siin} {\he} \n-go={\einV} \n-bury={\einV} {\on} place {\qq}-gather-{\qq}\\
			\glt `they were going to go bury him at the place (where graves are) gathered'}

	\ex{\glll	reʔ n-teek=ee n-ak, \sf{tee} \sf{pee} \sf{uu}\\
						reʔ n-teka=ee n-ak \sf{tee} \sf{pee} \sf{uu}\\
						{\req} \n-call={\eeV} {\n-\ak} T P U\\
			\glt `which is called T.P.U. (\it{tempat pemakaman umum} = public burial place)'}
\end{xlist}}

\ex{\ve{Hei, maans ee nmaeb ia te,}}
\exi{1.41}{\glll	heeʔ maans=ee n-maeb ia =te\\
									heeʔ manas=ee n-mabe ia =te\\
									hey sun={\ee} \n-afternoon {\ia} ={\te}\\
						\glt	`Well, late this afternoon' }

\ex{\ve{uab ii nfain suir jeen, nasurin. }}
\exi{1.43}{\glll	uab=ii n-fain suur\j=een, na-suri=n.\\
									uaba=ii n-fani suri=ena na-suri=n\\
									speech={\ii} \n-turn collide={\een} \na-collide={\einV}\\
						\glt	`the discussion had turned into a clash, they were at cross purposes.'}

\ex{\ve{Nasurin neu re{\Q} aam Fanu in neekn ii he nsuub nabaar re{\Q} kintal natuin}}
\exi{1.47}{\begin{xlist}
\ex{\glll	na-suri=n n-eu reʔ aam Fanu iin neek-n=ii \\
					na-suri=n n-eu reʔ ama Fanu ini neka-n=ii \\
					\na-collide={\einV} \n-{\eu} {\reqt} father{\M} Fanu {\iin} feelings-{\N}={\ii}\\}

\ex{\glll he n-suub na-baar reʔ \sf{kintal} na-tuin\\
					he n-suba  na-bara reʔ \sf{kintal} na-tuin\\
					{\he} \n-bury \na-forever {\reqt} yard \na-because\\
		\glt `They were at odds over father Fanu's desire to permanently bury (him), in the yard because' }
\end{xlist}}

\ex{\ve{in aamf ii es anrenu ma nhain re{\Q} nopu,}}
\exi{1.54}{\glll	iin aam-f=ii esa n-renu =ma n-hain reʔ nopu\\
									ini ama-f=ii esa n-renu =ma n-hani reʔ nopu \\
									{\iin} father-{\F}={\ii} one \n-order{\U} =and \n-dig {\reqt} hole \\
						\glt	`his father was the one who ordered (him), and he had dug the hole,' }

\ex{\ve{anraem je nok.}}
\exi{1.58}{\glll	a|n-raam\j=ee n-ok\\
									{\a}n-rame=ee n-oka\\
									\a\n-plaster={\eeV} {\n-\ok}\\
						\glt	\lh{\a}`and had also plastered it' }

\ex{\ve{Onaim ana{\Q}a preent ein nmatoof ein et re{\Q} nee nok are{\Q} mahoint eni ma}}
\exi{2.01}{\glll	{onai =m} anaʔapreent=ein n-ma-toof=ein et reʔ nee n-ok areʔ mahoint=eni =m\\
									{onai =ma} anaʔaprenat=eni n-ma-tofa=eni et reʔ nee n-oka areʔ mahonit=eni =ma\\
									and.so official={\ein} \n-\mak-quarrel={\ein} {\et} {\req} {\nee} {\n-\ok} every parents={\ein\U} =and\\
						\glt	`So (consequently) the government officials, they argued there with all the clan elders, and…' }

\ex{\ve{nuuk tuaf eni, nai{\Q} Fanu nok are{\Q} in tataf}}
\exi{2.05}{\glll	nuuk tua-f=eni, naiʔ Fanu n-ok areʔ iin tata-f\\
									nuka tua-f=eni naiʔ Fanu n-oka areʔ ini tata-f\\
									grief{\M} person-{\f}={\ein\U} {\naiq} Fanu {\n-\ok} every {\iin} eSi-{\f}\\
						\glt	`the bereaved, Mr. Fanu and with all his elder siblings, ' }

\ex{\ve{es-es ate nok in fee in mone,}}
\exi{2.09}{\glll	es{\tl}esa =t n-ok iin fee iin mone\\
									es{\tl}esa =te n-oka ini fee ini mone \\
									{\prd}one ={\te} {\n-\ok} {\iin} wife {\iin} husband \\
						\glt	`each one (of them) with his wife or her husband,' }

\ex{\ve{nasurin am ka tahiin he suubt ii on re{\Q} mee.}}
\exi{2.11}{\glll	na-suri=n =am ka= ta-hiin he suub-t=ii on reʔ mee\\
									na-suri=n =ma ka= ta-hini he suba-t=ii on reʔ mee\\
									\na-collide={\einV} =and {\ka}= {\ta}-know {\he} bury-{\at}={\ii} like {\reqt} how\\
						\glt	`they were at odds and we didn't know where we would bury him.' }

\ex{\ve{Ana{\Q}a preent ein naiti {\Q}niimk ein am ka tahiin he}}
\exi{2.16}{\glll	anaʔapreent=ein n-aiti ʔniim-k=ein =am ka= ta-hiin he\\
									anaʔaprenat=eni n-aiti ʔnima-k=eni =am ka= ta-hini he\\
									official={\ein} \n-lift hand-{\k}={\ein} and {\ka}= {\ta}-know {\he}\\
						\glt	`The government officials lifted their hands (= didn't want to have anything more to do with it) and we didn't know whether ' }

\ex{\ve{urusan he reek haef, are{\Q} tobiru on tiis raur gui he on re{\Q} mee?}}
\exi{2.28}{\begin{xlist}
	\ex{\glll	\sf{urusan} reʔ he reek hae-f \\
						\sf{urusan} reʔ he rekaʔ hae-f \\
						dealings {\req} {\he} order{\M} messenger-{\f} \\
			\glt `arrangements like sending messengers out (with news of the death),' }

	\ex{\glll areʔ tobiru on ahh tiis raargw=ii he on reʔ mee\\
						areʔ tobiru on {} tisi raru=ii he on reʔ mee\\
						every work {\on} {} pour{\M} palm.wine={\ii} {\he} like {\reqt} how\\
			\glt `every detail that had to be attended to, like (the ceremonial) pouring palm-wine, was going to happen how?'}
\end{xlist}}

\ex{\ve{Maut hena{\Q} tatniin sin.}}
\exi{2.33}{\glll	maut henaʔ ta-tniin =siin\\
									maut henaʔ ta-tnina =sini \\
									let {\he} {\ta}-listen {\siin}\\
						\glt `We really should listen to them.' (Deliberately vague.)}

\ex{\ve{Aam Nahor Bani nmaet, in raisn ii {\Q}tet-teta{\Q} kuun.}}
\exi{2.35}{\begin{xlist}
	\ex{\glll	aam Nahor Bani n-maet\\
						ama Nahor Bani n-mate \\
						father{\M} Nahor Bani \n-die \\}

	\ex{\glll	iin rais-n=ii ʔtet{\tl}tetaʔ kuu-n\\
						ini rasi-n=ii ʔtet{\tl}tetaʔ kuu-n\\
						{\iin} issue-{\N}={\ii} {\prd}different alone-{\N}\\
			\glt	`Father Nahor Bani died, and his issue (relating to his death) is entirely different by itself.' }
\end{xlist}}

\end{exe}

\section{Moo{\Q}hitu{\Q}}\label{sec:MooHit}

\subsection{Metadata}
\wg\begin{tabular}{lL{.67\textwidth}}
File-name:					& aaz20120715{\_}04\\
Archive-link:				& \url{http://catalog.paradisec.org.au/collections/OE1/items/aaz20120715_04}\\
Original Name:			& aaz-20120715-4-Nekmese-KusnawiBani-2\\
Language:						& Amarasi [aaz] \\
Dialect:						& Kotos; Koro{\Q}oto hamlet \\
Location:						& Nekmese{\Q}, Amarasi Selatan, Kupang \\
Date:								& 15/07/2015 \\
Speaker(s):					& Taniel Feni, Kusnawi Bani, Heronimus Bani\\
Recorded by: 				& Daniel Kaufman, Heronimus Bani\\
Transcribed by:			& Yedida Ora\\
Interlinear by:			& Owen Edwards \\
Indonesian/Kupang:	& Yedida Ora\\
Free English by:		& Owen Edwards\\
Genre:							& folk-tales\\
Summary:						& Story of Moo{\Q}hitu{\Q}, a mythical snake who created the world\\
Video:							& \url{https://www.youtube.com/watch?v=Z_2D9WhYuuM&list=PLcXFPx-z7B0q_2Ns3iYHigEY77DG4kXSU&index=15}\\
\end{tabular}

\subsection{Notes}
The original recording contains a number of separate stories.
I present here only the first story:
the story of Moo{\Q}hitu{\Q}.
This may be a conflation of two separate myths.
The first is a creation myth about how Moo{\Q}hitu{\Q}, a snake-like being,
separates the sky, land and sea.
The second myth is about how a python copulated with women
and is, perhaps, an explanation for the origin of men.
In Timorese thought the human world cannot exist without
women, who are the source of life.
This leaves unexplained the origin of men.

The information in the myth is incredibly dense
in parts and certain information is left unexplained and/or
assumed to be known by the hearers.
Footnotes provide additional explanations
as well as possible alternate readings.

\subsection{The text}
\begin{exe}

%\ex{\ve{Neno naa re{\Q} paha{\gap}{\Q}piin ii,}}
%\exi{0.00}{\glll	neno naa reʔ paha{\gap}ʔpiin-n=ii\\
%									neno naa reʔ paha{\gap}ʔpina-n=ii\\
%									day {\naa} {\reqt} land{\gap}below-{\N} ={\ii}\\
%						\glt	`At that time the world{\ldots}'}

%\ex[R:]{\ve{mee, mee, mee,}
%				\glt `Hey, wait.'}

%\ex[T:]{\ve{Oi, oi oi oi}
%				\glt `Oh.'}

%\ex[R:]{\ve{\sf{Yep.}}
%			\glt `Yep (start).'}

\ex{\ve{Neno naa pah-a{\Q}pinan ia, ankobub on bare mese{\Q}.}}
\exi{0.05}{\glll	neno naa paha{\gap}ʔpina-n ia a|n-kobub on bare meseʔ \\
									neno naa paha{\gap}ʔpina-n ia {\a}n-kobub on bare meseʔ \\
									day {\naa} land below-{\N} {\ia}	\a\n-piled.up {\on} place one \\
						\glt	`At that time this world was all piled up in one place' }

\ex{\ve{Ka nmui{\Q} fa mainuan.}}
\exi{0.10}{\glll	ka= n-muiʔ =fa maanuan\\
									ka= n-muʔi  =fa mainuan  \\
									{\ka}= \n-exist ={\fa} open \\
						\glt	`there was no openness/space.' }

\ex{\ve{Ka nmui{\Q} fa oe.}}
\exi{0.13}{\glll	ka= n-muiʔ =fa oe\\
									ka= n-muʔi  =fa oe\\
									{\ka}= \n-exist ={\fa} water\\
						\glt	`There was no water.' }

\ex{\label{ex:0.14}\ve{{\Q}Aa{\Q} oe ji nmees, ka tiit fa auf meto{\Q}.}}\footnote{
				Line \qf{ex:0.14}: The meaning of initial phonetic [ʔaːʔ] is currently unclear.
				It may be from the root \ve{{\rt}aʔa} `ritual speech, poetic speech' and
				could, perhaps, mean something like `I am telling it according to tradition'.
				In some other varieties of Meto \ve{{\rt}aʔa} simply means `speak, talk'.}
\exi{0.14}{\glll	ʔ-aaʔ oo\j=ii n-mees ka= tiit =fa auf metoʔ\\
									ʔ-aʔa oe=ii n-mese ka= tita =fa afu metoʔ\\
									\q-speak water={\ii} \n-alone {\ka}= exist ={\fa} ground{\M} dry\\
						\glt	`I say, there was only water, there was no dry ground.'}

\ex{\label{ex:0.17}\ve{Afu ma neno nmanaa{\Q}.}}
\exi{0.17}[K:]{\glll	afu =m neno n-ma-naaʔ\\
											afu =ma neno n-ma-naʔa \\
											ground =and sky \n-\mak-hold \\
								\glt	`The ground and sky held on to one another.' \footnote{
													Line \qf{ex:0.17} is spoken by Kusnawi Bani.}}

\newpage
\ex{\ve{Afu ma neno nmanaa{\Q}. Meis{\Q}ookn ii nnaa{\Q}.}}
\exi{0.18}[T:]{\glll	afu =m neno n-ma-naaʔ meisʔookn=ii n-naaʔ\\
											afu =ma neno n-ma-naʔa  meisʔokan=ii n-naʔa \\
											ground =and sky \n-\mak-hold dark={\ii} \n-hold \\
								\glt	`The ground and sky held on to one another, darkness held (fast).' }

\ex{\label{ex:0.20}\ve{Tapi re{\Q} kauna{\Q} ia, in nmoni nbi oe ji naan ii.}}
\exi{0.20}{\begin{xlist}
	\ex{\glll	\sf{tapi} ahh reʔ kaunaʔ ia \\
						\sf{tapi} {} reʔ kaunaʔ ia \\
						but ahh {\req} snake {\ia} \\}
	\ex{\glll	iin n-moni n-bi oo\j=ee naan-n=ii, \\
						ini n-moni n-bi oe=ee nana-n=ii \\
						{\iin} \n-live \n-loc water={\ee} inside-{\N}={\ii} \\
			\glt	`but as for this snake, he was living inside the water,'}\footnote{
								Line \qf{ex:0.20}:	The snake is Moo{\Q}hitu{\Q}.}
\end{xlist}}

\ex{\ve{Noki-noki te, in naskeke nfena nhake ma,}}
\exi{0.24}{\gll	noki-noki =te iin na-skeke n-fena n-hake =ma\\
									eventually ={\te} {\iin} \na-suddenly \n-rise \n-stand{\U} =and\\
						\glt	`after a while, he suddenly stood up and' }

\ex{\ve{in nfeen es mee te neon gui natsiri{\Q}, natsiri{\Q}, sampai in ntea re{\Q} aat neno nee msa{\Q}, in natuin ee ma,}}\label{ex:0.27}
\exi{0.27}{\begin{xlist}
	\ex{\glll	iin, iin n-feen es mee =t nee{\ng}gw=ii na-tsiriʔ na-tsiriʔ \\
						ini ini n-fena  es mee =te neno=ii na-tsiriʔ na-tsiriʔ \\
						{\iin} {\iin} \n-rise {\et} where ={\te} sky={\ii} \n-spread \n-spread \\
			\glt	`as he went up to somewhere, the sky kept spreading and spreading (upwards)'\footnote{
							Lines \qf{ex:0.27} and \qf{ex:0.34} explain how Moo{\Q}hitu{\Q}
							pushed up the sky, thus separating it from the water.}}

	\ex{\glll	\sf{sampe} iin n-tea reʔ aat neno nee msaʔ\\
						\sf{sampe} ini n-tea reʔ ata neno nee msaʔ \\
						until {\iin} \n-up.to {\req} up sky {\nee} also \\
			\glt `until when he arrived at where the top of the sky also is,'}

	\ex{\glll	iin na-tuin=ee =ma\\
						ini na-tuin=ee =ma\\
						{\iin} \n-follow={\eeV} =and\\
			\glt `he followed it and,'}

\end{xlist}}

\newpage
\ex{\ve{Anhake {\Q}roo-roo =te, es naa neon goe na{\Q} anmana{\Q}a ma,}}\label{ex:0.34}
\exi{0.34}{\begin{xlist}
	\ex{\glll	a|n-hake ʔro{\tl}roo =t \\
						n-hake ʔro{\tl}roo =te \\
						\n-stand {\prd}far ={\te} \\}
	\ex{\glll	es naa nee{\ng}gw=ee naʔ a|n-ma-naʔa =ma	\\
						es naa neno=ee naʔ n-ma-naʔa =ma	\\
						{\et} {\naa} sky={\ee} then \n-\mak-hold{\U} =and\\
			\glt	`when he had stood up for a long time, at that place
							only then the sky held fast (in relation to him) and,' }
\end{xlist}}

\ex{\ve{na{\Q} nsanu nfani kre{\Q}o-kre{\Q}o ma nfani nbi in baran.}}
\exi{0.37}{\begin{xlist}
	\ex{\glll	naʔ n-sanu n-fani kreʔo{\tl}kreʔo =ma\\
						naʔ n-sanu n-fani kreʔo{\tl}kreʔo =ma\\
						then \n-descend \n-return {\frd}a.bit =and\\}
	\ex{\glll	n-fani n-bi iin bara-n\\
						n-fani n-bi ini bara-n\\
						\n-return \n-{\bi} {\iin} place-{\N}\\
			\glt	`then (he) went back down bit by bit and returned to his place'}
\end{xlist}}

\ex{\ve{Nfani nbi in baarn ii. In baarn ee et oe je nanan.}}
\exi{0.40}{\begin{xlist}
	\ex{\glll	n-fani n-bi iin baar-n=ii \\
						n-fani n-bi ini bara-n=ii \\
						\n-return \n-{\bi} {\iin} place-\N={\ii}  \\}
	\ex{\glll	in baar-n=ee et oo\j=ee nana-n\\
						in bara-n=ee et oe=ee nana-n\\
						{\iin} place-\N={\ee} {\et} water={\ee} inside-{\N}\\
			\glt	`(He) went back to his place. His place was inside the water.'}
\end{xlist}}

\ex{\ve{Nbi-bi oe je naan ee onai te, anmo{\Q}e ma npoi jeen anbi meto{\Q}.}}
\exi{0.43}{\begin{xlist}
	\ex{\glll	n-bi{\tl}bi oo\j=ee naan-n=ee {onai =te}\\
						n-ni{\tl}bi oe=ee nana-n=ee {onai =te}\\
						\n-{\prd}loc water={\ee} inside\N={\ee} then\\
			\glt	`after he had been in the water for a while then,'}
	\ex{\glll	a|nmoʔe =ma npoo\j=ena n-bi metoʔ\\
						{\a}n-moʔe =ma n-poi=ena n-bi metoʔ \\
						{\a}\n-do{\U} and \n-exit={\een} \n-{\bi} dry \\
			\glt \lh{\a}`(he) made (dry land) and went out onto dry land,'}
\end{xlist}}

\ex{\ve{Npoi nbi meot{\Q} ee onai te, in ka nmui{\Q} fa bare he natua ma,}}\label{ex:0.47}
\exi{0.47}{\begin{xlist}
	\ex{\glll	n-poi n-bi meotʔ=ee {onai =te}\\
						n-poi n-bi metoʔ=ee {onai =te}\\
						\n-exit \n-{\bi} dry={\ee} then\\}
	\ex{\glll	iin ka= n-muiʔ =fa bare he na-tua =m\\
						ini ka= n-muʔi =fa bare he na-tua =ma\\
						{\iin} {\ka}= \n-exist ={\fa} place {\he} \na-live =and\\
			\glt	`having gone out onto the dry land, he didn't have a place to live and,' }
\end{xlist}}

\ex{\ve{he natua te, baer mainuan.}}
\exi{0.51}{\glll	he na-tua =te he-- baer mainuan\\
									he na-tua =te {} bare mainuan\\
									{\he} \n-live top {} place{\M} open\\
						\glt	`he would (have to) live in an open place,' }

\ex{\ve{Natua te, baer ko{\Q}u.}}
\exi{0.53}{\glll	na-tua =te baer koʔu\\
									na-tua =te bare koʔu \\
									\na-live ={\te} place{\M} big \\
						\glt	`live in a big place,' }

\ex{\ve{Akhirnya, naim naan baer jes am namaika{\Q} nbi Smara{\Q} tunan.}}\label{ex:0.57}
\exi{0.57}{\begin{xlist}
	\ex{\glll	\sf{ahirɲa} ahh n-aim naan baar\j=esa =m namaikaʔ an--,  \\
						\sf{ahirɲa} {} n-ami naan bare=esa =ma na-maikaʔ {} \\
						in.the.end {} \n-look.for {\naan} place={\es} =and \na-stay {}  \\
			\glt `in the end, (he) looked there for a place and settled,'}
	\ex{\glll	na-maikaʔ n-bi Smaraʔ tunan\\
						na-maikaʔ n-bi Smaraʔ tuna-n\\
						\na-stay \n-{\bi} Sm. top-{\N}\\
			\glt `(he) settled on top of Smara{\Q}.' (a headland on the southern coast)}
\end{xlist}}

\ex{\ve{Namaika{\Q} nbi Smara{\Q} tuun ee ma,}}
\exi{1.01}{\glll	na-maikaʔ n-bi Smaraʔ tuun-n=ee =ma\\
									na-maikaʔ n-bi Smaraʔ tuna-n=ee =ma\\
									\n-stay \n-{\bi} Sm. top-\N={\ee} =and\\
						\glt	`settled on top of Smara{\Q} and' }

\newpage
\ex{\ve{In re{\Q} fee mnais unu{\Q} ma nai{\Q} unu{\Q} nnao nakbatun anbi tasi.}}
\exi{1.05}{\begin{xlist}
	\ex{\glll	iin reʔ fee mnais unuʔ =ma naiʔ unuʔ \\
						ini reʔ fee mnasiʔ unuʔ =ma naʔi unuʔ \\
						3sg {\req} wife old{\M} past and grandfather{\M} past \\
			\glt	`he (was) where old women of past times and old men of past times'}
	\ex{\glll	n-nao na-kbatu=n a|n-bi tasi\\
						n-nao na-kbatu=n {\a}n-bi tasi\\
						\n-go \n-shell={\einV} {\a}\n-{\bi} sea\\
			\glt	`went and collected shells by the sea,'}
\end{xlist}}

\ex{\label{ex:1.07}\ve{Ntea uab reu{\Q}f ii jena ma.}}
\exi{1.07}{\glll	n-tea uab reuʔf=ii\j=ena =ma ahh\\
					n-tea uaba reʔuf=ii=ena =ma {}\\
					\n-arrive speech{\M} bad={\ii}={\een} =and {}\\
		\glt	`he went there (to do things which are) bad to talk about.'\footnote{
								Line \qf{ex:1.07} is obscure.
								It probably foreshadows that the actions
								Moo{\Q}hitu{\Q} is about to carry out are bad to talk about.	
								Just after this line Kusnawi Bani says one or two inaudible words.}}

\ex{\ve{In fee je msa{\Q} nua sin huma{\Q} mese{\Q} tapi bifee je bifee biasa.}}\label{ex:1.10}
\exi{1.10}{\begin{xlist}
	\ex{\glll	iin fee\j=ee msaʔ nua sin humaʔ meseʔ \\
						ini fee=ee msaʔ nua sin humaʔ meseʔ \\
						{\iin} wife={\ee} also two {\siin} kind one \\}
	\ex{\glll	\sf{tapi} bifee\j=ee bifee \sf{biasa}\\
						\sf{tapi} bifee=ee bifee \sf{biasa}\\
						but woman={\ee} woman normal\\
		\glt `he and his wife were the same, but the woman was a normal woman'\footnote{
						Line (\ref{ex:1.10}a):
						The wife of Moo{\Q}hitu{\Q} has not been introduced before.
						The reference to her being a `normal woman' is probably a contrast with
						the fact that Moo{\Q}hitu{\Q} is a snake-like being.}}
\end{xlist}}

\ex{\ve{Cuma atoin{\Q} ein ee nteek ee te nak: Moo{\Q}hitu{\Q}.}}\label{ex:1.14}
\exi{1.14}{\glll	\sf{suma} atoinʔ=ein=ee n-teek=ee =te n-ak: Mooʔhituʔ. \\
									\sf{suma} atoniʔ=eni=ee n-teka=ee =te n-ak Mooʔhituʔ \\
									only man=\ein={\ee}	\n-call={\eeV} ={\te} {\n-\ak} Moo{\Q}hitu{\Q} \\
						\glt	`Only the men called him Moo{\Q}hitu{\Q}.'\footnote{
										Line (\ref{ex:1.14}): that only the men call him Moo{\Q}hitu{\Q} is
										probably a reference to his phallic shape and/or nature.}}

\newpage
\ex{\ve{Moo{\Q}hitu{\Q} re{\Q} naan, in kauna{\Q}.}}
\exi{1.16}{\glll	Mooʔhituʔ reʔ naan iin kaunaʔ\\
									Mooʔhituʔ reʔ naan ini kaunaʔ \\
									Moo{\Q}hitu{\Q} {\req} {\naan} {\iin} snake \\
						\glt `That Moo{\Q}hitu{\Q} was/is a snake.' }

\ex{\ve{Kauna{\Q}, mes huum atoni{\Q} on re{\Q} hit.}}
\exi{1.19}{\glll	kaunaʔ mes huum atoniʔ on reʔ hiit\\
									kaunaʔ mes humaʔ atoniʔ on reʔ hiit\\
									snake but face{\M} man like {\reqt} {\hiit}\\
						\glt `(He was) a snake but (he had) a human face/form like us.' }

\ex{\ve{Cuma in kaan ee es re{\Q} nai{\Q} Moo{\Q}hitu{\Q}.}}
\exi{1.22}{\glll	\sf{ʧuma} iin kaan-n=ee eseʔ {} naiʔ Mooʔhitu\\
									\sf{ʧuma} ini kana-n=ee esa reʔ naiʔ Mooʔhituʔ \\
									only {\iin} name\N={\ee} {\esc} {\req} {\naiq} Moo{\Q}hitu{\Q} \\
						\glt	`It was only his name which was Moo{\Q}hitu{\Q}.' }

\ex{\ve{In nfena nhake te, mo{\Q}ok hitu, mes ho muhiin he moo{\Q}k es ate, he mnaun{\Q} ii ba{\Q}uk.}}
\exi{1.24}{\begin{xlist}
\ex{\glll	iin n-fena n-hake =t moʔok hitu\\
					ini n-fena n-hake =te moʔok hitu \\
					{\iin} \n-rise \n-stand{\U} ={\te} section seven{\U}\\
		\glt `If he stood up (there would be) seven sections,'\footnote{
						Line \qf{ex:1.24}: an explanation of the name Moo{\Q}hitu{\Q}.
						It is from the root \ve{moʔok}
						`section of something long, e.g. joints of a finger, nodes of bamboo'
						and \ve{hitu} `seven'.}}\label{ex:1.24}
\ex{\glll	mes hoo mu-hiin he mooʔk=esa =t he mnaunʔ=ii baʔuk\\
					mes hoo mu-hini he moʔok=esa =te he mnanuʔ=ii baʔuk \\
					but {\hoo} \mu-know {\he} section=one ={\te} {\he} long={\ii} several \\
		\glt `but if you know (the length of) one section, it would be very long.'\footnote{
						Line \qf{ex:1.28}: Moo{\Q}hitu{\Q} is so long, that it is hard to know
						how long even a single section of him would be.}}\label{ex:1.28}
\end{xlist}}

\ex{\ve{Akhirnya in nhake nbi Smara{\Q} tuun ee te, bifee ngguin nakbatun nbi nahen nee kboa{\Q} ko{\Q}u.}}
\exi{1.30}{\begin{xlist}
	\ex{\glll	\sf{ahirɲa} iin n-hake n-bi Smaraʔ tuun-n=ee =te  \\
						\sf{ahirɲa} ini n-hake n-bi Smaraʔ tuna-n=ee =te  \\
						in.the.end {\iin} \n-stand \n-{\bi} Sm. top-\N={\ee} ={\te}  \\
			\glt	`In the end while he was standing on top of Smaraʔ,'}
	\ex{\glll	bifee={\ng}gwein na-kbatu=n n-bi nahen nee kboaʔ koʔu\\
						bifee=eni na-kbatu=n n-bi nahe-n nee kboʔes koʔu \\
						woman={\ein} \na-shell-{\einV} \n-{\bi} down-{\N} {\nee} clump{\M} big \\
			\glt	`the women were collecting sea shells down there in a big clump.'}
\end{xlist}}

\ex{\ve{In naim ranan huma{\Q}-huma{\Q} akhirnya,}}
\exi{1.34}{\glll	iin n-aim ranan humaʔ{\tl}humaʔ \sf{ahirɲa}\\
									ini n-ami  ranan humaʔ{\tl}humaʔ \sf{ahirɲa}\\
									{\iin} \n-look.for road {\frd}kind in.the.end\\
						\glt	`he was looking for various ways, and in the end,' }

\ex{\ve{permisi, ma re{\Q} in nnao npeo{\Q} afu, nmoe{\Q} on umeek ji ma,}}
\exi{1.37}{\begin{xlist}
	\ex{\glll	\sf{parmisi} =m reʔ iin nahh hihh \\
						\sf{parmisi} =ma reʔ ini {} {} \\
						excuse.me =and {\req} {\iin} {} {}  \\
			\glt	`excuse me, and it was where he, uhh'\footnote{
							Line \qf{ex:1.37a}: The narrator uses \it{permisi}
							to signal that he is about to talk of sexual matters.}}\label{ex:1.37a}
	\ex{\glll	iin n-nao n-peoʔ afu n-moaʔ on umeek\j=ii =ma\\
						ini n-nao n-peʔo  afu n-moʔe on umeke=ii =ma\\
						{\iin} \n-go \n-go.by ground \n-do like wolf.snake={\ii} =and\\
			\glt	`he went along the ground he was doing it like the wolf snake,'\footnote{
							Line \qf{ex:1.37b}: \ve{umeke} = \it{Lycodon sp.},
							a kind of non-poisonous red snake}}\label{ex:1.37b}
\end{xlist}}

\ex{\ve{in tuan ii nbi ata {\Q}toe{\Q}f ee tuun ee te, in aon ee es anaot ma,}}
\exi{1.42}{\begin{xlist}
	\ex{\glll	iin, iin tua-n=ii n-bi ata ʔtoeʔf=ee tuun-n=ee =t \\
						ini ini  tua-n=ii n-bi ata ʔtoʔef=ee tuna-n=ee =te \\
						{\iin} {\iin} self-\N={\ii} \n-{\bi} up mountain={\ee} top-\N={\ee} ={\te} \\
			\glt `while his self was up on top of the mountain,'}
	\ex{\glll	iin ao-n=ee ees a-nao-t =ma\\
						ini ao-n=ee ees a-nao-t =ma\\
						{\iin} body-\N={\ee} {\esc} {\at}-go-{\at} =and\\
			\glt `his body (was the) one which went and'}
\end{xlist}}

\newpage
\ex{\ve{in nkoin re{\Q} bifee ngguin nbi tasi.}}
\exi{1.46}{\glll	iin n-koin reʔ bifee={\ng}gwein n-bi tasi\\
									ini n-koni  reʔ bifee=eni n-bi tasi \\
									{\iin} \n-copulate {\reqt} woman={\ein} \n-{\bi} sea \\
					\glt `he copulated with those women at the sea.'}

\ex{\ve{Ka nakeon fa.}}
\exi{1.48}{\glll	ka= na-keo=n =fa\\
									ka= na-keo=n =fa\\
									{\ka}= \n-aware={\einV} ={\fa}\\
						\glt `They weren't aware of it.' }

\ex{\ve{In a{\Q}maen ii es anpeo{\Q} afu. In nmoe{\Q} jon on kaun{\Q} ii ma, nnonok anpeo{\Q} auf gui ma,}}
\exi{1.49}{\begin{xlist}
	\ex{\glll	iin, iin, ina ʔ-mae-n=ii esa n-peoʔ afu \\
						ini ini ina ʔ-mae-n=ii esa n-peʔo  afu \\
						{\iin} {\iin} {\iin} {\qq}-shame-\N={\ii} {\esc} \n-go.by ground  \\
			\glt `his private part was the one which went along the ground.'}
	\ex{\glll	iin n-mooʔ\j=oo-n on kaunʔ=ii =ma \\
						ini n-moʔe=oo-n on kaunaʔ=ii =ma \\
						{\iin} \n-do={\oo-\N} like snake={\ii} =and  \\
			\glt `it made itself like a snake and,'}
	\ex{\glll	n-nonok a|n-peoʔ aafgw=ii =ma\\
						n-nonok {\a}n-peʔo afu=ii =ma\\
						\n-crawl \a\n-go.by ground={\ii} =and\\
			\glt `crawled along the ground and,'}
\end{xlist}}

\ex{\ve{nnaob antama ma, in nkoin re{\Q} bifee ngguin. Sin nakeon fa.}}
\exi{1.55}{\begin{xlist}
	\ex{\glll	n-nao-b a|n-tama =m, iin n-koin reʔ bifee={\ng}gwein\\
						n-nao-b {\a}n-tama =ma ini n-koni  reʔ bifee=eni \\
						\n-go-{\b} \a\n-enter{\U} =and {\iin} \n-copulate {\reqt} woman={\ein} \\
			\glt `(he) made (it) go (and) penetrate and he copulated with the women'}
	\ex{\glll	siin ka= na-keo=n =fa\\
						sini ka= na-keo=n	=fa \\
						{\siin} {\ka}= \na-aware={\einV}	={\fa} \\
			\glt	`they weren't aware of it'}
\end{xlist}}

%\ex{\ve{Nbi tais je ma,}}
%\exi{1.58}{\glll	n-bi taas\j=ee =m\\
%									n-bi tasi=ee =ma\\
%									\n-{\bi} sea={\ee} =and\\
%						\glt `He was at the sea and {\ldots}' }

\end{exe}

\section{A car accident}\label{sec:CarCrash}
\subsection{Metadata}

\wg\begin{tabular}{lL{.67\textwidth}}
File-name:					& aaz20130911{\_}02\\
Archive-link:				& \url{http://catalog.paradisec.org.au/collections/OE1/items/aaz20130911_02}\\
Original Name:			& aaz-20130911-2-DominggusBani-HenkiOra- CeritaOtoJato\\
Language:						& Amarasi [aaz] \\
Dialect:						& Kotos; Koro{\Q}oto hamlet \\
Location:						& Nekmese{\Q}, Amarasi Selatan, Kupang \\
Date:								&	11/09/2013\\
Speaker(s):					& Dominggus Bani (D), Heronimus Bani (R), Henki Ora (H), Sefnat Bois, and occasional others\\
Recorded by: 				& Heronimus Bani\\
Transcribed by:			& Heronimus Bani\\
Interlinear by:			& Owen Edwards \\
Indonesian/Kupang:	& Heronimus Bani\\
Free English by:		& Owen Edwards\\
Genre:							& conversation\\
Summary:						& conversation about a car accident\\
%Length:		& 1.43\\
\end{tabular}

\subsection{Notes}
This text is a conversation about a recent car crash.
As is to be expected from natural free-flowing conversation,
there are many instances in which more than one person is speaking at once.
Given this, it was not possible for the transcriber (Heronimus Bani)
to transcribe every voice at every point in the recording.
I have listened through the entire text several times and edited where necessary.
Where there is doubt over the exact transcription,
I have deferred to the original.

The three dominant participants are Dominggus Bani
(D), Heronimus Bani (R) and Henki Ora (H).
Names of other participants
are given in full before their contributions.
When a speaker makes multiple consecutive contributions,
only the first contribution is marked.
The recording begins after the conversation has begun
and the topic of conversation has been established.

\newpage
\subsection{The text}
\begin{exe}
\renewcommand{\N}{\tsc{3gen}}

\ex[R:]{\ve{Onai te ma saa{\Q}, naa,}}
\exi{0.00}[]{\glll	{onai =t}, {onai =t}, =ma, =ma, saaʔ \sf{naa} \\
									{onai =te} {onai =te} =ma =ma saaʔ \sf{naa}\\
									and.then and.then =and =and what well \\
						\glt	`and then, and then, and, and what, well'}

\ex[]{\ve{kedalaman ma{\Q}boik{\Q} ee, keefn ii mnanu{\Q}.}}
\exi{0.03}[]{\glll	\sf{kedalaman} maʔboikʔ=ee keefn=ii mnanuʔ \\
									\sf{kedalaman} maʔbokiʔ=ee kefan=ii mnanuʔ \\
									deep suspended={\ee} gap={\ii} deep \\
						\glt	`(it was) deep (the car) was suspended, the gap was deep.'}

\ex[D:]{\ve{Re{\Q} natoon ii nak am, pas anritu{\Q} neu re{\Q} mnaun{\Q} ii jeen am,
				ka tahini mnaun{\Q} ii basik ate, ka tahiin, neor hit toka ma, es he tahiin.}}
%\exi{0.05}[]{\begin{xlist}
\exi{0.05}{\begin{xlist}
	\ex{\glll	reʔ na-toon=ii n-aka =m, \\
						reʔ na-tona=ii n-aka =ma\\
						{\req} \na-tell={\ii} \n-say =and\\
			\glt	`That's what they said,'}
	
	\ex{\glll	\sf{pas} a|n-rituʔ n-eu reʔ mnaunʔ=ii\j=ena =m, \\
						\sf{pas} {\a}n-rituʔ n-eu reʔ mnanuʔ=ii=ena =ma \\
						exact \a\n-roll \n-{\eu} {\reqt} deep={\ii}={\een} =and \\
			\glt	`and it rolled exactly into the deep space'}
	
	\ex{\glll	ka= ta-hini mnaunʔ=ii basik =at, \\
						ka= ta-hini mnanuʔ=ii basik =te \\
						{\ka}= \ta-know depth={\ii} how.much ={\te} \\
			\glt	`we don't know how deep it was'}
	
	\ex{\glll	ka= ta-hiin, neor hiit t-oka =m es he ta-hiin\\
						ka= ta-hini nero hiit t-oka =ma es he ta-hini \\
						{\ka}= {\ta}-know not {\hiit} {\t-\ok\U} =and one {\he} {\ta}-know \\
			\glt	`we don't know, we weren't with (them) to know' \txrf{0.10}}
\end{xlist}}

\ex[]{\ve{Mnanu{\Q}, oot goe, nak sin na{\Q} nateut oto.}}
\exi{0.12}[]{
	\glll	mnanuʔ ootgw=ee n-ak siin naʔ na-teut oto\\
				mnanuʔ oto=ee n-ak sini naʔ na-tetu oto \\
				deep car={\ee} \n-say {\siin} then \nat-upright car \\
	\glt	`(It was) deep, the car, they said they then stood the car upright'}

\ex[R:]{\ve{Sekau es neki?}}
\exi{0.15}[]{\glll
	sekau ees n-eki\\
	sekau esa n-eki\\
	who {\esc} \n-bring{\U} \\
\glt `Who was the one driving?'}

\ex{Sefnat Bois: \ve{Cuma mana fa te, nmouf goen ate,}}\vspace{-4pt}
\exi{0.16 }[]{\glll
	\sf{suma} nehh, mana =fa =te n-moofgw=ena =te\\
	\sf{suma} {} mana =fa =te n-mofu=ena =te \\
	only {} like.that ={\fa} ={\te} \n-fall={\een} ={\te} \\
\glt `Only, umm, when (it was) like that, when it fell,'}

\ex[R:]{\ve{Rem ee naah mes,}}
\exi{0.18}[]{\glll
	reem=ee na-ah mes \\
	reem=ee na-ah mes\\
	brakes={\ee} \na-eat but\\
\glt `the brakes failed? but{\ldots}'}

\ex[H:]{\ve{Rem ee naah. Semantara n{\Q}antareek.}}
\exi{0.19}[]{\glll
	reem=ee na-ah \sf{sementara} n-\sf{ʔantareek} \\
	reem=ee na-ah \sf{sementara} n-\sf{ʔantareek} \\
	brakes={\ee} \na-eat during \n-backing \\
\glt `The brakes failed, while they were backing.' }

\ex[R:]{\ve{Ohh, semantara n{\Q}antareek.}}
\exi{0.21}[]{\glll
	ohh, \sf{sementara} n-\sf{ʔantareek} \\
	{} \sf{sementara} n-\sf{ʔantareek} \\
	oh during \n-backing \\
\glt `Oh, while they were backing.'}

\ex[H:]{\ve{Jadi, in ka nakeo fa mnaun he--}}
\exi{0.23}[]{\glll
	\sf{{\j}adi} iin ka= nauhh ka= na-keo =fa mnaun he--\\
	\sf{{\j}adi} ini ka= {} ka= na-keo =fa mnanuʔ {}\\
	so {\iin} {\ka}= umm {\ka}= \na-be.aware ={\fa} deep{\M} {}\\
\glt `So, he wasn't, wasn't aware (it was) deep'}

\newpage
\ex[]{\ve{Posisi n{\Q}antareek in ka bisa nbi fa nee, saap ma{\Q}bake{\Q}.}}
\exi{0.25}[]{\begin{xlist}
	\ex{\glll	\sf{posisi} n-\sf{ʔantareek} iin ka= bisa n-bi =fa nee\\
						\sf{posisi} n-\sf{ʔantareek} ini ka= bisa n-bi =fa nee\\
						posisi \n-backing {\iin} {\ka}= able \n-{\bi} ={\fa} {\nee}\\
			\glt	`His position was backing, he couldn't get there '}
	\ex{\glll	saap maʔbakeʔ, \\
						saap maʔbakeʔ \\
						because narrow \\
			\glt	`because it was narrow.'}
\end{xlist}}

\ex[]{\ve{Bait in he naim bare hena{\Q} n{\Q}antareek ate, bisa.}}
\exi{0.28}[]{\glll
	bait iin he n-aim bare henaʔ n-\sf{ʔantareek} =at, bisa. \\
	bait ini he n-ami bare henaʔ n-\sf{ʔantareek} =te bisa \\
	actually {\iin} {\he} \n-look.for place {\he} \n-backing ={\te} able \\
\glt `Actually if he had looked for a place to back, he could have'}

\ex[R:]{\ve{In nareen on ma n{\Q}antareek anbi n--}}
\exi{0.31}[]{\glll
	iin na-reen=oo-n =ma n-\sf{ʔantareek} a|n-bi n-- \\
	ini na-rena=oo-n =ma n-\sf{ʔantareek} {\a}n-bi {}\\
	{\iin} \na-force={\oo=\N} =and \n-backing \a\n-{\bi} {}\\
\glt `He forced himself, and went back into it, he was in{\ldots}'}

\ex[H:]{\ve{Nabara ma{\Q}bake{\Q}.}}
\exi{0.32}[]{\glll
	na-bara maʔbakeʔ \\
	na-bara maʔbakeʔ \\
	\na-forever{\U} narrow \\
\glt `He was stuck in the narrow (place)' \txrf{0.32}}

\ex{Sefnat Bois: \ve{In he nbibi.}}\vspace{-4pt}
\exi{0.34}[]{\glll
	iin he n-bibi \\
	ini he n-bibi \\
	{\iin} {\he} \n-shrink {\U} \\
\glt `He would've wanted to shrink (the car)'}

\ex[D:]{\ve{Nak, oot gui nasnii, mak, am, nakamaf am,}}
\exi{0.35}[]{\glll	n-ak, ootgw=ii na-snii m-ak, =am, na-kamaf =am\\
										n-ak oto=ii na-snii m-ak =ma na-kamaf =ma\\
										\n-say car={\ii} \na-slope \m-say and \na-what's.it =and\\
							\glt	`he said, the car was sloping, you think, and what's it and'}
\ex[]{\ve{nasnii, ntaikobi nkoon, na{\Q} natetu.}}
\exi{0.38}[]{\glll	na-snii n-taikobi n-koon, naʔ na-tetu\\
										na-snii n-taikobi n-kono naʔ na-tetu\\
										\na-slope \n-fall \n-keep.on then \nat-upright{\U}\\
							\glt	`it was sloping, fell over, kept on, and only then he got the car upright' }

\ex[H:]{\ve{Onai ma, srutun re{\Q} ia, in nmouf goen.}}
\exi{0.40}[]{\glll
	{onai =ma} srutun reʔ ia, iin n-moofgw=een \\
	{onai =ma} srutun reʔ ia ini n-mofu=ena \\
	and.so suddenly {\req} {\ia} {\iin} \n-fall={\een} \\
\glt `and suddenly like this, it fell down'}

\ex{Sam Ora: \ve{Oh, mak oot gui in nmese nnao kuun.}}\vspace{-4pt}
\exi{0.42}[]{\glll
	ohh, m-ak, ootgw=ii iin n-mese n-nao kuu-n \\
	{} m-ak oto=ii ini n-mese n-nao kuu-n \\
	oh \m-say car={\ii} {\iin} \n-alone \n-go alone-{\N} \\
\glt `Oh, you think the car went by itself.'}

\ex[R:]{\ve{Mak, sofir ii nmouf goen?}}
\exi{0.43}[]{\glll
	m-ak ahh, sofiir=ii n-moofgw=een\\
	m-ak {} sofiir=ii n-mofu=ena\\
	\m-say {} driver={\ii} \n-fall={\een} \\
\glt `Do you think the driver fell?' }

\ex{Stef Ora: \ve{Tua.}}\vspace{-4pt}
\exi{0.45}[]{\gll
	tua \\
%	tua \\
	{\tua} \\
\glt `yes'}

\ex[R:]{\ve{Tuan?}}
\exi{0.46}[]{\gll
	tua-n \\
	owner-{\N} \\
\glt `(did you say) its owner?'}

\ex[H:]{\ve{Onaim, in nmeo te, oot gui in nmese ntaikob-koib.}}
\exi{0.47}[]{\glll
	{onai =m} iin n-meo =t, ootgw=ii iin n-mese n-taikob{\tl}koib \\
	{onai =m} ini n-meo =te oto=ii ini n-mese n-taikob{\tl}kobi \\
	and.so {\iin} \n-see ={\te} car={\ii} {\iin} \n-alone \n-{\prd}fall \\
\glt `And so when he saw it, the car fell down by itself'}

\newpage
\ex[D:]{\ve{Onai te, oirf ii nok aanh ii sin nbin belakang.}}
\exi{0.52}[]{\glll
	{onai =te} oir-f=ii n-ok aanh=ii siin n-bi=n a|\sf{blaka\ng} \\
	{onai =te} ori-f=ii n-oka anah=ii sini n-bi=n {\a}\sf{blaka\ng} \\
	and.then ySi-\f={\ii} {\n-\ok} child={\ii} {\siin} \n-\bi={\einV} {\a}back \\
\glt `and his younger brother with his child were in the back (of the car)'}

\ex[R:]{\ve{Orif Joni.}}
\exi{0.52}[]{\glll
	ori-f Joni. \\
	ori-f Joni \\
	ySi-{\F} Joni \\
\glt `the younger brother was Johnny.' \txrf{0.52}}

\ex[D:]{\ve{Tuan ii nnaben ate oni{\Q} maineun{\Q} een ate,}}
\exi{0.52}[]{\glll
	tua-n=ii, n-naben =at oniʔ maineunʔ=ena =te. \\
	tua-n=ii n-naben =te oniʔ mainenuʔ=ena =te \\
	owner-{\N}={\ii} \n-feel ={\te} maybe wide.length={\een} ={\te} \\
\glt `The owner, maybe he felt as though there was enough space'}

\ex[]{\ve{Tuan ii nnaben ate mnaun{\Q} een, ro in nrete npoi kuun.}}
\exi{0.55}{\begin{xlist}

	\ex{\glll
		tua-n=ii n-naben =at mnaunʔ=een\\
		tua-n=ii n-naben =te mnanuʔ=ena\\
		owner-{\N}={\ii} \n-feel ={\te} deep={\een}  \\
	\glt `The owner felt it was (too) deep,'}
	
	\ex{\glll
		ro iin n-rete n-poi kuu-n. \\
		ro ini n-rete n-poi kuu-n \\
		must {\iin} \n-jump \a\n-exit alone-{\N} \\
	\glt `he had to jump out by himself'}
	
\end{xlist}}

\ex[R:]{\ve{Aina, in nasaeb ba{\Q}-ba{\Q}uk atoin{\Q} ein?}}
\exi{1.00}[]{\glll
	aina, iin na-sae-b baʔ{\tl}baʔuk atoinʔ=ein\\
	aina ini na-sae-b {\prd}baʔuk atoniʔ=eni \\
	mother {\iin} \na-go.up-{\b} prd-several man={\ein} \\
\glt `Oh my, how many people was he carrying?'}

\ex[D:]{\ve{Molak am, muhiin he,}}
\exi{1.01}[]{\glll
	\sf{molak} =am mu-hiin he \\
	\sf{molak} =ma mu-hini he \\
	log and \muu-know {\he} \\
\glt `(he was carrying) logs, and you know...'}

\ex[R:]{\ve{Ma{\Q}fena{\Q}.}}
\exi{1.04}[]{\glll
	maʔfenaʔ \\
	maʔfenaʔ \\
	heavy \\
\glt `heavy'}

\ex[H:]{\ve{In nak fe{\Q} nasaeban naan tuka{\Q} bo{\Q} esa?}}
\exi{1.05}[]{\glll
	iin n-ak feʔ na-sae-ba=n naan tukaʔ boʔ=esa\\
	ini n-ak feʔ na-sae-ba=n naan tukaʔ boʔ=esa \\
	{\iin} \n-say still \nat-go.up-{\b}={\einV} {\naan} slice ten=one{\U} \\
\glt `he said, he was carrying ten of them, right?'}

\ex[R:]{\ve{Tuka{\Q} bo{\Q} es, mes mainenu{\Q}!}}
\exi{1.09}[]{\glll
	tukaʔ boʔ ees, mes mainenuʔ\\
	tukaʔ boʔ esa mes mainenuʔ\\
	slice ten one but wide.length\\
\glt `Ten of them. But that's too much!'}

\ex[H:]{\ve{Onai te, nak posisi n{\Q}antareek in nasaeba{\Q} nteni{\Q}.}}
\exi{1.10}[]{\glll
	{onai =t} n-ak \sf{posisi} n-\sf{ʔantareek} iin na-sae-baʔ n-teniʔ, \\
	{onai =te} n-ak \sf{posisi} n-\sf{ʔantareek} ini na-sae-{\b} n-teniʔ \\
	and.then \n-say position \n-backing={\ii} {\iin} \nat-go.up-{\b} \n-again \\
\glt `And then he said he was backing, he was carrying more'}

\ex[R:]{\ve{He nteni{\Q}.}}
\exi{1.13}[]{\glll
	he n-teniʔ \\
	he n-teniʔ \\
	{\he} \n-again \\
\glt `He wanted more.'}

\ex[D:]{\ve{Tasaeba{\Q} molak on re{\Q} nee ja te, ma{\Q}fena{\Q}.}}
\exi{1.13}[]{\glll
	ta-sae-baʔ \sf{molak} on reʔ nee\j=aa =t maʔfenaʔ \\
	ta-sae-baʔ \sf{molak} on reʔ nee=aa =te maʔfenaʔ \\
	\tg-go.up-{\b} log like {\reqt} {\nee=\aa} ={\te} heavy \\
\glt `carrying logs like that, it's heavy'}

\ex{Rehuel Nakmofa: \ve{Ma{\Q}fena{\Q}, papan re{\Q},}}\vspace{-4pt}
\exi{1.15}[]{\gll
	maʔfenaʔ, papan reʔ \\
	heavy plank {\req}\\
\glt `heavy, planks which {\ldots}'}

\ex[D:]{\ve{Papan, fe{\Q} papan noo nautn es, ma{\Q}kafa{\Q} fe{\Q}.}}
\exi{1.17}[]{\glll
	mahh papan, feʔ {noo nautn--,} papan noo nautn=ees, \hspace{20mm} maʔkafaʔ feʔ \\
	{} papan feʔ noo papan noo natun=ees {} maʔkafaʔ feʔ \\
	{} plank still {\manaq} plank {\manaq} hundred=one {} light still \\
\glt `Umm, planks, still a hundred, a hundred planks is still light!'}

\ex[H:]{\ve{Onai te, hi misaah miit noo nautn es!}}
\exi{1.21}[]{\glll
	{onai =t} naʔ hii mi-saah m-iit noo nautn=ees \\
	{onai =te} naʔ hii mi-saha m-ita noo natun=ees \\
	and.then {} {\hii} \mi-carry \m-try {\manaq} hundred=one \\
\glt `Well then, why don't you try and carry a hundred planks?'}

\exi{1.23}{\emph{[laughter]}}

\ex[D:]{\ve{Aah, hit tareta{\Q} nok oot goe ma hi ta{\Q}uab,}}
\exi{1.24}[]{\glll
	aah, hiit ta-retaʔ n-ok ootgw=ee =ma hiit ta-ʔuab {\ldots}\\
	{} hiit ta-retaʔ n-oka oto=ee =ma hiit ta-uaba \\
	ah {\hiit} {\ta}-story {\n-\ok} car={\ee} =ma {\hiit} {\ta}-speak \\
\glt `Ah yes! But we're talking about the car! And we're talking {\ldots}'}

\exi{1.26}{\emph{[laughter]}}

\ex[H:]{\ve{Au {\Q}ak, hi misoba{\Q} noo nautn es.}}
\exi{1.28}[]{\glll
	au ʔ-ak hii m-sobaʔ noo nautn=ees \\
	au ʔ-ak hii m-sobaʔ noo natun=ees \\
	{\au} \q-say {\hii} \m-try {\manaq} hundred=one \\
\glt `I said, you try (and carry) a hundred of them'}

\ex[D:]{\ve{Sonde, no nautn es ate, oot gui ma{\Q}kaaf{\Q} ii, naena te, mainenu{\Q}.}}
\exi{1.30}[]{\glll
	\sf{sonde}, noo nautn=esa =t ootgw=ii maʔkaafʔ=ii \hspace{20mm} n-aena =t, mainenuʔ \\
	\sf{sonde} noo natun=esa =te oto=ii maʔkafaʔ=ii {} n-aena =te mainenuʔ \\
	not {\manaq} hundred=one{\U} ={\te} car={\ii} light={\ii} {} \n-run{\U} ={\te} excessive \\
\glt `No, a hundred of them, (in) the car is light, (it) goes quickly, too much'}

\newpage
\ex[R:]{\ve{Onaim, ameent ee neu haa nai{\Q} Firgo.}}
\exi{1.33}[]{\glll
	{onai =m}, mhh, a-meen-t=ee n-eu =ha naiʔ Firgo\\
	{onai =m} {} a-mena-t=ee n-eu =ha naiʔ Firgo \\
	and.so {} {\at}-sick-{\at}={\ee} \n-{\eu} =only {\naiq} Firgo \\
\glt `And so, umm, the only one injured is Firgo.'}

\ex{Rehuel Nakmofa: \ve{Firgo nmees.}}\vspace{-4pt}
\exi{1.35}[]{\glll
	Firgo n-mees \\
	Firgo n-mese \\
	Firgo \n-alone \\
\glt `Just Firgo.'}

\ex[R:]{\ve{On dusun, tak, asaunt ee, nua sin oirf ii ka saa{\Q}.}}
\exi{1.36}{\begin{xlist}

		\ex{\glll
			on nehh, \sf{dusun}, ehh, t-ak a-saun-t=ee\\
			on {} \sf{dusun} {} t-ak a-sanu-t=ee \\
			{\on} {} county {} {\t-\ak} {\at}-descend-{\at}={\ee} \\
		\glt `Like, umm, the county (head), the one who fell down,'}

		\ex{\glll
			nua siin oir-f=ii ka= saaʔ \\
			nua sini ori-f=ii ka= saaʔ \\
			two {\siin} ySi-{\F=\ii} {\ka}= what \\
		\glt `nothing happened to those two kids.'}

\end{xlist}}

\ex{Rehuel Nakmofa: \ve{Nak ka saa{\Q} fa.}}\vspace{-4pt}
\exi{1.40}[]{\glll
	n-ak, ka= saaʔ =fa\\
	n-ak ka= saaʔ =fa \\
	\n-say {\ka}= what ={\fa} \\
\glt `they said nothing happened (to them)'}

\ex{Adi Bani: \ve{Nok keun{\Q} aa te ean{\Q} ee nasoin.}}\vspace{-4pt}
\exi{1.41}[]{\glll
	n-ok keunʔ =at, eanʔ=ee na-soin \\
	n-oka kenuʔ =te enoʔ=ee na-soni \\
	{\n-\ok} fortune ={\te} door={\ee} \na-open \\
\glt `It's fortunate, the door opened'}

\ex[R:]{\ve{Neu reko.}}
\exi{1.43}[]{\glll
	neu reko\\
	neu reko \\
	already good \\
\glt `Well, good.'}

\end{exe}
\section{Gatmel Bana's family (Ro{\Q}is)}\label{sec:GatBanFam}
\subsection{Metadata}

\wg\begin{tabular}{lL{.67\textwidth}}
File-name:					& aazRO20170901{\_}GatmelFamily\\
Archive-link:				& \url{https://catalog.paradisec.org.au/collections/OE2/items/aazRO20170901_GatmelFamily} \\
Original Name:			& aaz-RO-20170901-1-GatmelBana-VillageFamily\\
Language:						& Amarasi [aaz] \\
Dialect:						& Ro{\Q}is; Batuna hamlet \\
Location:						& Desa Tunbaun, Amarasi Barat, Timor, Indonesia\\
Date:								&	01/09/2017\\
Speaker(s):					& Gatmel Daniel Bana', Melianus Obhetan (introduction 0.01--1.01)\\
Recorded by: 				& Owen Edwards, Melianus Obhetan\\
Transcribed by:			& Owen Edwards\\
Interlinear by:			& Owen Edwards \\
Free English by:		& Owen Edwards\\
Checked:						& Unclear sections checked by Owen Edwards with Melianus Obhetan. \\
Genre:							& narrative\\
Summary:						& Gatmel talks about his family and its history.\\
\end{tabular}

%\subsection{Notes}
%In the first part of the recording (until 1.07)
%Melianus Obhetan records the metadata and introduces the narrator and his topic.
%After this the narrator begins.

\renewcommand{\N}{\tsc{3psr;sg.psm}}
\newcommand{\R}{\tsc{3psr;pl.psm}}
\subsection{The text}

\noindent Melianus Obhetan:
\begin{exe}
\ex{\ve{Mabe{\Q} ai hai mi{\Q}euk ma miteef, hai mibua ek maam Rosalin Honin hin uim ji natuina{\Q}, hin ulang tahun}}
\exi{0.05}{\glll
		mabeʔ ai hai mi-ʔeuk =ma mi-teef hai mi-bua ek \sf{maam} Rosalin Honin hiin uum\j=ii na-tuinaʔ hiin \sf{ulaŋ tahun}\\
		mabeʔ ai hai mi-ʔeku =ma mi-tefa hai mi-bua ek \sf{mama} Rosalin Honin hini umi=ii na-tuinaʔ hini \sf{ulaŋ tahun}\\
		evening {\ia} {\hai} \mi-meet =and \mi-meet {\hai} \mi-gather {\ek} mum{\M} Rosalin Honin {\iin} house={\ii} \na-follow {\iin} birthday\\
\glt `This evening we have met together, we have gathered at Rosalin Honin's house because it is her birthday.'}

\newpage
\ex{\ve{Mese{\Q} nmui{\Q} tuaf nu na{\Q}uab naan,}}
\exi{0.14}{\glll
		meseʔ ahh n-muiʔ ahh tuaf nu na-ʔuab n-aan \\
		meseʔ {} n-muʔi {} tuaf nu na-ʔuaba n-ana \\
		but {} \n-exist {} person {\he} \na-speak {\n-\ana} \\
\glt `But there is a person who wants to speak,'}

\ex{\ve{Henati{\Q} nataam hin haann ii nbi{\Q}aka reta{\Q}}.}
\exi{0.20}{\glll
		henatiʔ na-taam hiin haan-n=ii n-biʔaka ahh retaʔ\\
		henatiʔ na-tama hini hana-n=ii n-biʔaka {} retaʔ\\
		{\he} \nat-enter {\iin} voice-{\N=\ii} {\n-\bi} {} story\\
\glt `to put his voice in a story.'}

\ex{\ve{Etu naa neno ai tanggal satu September.}}
\exi{0.25}{\gll
		etu{\gap}naa neno ai \sf{taŋgal} \sf{satu} \sf{September}\\
		therefore day {\ia} date one September\\
\glt `Because of that, today is the first of September.'}

\ex{\ve{Tanggal satu fuun se{\Q}o.}}
\exi{0.30}{\glll
		\sf{taŋgal} \sf{satu} fuun fe-- se\<ʔ\>o\\
		\sf{taŋgal} \sf{satu} funan {} nine\<\qnum\> {}\\
		\sf{taŋgal} \sf{satu} month{\M} {} nine\<\qnum\> {}\\
\glt `The first of September (\emph{lit.} ninth month).'}

\ex{\ve{Ka{\Q}uab ii hin kaann ii Gatmel Daniel Bana{\Q}.}}
\exi{0.33}{\glll
		ahh ka-ʔuab=ii hiin kaan-n=ii Gatmel Daniel Banaʔ\\
		{} ka-ʔuaba=ii hini kana-n=ii Gatmel Daniel Banaʔ\\
		{} {\at-speak=\ii} {\iin} {name-\N=\ii} Gatmel Daniel Bana{\Q}\\
\glt `The one speaking's name is Gatmel Daniel Bana{\Q}.'}

\ex{\ve{Maam Rosalin hin tuuhaon aa.}}
\exi{0.37}{\gll
		\sf{maam} Rosalin hiin tuuhao-n=aa\\
		mum{\M} Rosalin {\iin} {brother-\N=\aa}\\
\glt `Rosalin's brother.'}

\newpage
\ex{\ve{Na{\Q}uab natuina{\Q} sin moinr ini nok sin nai{\Q}ik, sin ama{\Q}.}}
\exi{0.41}{\glll
			na-ʔuab na-tuinaʔ siin moin-r=ini n-oko ahh \hspace{20mm} siin na͡{\i}ʔik siin amaʔ\\
			na-ʔuaba na-tuinaʔ sini moni-r=ini n-oko {} {} sini naiʔik sini amaʔ\\
			\na-speak \na-follow {\siin} {life-\R=\ein\U} {\n-\qko} {} {} {\siin} PF {\siin} father \\
	\glt `(He will) speak about their lives from their grandfather and their father,'}

\ex{\ve{Nu nai{\Q} nnao ntookn ek ahh preent Niin-Ri{\Q}in.}}
\exi{0.46}{\glll
		nu{\gap}naiʔ n-nao n-took=n ek ahh preent Niin-Riʔin\\
		nu{\gap}naiʔ n-nao n-toko=n ek {} prenat Niin-Riʔin\\
		{then} \n-go \n-sit={\einV} {\ek} {} government Niin-Ri{\Q}in\\
\glt `and then go on to (talk) about occupying the government of Niin-Ri{\Q}in.'}

\ex{\ve{Nu nai{\Q} oka te, sin maam Ros nai{\Q}{\ldots}}}
\exi{0.50}{\glll
		nu{\gap}naiʔ {oka   te} siin ahh \sf{maam} Ros naiʔ \\
		nu{\gap}naiʔ {okeʔ =te} sini { } \sf{mama} Ros naiʔ \\
		then after.that {\siin} {} mum{\M} Ros \naiq \\
\glt `Then after that, their ahh, Rosalin{\ldots}'}

\ex{\ve{Maam Rosalin he{\Q} ai nfain neem ek sin ruan ma sin bare ek ruan Nai{\Q}bana{\Q}.}}
\exi{0.55}{\glll
		ahh \sf{maam} Rosalin heʔ ai n-fain neem ek siin ruan =ma siin bare ek ruan Naiʔbanaʔ\\
		{} \sf{mama} Rosalin heʔ ai n-fani nema ek sini ruan =ma sini bare ek ruan Naiʔbanaʔ\\
		{} mum{\M} Rosalin {\req} {\ia} \n-return {\nema} {\ek} {\siin} village =and {\siin} place {\ek} village Nai{\Q}bana{\Q}\\
\glt `how Rosalin here came back to their village and their place at the village of Nai{\Q}bana{\Q}.'}

\ex{\ve{Silahkan.}}
\exi{1.01}{\gll
		\sf{silahkan}\\
		please\\
\glt `Please begin!'}
\end{exe}

\newpage
\noindent Gatmel Daniel Bana{\Q}:

\begin{exe}
\exi{1.07}{\ve{ahh}}

\ex{\ve{Eti fuun tabu ai, au nu {\Q}peo kutonon, hit aan moen ji, kaes muti{\Q}.}}
\exi{1.09}{\glll
		eti fuun tabu ai au nu ʔ-peo ku-tono=n hiit aan \hspace{10mm} moon\j=ii kaes mutiʔ\\
		eti funan tabu ai au nu ʔ-peo ku-tona=n hiti anaʔ {} mone=ii kase mutiʔ\\
		time month{\M} time {\ia} {\au} {\he} \q-speak {\qu}-tell={\einV} {\hiit} child {} male={\ii} foreign{\M} white\\
\glt `at this moment and time of month I want to talk and tell (a story to) our son, the European.'}

\exi{1.17}{\ve{ahh}

\ex{\ve{Yang sebenarnya hai moko hai ruan aa es he{\Q} ai.}}
\exi{1.19}{\glll
		\sf{jaŋ} \sf{sebenarɲa} hai m-oko hai ruan=aa eseʔ {} ai\\
		\sf{jaŋ} \sf{sebenarɲa} hai m-oko hai ruan=aa esa heʔ ai\\
		\tsc{rel} truly {\hai} {\m-\qko} {\hai} village={\aa} {\esc} {\req} {\ia}\\
\glt `In truth we are from our village which is here.'}

\ex{\ve{Tetapi, et au bapa sin, mana{\Q} niim, sin feotr iin teun.}}}
\exi{1.23}{\glll
		\sf{tetapi} et au \sf{bapa} siin manaʔ niim siin feot-r=iin teun\\
		\sf{tetapi} et au \sf{bapa} sini manaʔ nima sini feto-r=ini tenu\\
		but {\ek} {\au} dad {\siinN} {\manaq} five {\siin} mS-{\R=\ein} three\\
\glt `But with my dad (and family) there are five of them, they have three sisters'}

\ex{\ve{Au bapa es moen muni{\Q}.}}
\exi{1.32}{\glll
		au \sf{bapa} ees moen muniʔ\\
		au \sf{bapa} esa mone muniʔ\\
		{\au} father {\esc} male{\M} young\\
\glt `My dad is the youngest man.'}

\ex{\ve{Es nakauhub au bapa ek nakaf Niin-Ri{\Q}in}.}
\exi{1.36}{\glll
		ees na-ka͡uhu-b au \sf{bapa} ek naka-f Niin-Riʔin\\
		esa na-kahu-b au \sf{bapa} ek naka-f Niin-Riʔin\\
		{\es} \nat-adopt-{\b} {\au} dad {\ek} head-{\f} Niin-Ri{\Q}in\\
\glt `My father was adopted into the head (top) of the village of Niin-Ri{\Q}in.'}

\newpage
\ex{\ve{Hai mtook ek naa sampai toon bo{\Q} es am hiut eeh,}}
\exi{1.43}{\glll
		hai m-took ek naa \sf{sampai} toon boʔ esa =m hiut ehh\\
		hai m-toko ek naa \sf{sampai} toon boʔ esa =m hitu {}\\
		{\hai} \m-sit {\ek} {\naa} until year ten one{\U} =and seven {}\\
\glt `We lived there until the year of '17 (2017).'}

\ex{\ve{Hai ruan=ee n-reef.}}
\exi{1.49}{\glll
		hai ruan=ee n-reef\\
		hai ruan=ee n-refa\\
		{\hai} village={\ee} \n-landslide\\
\glt `Our village was affected/destroyed by a landslide.'}

\ex{\ve{Hai ruan ee nreef ma,}}
\exi{1.51}{\glll
		hai ruan=ee n-reef =ma \\
		hai ruan=ee n-refa =ma \\
		{\hai} village={\ee} \n-landslide =and \\
\glt `Our village was affected by a landslide and'}
 
\ex{\ve{oka te hai uim ji naann ii,}}
\exi{1.53}{\glll
		{oka =t} hai uum\j=ii naan-n=ii\\
		{okeʔ =te} hai umi=ii nana-n=ii\\
		{after.that} {\hai} house={\ii} inside-{\N=\ii}\\
\glt `after that those of us in our house,'}

\ex{\ve{hai iim ek hai ruan he{\Q} hai bapa nmoin je es he{\Q} ai.}}
\exi{1.54 }{\glll
		hai iim ek hai ruan heʔ hai \sf{bapa} n-moon\j=ee eseʔ {} ai\\
		hai ima ek hai ruan heʔ hai \sf{bapa} n-moni=ee esa heʔ ai\\
		{\hai} \tsc{1px}{\textbackslash}come {\ek} {\hai} village {\req} {\hai} dad \n-live={\eeV} {\esc} {\req} {\ia}\\
\glt `we came to our village where our dad lives which is here'}

\ex{\ve{Es hai mmoe{\Q} hai umi ai, meter bo{\Q} es nok meter nee.}}
\exi{1.58}{\glll
		ees hai m-moeʔ hai umi ai meter boʔ ees n-ook meter nee\\
		esa hai m-moʔe hai umi ai meter boʔ esa n-oka meter nee\\
		one {\hai} \m-make {\hai} house {\ia} metre ten {\es} \n-{\ok} metre six\\
\glt `We made our house here [points to house], ten metres by six metres.'}

\newpage
\ex{\ve{Oka te hai mitua ok-oke{\Q} ek he{\Q} ai.}}
\exi{2.02}{\glll
		{oka =te} hai mi-tua ok{\tl}okeʔ \hp{e}k heʔ ai\\
		{okeʔ =te} hai mi-tua ok{\tl}okeʔ ek heʔ ai\\
		after.that {\hai} \mi-settle {\prd}all {\ek} {\reqt} {\ia}\\
\glt `After that we all lived here.'}

\ex{\ve{Setelah hai mitua ok-oke{\Q} ek he{\Q} ai ma, }}
\exi{2.04}{\gll
		\sf{setela} hai mi-tua ok{\tl}okeʔ k heʔ ai =ma \\
		after {\hai} \mi-settle {\prd}all {\ek} {\reqt} {\ia} =and\\
\glt `After we all settled here and'}

\ex{\ve{reefk ii nasnaas te}}
\exi{2.08}{\glll
		reefk=ii na-snaas =te\\
		refek=ii na-snasa ={\te} \\
		landslide={\ii} \na-stop ={\te} \\
\glt `when the landslide stopped,'}

\ex{\ve{He{\Q} au tata{\Q} he{\Q} au feot kou{\Q} gui, es he{\Q} maam Ros,}}
\exi{2.11}{\glll
		ahh heʔ au tataʔ heʔ au feot kooʔgw=ii es heʔ \sf{maam} Ros\\
		{} heʔ au tataʔ heʔ au feto koʔu=ii es heʔ \sf{maam} Ros\\
		{} {\reqt} {\au} eSi {\req} {\au} mS{\M} big={\ii} {\esc} {\req} mum{\M} Ros\\
\glt `my older sibling who is the eldest daughter, which is mother Ros'}

\ex{\ve{hin nok hin aanr ini, sin natuan nabaarn ek hai uim ji ese{\Q} ai.}}
\exi{2.15}{\glll
		hiin n-ook hiin aan-r=ini siin na-tua=n na-baar=n ek hai uum\j=ii eseʔ {} ai\\
		hini n-oka hini ana-r=ini sini na-tua=n na-bara=n ek hai umi=ii esa heʔ ai\\
		{\iin} {\n-\ok} {\iin} child-{\R=\ein} {\siin} \na-settle={\einV} \na-stay={\einV} {\ek} {\hai} house={\ii} {\esc} {\req} {\ia}\\
\glt `she and her children have stayed living at our house which is here.'}

\newpage
\ex{\ve{Tetapi hai bian ii hai mfain ek he{\Q} ruan he{\Q} reefk ee hin naann ee.}}
\exi{2.19}{\glll
		\sf{tetapi} hai bian=ii hai m-fain ek heʔ ruan heʔ reefk=ee hiin naan-n=ee\\
		\sf{tetapi} hai bian=ii hai m-fani ek heʔ ruan heʔ refek=ee hini nana-n=ee\\
		but {\hai} other={\ii} {\hai} \m-return {\ek} {\reqt} village {\req} landslide={\ee} {\iin} inside-{\N=\ee}\\
\glt `But we others we went back to the village which the landslide was in.'
			(\emph{lit}. village where the landslide's inside was)}

\ex{\ve{Tetapi setelah hai mfain te reefk ii reko.}}
\exi{2.27}{\glll
		\sf{tetapi} \sf{setela} hai m-fain =t reefk=ii reko\\
		\sf{tetapi} \sf{setela} hai m-fani =te refek=ii reko\\
		but after {\hai} \m-return ={\te} landslide={\ii} good\\
\glt `But after we went back the landslide was fine.' (i.e. it no longer a problem)}

\exi{2.30}{[While recording another man arrived,
said \it{syalom} `greetings' and approached us to shake hands.
At 2.38 he said again \it{syalom bapa} `greetings dad'
and shook hands with the narrator (Gatmel Bana{\Q}),
Melinaus Obhetan, and myself.
The narrator was a little distracted until about 2.45 due to this.]}

\ex{\ve{Enai ma au bapa he{\Q} nahoni{\Q} kai ji,}}
\exi{2.32}{\glll
		{enai =ma} au \sf{bapa} heʔ na-honiʔ =kaa\j=ii \\
		{enai =ma} au \sf{bapa} heʔ na-honiʔ =kai=ii \\
		and.so {\au} dad {\req} \na-birth =\kai={\ii} \\
\glt `And so my dad (at the time) which he had us,'}

\ex{\ve{hai tuaf bo{\Q} es am niim.}}
\exi{2.37}{\glll
		hai tuaf boʔ esa =m niim\\
		hai tuaf boʔ esa =ma nima\\
		{\hai} person ten one =and five\\
\glt `there were fifteen of us.' (\emph{lit.} we people were fifteen)}

\ex{\ve{Bapa nahoni{\Q} kai hai tuaf bo{\Q} es am niim.}}
\exi{2.40}{\gll
		\sf{bapa} na-honiʔ =kai hai tuaf boʔ esa =m niim\\
		dad \na-birth ={\kai} {\hai} person ten {\es} =and five\\
\glt `Dad had fifteen of us.'}

\newpage
\ex{\ve{Enai te he{\Q} kamaets iin tuaf nua hen.}}
\exi{2.45}{\glll
		{enai =te} heʔ ka-maet-s=iin tuaf nua=heen\\
		{enai =te} heʔ ka-maet-s=ini tuaf nua=hena \\
		and.then {\req} {\at}-die-{\at}={\ein} person two={\een}\\
\glt `And then there are two which have died.'}

\ex{\ve{Atoin{\Q} es ma bifee jes.}}
\exi{2.48}{\glll
		atoonʔ=ees =ma bifee\j=ees\\
		atoniʔ=esa =and bifee=esa\\
		man={\es} =and woman={\es} \\
\glt `One man and one woman.'}

\ex{\ve{Hai kamoint ii tuaf bo{\Q} es am teun.}}
\exi{2.52}{\glll
		hai ka-moin-t=ii tuaf boʔ esa =m teun\\
		hai ka-moni-t=ii tuaf boʔ esa =m tenu\\
		{\hai} {\at}-live-{\at}={\ii} person ten one =and three\\
\glt `There are thirteen of us are alive.'}

\ex{\ve{Jadi, he{\Q} hai kamoint ii hai mtook mibaar ek nakaf Niin-Ri{\Q}in,}}
\exi{2.55}{\glll
		\sf{{\j}adi} heʔ hai ka-moin-t=ii hai m-took mi-baar ek \hspace{10mm} nakaf Niin-Riʔin\\
		\sf{{\j}adi} heʔ hai ka-moni-t=ii hai m-toko mi-bara ek {} nakaf Niin-Riʔin\\
		so {\reqt} {\hai} {\at}-live-{\at}={\ii} {\hai} \m-sit {\mi}-stay {\ek} {} head-{\f} Niin-Ri{\Q}in\\
\glt `So those of us who are alive, we stayed living at the head (top) of Niin-Ri{\Q}in.'}

\ex{\ve{Tetapi au tata{\Q} maam Ros he{\Q} ai hin nfain neem ek hai ruan aa ese{\Q} nakaf Batuun he{\Q} ai}}
\exi{3.02}{\glll
		\sf{tetapi} au tataʔ \sf{maam} Ros heʔ ai hiin n-fain neem ek hai ruan=aa eseʔ {} naka-f Batuun heʔ ai\\
		\sf{tetapi} au tataʔ \sf{mama} Ros heʔ ai hini n-fani nema ek hai ruan=aa esa heʔ naka-f Batuna heʔ ai\\
		but {\au} eSi mum{\M} Ros {\req} {\ia} {\iin} \n-return {\nema} {\ek} {\hai} village={\aa} {\esc} {\req} head-{\f} Batuna {\req} {\ia}\\
\glt `But my older sibling, mum Ros here, she came back to our village at the head (top) of Batuna which is here.'}

\newpage
\ex{\ve{Jadi fai ai hit t{\Q}oenn tabua ttoit Uisneno,}}
\exi{3.11}{\glll
		\sf{{\j}adi} fai ai hiit tʔoenn ta-bua t-toit Uisneno \\
		\sf{{\j}adi} fai ai hiit t-ʔonen ta-bua t-toti Uisneno \\
		so night {\ia} {\hiit} \t-pray \ta-gather \t-ask God\\
\glt `So this evening we prayed together we asked God,'}

\ex{\ve{nok Uisneno nmaneer nok kit.}}
\exi{3.19}{\glll
		n-ook Uisneno n-ma-neer n-ook =kiit\\
		n-oka Uisneno n-ma-nera n-oka =kiit\\
		\n-{\ok} God \n-\mak-love \n-{\ok} ={\kiit}\\
\glt `that God would love/bless us'}

\ex{\ve{Es hit mabe{\Q} ai te tabua,}}
\exi{3.22}{\glll
		ees hiit mabeʔ ai =t ta-bua\\
		esa hiit mabeʔ ai =te ta-bua\\
		{\esc} {\hiit} evening {\ia} ={\te} \ta-gather\\
\glt `It was this evening when we gathered,'}

\ex{\ve{hai ori{\Q} es he{\Q} ho, ai{\Q} hai tata{\Q} es he{\Q} ho, mook paah Australia,}}
\exi{3.25}{\glll
		hai oriʔ ees heʔ hoo aiʔ hai tataʔ eseʔ {} hoo \hspace{3cm} m-ook paah Australia\\
		hai oriʔ esa heʔ hoo aiʔ hai tataʔ esa heʔ hoo {} m-oko paha Australia\\
		{\hai} ySi {\esc} {\req} {\hoo} or {\hai} eSi {\esc} {\req} {\hoo} {} {\m-\qko} land{\M} Australia\\
\glt `(that) our younger brother, which is you, or our older brother which is you, (you are) from the land of Australia'}

\ex{\ve{tetapi mabe{\Q} ai hit tateef tatuina{\Q} maneret Uisneno.}}
\exi{3.28}{\glll
		\sf{tetapi} mabeʔ ai hiit ta-teef anene-- ta-tuinaʔ \hspace{30mm} ma-nere-t Uisneno\\
		\sf{tetapi} mabeʔ ai hiit ta-tefa {} ta-tuinaʔ {} ma-nera-t Uisneno\\
		but evening {\ia} {\hiit} \ta-meet {} \ta-follow {} \mak-love-{\at} God\\
\glt `but this evening we have met based on the love of God'}

\newpage
\ex{\ve{Jadi peot he{\Q} au {\Q}peo ek he{\Q} ai ji, }}
\exi{3.34}{\glll
		\sf{{\j}adi} peo-t heʔ au ʔ-peo \hp{e}k heʔ aa{\j}=ii \\
		\sf{{\j}adi} peo-t heʔ au ʔ-peo ek heʔ ai=ii \\
		so talk-{\at} {\req} {\au} \q-talk {\ek} {\req} {\ia}={\ii} \\
\glt `So the speech which I spoke here,'}

\ex{\ve{hanya natuuk ma napaar, ntua ek he{\Q} ai}}
\exi{3.37}{\glll
		\sf{haɲa} na-tuuk =ma na-paar, n-tua \hp{e}k heʔ ai\\
		\sf{haɲa} na-tuka =ma na-para n-tua ek heʔ ai\\
		only \na-short =and \na-short \n-fill {\ek} {\reqt} {\ia}\\
\glt `it is only short [doublet], (it) finishes here.'}

\ex{\ve{Au {\Q}toit makasi.}}
\exi{3.43}{\glll
		au ʔ-toit makasi\\
		au ʔ-toti makasi\\
		{\au} \q-ask thanks\\
\glt `Thank you.'}

\end{exe}


\section{Prefixes}\label{sec:Pre}

\subsection{Verbal agreement prefixes}\label{sec:VerAgrPre}
Amarasi has two sets of verbal agreement prefixes: vocalic prefixes,
given in \trf{tab:VocAgrPre},
and consonantal prefixes, given in \trf{tab:ConAgrPre}.
The consonantal prefixes consist of the initial consonant of the vocalic prefixes,
bearing in mind that the \tsc{1sg} vocalic prefix \ve{u-}
begins with a predictable glottal stop ({\S}\ref{sec:GloStoIns}).
In Ro{\Q}is Amarasi the \tsc{1sg} vocalic prefix is \ve{ku-}
and the \tsc{1sg} consonantal prefix is \ve{k-} before vowels and \ve{ʔ-} before consonants.

\begin{table}[h]
	\caption{Subject agreement prefixes}\label{tab:SubAgrPre}
  \begin{subtable}[b]{0.49\textwidth}
		\centering
		\caption{Vocalic}\label{tab:VocAgrPre}
			\begin{tabular}{rll} \lsptoprule
						& \tsc{sg}	& \tsc{pl} \\ \midrule
				1		& \ve{u-}		& \ve{mi-} \\
				1,2	& \ve{} 		& \ve{ta-} \\
				2		& \ve{mu-}	& \ve{mi-} \\
				3		& \ve{na-}	& \ve{na-} \\ \lspbottomrule
			\end{tabular}
  \end{subtable}
  \begin{subtable}[b]{0.49\textwidth}
		\centering
		\caption{Consonantal}\label{tab:ConAgrPre}
			\begin{tabular}{rll} \lsptoprule
						& \tsc{sg}& \tsc{pl} \\ \midrule
				1		& \ve{ʔ-}	& \ve{m-} \\
				1,2	& \ve{} 	& \ve{t-} \\
				2		& \ve{m-}	& \ve{m-} \\
				3		& \ve{n-}	& \ve{n-} \\ \lspbottomrule
			\end{tabular}
  \end{subtable}
\end{table}

Which prefix set a verb takes is partially determined by the
phonotactic shape of the verbal root, partially determined
by the semantics of the verb and partially lexically determined.
Which prefix set is taken by a verb root according to the
structure of its root is summarised in \trf{tab:PreSetAccRooSha} below.

\begin{table}[ht]
	\caption{Verbal agreement prefix sets according to root shape}\label{tab:PreSetAccRooSha}
	\centering\stl{0.2em}
		\begin{tabular}{lll@{\hspace{-1mm}}ll}\lsptoprule
			Shape		&															& Prefix Set 					&Example				&Gloss\\ \midrule
			(σ)σσσ	&three or more syllables			& consonantal					&\ve{n-ʔeusfani}&`sneeze'\\
			{\#}CC 	&cluster-initial disyllable		& vocalic							&\ve{na-mnaha}	&`hungry'\\
			{\#}V		&vowel-initial disyllable			& consonantal					&\ve{n-inu}			&`drink'\\
			\multirow{2}{*}{{\#}C}
							&\multirow{2}{*}{consonant-initial disyllable}
																						&\multirow{2}{*}{{$\left\{\hspace{-2mm}\begin{array}{l}{\textrm{consonantal 75{\%}}}\\{\textrm{vocalic 25{\%}}}\end{array}\right.$}}
																																	&\ve{na-sai}		&`flow'\\
							&															&											&\ve{n-sae}			&`go up'\\
			\multirow{2}{*}{{\#}C} 
							&\multirow{2}{*}{{$\left\{\hspace{-2mm}\begin{array}{l}{\textrm{transitive}}\\{\textrm{intransitive}}\end{array}\right.$}}
																						& vocalic							&\ve{na-tama}		&`make enter'\\
							&															& consonantal					&\ve{n-tama}		&`enter'\\
			{\#}C 	&loan disyllable							& consonantal					&\it{\sf{n-dukuŋ}}		&`support'\\ \lspbottomrule
		\end{tabular}
\end{table}

This table shows that consonantal prefixes are taken by roots which have
three or more syllables, vowel initial roots and all disyllabic loans.
Roots which begin with a consonant cluster take the vocalic set.
Disyllabic roots which begin with a consonant take either set,
with the choice mostly being lexically specified,
though some take either set with a difference in transitivity.
In my current database I have 140 disyllabic consonant initial verb roots
which take the vocalic prefix set and 417 which take the consonantal set.

There are also a number of verb roots which can take both
sets of prefixes with a difference in valency.
Such roots take the consonantal prefixes when
intransitive and vocalic prefixes when transitive.
Two such roots are \ve{{\rt}tama} `enter' and \ve{{\rt}ʔeka} `close'
each of which is intransitive with a consonantal prefix
and transitive with a vocalic prefix.
Thus, \ve{in n-taam} `s/he enters' alongside \ve{in na-taam=ee} `s/he makes him/her enter',
and \ve{in n-ʔeek} `it is closed' alongside \ve{in na-ʔeek=ee} `s/he closes it'.
In most cases the transitive derivation of such roots also 
takes a transitive suffix \ve{-ʔ} or \ve{-b} (\srf{sec:TraSuf}).
Examples include \ve{{\rt}fani} `return' {\ra} \ve{in n-fain}
`s/he goes back' {\ra} \ve{in na-fain-ʔ=ee} `s/he returns it'
and \ve{{\rt}nao} {\ra} \ve{in n-nao} `s/he goes' {\ra} \ve{in na-nao-b=ee} `s/he makes him/her go'.
See \srf{sec:Suff} below for more discussion of these transitive suffixes.

Any combination of a consonantal prefix followed by another consonant is an allowable stem initial consonant cluster,
even if it would violate the restrictions against root initial consonant clusters given in {\S}\ref{sec:RooIniConClu}.
The only exception is a combination of \ve{ʔ-} before a root which begins with /ʔ/.
Such instances always surface phonetically as a single glottal stop [ʔ] rather than geminate [ʔː].

There are two verbs in Amarasi which have irregular inflections.
Firstly, there is the verb for `come'.
which has partially suppletive forms.
The conjugation of `come' is given in \trf{tab:ConCome}.
Secondly, there is the verb for `eat (soft food)',
which is the only monosyllabic verb root in my
database which takes agreement prefixes.
This root takes vocalic prefixes, meaning that the resulting
inflected word comprises a disyllabic foot (\srf{sec:TheFoo}).
For the purposes of metathesis, the final CV sequence of the derived
word (which is equivalent to the root) undergoes metathesis.
Thus \ve{na-\tbr{ha}} `3-eat\U' {\ra} \ve{na-\tbr{ah}} `3-eat\M'.
The paradigm for \ve{{\rt}ha} `eat (soft food)' is given in \trf{tab:ConHa}.

\begin{table}
	\caption{Irregular verbal conjugations}\label{tab:IrrVerCon}
  \begin{subtable}[b]{0.49\textwidth}
		\centering
		\caption{\ve{{\rt}Vma} `come'}\label{tab:ConCome}
			\begin{tabular}{rll} \lsptoprule
						& \tsc{sg}	& \tsc{pl} \\ \midrule
				1		& \ve{uum}	& \ve{iim} \\
				1,2	& \ve{} 		& \ve{teem} \\
				2		& \ve{uum}	& \ve{iim} \\
				3		& \ve{neem}	& \ve{nema=n} \\ \lspbottomrule
			\end{tabular}
  \end{subtable}
  \begin{subtable}[b]{0.49\textwidth}
		\centering
		\caption{\ve{{\rt}ha} `eat (soft food)'}\label{tab:ConHa}
			\begin{tabular}{rll} \lsptoprule
						& \tsc{sg}		& \tsc{pl} \\ \midrule
				1		& \ve{u-ah}		& \ve{mi-ah} \\
				1,2	& \ve{} 			& \ve{ta-ah} \\
				2		& \ve{mu-ah}	& \ve{mi-ah} \\
				3		& \ve{na-ah}	& \ve{na-ha=n} \\ \lspbottomrule
			\end{tabular}
  \end{subtable}
\end{table}

\subsection{Reciprocal prefix}\label{sec:RecPre}
The reciprocal prefix is \ve{ma-}.
The addition of the reciprocal prefix to a verb
makes it longer than a single foot,
thus all verbs with this prefix take the consonantal agreement prefixes.
Examples of verbs with \ve{ma-} extracted from my corpus
are given in \qf{ex:RecMa-} below,
in which forms usually also occur with the plural
enclitic \ve{=ein/=n} (\srf{sec:PluEnc}).

\begin{exe}
	\ex{Reciprocal \ve{ma-}}\label{ex:RecMa-}
	\sn{\gw\stl{0.3em}\begin{tabular}{lllll}
			`hit'			&\ve{{\rt}bana} 	&+ \ve{ma-} \ra& \ve{n-ma-bana=n} & `hit one another'\\
			`hold'		&\ve{{\rt}naʔa} 	&+ \ve{ma-} \ra& \ve{n-ma-naʔa=n} & `hold on to one another'\\
%			`glare'		&\ve{{\rt}smeruʔ}	&+ \ve{ma-} \ra& \ve{n-ma-smeurʔ=ein} & `glare at one another'\\
			`shake		&\ve{{\rt}tabe} 	&+ \ve{ma-} \ra& \ve{n-ma-tabe=n} & `shake hands with \\ \hhline{~}
	\hp{`}hands'	&								 	&							 & 									& \hp{`}one another'\\
			`think'		&\ve{{\rt}tenab} 	&+ \ve{ma-} \ra& \ve{n-ma-tenab} & `think one by one'\\
			`quarrel'	&\ve{{\rt}toe} 		&+ \ve{ma-} \ra& \ve{n-ma-toe=n} & `quarrel with each o.'\\
%								&									& 				 		 &									& \hp{`}one another'\\
		\end{tabular}}
\end{exe}

This prefix has the allomorph \ve{mak-} before some,
but not all, roots which begin with /t/.
I have so far collected six /t/ initial roots
which take the allomorph \ve{mak-}.
These six roots are given in \qf{ex:RecMak} below.
These forms can be compared with the final three forms in \qf{ex:RecMa-}
which are all also roots with an initial /t/.

\newpage
\begin{exe}
	\ex{Reciprocal \ve{mak-}}\label{ex:RecMak}
	\sn{\gw\stl{0.4em}\begin{tabular}{lllll}
			`ask'			&\ve{{\rt}tana} &+ \ve{mak-} \ra& \ve{n-mak-tana=n}		& `ask one another'\\
			`meet'		&\ve{{\rt}tefa} &+ \ve{mak-} \ra& \ve{n-mak-tefa=n}		& `meet one another'\\
			`angry'		&\ve{{\rt}toʔo}	&+ \ve{mak-} \ra& \ve{n-mak-toʔo=n}			& `angry at one another'\\
			`tell'		&\ve{{\rt}tono} &+ \ve{mak-} \ra& \ve{n-mak-tono=n} 	& `tell one another'\\
			`follow'	&\ve{{\rt}tuin} &+ \ve{mak-} \ra& \ve{n-mak-tuin=ein} & `consecutively'\\
		\end{tabular}}
\end{exe}

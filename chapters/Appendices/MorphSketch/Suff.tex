\section{Suffixes}\label{sec:Suff}
Kotos Amarasi does not allow word-final consonant clusters.
Thus, the addition of consonantal suffixes
to consonant-final roots is not straightforward.
Such clusters are avoided through deletion of the root final
consonant, using an allomorph of the suffix which contains a vowel,
or by not using the suffix.

\subsection{Genitive suffixes}\label{sec:GenSuf}
The genitive suffixes are given in \trf{tab:GenSuf} below.
These suffixes only occur on nouns
which are in a part-whole relationship with the ``possessor''.
Such nouns in turn occur almost obligatorily with a genitive suffix.
Examples of each of these suffixes on a number of
nouns are given in \trf{tab:BodParGenSuf}.
The use of the genitive suffixes is discussed in more detail in \srf{sec:GenSuf ch:SynMet}.

\begin{table}[h]
		\caption{Kotos Amarasi genitive suffixes}
		\centering
			\begin{tabular}{rll} \lsptoprule
						& \tsc{sg}&	\tsc{pl}	\\ \midrule
				1		& \ve{-k}	& \ve{-m}		\\
				1,2	& 				& \ve{-k}		\\
				2		& \ve{-m}	& \ve{-m}		\\
				3		& \ve{-n}	& \ve{-k}		\\
				0		& \mc{2}{c}{\ve{-f}}	\\ \lspbottomrule
			\end{tabular}
		\label{tab:GenSuf}
\end{table}

\begin{table}[h]
	\caption[Body parts with genitive suffixes]
		{Body parts with genitive suffixes \citep[7]{gr12}}\label{tab:BodParGenSuf}
	\centering
		\begin{tabular}{r|lllllll} \lsptoprule
							&							&`body'			&`spirit'			&`eye'				&`foot, leg'&`ear'				&`face' \\ \midrule
			0 			&							&\ve{ao-f}	&\ve{smana-f}	&\ve{mata-f}	&\ve{hae-f}	&\ve{ruki-f}	&\ve{huma-f}\\
		\tsc{1sg} &\ve{au}			&\ve{ao-k}	&\ve{smana-k}	&\ve{mata-k}	&\ve{hae-k}	&\ve{ruki-k}	&\ve{huma-k}\\
		\tsc{2sg} &\ve{hoo}			&\ve{ao-m}	&\ve{smana-m}	&\ve{mata-m}	&\ve{hae-m}	&\ve{ruki-m}	&\ve{huma-m}\\
		\tsc{3sg} &\ve{iin}			&\ve{ao-n}	&\ve{smana-n}	&\ve{mata-n}	&\ve{hae-n}	&\ve{ruki-n}	&\ve{huma-n}\\
		\tsc{1in} &\ve{hiit}		&\ve{ao-k}	&\ve{smana-k}	&\ve{mata-k}	&\ve{hae-k}	&\ve{ruki-k}	&\ve{huma-k}\\
		\tsc{1ex} &\ve{hai}			&\ve{ao-m}	&\ve{smana-m}	&\ve{mata-m}	&\ve{hae-m}	&\ve{ruki-m}	&\ve{huma-m}\\
		\tsc{2pl} &\ve{hii}			&\ve{ao-m}	&\ve{smana-m}	&\ve{mata-m}	&\ve{hae-m}	&\ve{ruki-m}	&\ve{huma-m}\\
		\tsc{3pl} &\ve{siin}		&\ve{ao-k}	&\ve{smana-k}	&\ve{mata-k}	&\ve{hae-k}	&\ve{ruki-k}	&\ve{huma-k}\\
		\lspbottomrule
		\end{tabular}
\end{table}

The ``0 person'' suffix \ve{-f} occurs when the possessor
is irrelevant to the discourse, or it is not in a part-whole relationship, or its association is not in focus.
This includes the citation form, amputation, or when the part is being talked about in generic terms \citep{gr12}.
On kin terms the suffix \ve{-f} has a different function, discussed in \srf{sec:KinGenSuf} below.
%In Koro{\Q}oto it appears to occur on all kin terms when they are possessed,
%and is absent when they are used as a vocative.

I have collected less than a dozen words which
contain a vowel medial glottal stop when no genitive
suffix is attached, and contain a vowel sequence when
a genitive suffix is attached.
These words are given in \qf{ex:MedGloStoDel2} below
with both unsuffixed and suffixed forms.\footnote{
		In addition to the forms given in \qf{ex:MedGloStoDel2}
		there are two Ro{\Q}is Amarasi terms which are also known to
		exhibit such medial glottal stop deletion.
		These are \ve{tuuhaʔo} {\ra} \ve{tuuhao-f} `same sex sibling'
		and \ve{maʔo} {\ra} \ve{aam mao-f} `father's younger brother'
		or \ve{ain mao-f} `mother's younger sister'.}
When the words in \qf{ex:MedGloStoDel2}
have known Proto-Malayo-Polynesian reflexes,
this medial glottal stop is not an inheritance from any reconstructed consonant.
(See \srf{sec:NonEtyGloSto} for more details.)


\begin{exe}
	\ex{VʔV {\lra} VV-C\sub{\tsc{gen}}}\label{ex:MedGloStoDel2}
	\sn{\gw\begin{tabular}{rlll}
		\ve{taʔe}	&+ \ve{-f} {\lra}	& \ve{tae-f}			& `a branch'\\
		\ve{haʔe}	&+ \ve{-m} {\lra}	& \ve{hoo hae-m}	& `your leg'\\
		\ve{noʔo}	&+ \ve{-k} {\lra}	& \ve{siin noo-k}	& `their leaves'\\
		\ve{uʔu}	&+ \ve{-n} {\lra}	& \ve{iin uu-n} 	& `its source'\\
		\ve{baʔe}	&+ \ve{-f} {\lra}	& \ve{bae-f}			& `same sex male cross cousin'\\
		\ve{beʔi}	&+ \ve{-f} {\lra}	& \ve{bei-f}			& `grandmother'\\
		\ve{naʔo}	&+ \ve{-f} {\lra}	& \ve{nao-f}			& `woman's brother' \\
		\ve{koʔu}	&+ \ve{-f} {\lra}	& \ve{aam kou-f}	& `father's older brother'\\ \hhline{~}
							&  								& 								& \hp{`}(\it{lit.} `big father')\\
		\ve{koʔu}	&+ \ve{-f} {\lra}	& \ve{ain kou-f}	& `mother's older sister'\\ \hhline{~}
							&  								& 								& \hp{`}(\it{lit.} `big mother')\\
		\ve{feʔu}	&+ \ve{-f} {\lra}	& \ve{moen feu-f}	& `son-in-law' (\it{lit.} `new male')\\
	\end{tabular}}
\end{exe}

While the medial glottal stop of \ve{koʔu} `big'
is deleted in the phrases \ve{aam kou\=/f} `father's elder brother'
and \ve{ain kou-f} `mother's elder sister, in the phrase \ve{keo koʔu-f} `Achille's tendon'
(from \ve{keo-} `vein' + \ve{koʔu}) it is retained.

Historically, the medial glottal stop in such forms is
a nominal suffix which has metathesised with the final vowel.
This suffix is attested in the Rote languages,
as seen in for instance in Termanu \it{beu-k}, Dengka \it{feu-ʔ} `new',
Amarasi \ve{feʔu} `new', Termanu \it{doo-k}, Dengka \it{loo-ʔ}, Amarasi \ve{noʔo} `leaf', and
Termanu \it{huu-k}, Dengka \it{huu-ʔ}, Amarasi \ve{uʔu} `tree trunk, source'.

With the exception of the words given in \qf{ex:MedGloStoDel2},
other words with a medial glottal stop retain
this glottal stop when a genitive suffix occurs.
In such cases the medial glottal stop is a retention of
an earlier root-medial consonant.
Two examples are Amarasi \ve{naʔi-f} `grandfather', which is from Proto-Malayo-Polynesian *laki `man, male',
and \ve{ʔbaʔa-f} `roots' which can be compared with Funai Helong \it{kbakat} `roots',
possibly irregular reflexes of Proto-Malayo-Polynesian *wakaR.

\subsubsection{Kin terms}\label{sec:KinGenSuf}
Kin relations take a different set of genitive suffixes to other nouns.
In ``normal'' Amarasi, kin relations take the suffix \ve{-f}
when the possessor is \tsc{3sg} and take no suffix with other possessors
or when the stem is not possessed.
When the possessor is \tsc{3pl} there
is variation between no-suffix and \ve{-f}

The ``normal'' Amarasi kin genitive paradigm is given in \trf{tab:AmaKinGen}.
These are the forms used, for instance,
in the Amarasi Bible translation in order to
avoid forms specific to a particular dialect.
In the village of Koro{\Q}oto (where most of my data was gathered)
the suffix \ve{-f} is used on all possessed kin relations.
The Koro{\Q}oto kin genitive suffixes are given in \trf{tab:AmaKorKinGen}.

\begin{table}
	\caption{Kin possessive paradigms}
  \begin{subtable}[b]{0.49\textwidth}
		\centering
		\caption{General Kotos Amarasi}\label{tab:AmaKinGen}
			\begin{tabular}{rll} \lsptoprule
						& \tsc{sg}	& \tsc{pl} \\ \midrule
				1		& {\0}			& {\0} \\
				1,2	& \ve{} 		& {\0} \\
				2		& {\0}			& {\0} \\
				3		& \ve{-f}		& {\0}/\ve{-f} \\
				0		& \mc{2}{c}{{\0}/\ve{-f}}	\\ \lspbottomrule
			\end{tabular}
  \end{subtable}
  \begin{subtable}[b]{0.49\textwidth}
		\centering
		\caption{Koro{\Q}oto hamlet}\label{tab:AmaKorKinGen}
			\begin{tabular}{rll} \lsptoprule
						& \tsc{sg}& \tsc{pl} \\ \midrule
				1		& \ve{-f}	& \ve{-f} \\
				1,2	& \ve{} 	& \ve{-f} \\
				2		& \ve{-f}	& \ve{-f} \\
				3		& \ve{-f}	& \ve{-f} \\
				0		& \mc{2}{c}{\0/\ve{-f}}	\\ \lspbottomrule
			\end{tabular}
  \end{subtable}
\end{table}

The (Kotos) Amarasi kin terms which have been attested with
these possessive paradigms are given in \trf{tab:AmaKinTerGenSuf}.
Nearly all kin terms have a medial or final glottal stop
which is deleted when the suffix \ve{-f} is attached.\footnote{
		An alternate analysis of the data is to analyse
		this glottal stop as an affix which would replace the cells
		filled with {\0} in \trf{tab:AmaKinGen}. However, under
		this analysis we cannot explain why \ve{naʔi} `grandfather'
		and \ve{kaʔo} `ancestor' do not take this putative \ve{-ʔ}
		affix when \ve{-f} does not occur.}
Noun phrases in which one of these kin terms is the head
noun also take genitive suffixes, thus for instance,
\ve{ama-f} `father' + \ve{koʔu} `big' {\ra} \ve{aam kou-f} `father's older brother'.
Not all words which are semantically kin terms take kin genitive suffixes.
Three examples are \ve{anah} `child', \ve{mone} `husband' and \ve{fee} `wife'.

\begin{table}[ht]
	\caption{Kotos Amarasi kin terms with genitive suffixes}\label{tab:AmaKinTerGenSuf}
	\centering
		\begin{threeparttable}[b]
			\stl{0.28em}\begin{tabular}{llll}\lsptoprule
\mc{2}{l}{Amarasi}			&	Gloss	&	meaning \\ \midrule
\ve{naʔi}	&	\ve{naʔi-f}	&	PF	&	`grandfather' \\
\ve{beʔi}	&	\ve{bei-f}	&	PM	&	`grandmother'\\
\ve{kaʔo}\su{†}	&		&	PPP	&	`ancestor'\\
\ve{amaʔ}	&	\ve{ama-f}	&	F	&	`father, father's brother'\\
\ve{ainaʔ}	&	\ve{aina-f}	&	M	&	`mother, mother's sister'\\
\ve{babaʔ}	&	\ve{baba-f}	&	MB/FZ	&	`parent's opposite sex sibling'\\
\ve{bitoroʔ}\su{‡}	&		&	MB	&	`mother's brother'\\
\ve{tataʔ}	&	\ve{tata-f}	&	eSi	&	`same sex elder sibling'\\
\ve{oriʔ}	&	\ve{ori-f}	&	ySi	&	`same sex younger sibling'\\
\ve{naʔo}	&	\ve{nao-f}	&	fB	&	`woman's brother'\\
\ve{fetoʔ}	&	\ve{feto-f}	&	mZ	&	`man's sister'\\
\ve{moen feʔu}	&	\ve{moen feu-f}	&	DH	&	`daughter's husband, \\
	&		&		&	\hp{`}opposite sex sibling's son'\\
	&	\ve{nane-f}	&	SW	&	`son's wife, opposite\\
	&		&		&	\hp{`}sex sibling's daughter'\\
\ve{baʔe}	&	\ve{bae-f}	&	WB/ZH/	&	1) `same sex cross-cousin, same sex sibling\\
	&		&	MBD/FZS	&	\hp{1) `}of spouse, opposite sex sibling's spouse'\\
	&		&		&	2) `mate, friend' \\
\ve{upuʔ}	&	\ve{upu-f}	&	CC	&	`grandchild'\\
\ve{uup kaʔo}	&	\ve{uup kaʔo-f}	&	CCC	&	`great-grandchild'\\
			\lspbottomrule
			\end{tabular}
		\begin{tablenotes}
			\item [†] \ve{kaʔo} `ancestor' and \ve{bitoroʔ} `mothers' brother
								have not yet been attested with a \tsc{3sg} possessor
								and the suffix \ve{-f}. \ve{nane-f} `daughter in law'
								has not yet been attested without the suffix \ve{-f}.
			\item [‡] \ve{bitoroʔ} `maternal uncle' is specific to Koro{\Q}oto hamlet
								and had the variant \ve{toroʔ}. Some varieties of Kopas have
								\ve{toloʔ} `maternal uncle'.
		\end{tablenotes}
	\end{threeparttable}
\end{table}

Examples of the ``normal'' paradigm for kin terms 
are given in \qf{ex:130825-6, 17.22}--\qf{ex:Genesis 29:13} below.
Most such examples I have encountered occur in the Amarasi Bible translation.

\begin{exe}
	\ex{\gll	au \tbr{babaʔ} na-mena =m et uam menas\\
						{\au} FZ/MB \na-sick =and {\et} house sick\\
			\glt	`My aunt was sick and in the hospital.' \txrf{130825-6, 17.22} {\emb{130825-6-17-22.mp3}{\spk{}}{\apl}}}\label{ex:130825-6, 17.22}
	\ex{\gll	hoo ro he m-hormaat hoo \tbr{ainaʔ} =ma hoo \tbr{amaʔ}.\\
						{\hoo} must {\he} \m-honour {\hoo} mother =and {\hoo} father\\
			\glt	`You must honour your mother and your father.' \txrf{Ephesians 6:2}}\label{ex:Ephesians 6:2}
	\ex{\gll	mes hoo m-ak iin reʔ ia, hoo \tbr{fetoʔ}! \\
						but {\hoo} \m-say {\iin} {\req} {\ia} {\hoo} mZ \\
			\glt	`But you said that she is your sister!' \txrf{Genesis 12:19}}\label{ex:Genesis 12:19}
	\ex{\gll	naiʔ Yakop naan, iin \tbr{feto}-\tbr{f} bi Ripka anah\\
						{\naiq} Jacob {\naan} {\iin} mZ-{\F} {\BI} Rebecca child \\
			\glt	`Jacob\sub{\it{i}} was the son of his\sub{\it{j}} sister Rebecca.'
						\txrf{Genesis 29:13}}\label{ex:Genesis 29:13}
\end{exe}

Examples of the Koro{\Q}oto general kin suffix \ve{-f}
are given in \qf{ex:130825-6, 3.43} and \qf{ex:130825-6, 2.06} below.

\begin{exe}
	\ex{\gll	au feto-\tbr{f} nee \sf{mema\ng} iin n-\sf{sadar}.\\
						{\au} mZ-\tbr{\F} {\nee} indeed {\iin} \n-aware\\
			\glt	`My sister there certainly was aware.'
						\txrf{130825-6, 3.43} {\emb{130825-6-03-43.mp3}{\spk{}}{\apl}}}\label{ex:130825-6, 3.43}
	\ex{\gll	hoo feat-\tbr{f}=ii bi sekau?\\
						{\hoo} mZ-\tbr{\F}=ii {\BI} who\\
			\glt	`Who's your sister?/What's your sister's name?'
						\txrf{130825-6, 2.06} {\emb{130825-6-02-06.mp3}{\spk{}}{\apl}}}\label{ex:130825-6, 2.06}
\end{exe}

\subsubsection{Ro{\Q}is possession}\label{sec:RoqPoss}
Ro{\Q}is Amarasi has a different set of possessive
suffixes compared with Kotos.
The Ro{\Q}is genitive suffixes are given in \trf{tab:RoqGenSuf}.
Ro{\Q}is does not have a separate paradigm
for kin relations and kin terms either
take no suffix or take the suffixes
given in \trf{tab:RoqGenSuf}.

\begin{table}[ht]
		\caption{Ro{\Q}is Amarasi genitive suffixes}
		\centering
			\begin{tabular}{rll} \lsptoprule
						& \tsc{sg}		&	\tsc{pl}		\\ \midrule
				1		& \ve{-k}			& \ve{-m}			\\
				1,2	& 						& \ve{-k/-r}	\\
				2		& \ve{-m}			& \ve{-m}			\\
				3		& \ve{-n/-r}	& \ve{-n/-r}	\\
				0		& \mc{2}{c}{\ve{-f}}				\\ \lspbottomrule
			\end{tabular}
		\label{tab:RoqGenSuf}
\end{table}

The third person suffix \ve{-r} is used when the thing possessed is plural
and \ve{-n} is used when the thing possessed is singular.
This means that for these persons the suffix
indexes the person of the possessor and the number of the possessum.
Elicited examples are given in \qf{ex:hin-maatn ee}--\qf{ex:sin-moinr iin} below which
shows each combination of a singular possessor and plural posessum.
In these examples \tsc{psr} is used for \emph{possessor} and \tsc{psm}
is used for \emph{possessum} (the thing possessed).

\begin{multicols}{2}
	\begin{exe}
		\ex{\gll	hiin maat-\tbr{n}=ee\\
							{\iin} eye-\tsc{\tbr{3psr;sg.psm}}={\ee}\\
				\glt	`her/his eye' \txrf{}}\label{ex:hin-maatn ee}
		\ex{\gll	hiin maat-\tbr{r}=iin\\
							{\iin} eye-\tsc{\tbr{3psr;pl.psm}}={\ein}\\
				\glt	`her/his eyes' \txrf{}}\label{ex:hin-maatr iin}
	\end{exe}
\end{multicols}
\begin{multicols}{2}
	\begin{exe}
		\ex{\gll	siin moin-\tbr{n}=ee\\
							{\siin} life-\tsc{\tbr{3psr;sg.psm}}={\ee}\\
				\glt	`their life' \txrf{}}\label{ex:sin-moinn ee}
		\ex{\gll	siin moin-\tbr{r}=iin\\
							{\siin} life-\tsc{\tbr{3psr;pl.psm}}={\ein}\\
				\glt	`their lives' \txrf{}}\label{ex:sin-moinr iin}
	\end{exe}
\end{multicols}

The ungrammatical example in \qf{ex:*hin maatn iin} below
shows that the singular possessum suffix \ve{-n} cannot
co-occur with the plural enclitic \ve{=iin}.
The ungrammaticality of \qf{ex:*hiin haer bian=ee}
arises from the suffix \ve{-r} marking the possessum \ve{haʔe}
`leg/foot' as plural in combination with real world knowledge
that people only have two legs.\footnote{
		Presumably the phrase \ve{hiin hae-r bian=iin} `its other legs'
		could be used with reference to an animal with multiple legs,
		but this has not yet been tested.}

\begin{exe}
	\ex[*]{\gll	hiin maat-\tbr{n}=\tbr{iin}\\
							{\iin} eye-\tsc{3psr;\tbr{sg}.psm}=\tbr{\ein}\\
				\glt	`(her/his eyes)' \txrf{}}\label{ex:*hin maatn iin}
	\ex[*]{\gll	hiin hae-\tbr{r} bian=ee\\
							{\iin} leg-\tsc{3psr;\tbr{pl}.psm} other={\ee}\\
				\glt	`(her/his other legs)' \txrf{}}\label{ex:*hiin haer bian=ee}
\end{exe}

An example from a text with a singular referent
possessing a plural possessum which in turn
possesses a singular referent is given in \qf{ex:RO-170822-3, 2.12} below.

\begin{exe}
	\ex{\gll	hiin maat-\tbr{r}=ini poun-\tbr{n}=ii msaʔ mutiʔ okeʔ \\
						{\iin} {eye-\tsc{3psr;pl.psm}=\ein} {hair-\tsc{3psr;sg.psm}=\ii} also white all \\
			\glt	`The hair of the eyes of it (that tribe) are also all white.'
						\txrf{RO-170822-3, 2.12}{\emb{RO-170822-3-02-12.mp3}{\spk{}}{\apl}}}\label{ex:RO-170822-3, 2.12}
\end{exe}

That the use of \ve{-r} to index plural possessums
is not exclusive to entities which naturally come in
groups (such as eyes, legs etc.), is shown in \qf{ex:RO-170917-1, 8.06-8.11} below
in which both \ve{fetoʔ} `man's sister' and \ve{tuuhaʔo} `brother'
each occur with the suffix \ve{-r} despite having a singular possessor.

\begin{exe}
	\ex{\begin{xlist}
		\ex{\gll	hiin feot-\tbr{r}=iin ma--, manaʔ, feot-n=ii meseʔ\\
							{\iin} mS-\tsc{\tbr{3psr;pl.psm}}={\ein} {} {\manaq} mS-\tsc{3psr;sg.psm}={\ii} one \\
				\glt	`He had sisters, umm, one sister.'\\
							\emph{lit.} `His sisters were, (his) sister was one.'
							\txrf{RO-170917-1, 8.06}{\emb{RO-170917-1-08-06-08-11.mp3}{\spk{}}{\apl}}}
		\ex{\gll	enai tuuhao-\tbr{r}=ii nua.\\
							then brother-\tsc{\tbr{3psr;pl.psm}}={\ii} two\\
				\glt	`then (he had) two brothers.' \txrf{8.11}}
	\end{xlist}}\label{ex:RO-170917-1, 8.06-8.11}
\end{exe}

The use of the variant first person inclusive genitive suffixes
\ve{-r} and \ve{-k} is probably also connected with the number of the
thing possessed, but in this case \ve{-k} can occur with both
plural and singular possessums, as shown in \qf{ex:RO-170829-1, 0.51}
and \qf{ex:obs. 01/09/17, p.29} below.\footnote{
		Regarding example \qf{ex:RO-170829-1, 0.51},
		Ro{\Q}is \ve{anaʔ} `child' takes genitive suffixes,
		unlike Kotos \ve{anah} `child'.}

\begin{exe}
	\ex{\gll	ahh baap Melianus n-oka, hiit ana-\tbr{k}, hiit oriʔ, naiʔ Oen\\
						{} father Melianus {\n-\ok} {\hiit} child-\tsc{\tbr{1pi.psr}} {\hiit} ySi {\naiq} Owen\\
			\glt	`Melianus with our son, our brother, Owen.'
						\txrf{RO-170829-1, 0.51}{\emb{RO-170829-1-00-51.mp3}{\spk{}}{\apl}}}\label{ex:RO-170829-1, 0.51}
	\ex{\gll	hiit t-kius, hiit maat-\tbr{k}=iin na-mtau\\
						{\hiit} \t-see {\hiit} {eye-\tsc{\tbr{1pi.psr}}=\ein} \na-scare\\
			\glt	`(If) we see (it), our eyes are scared.'
						\txrf{observation 01/09/17, p.29}}\label{ex:obs. 01/09/17, p.29}
\end{exe}

When the possessor is first person plural inclusive,
and the possessum is plural, the suffix \ve{-r} occurs
as illustrated in \qf{ex:elicit. 31/08/17, p.28}.
This suffix also usually occurs on the citation form of
body parts, as illustrated in \qf{ex:elicit. 170819-1, 2.51}.
In such examples the referent of the possessum may be singular
with the pronoun \ve{hiit} functioning in a generic sense,
though I have not yet tested whether \ve{-r} can occur
with an unambiguously singular possessum and first person plural inclusive possessor.\footnote{
		The suffix \ve{-r} cannot occur with \tsc{1sg} possessors.
		Thus, \ve{*au maat-r=iin} {\au} eye-\tsc{1pi.psr}={\ein} `my eyes' is ungrammatical.
		The correct phrase is \ve{au maat-k=iin} {\au} eye-\tsc{1sg.psr}={\ein} `my eyes'.}

\begin{exe}
	\ex{\gll	hiit maat-r=iin na-meen\\
						{\hiit} {eye-\tsc{1pi.psr}=\ein} \na-hurt\\
			\glt	`Our eyes hurt.'
						\txrf{elicit. 31/08/17, p.28}}\label{ex:elicit. 31/08/17, p.28}
	\ex{\gll	hiit \sf{hiduŋ} ehh hiit paan-r=aa \\
						{\hiit} nose {} {\hiit} nose-\tsc{1pi.psr}={\aa} \\
			\glt	`Our noses is: our noses/someone's nose'
						\txrf{elicit. 170819-1, 2.51}{\emb{RO-170819-1-02-51.mp3}{\spk{}}{\apl}}}\label{ex:elicit. 170819-1, 2.51}
\end{exe}

\subsection{Transitive suffixes}\label{sec:TraSuf}
Amarasi has two productive transitive suffixes, \ve{-ʔ} and \ve{-b}.
Of these the suffix \ve{-b} is highly productive,
while \ve{-ʔ} is slightly less productive.
Examples of \ve{-b} are given in \qf{ex:TraSuf-b}
and examples of \ve{-ʔ} in \qf{ex:TraSuf-q} below.
Neither of these suffixes is attested attached to consonant-final roots/stems.

\begin{exe}
	\ex{Transitive suffix \ve{-b}}\label{ex:TraSuf-b}
	\sn{\gw\begin{tabular}{llcccll}
		`ascends'		&\ve{n-sae}		&+&\ve{-b}&{\ra}& \ve{na-sae-b} & `raises sth.'\\
		`sits'			&\ve{n-took}	&+&\ve{-b}&{\ra}& \ve{na-toko-b} & `makes sit'\\
		`name'			&\ve{kana-f}	&+&\ve{-b}&{\ra}& \ve{na-kana-b} & `names s.o.'\\
		`remembers'	&\ve{na-mnau}	&+&\ve{-b}&{\ra}& \ve{na-mnau-b} & `reminds s.o.'\\
		`stops'			&\ve{na-snaas}&+&\ve{-b}&{\ra}& \ve{na-snasa-b} & `stops s.o.'\\
		`goes'			&\ve{n-nao}		&+&\ve{-b}&{\ra}& \ve{na-nao-b} & `makes s.o. go'\\
		\end{tabular}}
\end{exe}

\begin{exe}
	\ex{Transitive suffix \ve{-ʔ}}\label{ex:TraSuf-q}
	\sn{\gw\begin{tabular}{llcccll}
		`good'		&\ve{reko}		&+&\ve{-ʔ}&{\ra}& \ve{na-reko-ʔ} & `fixes'\\
		`stands'	&\ve{n-haek}	&+&\ve{-ʔ}&{\ra}& \ve{na-hake-ʔ} & `establishes'\\
		`returns'	&\ve{n-fain}	&+&\ve{-ʔ}&{\ra}& \ve{na-fani-ʔ} & `returns to s.o.,\\ \hhline{~}
							&\ve{n-fain}	&+&\ve{-ʔ}&{\ra}& \ve{na-fani-ʔ} & \hp{`}repeats sth.'\\
		\end{tabular}}
\end{exe}

I have also collected two intransitive verbs which
have a transitive counterpart which ends in a final /s/.
These are \ve{na-mtau} `scared' with \ve{na-mtaus} `scared of'
and \ve{n-mani} `laugh' with \ve{n-manis} `laugh at'.\footnote{
		Somewhat unusually, transitive \ve{n-manis} `laugh at'
		does not take vocalic prefixes. (\srf{sec:VerAgrPre}).}

In the case of \ve{na-mtaus} `scared of' the final consonant
may be a retention of the original final *t of Proto-Malayo-Polynesian *takut,
with application of the rule realising suffixal \ve{-t} as \ve{-s}
after roots which contain a /t/ (\srf{sec:Nom-t}).
However, such an explanation is not posible for \ve{n-manis} `laugh at',
from Proto-Central-Eastern Malayo-Polynesian *malip \citep{bltr}.

\subsection{Nominalising \it{-t}}\label{sec:Nom-t}
The suffix \ve{-t} is a nominaliser which derives nouns from verbs.
The nouns derived refer to the activity of the verb or the results of this activity.
The suffix \ve{-t} has the allomorph \ve{-s} after stems which contain a /t/,
and is related to the suffixal element of
the nominalising circumfix \ve{a-{\ldots}-t} (\srf{sec:NomA--t}).
This suffix has not yet been clearly attested on consonant-final roots.
Examples of \ve{-t} and its allomorph \ve{-s} are given in \qf{ex:NomCSuf-...t} below.\footnote{
		The word pair \ve{n-mena} `sick' and \ve{menas} `sickness' appear to represent
		a root which irregularly takes the allomorph \ve{-s} despite the lack of any /t/.
		However, a comparison with cognate forms in related languages,
		such as Tetun \it{moras} `to be sick, to be in poor health' \citep[143]{mo84}
		-- ultimately both reflexes of Proto-Malayo-Polynesian *ma-hapəjəs --
		reveals that the Amarasi root is actually the consonant-final nominal \ve{{\rt}menas} `sickness',
		from which the verb \ve{n-mena} `sick' is derived via the regular process
		of root final consonant deletion (\srf{sec:BasVerDer}).}

\newpage
\begin{exe}
	\ex{Nominalising suffix \ve{-t}}\label{ex:NomCSuf-...t}
	\sn{\gw\begin{tabular}{lllll}
		`speak poetically'&\ve{{\rt}ʔaʔa}	&+ \ve{-t} \ra& \ve{ʔaʔa-t} & `poetry'\\
		`do, make'				&\ve{{\rt}moʔe}	&+ \ve{-t} \ra& \ve{moʔe-t}	& `deed, act'\\
		`live'						&\ve{{\rt}moni}		&+ \ve{-t} \ra& \ve{moni-t} 	& `life'\\
		`believe'					&\ve{{\rt}pirsai}	&+ \ve{-t} \ra& \ve{pirsai-t} & `belief, religion'\\
		`speak foreign		&\ve{{\rt}rabi}		&+ \ve{-t} \ra& \ve{rabi-t} & `foreign\\ \hhline{~}
		\hp{`}language'		&									&							& 						& \hp{`}language'\\
		`sing'						&\ve{{\rt}sii}		&+ \ve{-t} \ra& \ve{sii-t} 	& `song'\\
		`die'							&\ve{{\rt}mate}		&+ \ve{-s} \ra& \ve{mate-s} & `death'\\
		`stand upright'		&\ve{{\rt}tetu}		&+ \ve{-s} \ra& \ve{tetu-s} & `blessing'\\
		`ask'							&\ve{{\rt}toti}		&+ \ve{-s} \ra& \ve{toti-s} & `request'\\
		`marry'						&\ve{{\rt}matsao}	&+ \ve{-s} \ra& \ve{matsao-s} & `marriage'\\
		\end{tabular}}
\end{exe}

\subsection{People group suffix \it{-s}}\label{sec:PeoGroSuf}
The suffix \ve{-s} forms nouns referring to people groups.
After CVC{\#} final stems this suffix replaces the final consonant,
while after VVC{\#} final stems this suffix has the allomorph \ve{-as}.
These allomorphs mean that the final foot
of the derived people group noun fills the CVCVC foot structure (\srf{sec:TheFoo}),
with examples such as \ve{Naet-as} `person from \ve{Naet}'
having the initial vowel sequence assigned to a single V-slot (\srf{sec:SurVVCVWor}).
Examples of \ve{-s} are given in \qf{ex:PeoGroSuf} below.

\begin{exe}
	\ex{People group suffix \ve{-s}}\label{ex:PeoGroSuf}
	\sn{\gw\stl{0.35em}\begin{tabular}{lllll}
		`Sabu'				&\ve{Sapu}			&+ \ve{-s}\hp{a}\ra&\ve{Sapu-s} 		& `person from Sabu'\\
		`Rote'				&\ve{Rote}			&+ \ve{-s}\hp{a}\ra&\ve{Rote-s} 		& `person from Rote'\\
		`Koro{\Q}oto'	&\ve{Koorʔoto}	&+ \ve{-s}\hp{a}\ra&\ve{Koorʔoto-s}	& `person from Koro{\Q}oto'\\
		`Belu'				&\ve{Beru}			&+ \ve{-s}\hp{a}\ra&\ve{Beru-s}			& `person from Belu'\\
		`Kupang'			&\ve{Kopan} 		&+ \ve{-s}\hp{a}\ra&\ve{Kopa-s}			& `person from Kupang'\\
		`Helong'			&\ve{ʔHeroʔ}		&+ \ve{-s}\hp{a}\ra&\ve{ʔHero-s}		& `Helong person'\\
		`Buraen'			&\ve{Buraen} 		&+ \ve{-as} \ra&\ve{Buraen-as}	& `person from Buraen'\\
		`Naet'				&\ve{Naet} 			&+ \ve{-as} \ra&\ve{Naet-as}		& `person from Naet'\\
		`east'				&\ve{neon sae-t}&+ \ve{-as} \ra&\ve{neon sae-t-as} & `easterner'\footnotemark\\
		\end{tabular}}
\end{exe}
\footnotetext{
		The form \ve{neon sae-t-as} `easterner' specifically
		refers to someone from the north-eastern Meto
		speaking areas; Oecusse (Baikeno), Miomafo, Insana, and Biboki.}

\subsection{The suffix \it{-aʔ}}\label{sec:Suf-aq}
VVC{\#} final verbs appear to have two forms,
one ending in VVC and one ending in VVCaʔ{\#}.
The forms ending in /aʔ/ do not occur before enclitics,
but other than this environment both forms appear
to be in free variation with one another
with no difference in meaning currently apparent.
Examples are given in \qf{ex:VVC/VVCaqAlt} below.

\begin{exe}
	\ex{VVC{\#} {\tl} VVCaʔ{\#} alternation}\label{ex:VVC/VVCaqAlt}
	\sn{\gw\begin{tabular}{rcll}
		\ve{na-baen}	&\tl& \ve{na-baenaʔ} & `pays'\\
		\ve{na-kain}	&\tl& \ve{na-kainaʔ} & `rebukes'\\
		\ve{na-ʔuab} &\tl& \ve{na-ʔuabaʔ} & `speaks'\\
		\ve{na-maik} 	&\tl& \ve{na-maikaʔ} & `stay, remain behind'\\
		\ve{na-tuin}	&\tl& \ve{na-tuinaʔ} & `follows'\\
	\end{tabular}}
\end{exe}

This also includes stems which are VVC{\#} final due
to the addition of a consonantal suffix to a VV{\#} final root.
Examples are given in \qf{ex:VV-C/VV-CaqAlt} below with the transitive suffix \ve{-b}.

\begin{exe}
	\ex{VV-C{\#} {\tl} VV-Caʔ{\#} alternation}\label{ex:VV-C/VV-CaqAlt}
	\sn{\gw\stl{0.4em}\begin{tabular}{llllll}
		\ve{{\rt}hae}	&+ \ve{-b} \ra&\ve{na-hae-b}	&\tl& \ve{na-hae-baʔ} & `tires s.o. out'\\
		\ve{{\rt}mnau}&+ \ve{-b} \ra&\ve{na-mnau-b}	&\tl& \ve{na-mnau-baʔ}& `reminds'\\
		\ve{{\rt}sae}	&+ \ve{-b} \ra&\ve{na-sae-b}	&\tl& \ve{na-sae-baʔ} & `raises, picks up'\\
		\ve{{\rt}tea}	&+ \ve{-b} \ra&\ve{na-tea-b}	&\tl& \ve{na-tea-baʔ} & `makes s.o. arrive'\\
%		\ve{{\rt}}&+&\ve{}&\ra&\ve{na-}	&\tl& \ve{na-} & `'\\
	\end{tabular}}
\end{exe}

The reason for this alternation is currently unknown.
One hypothesis I considered was that
this alternation was comparable to the alternation between U\=/forms and M\=/forms.
However, the forms ending in /aʔ/ occur in many environments where
a U\=/form is unexpected, such as in simple declarative sentences (Chapter \ref{ch:DisMet}).

One possibility is that final /aʔ/ occurs in order
to provide such forms a complete foot with no empty medial C-slots.
As discussed in \srf{sec:SurVVCVWor}, words which surface as VVCVC{\#},
are best analysed as being assigned a CVCVC foot with the initial
two vowels being assigned to a single V-slot, as illustrated for \ve{na-maikaʔ}
`stay, remain behind' in \qf{as:maikaq} below.
Forms without a final /aʔ/ on the other hand would have an empty medial C-slot,
as illustrated in \qf{as:maik} below.

\newpage
\begin{multicols}{2}
\begin{exe}
	\exa{\xy
		<3em,4cm>*\as{PrWd}="PrWd1",<4em,3cm>*\as{Ft}="ft1",
		<1em,2cm>*\as{σ}="s3",<3em,2cm>*\as{σ}="s1",<5em,2cm>*\as{σ}="s2",
		<0em,1cm>*\as{C}="cp1",<1em,1cm>*\as{V}="vp1",<2em,1cm>*\as{C}="c1",<3em,1cm>*\as{V}="v1",<4em,1cm>*\as{C}="c2",<5em,1cm>*\as{V}="v2",<6em,1cm>*\as{C}="c3",
		<0em,0cm>*\as{n}="n1",<1em,0cm>*\as{a}="a3",<2em,0cm>*\as{m}="k",<2.75em,0cm>*\as{a}="a1",<3.25em,0cm>*\as{i}="u",<4em,0cm>*\as{k}="n",<5em,0cm>*\as{a}="a2",<6em,0cm>*\as{ʔ}="q",
		"k"+U;"c1"+D**\dir{-};"a1"+U;"v1"+D**\dir{-};"u"+U;"v1"+D**\dir{-};"n"+U;"c2"+D**\dir{-};"a2"+U;"v2"+D**\dir{-};"q"+U;"c3"+D**\dir{-};
		"a3"+U;"vp1"+D**\dir{-};"n1"+U;"cp1"+D**\dir{-};
		"cp1"+U;"s3"+D**\dir{-};"c1"+U;"s3"+D**\dir{-};"c1"+U;"s1"+D**\dir{-};"c2"+U;"s1"+D**\dir{-};"v1"+U;"s1"+D**\dir{-};
		"vp1"+U;"s3"+D**\dir{-};"c2"+U;"s2"+D**\dir{-};"c3"+U;"s2"+D**\dir{-};"v2"+U;"s2"+D**\dir{-};
		"s1"+U;"ft1"+D**\dir{-};"s2"+U;"ft1"+D**\dir{-};"ft1"+U;"PrWd1"+D**\dir{-};"s3"+U;"PrWd1"+D**\dir{-};
	\endxy}\label{as:maikaq}
	\exa{\xy
		<3em,4cm>*\as{PrWd}="PrWd1",<4em,3cm>*\as{Ft}="ft1",
		<1em,2cm>*\as{σ}="s3",<3em,2cm>*\as{σ}="s1",<5em,2cm>*\as{σ}="s2",
		<0em,1cm>*\as{C}="cp1",<1em,1cm>*\as{V}="vp1",<2em,1cm>*\as{C}="c1",<3em,1cm>*\as{V}="v1",<4em,1cm>*\as{C}="c2",<5em,1cm>*\as{V}="v2",<6em,1cm>*\as{C}="c3",
		<0em,0cm>*\as{n}="n1",<1em,0cm>*\as{a}="a3",<2em,0cm>*\as{m}="k",<3em,0cm>*\as{a}="a1",<4em,0cm>*\as{ }="n",<5em,0cm>*\as{i}="a2",<6em,0cm>*\as{k}="q",
		"k"+U;"c1"+D**\dir{-};"a1"+U;"v1"+D**\dir{-};"a2"+U;"v2"+D**\dir{-};"q"+U;"c3"+D**\dir{-};
		"a3"+U;"vp1"+D**\dir{-};"n1"+U;"cp1"+D**\dir{-};
		"cp1"+U;"s3"+D**\dir{-};"c1"+U;"s3"+D**\dir{-};"c1"+U;"s1"+D**\dir{-};"c2"+U;"s1"+D**\dir{-};"v1"+U;"s1"+D**\dir{-};
		"vp1"+U;"s3"+D**\dir{-};"c2"+U;"s2"+D**\dir{-};"c3"+U;"s2"+D**\dir{-};"v2"+U;"s2"+D**\dir{-};
		"s1"+U;"ft1"+D**\dir{-};"s2"+U;"ft1"+D**\dir{-};"ft1"+U;"PrWd1"+D**\dir{-};"s3"+U;"PrWd1"+D**\dir{-};
	\endxy}\label{as:maik}
\end{exe}
\end{multicols}

If it is the case that the suffix /aʔ/ occurs to provide
words with a complete foot with only filled C-slots,
then the segmental material of /aʔ/ is expected.
The vowel /a/ is the default vowel in Amarasi (\srf{sec:VowFre}, \srf{sec:Epe})
and /ʔ/ is the default consonant (\srf{sec:GloStoIns}).

%na-keo-na'
%na-pau-na'
%na-poi-na'
%na-kaena'
%saena'
%toena'
%sakoina'
%na'bauna'
\section{Data for this work}\label{sec:Meth}
The core of the Amarasi data on which this work is based
is a corpus of recorded texts totalling nearly nineteen hours
of which about five hours has been processed.
This includes a little more than three hours of
transcribed, translated, and glossed Kotos texts,
as well as just over two hours of transcribed and translated Ro{\Q}is texts.
These texts are of a variety of genres and include narratives, folk-tales,
conversations, and traditional poetry.

An index of the texts which comprise this corpus is given in Appendix \ref{app:TexInd}.
These texts are archived with the
Pacific And Regional Archive for Digital Sources in Endangered Cultures (PARADISEC)
and nearly all are freely downloadable.

My Kotos texts were collected in three
field trips totalling seven months I made
in 2013, 2014, and 2016 over the course of my PhD work.
During these field trips I was hosted in Timor by
Heronimus Bani (Roni), a native speaker of Amarasi, in the village of Nekmese{\Q}.
These texts were recorded either by me or by Roni
and then transcribed and translated by native speakers of Amarasi, either Roni or Yedida Ora (Oma).
I then checked the initial transcriptions against the recording and glossed the text in Toolbox.
All my Kotos Amarasi texts can be accessed
from \url{http://catalog.paradisec.org.au/collections/OE1}.

During 2012 I was a participant in a two week
language documentation workshop held in Kupang:
\emph{Preserving Knowledge through Recording and Writing Local Languages}.
During this workshop a number of additional Kotos Amarasi
texts were recorded and transcribed by Oma.

My Ro{\Q}is texts were collected during a
field trip at the end of 2018
while undertaking an Australia Awards Endeavour Fellowship.
During this trip I spent one week in Buraen
with Toni Buraen and his family, followed by
two weeks in Tunbaun with Melianus Obhetan and his family.
I transcribed my Ro{\Q}is Amarasi texts
and then checked them with native speakers.
My Ro{\Q}is Amarasi texts can be downloaded from
\url{http://catalog.paradisec.org.au/collections/OE2}.

In addition to this text corpus, I also conducted a number of
elicitation sessions with Roni in 2016.
This elicitation involved working through recorded texts with Roni
and manipulating individual parts of sentences for grammaticality judgements.
When a manipulated sentence was accepted as grammatical,
I would then have Roni say it back to me.
This often resulted in him rejecting a sentence he had originally accepted.
Elicitation was also carried out with Oma on a number of occasions.

This data is supplemented by a translation
of the New Testament and Genesis into Kotos Amarasi: \citet{UBB15}.\footnote{
		This translation can be accessed online at \url{www.e-alkitab.org}
		or downloaded for free on Android devices from Google Playstore (search: Amarasi Bible).}
This translation was carried out by native Amarasi speakers
and is completely natural and idiomatic Amarasi as evidenced
by the fact that it is full of grammatical constructions that differ
from both Indonesian and Kupang Malay (used as front translation).
Before publication this translation was checked with at least three different
groups of native speakers comprising three or more speakers in each group
(representing a good cross section of age, gender, and educational levels) for clarity and naturalness.
The material was tested and further refined with each successive group,
then followed by a smoothing read-through looking at naturalness and flow before publishing.

Data from this translation is presented
when it contains good, clear exemplars of rare constructions.
However, no part of my analysis rests
solely on data found only in the Amarasi Bible translation.
See \cite{hehawi11} and \citet[2]{dryer13} for discussion of
the use of Bible translations as sources of linguistic data.

A final source of Kotos Amarasi data is a series
of primary school readers translated from Kupang Malay
into Amarasi by Yedida Ora \citep{or16,or16b,or16c}.
These readers have also been checked and edited for naturalness and fluency.

In addition to all this Amarasi data, I also have also
collected data on the following varieties of Meto,
some of which appears at various points in this book:
Timaus (half an hour of transcribed, translated, and glossed texts,
as well as 1 hour 15 minutes untranscribed texts, lexicon of 685 headwords),
Kusa-Manea (four hours of untranscribed texts, lexicon of 488 headwords),
Amanuban (22 untranscribed texts, 8 wordlists),
Ketun (3 untranscribed texts, 3 wordlists),
Kopas (3 untranscribed texts, 5 wordlists),
Fatule{\Q}u (2 wordlists), and
Amfo{\Q}an (1 wordlist).
I also have Baikeno data collected during
the 2012 Kupang language documentation workshop,
as well as data collected and provided by Charles Grimes.
Unless otherwise cited, all Meto data in this book comes
from these sources.
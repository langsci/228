\subsubsection{/\j/ in native vocabulary}\label{sec:PhoJNatVoc}
An additional piece of evidence for empty C-slots comes
from the distribution of /\j/ in native Amarasi roots.
There are currently five words in my dictionary of 1,972 unique lexical roots
which are not obviously loans which contain /\j/.
In each instance /\j/ occurs after a front vowel.
These words are given in \qf{ex2:PosNatAttJ} below.

In addition to these five words the proximal demonstrative
\ve{ia} has the variant form \ve{i\j a}.
(There are twelve attestations of \ve{i\j a} in my corpus
against 267 attestations of \ve{ia}.),
and Ro{\Q}is Amarasi from Tunbaun has \ve{poe\j isaʔ} `centipede'
(Buraen has \ve{poeʔisiʔ} and Kotos Amarasi has \ve{ʔbeebnisaʔ} for `centipede'). 

\begin{exe}
	\ex{Attestations of native /\j/:}\label{ex2:PosNatAttJ}
	\sn{\gw\begin{tabular}{llll}
		\ve{ai{\j}oʔo}	&[ʔaj{\j}ˈɔʔɔ]&{\emb{aijoqo.mp3}{\spk{}}{\apl}}	& `casuarina tree' \\
		\ve{ai{\j}onuus}	&[ʔaj{\j}ɔ̝ˈnʊːs]&{\emb{aijonuus.mp3}{\spk{}}{\apl}}	& `k.o. herb' \\
		\ve{bi{\j}ae}	&[biˈ{\j}aɛ]&{\emb{bijae.mp3}{\spk{}}{\apl}}	& `cow' \\
		\ve{nai{\j}eer}	&[najˈ{\j}ɛːr]&{\emb{naijeer.mp3}{\spk{}}{\apl}}	& `ginger' \\
		\ve{tai{\j}onif}	&[tajˈ{\j}onif]&{\emb{taijonif.mp3}{\spk{}}{\apl}}	& `jackfruit' \\
	\end{tabular}}
\end{exe}

Of these, the word \ve{ai{\j}onuus} `cummin' is historically
a compound of \ve{ai{\j}oʔo} `iron-wood tree' and \ve{nuus}
which has no independent meaning in Amarasi.
(In Fatule{\Q}u \ve{nuus} is attested with the meaning `blue'.)

If the /\j/ were removed from the words in \qf{ex2:PosNatAttJ},
we would find a sequence of three or more vowels in each instance.
Given that sequences of more than two vowels are not found in Amarasi ({\S}\ref{sec:VowSeq}),
it is possible to analyse /\j/ in the examples in \qf{ex2:PosNatAttJ} as epenthetic,
occurring to break up the disallowed underlying trivocalic sequence.

Under this analysis, the place features of the vowel /i/
would spread rightwards to fill an adjacent empty C-slot.
%We could motivate the spread of \p{/i/} rather than another vowel
%by observing that the vowel \p{/a/} is essentially featureless ({\S}\ref{sec:Vow}, \srf{sec:AssFinA})
%and that \p{/i/} is the first vowel in the putative sequence with place features.
The way in which this analysis would work is shown for \ve{bi\j ae} in \qf{as:bijae} below.
In \qf{as:bijae1} we have an illicit sequence of three vowels.
The place feature \tsc{[+front]} of the vowel /i/
spreads in \qf{as:bijae2} to break up the VVV sequence,
thus producing the consonant /\j/ in \qf{as:bijae3}.

\begin{multicols}{3}
	\begin{exe}
		\ex{\begin{xlist}
			\exa{\xy
				<0.8em,2cm>*\as{C}="CV1",<1.6em,2cm>*\as{V}="CV2",<2.4em,2cm>*\as{C}="CV3",<3.2em,2cm>*\as{V}="CV4",<4em,2cm>*\as{C}="CV5",<4.8em,2cm>*\as{V}="CV6",<5.6em,2cm>*\as{C}="CV7",
				<0.8em,1cm>*\as{b}="cv1",<1.6em,1cm>*\as{i}="cv2",<2.4em,1cm>*\as{ }="cv3",<3.2em,1cm>*\as{a}="cv4",<4em,1cm>*\as{ }="cv5",<4.8em,1cm>*\as{e}="cv6",<5.6em,1cm>*\as{ }="cv7",
				<1.6em,0pt>*\as{\tsc{[+fr.]}}="f","f"+U;"cv2"+D**\dir{-};
				"cv1"+U;"CV1"+D**\dir{-};"cv2"+U;"CV2"+D**\dir{-};"cv3"+U;"CV3"+D**\dir{};"cv4"+U;"CV4"+D**\dir{-};"cv5"+U;"CV5"+D**\dir{};"cv6"+U;"CV6"+D**\dir{-};"cv7"+U;"CV7"+D**\dir{};
				<1.6em,1.5cm>*\as{\tikz[red,thick,dashed,baseline=0.9ex]\draw (0,0) rectangle (0.3cm,1.5cm);}="box",
			\endxy}\label{as:bijae1}
			\exa{\xy
				<0.8em,2cm>*\as{C}="CV1",<1.6em,2cm>*\as{V}="CV2",<2.4em,2cm>*\as{C}="CV3",<3.2em,2cm>*\as{V}="CV4",<4em,2cm>*\as{C}="CV5",<4.8em,2cm>*\as{V}="CV6",<5.6em,2cm>*\as{C}="CV7",
				<0.8em,1cm>*\as{b}="cv1",<1.6em,1cm>*\as{i}="cv2",<2.4em,1cm>*\as{ }="cv3",<3.2em,1cm>*\as{a}="cv4",<4em,1cm>*\as{ }="cv5",<4.8em,1cm>*\as{e}="cv6",<5.6em,1cm>*\as{ }="cv7",
				<1.6em,0pt>*\as{\tsc{[+fr.]}}="f","f"+U;"cv2"+D**\dir{-};"f"+U;"cv3"+D**\dir{.};"cv3"+D;"CV3"+D**\dir{.};
				"cv1"+U;"CV1"+D**\dir{-};"cv2"+U;"CV2"+D**\dir{-};"cv3"+U;"CV3"+D**\dir{};"cv4"+U;"CV4"+D**\dir{-};"cv5"+U;"CV5"+D**\dir{};"cv6"+U;"CV6"+D**\dir{-};"cv7"+U;"CV7"+D**\dir{};
				<2em,1.5cm>*\as{\tikz[red,thick,dashed,baseline=0.9ex]\draw (0,0) rectangle (0.65cm,1.5cm);}="box",
			\endxy}\label{as:bijae2}
			\exa{\xy
				<0.8em,2cm>*\as{C}="CV1",<1.6em,2cm>*\as{V}="CV2",<2.4em,2cm>*\as{C}="CV3",<3.2em,2cm>*\as{V}="CV4",<4em,2cm>*\as{C}="CV5",<4.8em,2cm>*\as{V}="CV6",<5.6em,2cm>*\as{C}="CV7",
				<0.8em,1cm>*\as{b}="cv1",<1.6em,1cm>*\as{i}="cv2",<2.4em,1cm>*\as{\j}="cv3",<3.2em,1cm>*\as{a}="cv4",<4em,1cm>*\as{ }="cv5",<4.8em,1cm>*\as{e}="cv6",<5.6em,1cm>*\as{ }="cv7",
				<2em,0pt>*\as{\tsc{[+fr.]}}="f","f"+U;"cv2"+D**\dir{-};"f"+U;"cv3"+D**\dir{-};
				"cv1"+U;"CV1"+D**\dir{-};"cv2"+U;"CV2"+D**\dir{-};"cv3"+U;"CV3"+D**\dir{-};"cv4"+U;"CV4"+D**\dir{-};"cv5"+U;"CV5"+D**\dir{};"cv6"+U;"CV6"+D**\dir{-};"cv7"+U;"CV7"+D**\dir{};
				<1.6em,1.5cm>*\as{\tikz[red,thick,dashed,baseline=0.9ex]\draw (0,0) rectangle (0.3cm,1.5cm);}="box",
			\endxy}\label{as:bijae3}
		\end{xlist}}\label{as:bijae}
	\end{exe}
\end{multicols}

Evidence that this process has operated historically
comes from cognates in other Meto varieties.
Thus, Amanuban has \ve{bia} or \ve{bie} `cow'
and Kusa-Manea has \ve{bea} or \ve{bia} `cow',
all without medial /\j/. 

While the /\j/ in the words in \qf{ex2:PosNatAttJ} is probably historically epenthetic,
arising through a process similar to that illustrated in \qf{as:bijae},
in the modern language /\j/ also occurs in other environments in recent loanwords
such as \ve{{\rt}\j ari} `to become' < Malay \it{jadi}
and \ve{\j eket} `jacket' < Malay \emph{jeket} < English \emph{jacket}.
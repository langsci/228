\subsection{The CVCVC foot}\label{sec:TheFoo}
One of the most important elements of word structure in Amarasi is the foot.
All lexical words in Amarasi contain at least one foot.
Amarasi has a syllabic foot structure in which a foot consists of two syllables.
The structure of the Amarasi foot is given in \qf{ex:Ft->CVCVC} below.

\begin{exe}
	\ex{Ft {\ra} CVCVC}\label{ex:Ft->CVCVC}
\end{exe}

In \qf{ex:Ft->CVCVC} `V' represents a V-slot
which is obligatorily filled in by one of the segmental vowels (\srf{sec:Vow}).
The letter \emph{C} represents a C-slot which is optionally
filled by one of the segmental consonants (\srf{sec:Con}).
Stress falls on the penultimate V-slot of the foot (\srf{sec:Str}).

Under my analysis, C-slots can be empty.
This means that a word such as \ve{muʔit} `animal'
has the same underlying structure as \ve{fatu} `stone', \ve{kaut} `papaya' or \ve{ai} `fire'.
In all cases these words map onto the same CVCVC foot structure.
Thus, \ve{fatu} `stone', \ve{kaut} `papaya'  and \ve{ai} `fire' have empty C-slots.
The underlying structures of these three words
are given in \qf{as:Ft:fatu}--\qf{as:Ft:ai} below
alongside \ve{muʔit} `animal' which has no empty C-slots.

\begin{multicols}{4}
	\begin{exe}
		\exa{\label{as:Ft:muqit}\xy
			<2.7em,3cm>*\as{Ft}="ft",
			<1.8em,2cm>*\as{σ}="s1",<3.6em,2cm>*\as{σ}="s2",
			<0.9em,1cm>*\as{C}="CV1",<1.8em,1cm>*\as{V}="CV2",<2.7em,1cm>*\as{C}="CV3",<3.6em,1cm>*\as{V}="CV4",<4.5em,1cm>*\as{C}="CV5",
			<0.9em,0cm>*\as{m}="cv1",<1.8em,0cm>*\as{u}="cv2",<2.7em,0cm>*\as{ʔ}="cv3",<3.6em,0cm>*\as{i}="cv4",<4.5em,0cm>*\as{t}="cv5",
			"cv1"+U;"CV1"+D**\dir{-};"cv2"+U;"CV2"+D**\dir{-};"cv3"+U;"CV3"+D**\dir{-};
			"cv4"+U;"CV4"+D**\dir{-};"cv5"+U;"CV5"+D**\dir{-};
			"CV1"+U;"s1"+D**\dir{-};"CV2"+U;"s1"+D**\dir{-};"CV3"+U;"s1"+D**\dir{-};
			"CV3"+U;"s2"+D**\dir{-};"CV4"+U;"s2"+D**\dir{-};"CV5"+U;"s2"+D**\dir{-};
			"s1"+U;"ft"+D**\dir{-};"s2"+U;"ft"+D**\dir{-};
		\endxy}
		\exa{\label{as:Ft:fatu}\xy
			<1.8em,3cm>*\as{Ft}="ft1",
			<0.9em,2cm>*\as{σ}="s1",<2.7em,2cm>*\as{σ}="s2",
			<0cm,1cm>*\as{C}="C1",<0.9em,1cm>*\as{V}="V1",<1.8em,1cm>*\as{C}="C2",<2.7em,1cm>*\as{V}="V2",<3.6em,1cm>*\as{C}="C3",
			<0em,0cm>*\as{f}="c1",<0.9em,0cm>*\as{a}="v1",<1.8em,0cm>*\as{t}="c2",<2.7em,0cm>*\as{u}="v2",
			"c1"+U;"C1"+D**\dir{-};"c2"+U;"C2"+D**\dir{-};"v1"+U;"V1"+D**\dir{-};"v2"+U;"V2"+D**\dir{-};
			"C1"+U;"s1"+D**\dir{-};"C2"+U;"s1"+D**\dir{-};"V1"+U;"s1"+D**\dir{-};
			"C2"+U;"s2"+D**\dir{-};"C3"+U;"s2"+D**\dir{-};"V2"+U;"s2"+D**\dir{-};
			"s1"+U;"ft1"+D**\dir{-};"s2"+U;"ft1"+D**\dir{-};
		\endxy}
		\exa{\xy
			<1.8em,3cm>*\as{Ft}="ft1",
			<0.9em,2cm>*\as{σ}="s1",<2.7em,2cm>*\as{σ}="s2",
			<0em,1cm>*\as{C}="C1",<0.9em,1cm>*\as{V}="V1",<1.8em,1cm>*\as{C}="C2",<2.7em,1cm>*\as{V}="V2",<3.6em,1cm>*\as{C}="C3",
			<0em,0cm>*\as{k}="c1",<0.9em,0cm>*\as{a}="v1",<2.7em,0cm>*\as{u}="v2",<3.6em,0cm>*\as{t}="c3",
			"c1"+U;"C1"+D**\dir{-};"c3"+U;"C3"+D**\dir{-};"v1"+U;"V1"+D**\dir{-};"v2"+U;"V2"+D**\dir{-};
			"C1"+U;"s1"+D**\dir{-};"C2"+U;"s1"+D**\dir{-};"V1"+U;"s1"+D**\dir{-};
			"C2"+U;"s2"+D**\dir{-};"C3"+U;"s2"+D**\dir{-};"V2"+U;"s2"+D**\dir{-};
			"s1"+U;"ft1"+D**\dir{-};"s2"+U;"ft1"+D**\dir{-};
		\endxy}
		\exa{\label{as:Ft:ai}\xy
			<1.8em,3cm>*\as{Ft}="ft1",
			<0.9em,2cm>*\as{σ}="s1",<2.7em,2cm>*\as{σ}="s2",
			<0em,1cm>*\as{C}="C1",<0.9em,1cm>*\as{V}="V1",<1.8em,1cm>*\as{C}="C2",<2.7em,1cm>*\as{V}="V2",<3.6em,1cm>*\as{C}="C3",
			<0.9em,0cm>*\as{a}="v1",<2.7em,0cm>*\as{i}="v2",
			"v1"+U;"V1"+D**\dir{-};"v2"+U;"V2"+D**\dir{-};
			"C1"+U;"s1"+D**\dir{-};"C2"+U;"s1"+D**\dir{-};"V1"+U;"s1"+D**\dir{-};
			"C2"+U;"s2"+D**\dir{-};"C3"+U;"s2"+D**\dir{-};"V2"+U;"s2"+D**\dir{-};
			"s1"+U;"ft1"+D**\dir{-};"s2"+U;"ft1"+D**\dir{-};
		\endxy}
	\end{exe}
\end{multicols}

Under certain conditions there are phonetic traces of actual consonants in these empty C-slots.
There are at least six morphological and/or phonological processes under
which phonetic traces of these empty C-slots can be identified.
These conditions are discussed in \srf{sec:EmpCSlo}.
In addition to these language-internal rules,
in other varieties of Meto there are
examples of actual consonants surfacing in environments
for which I posit empty C-slots in Amarasi.
This comparative data is also discussed in \srf{sec:EmpCSlo}.

As discussed in complete detail in Chapter \ref{ch:StrMetAma},
metathesis in Amarasi targets the final
CV sequence of a foot, usually with subsequent
deletion of the final C-slot.
This results in a derived foot structure with no medial C-slot.
I will refer to this derived foot structure as the M-foot (Ft\sub{\tsc{m}}).
The structures of the M\=/form of the words in \qf{as:Ft:muqit}--\qf{as:Ft:ai} above;
\ve{muʔit} {\ra} \ve{muiʔ} `animal', \ve{fatu} {\ra} \ve{faut} `stone',
\ve{kaut} {\ra} \ve{kau} `papaya', and \ve{ai} {\ra} \ve{ai} `fire'
are shown in \qf{as:Ftm:muiq}--\qf{as:Ftm:ai} below.

\begin{multicols}{4}
	\begin{exe}
		\exa{\label{as:Ftm:muiq}\xy
			<2.5em,3cm>*\as{\hp{\sub{\tsc{m}}}Ft\sub{\tsc{m}}}="ft1",
			<1.5em,2cm>*\as{σ}="s1",<3.5em,2cm>*\as{σ}="s2",
			<1em,1cm>*\as{C}="CV1",<2em,1cm>*\as{V}="CV2",<3em,1cm>*\as{V}="CV3",<4em,1cm>*\as{C}="CV4",
			<1em,0cm>*\as{m}="cv1",<2em,0cm>*\as{u}="cv2",<3em,0cm>*\as{i}="cv3",<4em,0cm>*\as{ʔ}="cv4",
			"cv1"+U;"CV1"+D**\dir{-};"cv2"+U;"CV2"+D**\dir{-};"cv3"+U;"CV3"+D**\dir{-};"cv4"+U;"CV4"+D**\dir{-};
			"CV1"+U;"s1"+D**\dir{-};"CV2"+U;"s1"+D**\dir{-};"CV3"+U;"s2"+D**\dir{-};"CV4"+U;"s2"+D**\dir{-};
			"s1"+U;"ft1"+D**\dir{-};"s2"+U;"ft1"+D**\dir{-};
		\endxy}
		\exa{\xy
			<2.5em,3cm>*\as{\hp{\sub{\tsc{m}}}Ft\sub{\tsc{m}}}="ft1",
			<1.5em,2cm>*\as{σ}="s1",<3.5em,2cm>*\as{σ}="s2",
			<1em,1cm>*\as{C}="CV1",<2em,1cm>*\as{V}="CV2",<3em,1cm>*\as{V}="CV3",<4em,1cm>*\as{C}="CV4",
			<1em,0cm>*\as{f}="cv1",<2em,0cm>*\as{a}="cv2",<3em,0cm>*\as{u}="cv3",<4em,0cm>*\as{t}="cv4",
			"cv1"+U;"CV1"+D**\dir{-};"cv2"+U;"CV2"+D**\dir{-};"cv3"+U;"CV3"+D**\dir{-};"cv4"+U;"CV4"+D**\dir{-};
			"CV1"+U;"s1"+D**\dir{-};"CV2"+U;"s1"+D**\dir{-};"CV3"+U;"s2"+D**\dir{-};"CV4"+U;"s2"+D**\dir{-};
			"s1"+U;"ft1"+D**\dir{-};"s2"+U;"ft1"+D**\dir{-};
		\endxy}
		\exa{\xy
			<2.5em,3cm>*\as{\hp{\sub{\tsc{m}}}Ft\sub{\tsc{m}}}="ft1",
			<1.5em,2cm>*\as{σ}="s1",<3.5em,2cm>*\as{σ}="s2",
			<1em,1cm>*\as{C}="CV1",<2em,1cm>*\as{V}="CV2",<3em,1cm>*\as{V}="CV3",<4em,1cm>*\as{C}="CV4",
			<1em,0cm>*\as{k}="cv1",<2em,0cm>*\as{a}="cv2",<3em,0cm>*\as{u}="cv3",<4em,0cm>*\as{}="cv4",
			"cv1"+U;"CV1"+D**\dir{-};"cv2"+U;"CV2"+D**\dir{-};"cv3"+U;"CV3"+D**\dir{-};
			"CV1"+U;"s1"+D**\dir{-};"CV2"+U;"s1"+D**\dir{-};"CV3"+U;"s2"+D**\dir{-};"CV4"+U;"s2"+D**\dir{-};
			"s1"+U;"ft1"+D**\dir{-};"s2"+U;"ft1"+D**\dir{-};
		\endxy}
		\exa{\label{as:Ftm:ai}\xy
			<2.5em,3cm>*\as{\hp{\sub{\tsc{m}}}Ft\sub{\tsc{m}}}="ft1",
			<1.5em,2cm>*\as{σ}="s1",<3.5em,2cm>*\as{σ}="s2",
			<1em,1cm>*\as{C}="CV1",<2em,1cm>*\as{V}="CV2",<3em,1cm>*\as{V}="CV3",<4em,1cm>*\as{C}="CV4",
			<1em,0cm>*\as{}="cv1",<2em,0cm>*\as{a}="cv2",<3em,0cm>*\as{i}="cv3",<4em,0cm>*\as{}="cv4",
			"cv2"+U;"CV2"+D**\dir{-};"cv3"+U;"CV3"+D**\dir{-};
			"CV1"+U;"s1"+D**\dir{-};"CV2"+U;"s1"+D**\dir{-};"CV3"+U;"s2"+D**\dir{-};"CV4"+U;"s2"+D**\dir{-};
			"s1"+U;"ft1"+D**\dir{-};"s2"+U;"ft1"+D**\dir{-};
		\endxy}
	\end{exe}
\end{multicols}

The only exceptions to this obligatory CVCVC foot
are vowel-initial enclitics which have a defective
onset-less VCVC foot. This structure is not problematic
as such enclitics always occur attached to a host,
the final C-slot of which supplies the onset C-slot for the enclitic.
The structures of three vowel-initial enclitics
metathesised and unmetathesised are given
in \qf{as:VowIniCl} and \qf{as:VowIniCl-M} below to illustrate.
These enlitics are \ve{=esa/=ees} `one',
\ve{=eni/=ein} `{\ein}' and \ve{=ee} `{\ee}/{\eeV}'.

\begin{multicols}{2}
	\begin{exe}
		\exa{\xy
			<2.25em,4cm>*\as{Ft}="Ft1",,
			<1.5em,3cm>*\as{σ}="s1",<3em,3cm>*\as{σ}="s2",
			<1em,2cm>*\as{V}="CV1",<2em,2cm>*\as{C}="CV2",<3em,2cm>*\as{V}="CV3",<4em,2cm>*\as{C}="CV4",
			<1em,1cm>*\as{e}="cv1",<2em,1cm>*\as{s}="cv2",<3em,1cm>*\as{a}="cv3",<4em,1cm>*\as{ }="cv4",
			<1em,0.5cm>*\as{e}="cv1.2",<2em,0.5cm>*\as{n}="cv2.2",<3em,0.5cm>*\as{i}="cv3.2",<4em,0.5cm>*\as{ }="cv4",
			<1em,0cm>*\as{e}="cv1.3",<2em,0cm>*\as{ }="cv2.3",<3em,0cm>*\as{e}="cv3.3",<4em,0cm>*\as{ }="cv4",
			"cv1"+U;"CV1"+D**\dir{-};"cv2"+U;"CV2"+D**\dir{-};"cv3"+U;"CV3"+D**\dir{-};
			"CV1"+U;"s1"+D**\dir{-};"CV2"+U;"s1"+D**\dir{-};"CV2"+U;"s2"+D**\dir{-};"CV3"+U;"s2"+D**\dir{-};"CV4"+U;"s2"+D**\dir{-};
			"s1"+U;"Ft1"+D**\dir{-};"s2"+U;"Ft1"+D**\dir{-};
		\endxy}\label{as:VowIniCl}
		\exa{\xy
			<2em,4cm>*\as{\hp{\sub{\tsc{m}}}Ft\sub{\tsc{m}}}="Ft1",,
			<1.5em,3cm>*\as{σ}="s1",<2.5em,3cm>*\as{σ}="s2",
			<1em,2cm>*\as{V}="CV1",<2em,2cm>*\as{V}="CV2",<3em,2cm>*\as{C}="CV3",
			<1em,1cm>*\as{e}="cv1",<2em,1cm>*\as{e}="cv2",<3em,1cm>*\as{s}="cv3",
			<1em,0.5cm>*\as{e}="cv1.2",<2em,0.5cm>*\as{i}="cv2.2",<3em,0.5cm>*\as{n}="cv3.2",
			<1em,0cm>*\as{e}="cv1.3",<2em,0cm>*\as{e}="cv2.3",<3em,0cm>*\as{ }="cv3.3",
			"cv1"+U;"CV1"+D**\dir{-};"cv2"+U;"CV2"+D**\dir{-};"cv3"+U;"CV3"+D**\dir{-};
			"CV1"+U;"s1"+D**\dir{-};"CV2"+U;"s1"+D**\dir{-};"CV2"+U;"s2"+D**\dir{-};"CV3"+U;"s2"+D**\dir{-};
			"s1"+U;"Ft1"+D**\dir{-};"s2"+U;"Ft1"+D**\dir{-};
		\endxy}\label{as:VowIniCl-M}
	\end{exe}
\end{multicols}

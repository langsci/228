\section{Prosodic structures}\label{sec:ProsStr}
Three distinct units of Amarasi prosodic structure
can be identified: the syllable (\srf{sec:Syl}),
a disyllabic foot (\srf{sec:TheFoo}),
and a prosodic word which is the locus of stress placement (\srf{sec:PrWd}).

\subsection{The CVC syllable}\label{sec:Syl}
The Amarasi syllable consists of an onset C-slot, a nucleus V-slot
and a coda C-slot, thus σ {\ra} CVC.
Syllable weight plays no role in the languages.
That is, Amarasi is not a quantity sensitive language.

C-slots which occur between two V-slots are ambisyllabic
(\citealp[36]{clke83}, \citealp[217ff]{du90}).
Such a C-slot is both the coda of the preceding syllable
and the onset of the following syllable.
Note the analysis of intervocalic consonants as
ambisyllabic is a crucial part of my analysis of
metathesis before vowel-initial syllables (\srf{sec:Met ch:PhoMet}).
Independent evidence for this analysis comes from reduplication (\srf{sec:Red}).

This syllable structure is identical for all feet regardless
of the contents of each of the C-slots and V-slots.
Thus, each segmental vowel of a word is the nucleus of a unique syllable.
The syllabification of \ve{muʔit} `animal', \ve{fatu} `stone', \ve{kaut} `papaya' and \ve{ai} `fire'
is shown in \qf{as:Syl:muqit}--\qf{as:Syl:ai} below.
Extensive evidence for empty C-slots is given in \srf{sec:EmpCSlo}.

\begin{multicols}{4}
	\begin{exe}
		\exa{\label{as:Syl:muqit}\xy
			<1.8em,2cm>*\as{σ}="s1",<3.6em,2cm>*\as{σ}="s2",
			<0.9em,1cm>*\as{C}="CV1",<1.8em,1cm>*\as{V}="CV2",<2.7em,1cm>*\as{C}="CV3",<3.6em,1cm>*\as{V}="CV4",<4.5em,1cm>*\as{C}="CV5",
			<0.9em,0cm>*\as{m}="cv1",<1.8em,0cm>*\as{u}="cv2",<2.7em,0cm>*\as{ʔ}="cv3",<3.6em,0cm>*\as{i}="cv4",<4.5em,0cm>*\as{t}="cv5",
			"cv1"+U;"CV1"+D**\dir{-};"cv2"+U;"CV2"+D**\dir{-};"cv3"+U;"CV3"+D**\dir{-};
			"cv4"+U;"CV4"+D**\dir{-};"cv5"+U;"CV5"+D**\dir{-};
			"CV1"+U;"s1"+D**\dir{-};"CV2"+U;"s1"+D**\dir{-};"CV3"+U;"s1"+D**\dir{-};
			"CV3"+U;"s2"+D**\dir{-};"CV4"+U;"s2"+D**\dir{-};"CV5"+U;"s2"+D**\dir{-};
		\endxy}
		\exa{\label{as:Syl:fatu}\xy
			<0.9em,2cm>*\as{σ}="s1",<2.7em,2cm>*\as{σ}="s2",
			<0cm,1cm>*\as{C}="C1",<0.9em,1cm>*\as{V}="V1",<1.8em,1cm>*\as{C}="C2",<2.7em,1cm>*\as{V}="V2",<3.6em,1cm>*\as{C}="C3",
			<0em,0cm>*\as{f}="c1",<0.9em,0cm>*\as{a}="v1",<1.8em,0cm>*\as{t}="c2",<2.7em,0cm>*\as{u}="v2",
			"c1"+U;"C1"+D**\dir{-};"c2"+U;"C2"+D**\dir{-};"v1"+U;"V1"+D**\dir{-};"v2"+U;"V2"+D**\dir{-};
			"C1"+U;"s1"+D**\dir{-};"C2"+U;"s1"+D**\dir{-};"V1"+U;"s1"+D**\dir{-};
			"C2"+U;"s2"+D**\dir{-};"C3"+U;"s2"+D**\dir{-};"V2"+U;"s2"+D**\dir{-};
		\endxy}
		\exa{\xy
			<0.9em,2cm>*\as{σ}="s1",<2.7em,2cm>*\as{σ}="s2",
			<0em,1cm>*\as{C}="C1",<0.9em,1cm>*\as{V}="V1",<1.8em,1cm>*\as{C}="C2",<2.7em,1cm>*\as{V}="V2",<3.6em,1cm>*\as{C}="C3",
			<0em,0cm>*\as{k}="c1",<0.9em,0cm>*\as{a}="v1",<2.7em,0cm>*\as{u}="v2",<3.6em,0cm>*\as{t}="c3",
			"c1"+U;"C1"+D**\dir{-};"c3"+U;"C3"+D**\dir{-};"v1"+U;"V1"+D**\dir{-};"v2"+U;"V2"+D**\dir{-};
			"C1"+U;"s1"+D**\dir{-};"C2"+U;"s1"+D**\dir{-};"V1"+U;"s1"+D**\dir{-};
			"C2"+U;"s2"+D**\dir{-};"C3"+U;"s2"+D**\dir{-};"V2"+U;"s2"+D**\dir{-};
		\endxy}
		\exa{\label{as:Syl:ai}\xy
			<0.9em,2cm>*\as{σ}="s1",<2.7em,2cm>*\as{σ}="s2",
			<0em,1cm>*\as{C}="C1",<0.9em,1cm>*\as{V}="V1",<1.8em,1cm>*\as{C}="C2",<2.7em,1cm>*\as{V}="V2",<3.6em,1cm>*\as{C}="C3",
			<0.9em,0cm>*\as{a}="v1",<2.7em,0cm>*\as{i}="v2",
			"v1"+U;"V1"+D**\dir{-};"v2"+U;"V2"+D**\dir{-};
			"C1"+U;"s1"+D**\dir{-};"C2"+U;"s1"+D**\dir{-};"V1"+U;"s1"+D**\dir{-};
			"C2"+U;"s2"+D**\dir{-};"C3"+U;"s2"+D**\dir{-};"V2"+U;"s2"+D**\dir{-};
		\endxy}
	\end{exe}
\end{multicols}

The only case in which the syllable structure is not CVC
is in the derived CVVC M-foot and the initial syllable
of vowel-initial enclitics,
both of which are discussed in (\srf{sec:TheFoo}).

Words with the surface structure (C)VVCV(C){\#}, such as \ve{kaunaʔ} `snake; creature'
are the only cases in which a sequence of two vowels is the nucleus of a single phonemic syllable.
The first two vowels of such words are assigned to a single V-slot
and thus by extrapolation form the nucleus
of the syllable to which that V-slot belongs.
This is discussed in more detail in \srf{sec:SurVVCVWor} below.
The syllabification of \ve{kaunaʔ} `snake; creature' is shown in \qf{as:Syl:kaunaq} below.

\begin{exe}
	\exa{\xy
		<1em,2.5cm>*\as{σ}="s1",<3em,2.5cm>*\as{σ}="s2",
		<0em,1.5cm>*\as{C}="c1",<1em,1.5cm>*\as{V}="v1",<2em,1.5cm>*\as{C}="c2",<3em,1.5cm>*\as{V}="v2",<4em,1.5cm>*\as{C}="c3",
		<0em,0.5cm>*\as{k}="k",<0.7em,0.5cm>*\as{a}="a1",<1.3em,0.5cm>*\as{u}="u",<2em,0.5cm>*\as{n}="n",<3em,0.5cm>*\as{a}="a2",<4em,0.5cm>*\as{ʔ}="q",
		<-0.5em,0cm>*\as{[ˈ}="[",
		<0em,0cm>*\as{k}="kp",<0.7em,0cm>*\as{ɐ}="a1p",<1.3em,0cm>*\as{w}="up",
		<2em,0cm>*\as{n}="np",<3em,0cm>*\as{ɐ}="a2p",<4em,0cm>*\as{ʔ}="qp",<4.5em,0cm>*\as{]}="]",
		<5.5em,0cm>*\as{{\emb{kaunaq.mp3}{\spk{}}{\apl}}}="spk",
		"k"+U;"c1"+D**\dir{-};"a1"+U;"v1"+D**\dir{-};"u"+U;"v1"+D**\dir{-};"n"+U;"c2"+D**\dir{-};"a2"+U;"v2"+D**\dir{-};"q"+U;"c3"+D**\dir{-};
		"c1"+U;"s1"+D**\dir{-};"c2"+U;"s1"+D**\dir{-};"v1"+U;"s1"+D**\dir{-};
		"c2"+U;"s2"+D**\dir{-};"c3"+U;"s2"+D**\dir{-};"v2"+U;"s2"+D**\dir{-};
	\endxy}\label{as:Syl:kaunaq}
\end{exe}

While each V-slot is phonemically the nucleus of its own syllable
(with the exception of surface (C)VVCV(C) words),
there are some situations in which a vowel sequence (\srf{sec:VowSeq})
can optionally coalesce into a single phonetic syllable.
This optional phonetic coalescence does not in any way affect the underlying phonemic structures.
Two vowels which have coalesced into a single phonetic syllable
remain the peak of two phonemic syllables for the purposes of
stress assignment, reduplication, metathesis,
and all other morphophonemic processes of the language.

Firstly, as discussed in \srf{sec:DouVow},
in normal speech a sequence of two identical vowels usually
coalesces into a single phonetic syllable with a single
intensity peak at the beginning of the vowel sequence.
The examples from \srf{sec:DouVow}
are repeated in \qf{ex:VV>V:2} below.

\newpage
\begin{exe}
	\ex{/V{\sA}V{\sA}/ {\ra} [Vː] \label{ex:VV>V:2}}
	\sn{\gw\begin{tabular}{llll}
		\ve{a|n-s\tbr{ii}}	&[ʔanˈs\tbr{iː}]	&{\emb{ansii.mp3}{\spk{}}{\apl}}& `sings'  \\
		\ve{f\tbr{ee}}			&[f\tbr{ɛː}]			&{\emb{fee.mp3}{\spk{}}{\apl}}& `wife' \\
		\ve{h\tbr{aa}}			&[h\tbr{aː}]			&{\emb{haa.mp3}{\spk{}}{\apl}}& `four' \\
		\ve{\tbr{oo}}				&[ʔ\tbr{ɔː}]			&{\emb{oo.mp3}{\spk{}}{\apl}}& `bamboo' \\
		\ve{t\tbr{uu}-f}		&[t̪\tbr{ʊˑ}f]	&{\emb{tuuf.mp3}{\spk{}}{\apl}}& `knee' \\
	\end{tabular}}
\end{exe}

Another situation in which two vowels often (though not always)
are realised as a single phonetic syllable with only a single intensity peak
at the beginning of the vowel sequence is when the second vowel
is higher than the first.
When this is the case the second vowel can be realised as an off-glide.
Examples are given in \qf{ex:VVSyl} below.

\begin{exe}
	\ex{/VV/ {\ra} [V\sarc{V}] }\label{ex:VVSyl}
	\sn{\gw\begin{tabular}{llll}
			\ve{a|n-\tbr{tou}p}	&[ʔan̪ˈt̪\tbr{ɘw}p]	&{\emb{antoup.mp3}{\spk{}}{\apl}}& `receives' \\
			\ve{n-\tbr{ei}k}		&[n\tbr{ej}kʲ]		&{\emb{neik.mp3}{\spk{}}{\apl}}& `takes' \\
			\ve{t\tbr{ei}}			&[t̪\tbr{ej}]			&{\emb{tei-diph.mp3}{\spk{}}{\apl}}& `faeces' \\
			\ve{f\tbr{au}k}			&[f\tbr{ɐw}k]			&{\emb{fauk.mp3}{\spk{}}{\apl}}& `how many' \\
	\end{tabular}}
\end{exe}

This realisation is entirely optional,
and many instances of a vowel followed by a higher vowel are
realised transparently as two phonetic syllables.
Examples are given in \qf{ex:/V.V/->[V.V]} below.

\begin{exe}
	\ex{/VV/ {\ra} [V.V] }\label{ex:/V.V/->[V.V]}
	\sn{\gw\begin{tabular}{llll}
			\ve{t\tbr{ai}-f}		&[ˈt̪\tbr{a.i}f]	&{\emb{taif.mp3}{\spk{}}{\apl}}	& `belly' \\
			\ve{sn\tbr{ae}n}		&[ˈsn\tbr{a.ɛ}n]		&{\emb{snaen.mp3}{\spk{}}{\apl}}	& `sand' \\
			\ve{ans\tbr{ao}-f}	&[ʔanˈs\tbr{a.ɔ}f]	&{\emb{ansaof.mp3}{\spk{}}{\apl}}	& `solar plexus' \\
			\ve{t\tbr{ei}}			&[ˈt̪\tbr{e.i}]	&{\emb{tei-mono.mp3}{\spk{}}{\apl}}		& `faeces' \\
%			\ve{}	&[]	&{\emb{.mp3}{\spk{}}{\apl}}& `' \\
%			\ve{}	&[]	&{\emb{.mp3}{\spk{}}{\apl}}& `' \\
%			\ve{}	&[]	&{\emb{.mp3}{\spk{}}{\apl}}& `' \\
	\end{tabular}}
\end{exe}

Realisation as a single phonetic syllable
rarely occurs when both vowels
of a sequence are of equal height,
or when the first vowel is higher than the second.
Examples are given in \qf{ex:/V.V/->[V.V]2} below.

\begin{exe}
	\ex{/VV/ {\ra} [V.V] }\label{ex:/V.V/->[V.V]2}
	\sn{\gw\begin{tabular}{llll}
			\ve{\tbr{oe} kmii}&[ʔ\tbr{ɔ.ɛ}kˈmi:]	&{\emb{oe-kmii.mp3}{\spk{}}{\apl}}& `urine' \\
			\ve{n\tbr{oa}h}		&[ˈn\tbr{ɔ.ɐ}h]	&{\emb{noah.mp3}{\spk{}}{\apl}}& `coconut' \\
			\ve{f\tbr{ua}-f}	&[ˈf\tbr{ʊ.ɐ}f]	&{\emb{fuaf.mp3}{\spk{}}{\apl}}& `fruit' \\
			\ve{\tbr{ia}}			&[ˈʔ\tbr{i.a}]	&{\emb{ia.mp3}{\spk{}}{\apl}}& `here' \\
			\ve{mn\tbr{ea}s}	&[ˈm͡n\tbr{ɛ.a}s]	&{\emb{mneas.mp3}{\spk{}}{\apl}}& `hulled rice' \\
%			\ve{}	&[]	&{\emb{.mp3}{\spk{}}{\apl}}& `' \\
	\end{tabular}}
\end{exe}

Importantly for any analysis of metathesis in Amarasi,
vowel sequences created through metathesis \it{do not} obligatorily coalesce.
This means that an account of Amarasi metathesis
in which metathesis is driven by the need for stressed syllables to be heavy
(as has been proposed for Kwara'ae -- see \srf{sec:Kwa})
cannot account for all the data.

Examples of vowel sequences created through metathesis
in which phonetic coalescence has not occurred are given in \qf{ex:VCV->VVC->[V.VC]} below.
Additionally, in each example in \qf{ex:VCV->VVC->[V.VC]} the second vowel is higher than the first;
the kind of vowel sequence which most commonly coalesces.

\begin{exe}
	\ex{V\sub{1}CV\sub{2}{\#} {\ra} V\sub{1}V\sub{2}C{\#} {\ra} [V.VC] }\label{ex:VCV->VVC->[V.VC]}
	\sn{\gw\begin{tabular}{lllll}
		\ve{{\rt}toti}&\ve{a|n-t\tbr{oi}t}	&[ʔan̪ˈt̪\tbr{ɵ.i}t̪]&{\emb{antoit.mp3}{\spk{}}{\apl}}& `asks' \\
			\ve{{\rt}mani}&\ve{a|n-m\tbr{ai}n}	&[ʔanˈm\tbr{a.i}n]	&{\emb{anmain.mp3}{\spk{}}{\apl}}	& `laughs' \\
			\ve{{\rt}hake}&\ve{a|n-h\tbr{ae}k}	&[ʔanˈh\tbr{a.ɛ}kʲ]	&{\emb{anhaek.mp3}{\spk{}}{\apl}}	& `stands' \\
			\ve{{\rt}fanu}&\ve{f\tbr{au}n}	&[f\tbr{a.ʊ}n]			&{\emb{faun.mp3}{\spk{}}{\apl}}	& `eight' \\ 
			\ve{{\rt}tenu}&\ve{t\tbr{eu}n}	&[t̪\tbr{ɛ.ʊ}n]	&{\emb{teun.mp3}{\spk{}}{\apl}}	& `three' \\
	\end{tabular}}
\end{exe}

Coalescence of two vowels into a single phonetic syllable
is more frequent in rapid speech and when the vowel sequence does not bear primary stress.
Thus, in a particular wordlist, the word \ve{hau} `tree, wood' occurs in isolation as
[ˈha.ʊ] {\emb{hau.mp3}{\spk{}}{\apl}},
without the second vowel being realised as an off-glide.
However, in the same wordlist when the same word occurs in the compound \ve{hau noʔo} `tree leaf'
it is realised as [hawˈnɔʔɔ]{\emb{hau-noqo.mp3}{\spk{}}{\apl}},
with the second vowel desyllabified.
Again, such desyllabification is \it{not} obligatory
and vowel sequences which do not have primary stress also
often surface with two phonetic syllables.
One example is \ve{oe mninuʔ} `water (for) drinking'
{\ra} [ʔɔ.ɛmˈninʊʔ]{\emb{oe-mninuq.mp3}{\spk{}}{\apl}}.

\subsection{Roots with one foot (root {\ra} Ft)}\label{sec:RooOneFoo}
Roots consisting of a single foot are the most common kind of root in my corpus
with 64\% (1,223/1,913) of all lexical roots containing a single foot.\footnote{
		There is only one root in my entire corpus which
		has two syllables and a non-CVCVC foot.
		This is the loan \ve{maski} `even though' from Portuguese \it{mas que}.}

Given that C-slots may be empty in Amarasi (\srf{sec:EmpCSlo}),
a root with a single foot may surface maximally as CVCVC, with all C-slots filled,
and minimally as VV, with all C-slots empty.
Word-initial empty C-slots are automatically filled by a phonetic glottal stop (\srf{sec:GloStoIns}).
An example of every attested structure for words with a single foot is given in \trf{tab:RooSinFoo}.

\begin{table}[ht]
	\centering\caption{Roots with a single foot}\label{tab:RooSinFoo}
		{\begin{tabular}{llllllr}\lsptoprule
			Structure						&Root				&Phonetic	&																&gloss					&no.&\%\\\midrule
			CVCV\hp{C}					&\ve{fatu}	&[ˈfat̪ʊ]	&{\emb{fatu.mp3}{\spk{}}{\apl}}	&`stone, rock'	&547&45\%\\
			CVCVC								&\ve{manas}	&[ˈmanɐs]	&{\emb{manas.mp3}{\spk{}}{\apl}}&`sun'					&345&28\%\\
			CV\hp{C}V\hp{C}			&\ve{hau}		&[ˈhaʊ]		&{\emb{hau.mp3}{\spk{}}{\apl}}	&`wood, tree'		&133&11\%\\
			CV\hp{C}VC					&\ve{puah}	&[ˈpʊɐh]	&{\emb{puah.mp3}{\spk{}}{\apl}}	&`betel nut'		&77	&6\%\\
			\hp{C}VCV\hp{C}			&\ve{asu}		&[ˈʔasʊ]	&{\emb{asu.mp3}{\spk{}}{\apl}}	&`dog'					&54	&4\%\\
			\hp{C}VCVC					&\ve{anin}	&[ˈʔanin]	&{\emb{anin.mp3}{\spk{}}{\apl}}	&`wind'					&51	&4\%\\
			\hp{C}V\hp{C}V\hp{C}&\ve{ai}		&[ˈʔai]		&{\emb{ai.mp3}{\spk{}}{\apl}}		&`fire'					&10	&1\%\\
			\hp{C}V\hp{C}VC			&\ve{uat}		&[ˈʔʊɐt̪]	&{\emb{uat.mp3}{\spk{}}{\apl}}	&`veins'				&6	&\\ \lspbottomrule
		\end{tabular}
		}
\end{table}

\subsubsection{Surface VVCV(C){\#} words}\label{sec:SurVVCVWor}
Among disyllables, there are also 32 words with the structure VVCV(C){\#} in my dictionary.
I analyse such words as consisting of a single foot.
All Amarasi words so far attested with this shape
have identical final and penultimate vowels, or the final vowel is /a/.
These words are all given in \ref{tab:AmaSurVVCWor} below.

\begin{table}[ht]
	\caption{Amarasi surface VVCV(C){\#} words}\label{tab:AmaSurVVCWor}
	\centering
			\begin{tabular}{lll|lll}\lsptoprule
				Amarasi	&	Gloss	&	VV	&	Amarasi	&	Gloss	&	VV	\\ \midrule
				\ve{aikaʔ}	&	`thorn'	&	\ve{ai}	&	\ve{n-auban}	&	`crowd'	&	\ve{au}	\\
				\ve{n-aikas}	&	`praise'	&	\ve{ai}	&	\ve{aunu}	&	`spear'	&	\ve{au}	\\
				\ve{aina-f}	&	`mother'	&	\ve{ai}	&	\ve{kaunaʔ}	&	`snake; creature'	&	\ve{au}	\\
				\ve{n-aini}	&	`mourn'	&	\ve{ai}	&	\ve{n-ʔaubar}	&	`unify'	&	\ve{au}	\\
				\ve{n-aiti}	&	`pick up'	&	\ve{ai}	&	\ve{na-ʔkaunuʔ}	&	`bother'	&	\ve{au}	\\
				\ve{baitiʔ}	&	`should'	&	\ve{ai}	&	\ve{maukuʔ}	&	`cuscus'	&	\ve{au}	\\
				\ve{na-kainaʔ}	&	`forbid'	&	\ve{ai}	&	\ve{na-maunu}	&	`crazy'	&	\ve{au}	\\
				\ve{na-ʔaisa}	&	`tie'	&	\ve{ai}	&	\ve{na-mausa-b}	&	`domesticate'	&	\ve{au}	\\
				\ve{na-maikaʔ}	&	`remain'	&	\ve{ai}	&	\ve{mautu}	&	`allow'	&	\ve{au}	\\
				\ve{na-saitan}	&	`leave'	&	\ve{ai}	&	\ve{naunuʔ}	&	`breadfruit'	&	\ve{au}	\\
				\ve{n-aena}	&	`run'	&	\ve{ae}	&	\ve{nautus}	&	`beetle'	&	\ve{au}	\\
				\ve{n-aesa}	&	`squeeze'	&	\ve{ae}	&	\ve{na-noebaʔ}	&	`listless'	&	\ve{oe}	\\
				\ve{na-ʔaekaʔ}	&	`soak'	&	\ve{ae}	&	\ve{na-roitan}	&	`prepare'	&	\ve{oi}	\\
				\ve{na-taekaʔ}	&	`puddle'	&	\ve{ae}	&	\ve{na-soitan}	&	`open'	&	\ve{oi}	\\
				\ve{n-eiti}	&	`travel'	&	\ve{ei}	&	\ve{uabaʔ}	&	`speech'	&	\ve{ua}	\\
				\ve{n-meiti}	&	`dry up'	&	\ve{ei}	&	\ve{na-kaaka}	&	`howl'	&	\ve{aa}	\\
			\lspbottomrule
				\end{tabular}
\end{table}

Under this analysis the first two vowel segments of such words are assigned to a single V-slot,
thus forming a kind of phonetic diphthong.
The proposed structures of \ve{kaunaʔ} {\ra} [ˈkɐwnɐʔ]
{\emb{kaunaq.mp3}{\spk{}}{\apl}} `snake; creature',
\ve{aikaʔ} {\ra} [ˈʔajkaʔ] {\emb{aikaq.mp3}{\spk{}}{\apl}} `thorn'
and \ve{aina-f} {\ra} [ˈʔajnɐf] {\emb{ainaf.mp3}{\spk{}}{\apl}} `mother'
are given in \qf{as:(C)VVCV(C)} below.

\begin{multicols}{3}
	\begin{exe}\ex{
		\begin{xlist}
			\exa{\xy
				<1em,2.5cm>*\as{σ}="s1",<3em,2.5cm>*\as{σ}="s2",
				<0em,1.5cm>*\as{C}="c1",<1em,1.5cm>*\as{V}="v1",<2em,1.5cm>*\as{C}="c2",<3em,1.5cm>*\as{V}="v2",<4em,1.5cm>*\as{C}="c3",
				<0em,0.5cm>*\as{k}="k",<0.7em,0.5cm>*\as{a}="a1",<1.3em,0.5cm>*\as{u}="u",<2em,0.5cm>*\as{n}="n",<3em,0.5cm>*\as{a}="a2",<4em,0.5cm>*\as{ʔ}="q",
				<-0.5em,0cm>*\as{[ˈ}="[",<0em,0cm>*\as{k}="kp",<0.7em,0cm>*\as{ɐ}="a1p",<1.3em,0cm>*\as{w}="up",
				<2em,0cm>*\as{n}="np",<3em,0cm>*\as{ɐ}="a2p",<4em,0cm>*\as{ʔ}="qp",<4.5em,0cm>*\as{]}="]",
				"k"+U;"c1"+D**\dir{-};"a1"+U;"v1"+D**\dir{-};"u"+U;"v1"+D**\dir{-};"n"+U;"c2"+D**\dir{-};"a2"+U;"v2"+D**\dir{-};"q"+U;"c3"+D**\dir{-};
				"c1"+U;"s1"+D**\dir{-};"c2"+U;"s1"+D**\dir{-};"v1"+U;"s1"+D**\dir{-};
				"c2"+U;"s2"+D**\dir{-};"c3"+U;"s2"+D**\dir{-};"v2"+U;"s2"+D**\dir{-};
			\endxy}\label{as:kaunaq}
			\exa{\xy
				<1em,2.5cm>*\as{σ}="s1",<3em,2.5cm>*\as{σ}="s2",
				<0em,1.5cm>*\as{C}="c1",<1em,1.5cm>*\as{V}="v1",<2em,1.5cm>*\as{C}="c2",<3em,1.5cm>*\as{V}="v2",<4em,1.5cm>*\as{C}="c3",
				<0em,0.5cm>*\as{ }="k",<0.7em,0.5cm>*\as{a}="a1",<1.3em,0.5cm>*\as{i}="u",<2em,0.5cm>*\as{k}="n",<3em,0.5cm>*\as{a}="a2",<4em,0.5cm>*\as{ʔ}="q",
				<-0.5em,0cm>*\as{[ˈ}="[",<0em,0cm>*\as{ʔ}="kp",<0.7em,0cm>*\as{a}="a1p",<1.3em,0cm>*\as{j}="up",
				<2em,0cm>*\as{k}="np",<3em,0cm>*\as{a}="a2p",<4em,0cm>*\as{ʔ}="qp",<4.5em,0cm>*\as{]}="]",
				"a1"+U;"v1"+D**\dir{-};"u"+U;"v1"+D**\dir{-};"n"+U;"c2"+D**\dir{-};"a2"+U;"v2"+D**\dir{-};"q"+U;"c3"+D**\dir{-};
				"c1"+U;"s1"+D**\dir{-};"c2"+U;"s1"+D**\dir{-};"v1"+U;"s1"+D**\dir{-};
				"c2"+U;"s2"+D**\dir{-};"c3"+U;"s2"+D**\dir{-};"v2"+U;"s2"+D**\dir{-};
			\endxy}\label{as:aikaq}
			\exa{\xy
				<1em,2.5cm>*\as{σ}="s1",<3em,2.5cm>*\as{σ}="s2",
				<0em,1.5cm>*\as{C}="c1",<1em,1.5cm>*\as{V}="v1",<2em,1.5cm>*\as{C}="c2",<3em,1.5cm>*\as{V}="v2",<4em,1.5cm>*\as{C}="c3",
				<0em,0.5cm>*\as{ }="k",<0.7em,0.5cm>*\as{a}="a1",<1.3em,0.5cm>*\as{i}="u",<2em,0.5cm>*\as{n}="n",<3em,0.5cm>*\as{a}="a2",<4em,0.5cm>*\as{f}="q",
				<-0.5em,0cm>*\as{[ˈ}="[",<0em,0cm>*\as{ʔ}="kp",<0.7em,0cm>*\as{a}="a1p",<1.3em,0cm>*\as{j}="up",
				<2em,0cm>*\as{n}="np",<3em,0cm>*\as{ɐ}="a2p",<4em,0cm>*\as{f}="qp",<4.5em,0cm>*\as{]}="]",
				<3.5em,0.5cm>*\as{-}="-",
				"a1"+U;"v1"+D**\dir{-};"u"+U;"v1"+D**\dir{-};"n"+U;"c2"+D**\dir{-};"a2"+U;"v2"+D**\dir{-};"q"+U;"c3"+D**\dir{-};
				"c1"+U;"s1"+D**\dir{-};"c2"+U;"s1"+D**\dir{-};"v1"+U;"s1"+D**\dir{-};
				"c2"+U;"s2"+D**\dir{-};"c3"+U;"s2"+D**\dir{-};"v2"+U;"s2"+D**\dir{-};
			\endxy}\label{as:ainaf}
		\end{xlist}}\label{as:(C)VVCV(C)}
	\end{exe}
\end{multicols}

There are four observations which support this analysis.
Firstly, as discussed in \srf{sec:Str} (\prf{ex:(C)VVCV(C)->"(C)VVCV(C)}),
stress falls on the penultimate segmental vowel of a word in Amarasi.
For VVCV(C){\#} words, however, stress falls on the antepenultimate segmental vowel.
This otherwise aberrant stress pattern can be explained by positing that
stress is assigned to the penultimate V-slot of the foot,
rather than being assigned to any specific segmental vowel.

Secondly, in almost all cases the initial vowel sequence of a VVCV(C){\#}
word forms a phonetic diphthong and the second vowel is realised as a glide,
as illustrated with the three examples in \qf{as:(C)VVCV(C)} above.

The only word for which I am aware that a phonetic diphthong
is not always found is \ve{uabaʔ} `speech, to speak'.
There are seven instances of the U\=/form of this word in my corpus
(with a verbal agreement prefix in six instances).
Five have a phonetic diphthong [ˈʔwɐbɐʔ]
and two cases have two full vowels [ˈʔʊ.ɐbɐʔ].
An example of the latter pronunciation is given in \qf{ex:UsiTheLan} below.
However, even in such instances stress falls on the antepenultimate segmental vowel. 

\begin{exe}
\let\eachwordone=\textnormal \let\eachwordtwo=\itshape
	\ex{\glll	[ʔanˈpa\sarc{ɛ}k sɪn \tbr{ˌ}ʔ\tbr{ʊ.ɐ}bɐʔ]\\
						\hp{[ʔ}a|n-paek siin uabaʔ \\
						\hp{[ʔ}{\a\n}-use {\siin} speech	\\
			\glt	\lh{[ʔa|}`{\ldots} using their language' \txrf{130920-1, 4.18}
						{\emb{130920-1-04-18.mp3}{\spk{}}{\apl}}}\label{ex:UsiTheLan}
\end{exe}

Thirdly, when we examine \emph{which} vowel sequences occupy the initial V-slot in such words,
we find a preference for the VV sequence to be /au/ (11/32) or /ai/ (10/32),
with 21/32 words having either of these sequences; 66\%.
Such sequences represent the most common kinds of diphthongs in languages of the world
(\citealt[36]{li86}, \citealt[40]{mi98}).

Fourthly, if surface VVCV(C){\#} words did in fact consist of a syllable and a foot,
they would be the only words whose final foot was not preceded by a consonant,
either a phonemic consonant (\srf{sec:Con})
or a predictable glottal stop (\srf{sec:GloStoIns}).
For these reasons I analyse the initial vowel sequence
in surface VVCV(C){\#} as being assigned to a single V-slot.

\subsubsection{Ro{\Q}is Amarasi diphthongisation}\label{sec:RoqAmaDip}
In Ro{\Q}is Amarasi there is a regular process whereby
the penultimate vowel of a word becomes a diphthong
by copying the final vowel into the penultimate V-slot
after the underlying penultimate vowel.\footnote{
		The same process of diphthongisation described
		in this section also occurs in the variety
		of Meto spoken in the village of Oepaha immediately
		to the east of the Ro{\Q}is area.}
This only operates for CVC{\#} final words.
Examples are given in \trf{tab:RoqStrVSloDip}
alongside cognates in Kotos for comparison.
I transcribe diphthongs formed by this automatic
process with the tie-bar [\, ͡ \,] to distinguish
them from underlying vowel sequences.

\begin{table}[h]
	\centering\caption{Ro{\Q}is Amarasi stressed diphthongisation}\label{tab:RoqStrVSloDip}
	\begin{threeparttable}[b]
		\begin{tabular}{lll|lll}\lsptoprule
			Kotos							&	Ro{\Q}is								&	gloss				&	Kotos									&	Ro{\Q}is										&	gloss	\\ \midrule
			\ve{t\tbr{e}fis}	&	\ve{t\tbrtb{e}{\i}fik}	&	`roof'			&	\ve{niis \tbr{e}no-f}	&	\ve{niis \tbrtb{e}{o}no-f}	&	`incisors'	\\
			\ve{m\tbr{a}sik}	&	\ve{m\tbrtb{a}{\i}sik}	&	`salt'			&	\ve{n-ʔ\tbr{a}tor}		&	\ve{n-ʔ\tbrtb{a}{o}tor}			&	`arrange'	\\
			\ve{t\tbr{o}ʔis}	&	\ve{t\tbrtb{o}{\i}ʔis}	&	`trumpet'		&	\ve{s\tbr{i}ʔu-f}			&	\ve{s\tbrtb{\i}{u}ʔu-f}			&	`elbow'	\\
			\ve{h\tbr{u}nik}	&	\ve{h\tbrtb{u}{\i}nik}	&	`turmeric'	&	\ve{\tbr{e}suk}				&	\ve{\tbrtb{e}{u}suk}				&	`mortar'	\\
			\ve{\tbr{a}net}		&	\ve{\tbrtb{a}{e}net}		&	`needle'		&	\ve{m\tbr{a}nus}			&	\ve{m\tbrtb{a}{u}nus}				&	`beetle vine'	\\
			\ve{r\tbr{o}ne-f}	&	\ve{r\tbrtb{o}{e}ne-f}	&	`brain'			&	\ve{p\tbr{o}nu-f}			&	\ve{p\tbrtb{o}{u}nu-f}			&	`body hair'\su{†}	\\
			\lspbottomrule
		\end{tabular}
			\begin{tablenotes}
				\item [†] Kotos \ve{ponu-f} is `moustache'
									and Ro{\Q}is \ve{po͡unu-f} is `body hair'.
			\end{tablenotes}
		\end{threeparttable}
\end{table}

In my Ro{\Q}is data this process does not affect words
in which the penultimate and final vowels are the same,
words in which the final vowel is /a/,
or words the final consonant of which is /ʔ/.\footnote{
		There is evidence that this process operates
		for some words with final /a/ in some other varieties of Ro{\Q}is Amarasi.
		Thus, I collected the words \ve{p\tbrtb{o}{a}ʔan} {\la} \ve{p\tbr{o}ʔan} `orchard'
		and \ve{p\tbrtb{u}{a}ʔat} {\la} \ve{p\tbr{u}at} `wave' from a Ro{\Q}is speaker
		from Baun. Kotos has \ve{poʔon} `orchard' and \ve{okin} `wave'.}
		
This process is highly productive
and applies to loans (e.g. Malay \it{ator} {\ra} \ve{n-ʔa͡otor} `arrange'),
words whose final consonant is a suffix (e.g. \ve{enoʔ} `door' {\ra}
\ve{niis e͡ono-f} `incisors' lit. `door teeth')
and even across clitic and word boundaries
so long as the consonant of the following morpheme
is the coda of the morpheme which undergoes diphthongisation.
Examples of Ro{\Q}is diphthongisation operating across morpheme boundaries
are given in \qf{ex:RO-170830-1, 4.26}--\qf{ex:RO-170829-1, 17.13} below.
In \qf{ex:RO-170830-1, 4.26} the consonant triggering diphthongisation
is an enclitic, in \qf{ex:RO-170829-1, 19.03} it is the prefix of the following
word and in \qf{ex:RO-170829-1, 17.13} it is the first member of root-initial cluster.
		
\begin{exe}
	\ex{\glll	\sf{\j adi} n-tean-aʔ toon boaʔ h\tbrtb{\i}{u}\tbr{tu} =m niim =te, naiʔ au ku-snaas\\
						\sf{\j adi} n-tean-aʔ toon boaʔ h\tbr{itu} =ma niim =te naiʔ au ku-snasa\\
						so \n-until-{\qV} year ten seven =and five ={\te} then {\au} {\qu}-stop\\
			\glt	`so (I did it) up until the year 1975, then I stopped'\txrf{RO-170830-1, 4.26}
						{\emb{RO-170830-1-04-26.mp3}{\spk{}}{\apl}}}\label{ex:RO-170830-1, 4.26}
	\ex{\glll	nunatiʔ hiin n-paak\j=ee, n-f\tbrtb{a}{\i}\tbr{ni} n-biʔaak heʔ hiin moin-n=ii,\\
						nunatiʔ hini n-pake=ee n-f\tbr{ani} n-biʔaka heʔ hini moni-n=ii\\
						{\he} {\iin} \n-use={\eeV} \n-again \n-{\bi} {\reqt} {\iin} life-{\N=\ii}\\
			\glt	`so that he can use it again in his life'\txrf{RO-170829-1, 19.03}
						{\emb{RO-170829-1-19-03.mp3}{\spk{}}{\apl}}}\label{ex:RO-170829-1, 19.03}
	\ex{\glll	\sf{bahwa}, toiʔs=ee na-haan ek n\tbrtb{e}{o}\tbr{no} tnanaʔ\\
						\sf{bahwa} toʔis=ee na-hana ek n\tbr{eno} tnanaʔ\\
						so.that trumpet={\ee} \na-sound {\ek} day middle\\
			\glt	`the trumpet sounds in the middle of the day'\txrf{RO-170829-1, 17.13}
						{\emb{RO-170829-1-17-13.mp3}{\spk{}}{\apl}}}\label{ex:RO-170829-1, 17.13}
\end{exe}

The prosodic and morphological structure of Ro{\Q}is \ve{ne͡ono tnanaʔ} `middle of the day'
is shown in \qf{as:PrWd:neono-tnanaq} below
alongside the Kotos equivalent \ve{neno tnanaʔ} in \qf{as:PrWd:neno-tnanaq}.
While morphologically the initial consonant of \ve{tnanaʔ} `middle'
is part of the second word, prosodically it is the coda of the previous word,
thus triggering diphthongisation of the penultimate vowel of this word
in Ro{\Q}is \qf{as:PrWd:neono-tnanaq}.

\begin{multicols}{2}
	\begin{exe}\ex{\label{as:PrWd:neno-tnanaq->neono-tnanaq}
		\begin{xlist}
			\exa{\label{as:PrWd:neno-tnanaq}\xy
					<3em,5cm>*\as{PrWd}="PrWd1",<8em,5cm>*\as{PrWd}="PrWd2",
					<3em,4cm>*\as{Ft}="ft1",<8em,4cm>*\as{Ft}="ft2",
					<2em,3cm>*\as{σ}="s1",<4em,3cm>*\as{σ}="s2",<7em,3cm>*\as{σ}="s3",<9em,3cm>*\as{σ}="s4",
					<1em,2cm>*\as{C}="CV1",<2em,2cm>*\as{V}="CV2",<3em,2cm>*\as{C}="CV3",
					<4em,2cm>*\as{V}="CV4",<5em,2cm>*\as{C}="CV5",
					<6em,2cm>*\as{C}="CV6",<7em,2cm>*\as{V}="CV7",<8em,2cm>*\as{C}="CV8",
					<9em,2cm>*\as{V}="CV9",<10em,2cm>*\as{C}="CV10",
					<1em,1cm>*\as{n}="cv1",<2em,1cm>*\as{e}="cv2",<3em,1cm>*\as{n}="cv3",<4em,1cm>*\as{o}="cv4",
					<5em,1cm>*\as{t}="cv5",<6em,1cm>*\as{n}="cv6",<7em,1cm>*\as{a}="cv7",<8em,1cm>*\as{n}="cv8",
					<9em,1cm>*\as{a}="cv9",<10em,1cm>*\as{ʔ}="cv10",
					<2.5em,0cm>*\as{M}="m1",<7.5em,0cm>*\as{M}="m2",
					"m1"+U;"cv2"+D**\dir{-};"m1"+U;"cv1"+D**\dir{-};"m1"+U;"cv3"+D**\dir{-};"m1"+U;"cv4"+D**\dir{-};
					"m2"+U;"cv5"+D**\dir{-};"m2"+U;"cv6"+D**\dir{-};"m2"+U;"cv7"+D**\dir{-};
					"m2"+U;"cv8"+D**\dir{-};"m2"+U;"cv9"+D**\dir{-};"m2"+U;"cv10"+D**\dir{-};
					"cv1"+U;"CV1"+D**\dir{-};"cv2"+U;"CV2"+D**\dir{-};
					"cv3"+U;"CV3"+D**\dir{-};"cv4"+U;"CV4"+D**\dir{-};"cv5"+U;"CV5"+D**\dir{-};
					"cv6"+U;"CV6"+D**\dir{-};"cv7"+U;"CV7"+D**\dir{-};"cv8"+U;"CV8"+D**\dir{-};"cv9"+U;"CV9"+D**\dir{-};
					"cv10"+U;"CV10"+D**\dir{-};
					"CV1"+U;"s1"+D**\dir{-};"CV2"+U;"s1"+D**\dir{-};"CV3"+U;"s1"+D**\dir{-};
					"CV3"+U;"s2"+D**\dir{-};"CV4"+U;"s2"+D**\dir{-};"CV5"+U;"s2"+D**\dir{-};
					"CV6"+U;"s3"+D**\dir{-};"CV7"+U;"s3"+D**\dir{-};"CV8"+U;"s3"+D**\dir{-};
					"CV8"+U;"s4"+D**\dir{-};"CV9"+U;"s4"+D**\dir{-};"CV10"+U;"s4"+D**\dir{-};
					"s1"+U;"ft1"+D**\dir{-};"s2"+U;"ft1"+D**\dir{-};"s3"+U;"ft2"+D**\dir{-};"s4"+U;"ft2"+D**\dir{-};
					"ft1"+U;"PrWd1"+D**\dir{-};"ft2"+U;"PrWd2"+D**\dir{-};
			\endxy}
			\exa{\label{as:PrWd:neono-tnanaq}\xy
					<3em,5cm>*\as{PrWd}="PrWd1",<8em,5cm>*\as{PrWd}="PrWd2",
					<3em,4cm>*\as{Ft}="ft1",<8em,4cm>*\as{Ft}="ft2",
					<2em,3cm>*\as{σ}="s1",<4em,3cm>*\as{σ}="s2",<7em,3cm>*\as{σ}="s3",<9em,3cm>*\as{σ}="s4",
					<1em,2cm>*\as{C}="CV1",<2em,2cm>*\as{V}="CV2",<3em,2cm>*\as{C}="CV3",
					<4em,2cm>*\as{V}="CV4",<5em,2cm>*\as{C}="CV5",
					<6em,2cm>*\as{C}="CV6",<7em,2cm>*\as{V}="CV7",<8em,2cm>*\as{C}="CV8",
					<9em,2cm>*\as{V}="CV9",<10em,2cm>*\as{C}="CV10",
					<1em,1cm>*\as{n}="cv1",<1.666em,1cm>*\as{e}="cv2",<2.333em,1cm>*\as{o}="cv2.5",<3em,1cm>*\as{n}="cv3",<4em,1cm>*\as{o}="cv4",
					<5em,1cm>*\as{t}="cv5",<6em,1cm>*\as{n}="cv6",<7em,1cm>*\as{a}="cv7",<8em,1cm>*\as{n}="cv8",
					<9em,1cm>*\as{a}="cv9",<10em,1cm>*\as{ʔ}="cv10",
					<2.5em,0cm>*\as{M}="m1",<7.5em,0cm>*\as{M}="m2",
					"m1"+U;"cv2"+D**\dir{-};"m1"+U;"cv1"+D**\dir{-};"m1"+U;"cv3"+D**\dir{-};"m1"+U;"cv4"+D**\dir{-};
					"m2"+U;"cv5"+D**\dir{-};"m2"+U;"cv6"+D**\dir{-};"m2"+U;"cv7"+D**\dir{-};
					"m2"+U;"cv8"+D**\dir{-};"m2"+U;"cv9"+D**\dir{-};"m2"+U;"cv10"+D**\dir{-};
					"cv1"+U;"CV1"+D**\dir{-};"cv2"+U;"CV2"+D**\dir{-};"cv2.5"+U;"CV2"+D**\dir{-};
					"cv3"+U;"CV3"+D**\dir{-};"cv4"+U;"CV4"+D**\dir{-};"cv5"+U;"CV5"+D**\dir{-};
					"cv6"+U;"CV6"+D**\dir{-};"cv7"+U;"CV7"+D**\dir{-};"cv8"+U;"CV8"+D**\dir{-};"cv9"+U;"CV9"+D**\dir{-};
					"cv10"+U;"CV10"+D**\dir{-};
					"CV1"+U;"s1"+D**\dir{-};"CV2"+U;"s1"+D**\dir{-};"CV3"+U;"s1"+D**\dir{-};
					"CV3"+U;"s2"+D**\dir{-};"CV4"+U;"s2"+D**\dir{-};"CV5"+U;"s2"+D**\dir{-};
					"CV6"+U;"s3"+D**\dir{-};"CV7"+U;"s3"+D**\dir{-};"CV8"+U;"s3"+D**\dir{-};
					"CV8"+U;"s4"+D**\dir{-};"CV9"+U;"s4"+D**\dir{-};"CV10"+U;"s4"+D**\dir{-};
					"s1"+U;"ft1"+D**\dir{-};"s2"+U;"ft1"+D**\dir{-};"s3"+U;"ft2"+D**\dir{-};"s4"+U;"ft2"+D**\dir{-};
					"ft1"+U;"PrWd1"+D**\dir{-};"ft2"+U;"PrWd2"+D**\dir{-};
			\endxy}
		\end{xlist}}
	\end{exe}
\end{multicols}

%Finally, there is at least one example of such diphthongisation
%in my Kotos Amarasi data. This example is given in \qf{ex:130907-3, 0.19} below.
%XXX but different 'coz this wouldn't happen before glottal in Ro'is
%
%\begin{exe}
	%\ex{\glll	au ʔ-t\tbr{oiti} ʔnaka skoor =ma n-fee =kau \sf{surat} \sf{pindah}.\\
						%au ʔ-t\tbr{oti} ʔnaka skoor =ma n-fee =kau \sf{surat} \sf{pindah}\\
						%{\au} \q-ask head school =and \n-give ={\kau} letter move\\
			%\glt	`I asked the headmaster, and s/he gave me a letter to move.'
						%\txrf{130907-3, 0.19}{\emb{130907-3-00-19.mp3}{\spk{}}{\apl}}}\label{ex:130907-3, 0.19}
%\end{exe}
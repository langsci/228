\subsection{Root-final consonants}\label{sec:RooFinCon}
All consonants have been attested in root final position
except for the marginal phonemes /\j/ and /ɡw/.
\trf{tab:FreConWorCod} gives the frequency of consonants in root final
position compared to their frequency in other positions,
arranged by frequency in final position.
This count was made on my current dictionary of 2,005 unique roots.
This yielded 849 roots which ended in a consonant (42{\%} of all roots).

\begin{table}[h]
	\centering\caption{Frequency of root final consonants}\label{tab:FreConWorCod}
		\stl{0.35em}\begin{tabular}{r|ccccccccccccc}\lsptoprule
				C						&	\ve{ʔ}&	\ve{n}&	\ve{t}&	\ve{s}&	\ve{k}&	\ve{r}&	\ve{f}&	\ve{h}&	\ve{m}&	\ve{b}&	\ve{p}	&	\ve{\j}	&	\ve{gw}	\\ \midrule
				{\gap}{\#}	&	387		&	132		&	88		&	80		&	54		&	38		&	29		&	14		&	13		&	9			&	2				&	0				&	0				\\
										&	46{\%}&	16{\%}&	10{\%}&	9{\%}	&	6{\%}	&	4{\%}	&	3{\%}	&	2{\%}	&	2{\%}	&	1{\%}	&	0.2{\%}	&	0{\%}		&	0{\%}		\\ \midrule
				else.				&	456		&	671		&	500		&	468		&	440		&	428		&	300		&	282		&	235		&	196		&	138			&	11			&	2				\\
										&	11{\%}&	16{\%}&	12{\%}&	11{\%}&	11{\%}&	10{\%}&	7{\%}	&	7{\%}	&	6{\%}	&	5{\%}	&	3{\%}		&	0.3{\%}	&	0.05{\%}\\ \lspbottomrule
		\end{tabular}
\end{table}

While all consonants (except /\j/ and /ɡw/) are attested root finally,
there is a statistical skewing of which consonants do so,
with the glottal stop constituting 46{\%} of root final
consonants compared to 11{\%} of consonant phonemes in other root positions.

The labial stops /p/, /b/ and /m/
do not occur finally in any roots with more than two syllables.
This apparent restriction is probably the result of the
small number of roots of this size combined with the scarcity of 
the labial consonants root finally.
The labial fricative /f/ does occur on roots greater than two syllables
and there is a verbal suffix \ve{-b} `{\b}' and a nominal suffix \ve{-m} `{\mg}'
which freely attach to roots of more than two vowels.
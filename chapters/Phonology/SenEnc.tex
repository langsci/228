\subsection{Sentence enclitics}\label{sec:SenEnc}
Amarasi has three enclitics which have multiple forms:
the connectors \ve{=ma} `and' \ve{=te} `{\te}', as well as the negator enclitic \ve{=fa} `{\fa}'.
The connector \ve{=ma} is a general conjunction `and',
while \ve{=te} is a temporal subordinator.
Normally these enclitics occur with the forms given above.
An example of each is given in
\qf{ex:130920-1, 0.51 ch:ph}--\qf{ex:130825-6, 17.02 ch:DetPhoPho} below.

\begin{exe}
	\ex{\gll		n-reuk fanu =\tbr{te}, \\
							\n-pluck eight ={\te} \\
			\glt	`As it struck eight o'clock, {\ldots}' \txrf{130920-1, 0.51}
						\emb{130920-1-00-51-part.mp3}{\spk{}}{\apl}}\label{ex:130920-1, 0.51 ch:ph}
	\ex{\gll		iin n-nao n-bi Tofaʔ na-teef n-ok iin bae-f=ein=ee =\tbr{ma}\\
							{\iin} \n-nao {\n-\bi} Tofa{\Q} \na-meet {\n-\ok} {\iin} mate-\F=\ein={\ee} =and\\
			\glt	`He went to Tofa{\Q}, met with his mates and,' \txrf{130920-1, 2.18}
						\emb{130920-1-02-18.mp3}{\spk{}}{\apl}}\label{ex:130920-1, 2.18 ch:ph}
		\ex{\gll	au ka= amna-ah \sf{bubur} =kau =\tbr{fa}! \\
							{\au} {\ka}= {\at}-eat{\M} porridge ={\kau} ={\fa} \\
				\glt	`I don't eat porridge!' (\emph{lit.} `I'm not a porridge eater!') \txrf{130825-6, 17.02}
							{\emb{130825-6-17-02.mp3}{\spk{}}{\apl}}}\label{ex:130825-6, 17.02 ch:DetPhoPho}
\end{exe}

The final vowel of these enclitics is often deleted
as long as the following word does not begin with a consonant cluster.
Examples of each enclitic with the final vowel deleted
are given in \qf{ex:160326, 10.36}--\qf{ex:130825-7, 1.32} below.

\begin{exe}
	\ex{\glll		siin n-topu srainʔ=ii =\tbr{m}, na-srain sini =\tbr{m}\\
							sini n-topu sraniʔ=ii =\tbr{ma} na-srani sini =\tbr{ma}\\
							{\siin} \n-receive baptism={\ii} =and \na-baptise {\siin} =and\\
			\glt		`they received baptism and they were baptised and {\ldots}' \txrf{160326, 10.36}
							\emb{160326-10-36.mp3}{\spk{}}{\apl}}\label{ex:160326, 10.36}
	\ex{\glll		\sf{per} a|krei. {} kree\j=esa =\tbr{t}, ees.\\
							\sf{per} {\a}krei {} krei=esa =\tbr{te} esa\\
							per {\a}week {} week={\es} ={\te} {\es}\\
			\glt		`Per week. One (every) week.' \txrf{120923-1, 12.40}
							\emb{120923-1-12-40.mp3}{\spk{}}{\apl}}\label{ex:120923-1, 12.40 ch:ph}
	\ex{\glll	ka= n-heek =kau =\tbr{f} \\
						ka= n-heke =kau =\tbr{fa} \\
						{\ka}= \n-catch ={\kau} ={\fa}\\
			\glt	`I didn't get caught!'\txrf{130825-7, 1.32}
						{\emb{130825-7-01-32.mp3}{\spk{}}{\apl}}}\label{ex:130825-7, 1.32}
\end{exe}

For numerals greater than ten
deletion of the final vowel of \ve{=ma} `and'
is obligatory and the full form is no longer allowed.
A selection of such numerals is given in \qf{ex:21}--\qf{ex:5,768} below to illustrate.
While the full form is ungrammatical in such examples,
speakers recognise \ve{=m} as an allomorph of \ve{=ma}.

\begin{exe}
	\ex{\gll%	[bɔʔ nʊɐ \hp{=}m mɛsɛʔ]  \\
						boʔ nua =\tbr{m} meseʔ\\
						ten two =and one\\
			\glt `twenty-one' (21) {\emb{boq-nua-m-meseq.mp3}{\spk{}}{\apl}}}\label{ex:21}
	\ex{\gll%	[bɔʔ fanʊ \hp{=}m t̪eʊn]  \\
						boʔ fanu =\tbr{m}  teun \\
						ten eight =and three\\
			\glt `eighty-three' (83) {\emb{boq-fanu-m-teun.mp3}{\spk{}}{\apl}}}
	\ex{\gll%	[nat̪ʊn sɛɔ \hp{=}m bɔʔ \hp{=}ɛsɐ \hp{=}m haː]	\\
						natun seo =\tbr{m} boʔ =esa =\tbr{m} haa \\
						hundred nine =and ten =one =and four\\
			\glt `nine hundred and fourteen' (914) {\emb{natun-seo-m-boq-esa-m-haa.mp3}{\spk{}}{\apl}} }
	\ex{\gll%	[nifʊn nimɐ \hp{=}m nat̪ʊn hit̪ʊ \hp{=}m bɔʔ nɛː \hp{=}m faʊn]\\
						nifun nima =\tbr{m} natun hitu =\tbr{m} boʔ nee =\tbr{m} faun \\
						thousand five =and hundred seven =and ten six and eight\\
			\glt `five thousand seven hundred and sixty-eight' (5,768)
						{\emb{nifun-nima-m-natun-hitu-m-boq-neem-faun.mp3}{\spk{}}{\apl}}}\label{ex:5,768}
\end{exe}

When these enclitics attach to a consonant-final stem,
they optionally have vowel-initial forms which begin with [a].
These forms with an initial [a] have some parallels to epenthesis of [a]
before consonant clusters (\srf{sec:Epe}),
though unlike epenthesis glottal stop insertion does not occur
with the vowel-initial form of these enclitics.

The negator is only attested with this extra vowel in my
natural texts when the final vowel is also deleted
while the connectors have been attested with an initial
and final vowel, as well as just with an initial vowel.
Examples of the connectors with both vowels are given in
\qf{ex:130821-1, 7.20} and \qf{ex:130902-1, 3.15 ch:ph},
while examples of each enclitic only with an initial vowel
are given in \qf{ex:130821-1, 7.20}--\qf{ex:130825-6, 2.50}.

\begin{exe}
	\ex{\glll		ees naiʔ Nimrot =\tbr{ama} ain Fina\\
							esa naiʔ Nimrot =\tbr{ma} ain Fina\\
							{\esc} {\naiq} Nimrod =and mother Fina\\
			\glt		`It was Nimrod and Fina.' \txrf{130821-1, 7.20}
							\emb{130821-1-07-20.mp3}{\spk{}}{\apl}}\label{ex:130821-1, 7.20}
	\ex{\glll				ta-bsooʔ ta-mfa{\tl}faun =\tbr{ate} es{\tl}ees reʔ ia, \\
							ta-bsoʔo ta-mfa{\tl}faun =\tbr{te} es{\tl}esa reʔ ia\\
							\ta-dance \ta-{\prd}many ={\te} {\prd}{\es} {\req} {\ia}\\
			\glt		`When we all dance together one by one like this,'\txrf{130902-1, 3.15}
							\emb{130902-1-03-15-part.mp3}{\spk{}}{\apl}}\label{ex:130902-1, 3.15 ch:ph}
	\ex{\glll				neki=n =\tbr{am} na-ʔsoosʔ=ein a|n-bi Oeʔsao\\
							neki=n =\tbr{ma} na-ʔsosaʔ=eni {\a}n-bi Oeʔsao\\
							take={\einV} =and \na-sell={\ein} \a\n-{\bi} Oe{\Q}sao\\
			\glt		`(they) take them and sell them in Oe{\Q}sao.' \txrf{120715-1, 1.14}
							\emb{120715-1-01-14.mp3}{\spk{}}{\apl}}\label{ex:120715-1, 1.14 ch:ph}
	\ex{\glll		ʔnakaʔ fauk =\tbr{at} nine-f esa =t hoo m-ak: teun\\
							ʔnakaʔ fauk =\tbr{te} nine-f esa =t hoo m-ak tenu\\
							head how.many ={\te} wing-{\f} one ={\te} {\hoo} \m-say three\\
			\glt		`(I asked) How many heads, then one wing? you said: three.' \txrf{130914-1, 0.47}
							{\emb{130914-1-00-47.mp3}{\spk{}}{\apl}}}\label{ex:130914-1, 0.47}
	\ex{\glll		au, au u-krei, au ʔ-\sf{kisasi} =t ka= batuur =\tbr{af}.\\
							au au u-krei, au ʔ-\sf{kisasi} =te ka= batuur =\tbr{fa}.\\
							{\au} {\au} \q-church {\au} \q-service ={\te} {\ka}= true ={\fa}\\
			\glt		`I, I went to church, I went to services, it's not true.'\txrf{130825-6, 2.50}
							{\emb{130825-6-02-50.mp3}{\spk{}}{\apl}}}\label{ex:130825-6, 2.50}
\end{exe}

The vowel-initial forms of the connectors occasionally induce metathesis on their host
in the same way as vowel-initial enclitics (\srf{sec:VowIniEnc}).
This is uncommon in my database,
with only about half a dozen unambiguous cases.\footnote{
		Many putative cases of \ve{=ama} and \ve{=ate}
		inducing metathesis on their host are ambiguous as they could be analysed
		as a combination of the determiner \ve{=aa} followed by the connector.}
Two examples are given in \qf{ex:130913-1, 0.35} and \qf{ex:120923-2, 5.38} below.

\begin{exe}
	\ex{\glll	iin aanh=ein na-ʔroo=n =am of ne\tbr{em}=n =\tbr{at} of he m--\\
						ini anah=eni na-ʔroo=n =ma of ne\tbr{ma}=n =te of he m--\\
						{\iin} child={\ein} \na-far={\einV} =and later \nema\tbrMv={\einV} ={\te} later {\he} {}\\
			\glt	`his children are far away, later when they come,{\ldots}'
						\txrf{130913-1, 0.35}{\emb{130913-1-00-35.mp3}{\spk{}}{\apl}}}\label{ex:130913-1, 0.35}
	\ex{\glll	kaah =te, reʔ iin na-papaʔ =ma iin n-si\tbr{ir}k =\tbr{am} iin n-nao piut\\
						kaah =te reʔ ini na-papaʔ =ma iin n-si\tbr{ri}k =ma ini n-nao piut\\
						{\kaah} ={\te} {\req} {\iin} \na-wound =and {\iin} \n-spread{\tbrMv} =and {\iin} \n-go forever\\
			\glt	`If not, s/he is wounded and it (the wound) spreads and keeps going (= growing).'
						\txrf{120923-2, 5.38}{\emb{120923-2-05-38.mp3}{\spk{}}{\apl}}}\label{ex:120923-2, 5.38}
\end{exe}

The frequencies of the different forms of each of these enclitics
after vowel-final and consonant-final hosts
in my corpus of texts are given in \trf{tab:FreForSenEnc}.\
As can be seen from this table, the full form of the negator \ve{=fa}
and the connector \ve{=te} are most frequent in all environments.
While the full form of \ve{=ma} is overall most common,
after vowels the form \ve{=m} is slightly more common.

\begin{table}[h]
	\centering\caption{Frequency of the forms of sentence enclitics}\label{tab:FreForSenEnc}
	\stl{0.4em}\begin{tabular}{llllll|lllll|llll}\lsptoprule
	&	\ve{ma}	&	\ve{m}	&	\ve{ama}	&	\ve{am}	& tot.&	\ve{te}	&	\ve{t}	&	\ve{ate}	&	\ve{at}	& tot.&	\ve{fa}	&	\ve{f}	&	\ve{af}	&tot.\\ \midrule
	V{\gap}	&	233	&	263	&	0	&	0	&	496	&	332	&	233	&	0	&	0	&	565	&	70	&	38	&	0	&	108	\\
	C{\gap}	&	121	&	18	&	17	&	73	&	229	&	66	&	18	&	38	&	30	&	151	&	153	&	0	&	2	&	155	\\
	all	&	354	&	281	&	17	&	72	&	724	&	398	&	251	&	38	&	30	&	716	&	223	&	38	&	2	&	263	\\
		\lspbottomrule
	\end{tabular}
\end{table}
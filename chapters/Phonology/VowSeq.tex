\subsubsection{Vowel sequences}\label{sec:VowSeq}
Amarasi allows a maximum of two vowels to surface adjacent to one another.
Every sequence of two vowels occurs in Amarasi,
with the exception of a high vowel followed by a mid vowel.
Attested sequences are given in \trf{tab:AmaVowSeq} below,
with frequencies in my dictionary of 1,975 unique roots given in \trf{tab:VowSeqFre}.
All the sequences given in \trf{tab:AmaVowSeq},
with the exception of /ou/,
have been attested in underlying U-forms.
That is, only the sequence /ou/ has so far been
attested exclusively in metathesised words.

\begin{table}[h]
	\caption{Amarasi vowel sequences}\label{tab:AmaVowSeq}
	\begin{subtable}[b]{0.49\textwidth}
		\centering\caption{Attested vowel sequences}\label{tab:AttVowSeq}
			\stl{0.5em}\begin{tabular}{c|cccccl} \lsptoprule
		V\sub{1}{\da}	&i	&e	&a	&o	&u	&	$\underleftarrow{\textrm{V\sub{2}}}$\\ \midrule
				i				&\ve{ii}&\ve{	}&\ve{ia}&\ve{	}	&\ve{iu}&\\
				e				&\ve{ei}&\ve{ee}&\ve{ea}&\ve{eo}&\ve{eu}&\\
				a				&\ve{ai}&\ve{ae}&\ve{aa}&\ve{ao}&\ve{au}&\\
				o				&\ve{oi}&\ve{oe}&\ve{oa}&\ve{oo}&\ve{ou}&\\
				u				&\ve{ui}&\ve{	}&\ve{ua}&\ve{	}	&\ve{uu}&\\	\lspbottomrule
			\end{tabular}
	\end{subtable}
	\begin{subtable}[b]{0.49\textwidth}
		\centering\caption{Frequencies}\label{tab:VowSeqFre}
			\stl{0.5em}\begin{tabular}{c|cccccl} \lsptoprule
		V\sub{1}{\da}	&i	&e	&a	&o	&u	&	$\underleftarrow{\textrm{V\sub{2}}}$\\ \midrule
				i					&19	&		&9	&		&10	&	\\
				e					&16	&28	&5	&27	&13	&	\\
				a					&80	&37	&35	&19	&49	&	\\
				o					&17	&39	&6	&37	& 	&	\\
				u					&17	&		&35	&		&26	&	\\	\lspbottomrule
			\end{tabular}
	\end{subtable}
\end{table}

One distinctive phonetic feature of both varieties of Meto
spoken around the Amarasi area,
is centralisation of /a/ when followed by a high vowel.
This is most common in the sequence /au/,
but does also occur in the sequence /ai/.
This centralisation does not affect sequences of /au/ or /ai/ resulting from metathesis.
Examples are given in \qf{ex:aCen} below.

\begin{exe}
	\ex{/a/ {\ra} [ə] / {\gap}V\tsc{[+high]}\label{ex:aCen}}
		\sn{\gw\begin{tabular}{llll}
			\ve{\tbr{a}u}		&[ˈʔ\tbr{ə}ʊ]		&{\emb{au.mp3}{\spk{}}{\apl}}&`\tsc{1sg}' \\
			\ve{sek\tbr{a}u}&[sɛˈk\tbr{ə}w] &{\emb{sekau.mp3}{\spk{}}{\apl}}&`who' \\
		\end{tabular}}
\end{exe}

Alternately, the first element of the sequence /ai/ is often fronted to [ɛ].
These sequences are generally kept distinct from underlying sequences of /e/+/i/,
which are usually realised as [ei] according to the regular rule of mid vowel raising
before high vowels (see rule \qf{ex:VRaising} on \prf{ex:VRaising}).
Raising of /a/ to [ɛ] before /i/ does not occur in careful speech.
The examples in \qf{ex:aRai} below were extracted from texts.

\begin{exe}
	\ex{/a/ {\ra} [ɛ] / {\gap}i \label{ex:aRai}}
		\sn{\gw\begin{tabular}{llll}
			\ve{n-mur\tbr{ai}}&[n͡mʊˈɾ\tbr{ɛj}]	&{\emb{nmurai.mp3}{\spk{}}{\apl}}	& `begins' \\
			\ve{m\tbr{ai}nuan}&[m\tbr{ɛj}ˈnʊɐn]	&{\emb{mainuan.mp3}{\spk{}}{\apl}}& `open(ness), freedom' \\
		\end{tabular}}
\end{exe}

The second vowel of sequences beginning with /i/ is often fronted.
This might only happen before apical consonants,
seen in \qf{ex:VowFro} before the voiceless apical sibilant /s/.

\begin{exe}
	\ex{/V/ {\ra} [V̙] / i{\gap}\label{ex:VowFro}}
	\sn{\gw\begin{tabular}{llll}
		\ve{a|n-ki\tbr{u}s}	&[ʔanˈki\tbr{ʉ}s]	&{\emb{ankius.mp3}{\spk{}}{\apl}}& `sees' \\
		\ve{a|n-ki\tbr{a}s}	&[ʔanˈki\tbr{æ}s]	&{\emb{ankias.mp3}{\spk{}}{\apl}}& `sees' (see {\S}\ref{sec:HeiDisKor}) \\
	\end{tabular}}
\end{exe}

The mid-back vowel /o/ often dissimilates in backness and rounding from a following high vowel.
This results in either a centralised rounded or unrounded vowel,
as conditioned by the rounding quality of the following high vowel:

\begin{exe}
	\ex{/o/ {\ra} [β\tsc{back}, β\tsc{round}] /
	{\gap}V[+\tsc{high}, {\A}\tsc{back}, {\A}\tsc{round}] \label{ex:oFro}}
		\sn{\gw\begin{tabular}{llll}
			\ve{a|n-t\tbr{o}it}&[ʔan̪ˈt̪\tbr{ɵ}it̪]&{\emb{antoit.mp3}{\spk{}}{\apl}}& `asks' \\
			\ve{a|n-t\tbr{o}up}&[ʔan̪ˈt̪{\tbr{\"ɤ}}ʊp]&{\emb{antoup-Roni.mp3}{\spk{}}{\apl}}& `receives'  \\
	\end{tabular}}
\end{exe}

\paragraph{Double Vowels}\label{sec:DouVow}
In normal speech a sequence of two identical vowels always
coalesces into a single phonetic syllable with a
single phonetically long or half-long vowel.
Examples are given in \qf{ex:VV>V:} below.

\begin{exe}
	\ex{/V{\sA}V{\sA}/ {\ra} [Vː] \label{ex:VV>V:}}
	\sn{\gw\begin{tabular}{llll}
		\ve{a|n-s\tbr{ii}}	&[ʔanˈs\tbr{iː}]	&{\emb{ansii.mp3}{\spk{}}{\apl}}& `sings'  \\
		\ve{f\tbr{ee}}			&[f\tbr{ɛː}]			&{\emb{fee.mp3}{\spk{}}{\apl}}& `wife' \\
		\ve{h\tbr{aa}}			&[h\tbr{aː}]			&{\emb{haa.mp3}{\spk{}}{\apl}}	& `four' \\
		\ve{\tbr{oo}}				&[ʔ\tbr{ɔː}]			&{\emb{oo.mp3}{\spk{}}{\apl}}		& `bamboo' \\
		\ve{t\tbr{uu}-f}		&[t̪\tbr{ʊˑ}f]	&{\emb{tuuf.mp3}{\spk{}}{\apl}}	& `knee' \\
	\end{tabular}}
\end{exe}

An alternate analysis of such data would be to propose that
sequences of two identical vowels are underlyingly long vowels;
that is a single vowel linked to two morae.
Each of these analyses is shown in for \ve{fee} `wife'
in \qf{as:Syl:fee} and \qf{as:Syl:fe:} respectively.

\begin{multicols}{2}
	\begin{exe}
			\ex{Analysis 1: /fee/ `wife'}\label{as:Syl:fee}
			\sna{\xy
				<0.5em,3cm>*\as{σ}="s1",<2em,3cm>*\as{σ}="s2",
				<0em,2cm>*\as{C}="C1",<1em,2cm>*\as{V}="V1",<2em,2cm>*\as{V}="V2",
				<0em,1cm>*\as{f}="c1",<1em,1cm>*\as{e}="v1",<2em,1cm>*\as{e}="v2",
				<2em,0cm>*\as{}="empty",
				"c1"+U;"C1"+D**\dir{-};"v1"+U;"V1"+D**\dir{-};"v2"+U;"V2"+D**\dir{-};
				"C1"+U;"s1"+D**\dir{-};"V1"+U;"s1"+D**\dir{-};"V2"+U;"s2"+D**\dir{-};
			\endxy}
			\ex{Analysis 2: /feː/ `wife'}\label{as:Syl:fe:}
			\sna{\xy
				<1em,3cm>*\as{σ}="s1",<1em,2cm>*\as{μ}="m1",<2em,2cm>*\as{μ}="m2",
				<0em,1cm>*\as{C}="C1",<1.5em,1cm>*\as{V}="V1",
				<0em,0cm>*\as{f}="c1",<1.5em,0cm>*\as{e}="v1",
				"m1"+U;"s1"+D**\dir{-};"m2"+U;"s1"+D**\dir{-};
				"c1"+U;"C1"+D**\dir{-};"v1"+U;"V1"+D**\dir{-};
				"C1"+U;"s1"+D**\dir{-};"V1"+U;"m1"+D**\dir{-};"V1"+U;"m2"+D**\dir{-};
			\endxy}
	\end{exe}
\end{multicols}

The reason for analysing such data as representing
a sequence of two identical vowels rather than a single long vowel,
is that, with the exception of their phonetic realisation,
sequences of two identical vowels behave identically
in every respect to sequences of two different vowels.
This is true of stress assignment (\srf{sec:Str}), reduplication (\srf{sec:Red})
and every other process of the language.

One process which illustrates well the fact that 
sequences of two identical vowels behave
identically to sequences of two different vowels
is glottal stop infixation whereby the second part of each of
the nominalising circumfixes
\ve{ma-{\ldots}-ʔ} `property nominalisation' (\srf{sec:PropCir})
and \ve{ʔ-{\ldots}-ʔ} `object nominalisation' (\srf{sec:NomQ--q})
occurs as an infix between the vowels of a final vowel sequence.
Examples are given in \qf{ex1:NomCirInf} below.

\begin{exe}
	\ex{Circum-/Infixes \ve{ʔ-{\ldots}\<ʔ\>} and \ve{ma-{\ldots}\<ʔ\>}}\label{ex1:NomCirInf}
	\sn{\gw\stl{0.45em}\begin{tabular}{rlcrcll}
			`covers'			&\ve{n-neo} 	&+&\ve{ʔ-{\ldots}-ʔ}	&\ra& \ve{ʔ-ne\<ʔ\>o} & `umbrella'\\
%			`pounds'			&\ve{n-pau} 	&+&\ve{ʔ-{\ldots}-ʔ}	&\ra& \ve{ʔ-pa\<ʔ\>u} & `mortar and pestle'\\
			`writes'			&\ve{n-tui} 	&+&\ve{ʔ-{\ldots}-ʔ}	&\ra& \ve{ʔ-tu\<ʔ\>i} & `pen'\\
			`writes'			&\ve{n-tui} 	&+&\ve{ma-{\ldots}-ʔ}	&\ra& \ve{ma-tu\<ʔ\>i} & `written'\\
			`be aware'		&\ve{na-keo} 	&+&\ve{ma-{\ldots}-ʔ}	&\ra& \ve{ma-ke\<ʔ\>o} & `aware'\\
			`believes'		&\ve{n-pirsai}&+&\ve{ma-{\ldots}-ʔ}	&\ra& \ve{ma-pirsa\<ʔ\>i} & `believing'\\
			`sings'				&\ve{n-sii} 	&+&\ve{ʔ-{\ldots}-ʔ}	&\ra& \ve{ʔ-si\<ʔ\>i} & `song'\\
			`wife'				&\ve{fee} 		&+&\ve{ma-{\ldots}-ʔ}	&\ra& \ve{ma-fe\<ʔ\>e} & `having a wife'\\
			`leaf'				&\ve{noo-f} 	&+&\ve{ma-{\ldots}-ʔ}	&\ra& \ve{ma-no\<ʔ\>o} & `leafy'\\
			`base'				&\ve{uu-f} 		&+&\ve{ma-{\ldots}-ʔ}	&\ra& \ve{ma-ʔu\<ʔ\>u} & `based'\\
	\end{tabular}}
\end{exe}

If words with a sequence of two identical vowels such as \ve{fee} `wife'
were analysed as having a single long vowel, the insertion of a glottal
stop in forms such as \ve{ma-fe\<ʔ\>e} `having a wife'
is completely unexpected; one segment should not be able to occur inside another.
However, if such words have a sequence of two vowels,
then this behaviour is simply explained by the second element of these prefixes
occurring between the two vowel segments.

A second reason for not analysing sequences of two identical
vowels as phonemically long vowels is that under such an analysis
every other vowel sequence (except for high vowels
followed by mid vowels) would be attested
with no apparent reason why sequences of two identical vowels do not occur.

While sequences of two identical vowels usually coalesce into a single
phonetic syllable, each vowel is still treated as the nucleus of a separate syllable
as regards to every phonological and morphophonemic process of the language.
The only difference between sequences of two identical vowels
and sequences of two different vowels is the frequency with which phonetic coalescence occurs:
coalescence is almost universal for sequences of two identical vowels
and only optional for sequences of two different vowels.
Vowel coalescence is discussed further in \srf{sec:Syl}

\paragraph{Kotos height dissimilation}\label{sec:HeiDisKor}
In Kotos Amarasi the second vowel of a sequence in which both vowels
have the same height but different backness is often realised as /a/.
This rule can apply to all sequences of two mid vowels,
but only to sequences of two high vowels followed by a consonant.
Examples are given in \qf{ex:V->a} below.

\begin{exe}
	\ex{ V$\left[\begin{array}{l} 
					\textnormal{\A}\tsc{high} \\
					\textnormal{β}\tsc{back} \\ \end{array} \right]$
					{\ra} /a/ /V$\left[\begin{array}{l}
					\textnormal{-\A}\tsc{high} \\
					\textnormal{-β}\tsc{back} \\ \end{array} \right]${\gap}}\label{ex:V->a}
	\sn{\gw\begin{tabular}{rll|rlll}
		\mc{3}{l|}{General Amarasi}	&\multicolumn{3}{l}{Kotos Amarasi}					& \\
		\ve{ri\tbr{u}ksaen}&[ri\tbr{ʊ}kˈsaɛn]	&{\emb{riuksaen.mp3}{\spk{}}{\apl}}&\ve{ri\tbr{a}ksaen}	&[ri\tbr{a}kˈsaɛn]&{\emb{riaksaen.mp3}{\spk{}}{\apl}}&`python' \\
		\ve{se\tbr{o}}			&[ˈsɛ\tbr{ɔ}]			&{\emb{seo.mp3}{\spk{}}{\apl}}&\ve{se\tbr{a}}			&[ˈsɛ\tbr{a}]			&{\emb{sea.mp3}{\spk{}}{\apl}}&`nine' \\
		\ve{o\tbr{e}}			&[ˈʔɔ\tbr{ɛ}]				&{\emb{oe.mp3}{\spk{}}{\apl}}&\ve{o\tbr{a}}				&[ˈʔɔ\tbr{ɐ}]			&{\emb{oa.mp3}{\spk{}}{\apl}}&`water' \\
	\end{tabular}}
\end{exe}

This vowel dissimilation is perceived as distinctly peculiar to Kotos Amarasi
and words such as \ve{oa} `water' are viewed by Kotos speakers,
as well as outsiders, as emblematic of this variety. %\footnote{
		%The distinctive and emblematic nature of Kotos height dissimilation
		%is further shown by the fact that Kotos Amarasi speakers sometimes
		%hyper correct forms with an original /Va/ sequence when speaking to
		%speakers of other varieties of Meto.
		%One observed example is \ve{noe metoʔ} `dry coconut' {\la} \ve{noa metoʔ}
		%{\la} \ve{noah} `coconut' + \ve{metoʔ} `dry'
		%uttered by an Amarasi speaker when talking to a speaker of Amfo{\Q}an.}
This height dissimilation does not occur in Ro{\Q}is Amarasi or Amabi.\footnote{
		Speakers of Kotos Amarasi report that this vowel dissimilation operates
		to different extents in different Kotos speaking villages and hamlets.
		Thus, for instances, inhabitants of the hamlet of Koro{\Q}oto
		have \ve{oef}{\tl}\ve{oaf} `soup' while inhabitants of Fo{\Q}asa{\Q}
		are reported to only have \ve{oef} `soup'.}

In some lexemes this rule also operates across an intervening glottal stop.
The lexemes in my database in which this has been recorded are \ve{kreʔo} {\ra} \ve{kreʔa} `a bit',
and \ve{seʔo} {\ra} \ve{seʔa} `ninth'.

\paragraph{Quantification of vowel sequence length}\label{sec:QuaLenVowSeq}
The lengths of vowels and vowel sequences where one of the vowels of the sequence was stressed
were measured in polysyllabic words from four texts of a single speaker.
The vowels to be measured were marked in Praat with a text-grid and the lengths extracted with a script.
The measurements for vowels of words with a distinctive pause intonation, as well as pronouns,
which are often unstressed, were excluded from the data set.

This yielded a total 1,249 measurements.
Of these 472 tokens were of a single vowel,
314 represented a sequence of two identical vowels
and 463 represented a sequence of two different vowels.
The results are summarised in \trf{tab:VowLenAma}.

\begin{table}[h]%For significance http://xkcd.com/1478/
	\centering\caption{Vowel lengths in Amarasi}\label{tab:VowLenAma}
		\begin{tabular}{rrrrr}\lsptoprule
								&\multicolumn{1}{c}{V}& \multicolumn{1}{r}{V{\sA}V{\sA}}
								&\multicolumn{1}{r}{V{\sA}V{\sB}}& \multicolumn{1}{c}{all}\\ \midrule
%			sum all tokens (sec.)	&46.414			&40.373			&64.006			&150.793	\\
			average length (sec.)	&\tbr{0.098}&\tbr{0.129}&\tbr{0.138}&0.121	\\
			number of tokens			&472				&314				&463				&1,249		\\
			standard deviation		&0.034			&0.05				&0.061			&0.055		\\ 
			t-test (vs. V)				&						&p <0.001	&p <0.001		&\\ \lspbottomrule
%			t-test (V{\sA}V{\sA})	&						&						&p =0.0159 		&\\
		\end{tabular}
\end{table}

This table shows that a sequence of two different vowels is on average 41{\%} longer than a single vowel,
while a sequence of two identical vowels is on average 31{\%} longer than a single vowel.
These differences are statistically significant as shown by a two tailed t-test.

A sequence of two identical vowels is distinct from a single vowel.
However, it is not the case that every instance of a sequence of two identical vowels
will always be phonetically longer than a single vowel.
Other factors, such as sentence stress and intonation,
can conspire to increase or decrease the phonetic length of any particular token of a vowel or vowel sequence.

Compare examples \qf{ex:EacOfThe} and \qf{ex:TheOldMen} from the same speaker.
In example \qf{ex:EacOfThe} the vowel sequence of the word \ve{fee} `wife' measures 0.141 seconds;
above the average for a sequence of two identical vowels.
However, the same vowel sequence in the same word
in sentence \qf{ex:TheOldMen} measures 0.083 seconds;
below the average for a single vowel.

\begin{exe}
\let\eachwordone=\textnormal \let\eachwordtwo=\ve
\ex{\glll	[ʔɛsʔɛsə \hp{=}{t̪}̚  nɔk ʔɪ̰n ˈ\tbr{fɛː}\sub{0.141} ɪ̰n mɔnɛ̤]\\
						\hp{[}es{\tl}esa =t n-ok iin \tbr{fee} iin mone \\
						\hp{[}{\frd}one ={\te} {\n-\ok} {\iin} wife {\iin} man \\
				\glt \lh{[}`each [of them] with their wife or their husband{\ldots}' \txrf{130928-1, 2.09}
						{\emb{130928-1-02-09.mp3}{\spk{}}{\apl}}}\label{ex:EacOfThe}
\ex{\glll	[wə̪ n̰a̰\sarc{ɛ}f \hp{=}m̩ \tbr{f\"ɛ}\sub{0.083} mnasɪ̰ ʔa̰ɾɛ̰ a̰nɐ̤ˈanɐ̤ nɐβ̞o\sarc{ʌ}n m̩]\\
					\hp{[}ahh ʔnaef =am \tbr{fee} mnasiʔ areʔ anah{\tl}anah na-bua=n{\ldots} \\
						{} old.man and wife old all {\frd}child {\na}-gather={\einV} \\
				`the old men and woman, all the children gathered' \txrf{130902-1, 3.52}
						{\emb{130902-1-03-52.mp3}{\spk{}}{\apl}}}\label{ex:TheOldMen}
\end{exe}

The word \ve{fee} is shortened in \qf{ex:TheOldMen}
as it is the first word of a modified noun phrase,
and thus does not take primary stress (\srf{sec:Str}).

\subsubsection{Loan vowel nativisation}\label{sec:LoaVowNat}
The most common non-native vowel which occurs in loan-words is the vowel /ə/.
This vowel is reflected as /a/ in Amarasi as shown by
Dutch \emph{lezen} /leːz\tbr{ə}/ {\textgreater} Amarasi \ve{n-res\tbr{a}} `reads'.
Instances of Malay /ə/ are also reflected as /a/,
though in many cases these could be borrowings from Kupang Malay
in which proto-Malay *ə usually became /a/.
One example is \ve{p\tbr{a}rikas} `to examine' < Malay \emph{p\tbr{e}riksa}
/p\tbr{ə}riksa/ or Kupang Malay \emph{p\tbr{a}riksa}.
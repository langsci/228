\subsection{Consonant inventory}\label{sec:Con}
Amarasi has thirteen phonemic consonants to draw on to fill a C-slot.
These consonants are shown in \trf{tab:AmaCon}.
The symbols used in my phonemic transcription are given in \trf{tab:AmaConBroTra}.
These consonants are phonetically realised with the standard IPA values
associated with the symbols given in \trf{tab:AmaConNarTra},
with common allophones discussed below.

\begin{table}[h]
	\caption{Amarasi consonants}\label{tab:AmaCon}
	\begin{subtable}[b]{0.49\textwidth}
		\caption{Broad transcription}\label{tab:AmaConBroTra}
		\centering
			\begin{tabular}{r|cc@{\hspace{2.3mm}}c@{\hspace{2.3mm}}cc} \lsptoprule
				\begin{sideways}\phantom{Alveolar}\end{sideways}&\begin{sideways}Labial\end{sideways}&\begin{sideways}Coronal\end{sideways}&
				\begin{sideways}Dorsal\end{sideways}&\begin{sideways}Glottal\end{sideways}  \\ \midrule
				Plosives		&\ve{p}	&\ve{t}		&\ve{k}		&\ve{ʔ} \\
				Obstruents	&\ve{b}	&\ve{\j}	&\ve{gw}	&			\\
				Fricatives	&\ve{f}	&\ve{s}		&					&\ve{h}\\
				Nasals			&\ve{m}	&\ve{n}		&					&			\\
				Liquid			&				&\ve{r}		&					&			\\	\lspbottomrule
			\end{tabular}
	\end{subtable}
	\begin{subtable}[b]{0.49\textwidth}
		\caption{Narrow transcription}\label{tab:AmaConNarTra}
		\centering
			\begin{tabular}{ccc@{\hspace{2mm}}c@{\hspace{2mm}}c@{\hspace{2mm}}c} \lsptoprule
				\begin{sideways}Labial\end{sideways}&\begin{sideways}Dental\end{sideways}&\begin{sideways}Alveolar\end{sideways}&
				\begin{sideways}Palatal\end{sideways}&\begin{sideways}Velar\end{sideways}&\begin{sideways}Glottal\end{sideways}  \\ \midrule
						p		&t̪	&		&			&k			&ʔ	\\
						b/β	&			&		&\j/ʒ	&ɡw/ɣw	&		\\
						f		&			&s	&			&				&h	\\
						m		&			&n	&			&				&		\\
								&			&r	&			&				&		\\	\lspbottomrule
			\end{tabular}
	\end{subtable}
\end{table}

The liquid /r/ is realised as an alveolar trill [r], tap [ɾ],
or occasionally as an alveolar approximant [ɹ].
In the speech of some speakers it is usually preceded by a voiceless component phrase initially,
as shown in \qf{ex:VoiclessR}.

\newpage
\begin{exe}
	\ex{/r/ {\ra} [hr] {\tl} [r] /{\#}{\gap}\label{ex:VoiclessR}}
		\sn{\gw\begin{tabular}{llll} 
						\ve{\tbr{r}uman}	& [ˈ\tbr{hr}ʊmɐn] &{\emb{ruman.mp3}{\spk{}}{\apl}}& `empty'  \\
						\ve{\tbr{r}uru-f} & [ˈ\tbr{hɾ}ʊɾʊf] &{\emb{ruruf.mp3}{\spk{}}{\apl}}& `lips' \\
						\ve{\tbr{r}ekaʔ}	& [ˈ\tbr{hɾ}ɛkɐʔ] &{\emb{rekaq.mp3}{\spk{}}{\apl}}& `when?' \\
		\end{tabular}}
\end{exe}

No known Meto variety has a voiced alveolar plosive /d/ in native vocabulary.
[d] only occurs in Amarasi epenthetically between /n/ and /r/.
Likewise, epenthetic [b] often occurs between /m/ and /r/.
Examples are given in \qf{ex:dEpenthesis}.

\begin{exe}
	\ex{N[{\A}\tsc{place}] {\ra} N[{\A}\tsc{place}]P[{\A}\tsc{place}] /{\gap}r \label{ex:dEpenthesis}}
		\sn{\gw\begin{tabular}{llll}
			\ve{a|\tbr{n}-\tbr{r}ooʔ} 	& [ʔaˈ\tbr{ndɾ}ɔːʔ]				&{\emb{anrooq.mp3}{\spk{}}{\apl}}					& `spews' \\
			\ve{a|\tbr{n}-\tbr{r}eruʔ}	& [ʔaˈ\tbr{ndɾ}eɾʊʔ]			&{\emb{anreruq.mp3}{\spk{}}{\apl}}				& `is tired' \\
			\ve{ʔmuik su\tbr{mr}iriʔ} 	& [ʔmʊiksuˈ\tbr{mbr}irɪʔ]	&{\emb{qmuik-sumririq.mp3}{\spk{}}{\apl}}	& `k.o. small lime' \\
		\end{tabular}}
\end{exe}

The alveolar nasal /n/ assimilates to the place of a following
obstruent in non-careful speech,
with the exception of the labial plosives /p/ and /b/,
before which such assimilation has not been observed in Amarasi.\footnote{
		Assimilation of /n/ to [m] or [ɱ] before labial
		obstruents occurs in other Meto varieties.}
Examples are given in \qf{ex:nAss} below.

\begin{exe}
	\ex{/n/ {\ra} [{\A}\tsc{place}] / {\gap}P[{\A}\tsc{place}] \label{ex:nAss}}
		\sn{\gw\begin{tabular}{llll}
				\ve{a|\tbr{n}-tuup} 		& [ʔa\tbr{n̪}ˈt̪ʊːp]		&{\emb{antuup.mp3}{\spk{}}{\apl}}& `sleeps'  \\
				\ve{a|\tbr{n}-{\j}air} 	& [ʔa\tbr{ɲ}ˈ{\j}aer]	&{\emb{anjair.mp3}{\spk{}}{\apl}}& `becomes' \\
				\ve{ba\tbr{n}kofaʔ} 		& [bɐ\tbr{ŋ}ˈkɔfɐʔ]		&{\emb{bankofaq.mp3}{\spk{}}{\apl}}& `caterpillar' \\
				\ve{tu\tbr{n}gwuru} 		& [t̪ʊ\tbr{ŋ}ˈɡʊɾʊ]		&{\emb{tungguru.mp3}{\spk{}}{\apl}}& `teacher' \\
		\end{tabular}}
\end{exe}

The voiceless dorsal plosive /k/ is often palatalised before or after a front vowel.
Two examples are given in \qf{ex:PalatalK} below.

\begin{exe}
	\ex{/k/ {\ra} [kʲ] /{\gap}V\tsc{[+fr]}, V\tsc{[+fr]}{\#}{\gap} \label{ex:PalatalK}}
		\sn{\gw\begin{tabular}{llll}
			\ve{u\tbr{k}i}		& [ˈʔʊ\tbr{kʲ}i]	&{\emb{uki.mp3}{\spk{}}{\apl}}& `banana'  \\
			\ve{n-ei\tbr{k}}	& [nej\tbr{kʲ}]		&{\emb{neik.mp3}{\spk{}}{\apl}}& `takes' \\
		\end{tabular}}
\end{exe}

The glottal stop /ʔ/ can be reduced to
creaky voice on surrounding voiced segments.
This is most common in rapid speech.
Two examples from texts are given in
\qf{ex:LitBitAft} and \qf{ex:WhiSenTo} below.

\begin{exe}
\let\eachwordone=\textnormal \let\eachwordtwo=\itshape
	\ex{\glll [ɛ̰ːː ndɹɛʊk hit̪ʊ ŋkɔnɔ kɾ\tbr{ɛ̰ɔ̰}] \\
						\hp{[}ehh n-reuk, hitu n-kono kre\tbr{ʔ}o \\
					%	{} n-reku hitu n-kono kre\tbr{ʔ}o \\
						{} {\n}-pluck seven {\n}-pass little\\
			\glt \lh{[ehh}`a little bit after it struck seven o'clock.'  \txrf{130920-1, 0.47}
			\emb{130920-1-00-47.mp3}{\spk{}}{\apl}}\label{ex:LitBitAft}
	\ex{\glll [ɾ\tbr{ɛ̰} {\a}n̩sɔo\tbr{n̰}ɛ nɛʊ ɐ̰bit̪ɐn hɾɔmɐ] \\
						\hp{[}re\tbr{ʔ} a|n-soun\tbr{ʔ}=ee n-eu a-bi-t=an Roma \\
					%	\hp{[}re\tbr{ʔ} {\a}n-sonu\tbr{ʔ}=ee n-eu a-bi-t=an \*Roma \\
						\hp{[}{\req} {\a\n}-send={\eeV} {\n}-{\eu} {\at}-{\bi}-{\at}={\einV} Roman \\
			\glt \lh{[}`which [he] sent to the inhabitants of Rome.'  \txrf{130920-1, 0.27}
			\emb{130920-1-00-27.mp3}{\spk{}}{\apl}} \label{ex:WhiSenTo}
\end{exe}

The labio-dental fricative /f/ in Amarasi is usually articulated
with the lower part of the lip touching the teeth,
rather than with the top/outer part of the lip, as in English.

\subsubsection{Voiced obstruents}\label{sec:VoiObs}
The voiced obstruents /\j/ and /ɡw/ are marginal phonemes with a limited distribution.
In native vocabulary they only occur as a result of vowel features
spreading into empty \mbox{C-slots},
under the process of consonant insertion at clitic boundaries
(\srf{sec:EmpCSloConIns}, \srf{sec:ConIns}).

In Koro{\Q}oto the voiced velar obstruent /ɡw/
is not followed by a labio-velar glide before
the back rounded vowels /u/ and /o/.
Examples are given in \qf{ex:gw->g}.

\begin{exe}
	\ex{/ɡw/ {\ra} [ɡ] /{\gap}V\tsc{[+round]}}\label{ex:gw->g}
		\sn{\gw\stl{0.275em}\begin{tabular}{rcllllll}
			\ve{na-kneʔo}&+&\ve{=oo-n}	&\ra&\ve{na-kneeʔ\tbr{gw}=oo-n}	&[nakˈnɛːʔ\tbr{ɡ}ɔn]		&{\emb{nakneeqg-on.mp3}{\spk{}}{\apl}}&`twisted'\\
			\ve{na-tinu}&+&\ve{=oo-n}	&\ra&\ve{na-tiin\tbr{gw}=oo-n}		&[naˈt̪iːŋ\tbr{ɡ}ɔn]	&{\emb{natiingg-on.mp3}{\spk{}}{\apl}}&`worries'\\
																&&& &\ve{tun\tbr{gw}uru}				&[t̪ʊŋˈ\tbr{ɡ}ʊɾʊ]		&{\emb{tungguru.mp3}{\spk{}}{\apl}}&`teacher'\\
		\end{tabular}}
\end{exe}

An alternate analysis of the same data would be to posit that
this obstruent is underlyingly unrounded,
and acquires rounding before unrounded vowels:
/ɡ/ {\ra} [ɡw] /{\gap}V\tsc{[-round]}.
However, such a rule is phonetically unmotivated,
while the rule in \qf{ex:gw->g} in which a rounded obstruent
is de-rounded before rounded vowels is a phonetically natural rule of dissimilation.
Despite the fact that it is non-distinctive,
from now on I transcribe the unrounded allophone
of /ɡw/ as \it{<}\ve{g}\it{>} throughout this book.\footnote{
		In the variety of Kotos Amarasi spoken in the hamlet of Fo{\Q}asa{\Q} the voiced velar obstruent is never rounded,
		and for this variety of Amarasi I posit the phoneme /ɡ/ rather than /ɡw/.
		Fo{\Q}asa{\Q} /ɡ/ also occurs in a wider range of environments than Koro{\Q}oto /ɡw/.
		In Fo{\Q}asa{\Q} Kotos Amarasi /ɡ/
		is inserted at clitic boundaries after vowel-final stems.
		See \srf{sec:FoqConIns} for more details.}

Apart from instances arising from consonant insertion,
the voiced obstruents /\j/ and /ɡw/ also occur in loan words.
Examples include \ve{a|n-\j air} `become' {\la} Malay \emph{jadi}
and \ve{tu{\ng}guru} `teacher' {\la} Malay \emph{tuan + guru}.
In some loans /\j/ is adapted as /r/
and /ɡ/ as /k/ (\srf{sec:LoaConNat}).

The voiced obstruents are realised as stops [b {\j} ɡw],
fricatives [β ʒ ɣw], or approximants [β̞ j ɰw].
In many environments the alternation is a case of free variation,
however, in certain environments either the stop or the continuant
(fricative and approximant) allophones are more common.
A count was made of the realisations of every voiced
obstruent in three texts for my main consultant, Roni.
The results are summarised in \trf{tab:FreStoConRea} below.

\begin{table}[h]
	\centering\caption[Frequency of stop and continuant realisations]
	{Frequency of stop and continuant realisations\su{†}}\label{tab:FreStoConRea}
		\begin{threeparttable}[b]
		\begin{tabular}{rrcccc} \lsptoprule
									&							&V{\gap}	& N{\gap} & C{\gap} & {\#}{\gap} \\ \midrule
			continuant:	&[β ʒ ɣw]			&61				&0				&23				&5 \\
			stop: 			&[b {\j} ɡw]	&23				&12				&9				&7 \\
			stop \%			&							&27\%			&100\%		&28\%			&58\% \\ \lspbottomrule
		\end{tabular}
		\begin{tablenotes}
		\item [†] V{\gap} is post-vocalic; both V{\gap}C and V{\gap}V,
							N{\gap} is after a homorganic nasal and C{\gap} is
							after other consonants
		\end{tablenotes}
	\end{threeparttable}
\end{table}

\trf{tab:FreStoConRea} shows that, for Roni,
continuant allophones are dominant after vowels and consonants,
while they are do not occur after homorganic nasals.
Only phrase initially are stop allophones slightly more common,
though this could be an artefact of the tiny data sample in this environment.

Examples of both realisations of the bilabial obstruent /b/
taken from Roni's speech are given in \qf{ex:WePre}--\qf{ex:InAFor} below.
In \qf{ex:WePre} and \qf{ex:WeWenAlo} the bilabial
obstruent /b/ is pronounced as a plosive [b].
In \qf{ex:WePre} the plosive occurs between two vowels
and in \qf{ex:WeWenAlo} it occurs after a homorganic nasal.

\begin{exe}
\let\eachwordone=\textnormal \let\eachwordtwo=\itshape
\ex{\glll [hɛj mi\tbr{b}aɾɐβ] \\
					\hp{[}hai mi-\tbr{b}arab\\
					\hp{[}{\hai} {\mi}-prepare\\
			\glt {\leavevmode\hp{[}}`We prepared,' \txrf{130902-1, 4.23}
			\emb{130902-1-04-23.mp3}{\spk{}}{\apl}}\label{ex:WePre}
\ex{\glll [haj mɔkə m\tbr{b}i ɾɛ̰ ɛ̰æː kosʊʔ] \\
					\hp{[}hai m-oka m-\tbr{b}i reʔ ahh kosuʔ\\
					\hp{[}{\hai} {\m}-{\ok} {\m}-{\bi} {\reqt} {} dance.kind\\
			\glt {\leavevmode\hp{[}}`We joined in with the \it{kosu{\Q}} dance.' \txrf{130902-1, 2.59}
			\emb{130902-1-02-59.mp3}{\spk{}}{\apl}}\label{ex:WeWenAlo}
\end{exe}

Examples of the bilabial obstruent /b/ realised as a fricative [\B]
are given in \qf{ex:HeyWheIts} and \qf{ex:InAFor} below.
In \qf{ex:HeyWheIts} it occurs between two vowels
and in \qf{ex:InAFor} it occurs before another consonant.
Example \qf{ex:InAFor} also shows a both an affricate and fricative realisation of /\j/.

\begin{exe}
\let\eachwordone=\textnormal \let\eachwordtwo=\itshape
\ex{\glll [hɛ mansɛ nma\sarc{ɛ}\tbr{β}ɛ \hp{=}t̪ɛ] \\
						\hp{[}heʔ maans=ee n-mae\tbr{b}=ee =te\\
					%	\hp{[}heʔ manas=ee n-mabeʔ=ee =te \\
						\hp{[}hey sun={\ee} {\n}-afternoon={\eeV} ={\te}\\
			\glt	\lh{[}`hey, when it was the afternoon {\ldots}' \txrf{130928-1, 1.41}
			\emb{130928-1-01-41.mp3}{\spk{}}{\apl}}\label{ex:HeyWheIts}
\ex{\glll [kɐs{\tS}ɛ nɾaˑ\tbr{β}ʒɛ \hp{=}t̪ nak \hp{``}masɔ min̪t̪a] \\
						\hp{[}kaas\j=ee n-raa\tbr{b}\j=ee =t n-ak: \sf{``maso} \sf{minta}''\\
					%	\hp{[}kase=ee n-rabi=ee =te n-ak \sf{\hp{``}maso} \sf{minta}\\
						\hp{[}foreign={\ee} {\n}-speak.foreign={\eeV} ={\te} {\n-\ak} \hp{``}enter ask\\
			\glt	\lh{[}`In a foreign language they call it
						`enter to ask'.' \txrf{130902-1, 0.35}
			\emb{130902-1-00-35.mp3}{\spk{}}{\apl}}\label{ex:InAFor}
\end{exe}

\subsubsection{Consonant frequencies}\label{sec:ConCou}

A count of the frequency of each consonant was
carried out on my current dictionary of 2,005
unique roots (including bound morphemes).
This yielded a total of 5,063 consonants,
the frequencies of which are given in \trf{tab:ConFre} in order of frequency.

\begin{table}[h]
	\centering\caption{Consonant frequencies}\label{tab:ConFre}
		\stl{0.4em}\begin{tabular}{rccccccccccccc} \lsptoprule
	C		&\ve{ʔ}	&\ve{n}	&\ve{k}	&\ve{t}	&\ve{s}	&\ve{r}	&\ve{b}	&\ve{m}	&\ve{p}	&\ve{f}	&\ve{h}	&\ve{\j}&\ve{gw}	\\ \midrule
	no.	&	858		&	816		&	601		&	560		&	503		&	474		&	332		&	305		&	251		&	208		&	142		&	11		&	2				\\
			&	17\%	&	16\%	&	12\%	&	11\%	&	10\%	&	9\%		&	7\%		&	6\%		&	5\%		&	4\%		&	3\%		&	0.2\%	&	0.04\%	\\	\lspbottomrule
		\end{tabular}
\end{table}

As can be seen from \trf{tab:ConFre}, the voiced obstruents /\j/ and /ɡw/
are extremely infrequent in my corpus.
This provides additional evidence for their marginal status within the phoneme inventory.
This table also shows that the glottal stop /ʔ/ is the most common consonant.
This is despite the fact that it was not consistently transcribed
in some earlier descriptions of Meto, notably those of \citeauthor{mi39}.

\subsubsection{Loan consonant naturalisation}\label{sec:LoaConNat}
The naturalisation of non-native consonants in Amarasi is summarised in \trf{tab:NatForPho}.
The phonemes /{\j}/ and /ɡ/ in loanwords are usually adapted into Amarasi as
/r/ or /k/ respectively, though in a small number of cases they undergo no change.
Concerning the phoneme /\j/ (for which more examples are available),
some words, such as `become' shown in \trf{tab:NatForPho},
have variants reflecting both /r/ and /\j/,
while other words such as \ve{baru} {\textless} Malay \emph{baju} `shirt' (ultimately from Persian) and
\ve{\j eket} {\textless} Malay \emph{jeket} {\textless} English \emph{jacket} have only one form.
That these phonemes are often naturalised in Amarasi is additional evidence that they are marginal phonemes.

\begin{table}[h]
	\centering\caption{Naturalisation of foreign consonants in Amarasi}\label{tab:NatForPho}
		{\begin{tabular}{cllllll} \lsptoprule
						&			& 				& source 					& Amarasi 			& Donor 		& Meaning		\\ \midrule
			/w/		&\ra	&\ve{b}		& \emph{kawin}		& \ve{kabin}		& via Malay	& `wedding'	\\ 
			/ŋ/		&\ra	&\ve{n}		& \emph{sidang}		& \ve{siran}		& via Malay	& `meeting'	\\ 
			/d/		&\ra	&\ve{r}		& \emph{duit}			& \ve{roit}			& Dutch			& `money'		\\ 
			/l/		&\ra	&\ve{r}		& \emph{lezen}		& \ve{n-resa}		& Dutch			& `read'		\\ 
			/\j/	&\ra	&\ve{r}		& \emph{jadi}			& \ve{n-rari}		& via Malay & `become'	\\
			/\j/	&\ra	&\ve{\j}	& \emph{jadi}			& \ve{n-\j ari}	& via Malay & `become'	\\  
			/ɡ/		&\ra	&\ve{k}		& \emph{igreja}		& \ve{krei}			& Portuguese& `church'	\\ 
			/ɡ/		&\ra	&\ve{gw}	& \emph{tuan guru}& \ve{tuŋguru}	& via Malay & `teacher'	\\ 
			/\tS/	&\ra	&\ve{s}		& \emph{percaya}	& \ve{n-pirsai}	& via Malay & `believe'	\\	\lspbottomrule
		\end{tabular}
	}
\end{table}
\subsection{Roots with two feet}\label{sec:RooTwoFee}
Roots with two feet constitute 6{\%} (106/1,913) of my current corpus.
Of such roots, the medial C-slot of the initial foot
is usually unfilled surfacing as (C)VV(C).
This is attested in 69{\%} (73/106) of instances in my corpus.
With the exception of loans, all other roots with
two feet have an initial foot with the structure (C)VCV
which is followed by a medial consonant cluster.

With these considerations in mind,
and observing the constraint against sequences of three vowels,
the maximal structure of words with two feet is either
(C)VV(C)|CV(C)V(C) or (C)VCVC|CV(C)V(C).
Examples are given in \trf{tab:WorTwoFee}.

\begin{table}[h]
	\centering
	\caption[Roots with two feet]{Roots with two feet\su{†}}\label{tab:WorTwoFee}
		\begin{threeparttable}
				\stl{0.26em}\begin{tabular}{llllllr} \lsptoprule
				Structure						&Phonemic				&Phonetic				&		&gloss					&no.&\%\\ \midrule
				(C)VV|CVCV(C)				&\ve{paumakaʔ}	&[ˌpɐwˈmakɐʔ]		&{\emb{paumakaq.mp3}{\spk{}}{\apl}}	&`near'						&38	&36\%\\
				(C)VVC|CVCV(C)			&\ve{meisʔokan}	&[ˌmɛ̞jsˈʔɔkɐn]	&{\emb{meisqokan.mp3}{\spk{}}{\apl}}&`dark(ness)'			&21	&20\%\\
				(C)VVC|CV\hp{C}V(C)	&\ve{riuksaen}	&[ˌriʊkˈsaɛn]		&{\emb{riuksaen.mp3}{\spk{}}{\apl}}	&`python'					&10	&9\%\\
				(C)VCVC|CVCV(C)			&\ve{ataʔraʔe}	&[ˌʔat̪aʔˈraʔɛ]	&{\emb{ataqraqe.mp3}{\spk{}}{\apl}}	&`praying mantis'	&17	&16\%\\
				(C)VCVC|CV\hp{C}V(C)&\ve{paratrao}	&[ˌparat̪ˈraɔ]		&{\emb{paratrao.mp3}{\spk{}}{\apl}}	&`kingfisher'			&9	&8\%\\
				(C)VV|CV\hp{C}V(C)	&\ve{nai\j eer}	&[ˌnajˈ\j ɛːr]	&{\emb{naijeer.mp3}{\spk{}}{\apl}}	&`ginger'					&5	&5\%\\ \lspbottomrule
				\end{tabular}%}
			\begin{tablenotes}
				\item [†]	In addition to the structures given in this Table,
									there are three words with two feet and an initial cluster:
									\ve{ʔbeebnisaʔ} `centipede', \ve{ʔhoesaif} `ditch',
									and \ve{ʔkauboe} `rattan goad'. There are also three
									words with an exceptional root structure:
									\ve{n-ʔantareek} `reverse' from Dutch \it{achteruit/aantrekken},
									\ve{n-ʔistarika} `ironing' from Dutch \it{strijken} and
									\ve{n-sikarotiʔ} `hyperactive' from one of the Rote
									languages, e.g. Termanu \it{sikiloti, sikaloto} \citep[544]{jo08}.
			\end{tablenotes}
		\end{threeparttable}
\end{table}

The constraints which apply to the initial foot in words with two feet
are due to this foot being an M\=/form; that is,
the form taken by nouns with a following attributive modifier.
This means that roots with two feet have a prosodic structure
identical to that of a modified nominal phrase (\srf{sec:Str})
and all roots with two feet are probably historic compounds.

In some instances one element of the historic phrase is still attested in Amarasi as an independent root.
Three probable examples include \ve{saanʔoo} `stick insect',
from unattested \ve{\tcb{*}sana} with \ve{oo} `bamboo',\footnote{
		Charles Grimes (p.c. July 2016) points out that
		initial \ve{\tcb{*}sana} could be connected with PMP
		*saŋa `bifurcation, fork of a branch' \cite{bltr}.}
\ve{faifsosoʔ} `kind of plant fed to pigs', from \ve{fafi}
`pig' with unattested \ve{\tcb{*}sosoʔ},
and \ve{enosneer} `window', from \ve{enoʔ} `door'
with unattested independent \ve{\tcb{*}sneer}.\footnote{
		The final part of \ve{enosneer} `window' is
		from Portuguese \it{janela} /ʒanɛla/ `window' (\srf{sec:PhoNat}).}

However, in many cases neither of the putative compound elements
are currently attested elsewhere in Amarasi.
Two examples are \ve{suufneneʔ} `tree snake' and \ve{meisʔokan} `dark(ness)'.
More exhaustive data on other varieties of Meto and languages of the region
may reveal cognates for some of these otherwise unattested elements.

Finally, there are four roots in my current database with five syllables.
These are: \ve{baatbosʔoo} `antlion', \ve{n-maʔautuu} `ram (verb)',
\ve{n-ʔakaʔbi\j aʔe} `walk on one's hands and feet' and \ve{ai\j onuus} `kind of herb'.
Of these the last two are clearly historically polymorphemeic.
The root \ve{n-ʔakaʔbi\j aʔe} is from a prefix \ve{ʔakaʔ},
attested unproductively on a few other verbs and \ve{bi\j ae} `cow',
while \ve{ai\j onuus} `kind of herb' is from \ve{ai\j oʔo} `casuarina tree'
combined with \ve{nuus} which has no independent use in Amarasi but
means `blue' in some other varieties of Meto.

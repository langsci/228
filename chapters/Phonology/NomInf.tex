\subsubsection{Glottal stop infixation}\label{sec:NomInf}
One piece of evidence for empty C-slots in Amarasi
is the behaviour of the nominalising circumfix \ve{ʔ-{\ldots}-ʔ} (\srf{sec:NomQ--q})
and the property circumfix \ve{ma-{\ldots}-ʔ} (\srf{sec:PropCir}).
When these circumfixes attach to a surface CVCV root,
the initial element occurs as a prefix and the second element as a suffix.
Examples are given in \qf{ex:NomCir}.

\begin{exe}
	\ex{Circumfixes \ve{ʔ-{\ldots}-ʔ} and \ve{ma-{\ldots}-ʔ}}\label{ex:NomCir}
	\sn{\stl{0.45em}\gw\begin{tabular}{rlcrcll}
			`grate'				&\ve{		{\rt}fona}	&+&\ve{ʔ-{\ldots}-ʔ}	&\ra& \ve{ʔ-fona-ʔ}		&`grater'\\
			`bind' 				&\ve{		{\rt}futu}	&+&\ve{ʔ-{\ldots}-ʔ}	&\ra& \ve{ʔ-futu-ʔ}		&`cloth band'\\
			`sit'				 	&\ve{		{\rt}toko}	&+&\ve{ʔ-{\ldots}-ʔ}	&\ra& \ve{ʔ-toko-ʔ}		&`chair'\\
			`sweep' 			&\ve{		{\rt}sapu}	&+&\ve{ʔ-{\ldots}-ʔ}	&\ra& \ve{ʔ-sapu-ʔ}		&`broom'\\
			`hear'				&\ve{		{\rt}nena}	&+&\ve{ma-{\ldots}-ʔ}	&\ra& \ve{ma-nena-ʔ}	& `heard'\\
			`receive'			&\ve{		{\rt}topu}	&+&\ve{ma-{\ldots}-ʔ}	&\ra& \ve{ma-topu-ʔ}	& `received'\\
			`stone, rock' &\ve{\hp{\rt}fatu}	&+&\ve{ma-{\ldots}-ʔ}	&\ra& \ve{ma-fatu-ʔ}	&`stony, rocky'\\
			`hair' 				&\ve{\hp{\rt}funu-}	&+&\ve{ma-{\ldots}-ʔ}	&\ra& \ve{ma-funu-ʔ}	&`hairy'\\
			`key'					&\ve{\hp{\rt}retuʔ}	&+&\ve{ma-{\ldots}-ʔ}	&\ra& \ve{ma-retu-ʔ}	& `locked'\\
			`thorn'				&\ve{\hp{\rt}aikaʔ}	&+&\ve{ma-{\ldots}-ʔ}	&\ra& \ve{ma-ʔaika-ʔ}	& `thorny'\\
		\end{tabular}}
\end{exe}

When these circumfixes occur on a root with a final vowel sequence,
the second glottal stop occurs between these two vowels as an infix.
Examples are given in \qf{ex:NomCirInf} to illustrate.

\begin{exe}
	\ex{Circum-/Infixes \ve{ʔ-{\ldots}\<ʔ\>} and \ve{ma-{\ldots}\<ʔ\>}}\label{ex:NomCirInf}
	\sn{\stl{0.35em}\gw\begin{tabular}{rlcrcll}
			`cover'		&\ve{		{\rt}neo}		&+&\ve{ʔ-{\ldots}-ʔ}	&\ra& \ve{ʔ-ne\<ʔ\>o}			& `umbrella'\\
			`pound'		&\ve{		{\rt}pau}		&+&\ve{ʔ-{\ldots}-ʔ}	&\ra& \ve{ʔ-pa\<ʔ\>u}			& `mortar and pestle'\\
			`exit'		&\ve{		{\rt}poi}		&+&\ve{ʔ-{\ldots}-ʔ}	&\ra& \ve{ʔ-po\<ʔ\>i}			& `exit (noun)'\\
			`sing'		&\ve{		{\rt}sii}		&+&\ve{ʔ-{\ldots}-ʔ}	&\ra& \ve{ʔ-si\<ʔ\>i}			& `song'\\
			`write'		&\ve{		{\rt}tui}		&+&\ve{ʔ-{\ldots}-ʔ}	&\ra& \ve{ʔ-tu\<ʔ\>i}			& `pen'\\
			`write'		&\ve{		{\rt}tui}		&+&\ve{ma-{\ldots}-ʔ}	&\ra& \ve{ma-tu\<ʔ\>i}		& `written'\\
			`aware'		&\ve{		{\rt}keo}		&+&\ve{ma-{\ldots}-ʔ}	&\ra& \ve{ma-ke\<ʔ\>o}		& `aware'\\
			`believe'	&\ve{		{\rt}pirsai}&+&\ve{ma-{\ldots}-ʔ}	&\ra& \ve{ma-pirsa\<ʔ\>i}	& `believing'\\
			`wife'		&\ve{\hp{\rt}fee} 	&+&\ve{ma-{\ldots}-ʔ}	&\ra& \ve{ma-fe\<ʔ\>e}		& `having a wife'\\
			`leaf'		&\ve{\hp{\rt}noo-f}	&+&\ve{ma-{\ldots}-ʔ}	&\ra& \ve{ma-no\<ʔ\>o}		& `leafy'\\
			`base'		&\ve{\hp{\rt}uu-f} 	&+&\ve{ma-{\ldots}-ʔ}	&\ra& \ve{ma-ʔu\<ʔ\>u}		& `based'\\
	\end{tabular}}
\end{exe}

Under an analysis involving empty C-slots,
the infixed allomorph can be captured by proposing
that the circumfix is fundamentally a prefix
with the second element occupying the first available
empty C-slot from the left edge of the word.

When the medial C-slot of a root is already filled
the first available empty C-slot is word final,
as shown in \qf{as:qtokoq} below for \ve {ʔ-toko-ʔ} `chair'.
When the root contains a vowel sequence
the first available empty C-slot is root medial,
as shown in \qf{as:qsiqi} below for \ve {ʔ-si\<ʔ\>i} `song'.

\newpage
\begin{multicols}{2}
\begin{exe}
	\exa{\xy
		<0pt,3cm>*\as{ʔ}="q1",<5em,3cm>*\as{ʔ}="q2",
		<0pt,2cm>*\as{C}="c0",<0.5em,2cm>*\as{|}="|",<1em,2cm>*\as{C}="c1",<2em,2cm>*\as{V}="v1",<3em,2cm>*\as{C}="c2",<4em,2cm>*\as{V}="v2",<5em,2cm>*\as{C}="c3",
		<1em,1cm>*\as{t}="pc1",<2em,1cm>*\as{o}="pv1",<3em,1cm>*\as{k}="pc2",<4em,1cm>*\as{o}="pv2",
		<2.5em,0cm>*\as{`sit'}="m1",<2.5em,4cm>*\as{\tsc{nmlz}}="m2",
		"q1"+U;"m2"+D**\dir{-};"q2"+U;"m2"+D**\dir{-};
		"m1"+U;"pc1"+D**\dir{-};"m1"+U;"pc2"+D**\dir{-};"m1"+U;"pv1"+D**\dir{-};"m1"+U;"pv2"+D**\dir{-};
		"c0"+U;"q1"+D**\dir{-};"c3"+U;"q2"+D**\dir{-};
		"pc1"+U;"c1"+D**\dir{-};"pc2"+U;"c2"+D**\dir{-};"pv1"+U;"v1"+D**\dir{-};"pv2"+U;"v2"+D**\dir{-};
	\endxy}\label{as:qtokoq}
	\exa{\xy
		<0pt,3cm>*\as{ʔ}="q1",<3em,3cm>*\as{ʔ}="q2",
		<0pt,2cm>*\as{C}="c0",<0.5em,2cm>*\as{|}="|",<1em,2cm>*\as{C}="c1",<2em,2cm>*\as{V}="v1",<3em,2cm>*\as{C}="c2",<4em,2cm>*\as{V}="v2",<5em,2cm>*\as{C}="c3",
		<1em,1cm>*\as{s}="pc1",<2em,1cm>*\as{i}="pv1",<3em,1cm>*\as{}="pc2",<4em,1cm>*\as{i}="pv2",
		<2.5em,0cm>*\as{`sing'}="m1",<1.5em,4cm>*\as{\tsc{nmlz}}="m2",
		"q1"+U;"m2"+D**\dir{-};"q2"+U;"m2"+D**\dir{-};
		"m1"+U;"pc1"+D**\dir{-};"m1"+U;"pv1"+D**\dir{-};"m1"+U;"pv2"+D**\dir{-};
		"c0"+U;"q1"+D**\dir{-};"c2"+U;"q2"+D**\dir{-};
		"pc1"+U;"c1"+D**\dir{-};"pv1"+U;"v1"+D**\dir{-};"pv2"+U;"v2"+D**\dir{-};
	\endxy}\label{as:qsiqi}
\end{exe}
\end{multicols}
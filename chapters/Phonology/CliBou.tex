\section{Enclitics}\label{sec:CliBou}
There are a number of morphophonemic processes
associated with enclitics in Amarasi.
Firstly, the host of a vowel-initial enclitic
undergoes a number of processes including
consonant insertion, metathesis, and vowel assimilation.
These processes are the focus of Chapter \ref{ch:PhoMet},
but I provide a brief overview in this section.

Secondly, some enclitics have multiple forms
partly dependent on whether they attach to a vowel-final
or consonant-final host. Clitics with multiple forms
include the plural enclitic \ve{=ein} (\srf{sec:PluEnc2})
and the sentence enclitics \ve{=ma} `and', \ve{=te}
`\tsc{sub}', and \ve{=fa} `\tsc{neg}'.

\subsection{Vowel-initial enclitics}\label{sec:VowIniEnc}
When vowel-initial enclitics are attached to a CVC{\#} final stem,
the stem undergoes metathesis. Examples are given in \qf{ex:CVC=V->VCC=V} below.

\begin{exe}
	\ex{C\sub{1}VC\sub{2}+=V {\ra} VC\sub{1}C\sub{2}+=V}\label{ex:CVC=V->VCC=V}
		\sn{\gw\begin{tabular}{rlllll}
			\ve{ra\tbr{mu}p}&+&\ve{=ee}&{\ra}&\ve{ra\tbr{um}p=ee}	& `the light' \\
			\ve{mu\tbr{ʔi}t}&+&\ve{=ee}&{\ra}&\ve{mu\tbr{iʔ}t=ee}	& `the animal' \\
			\ve{te\tbr{nu}k}&+&\ve{=ee}&{\ra}&\ve{te\tbr{un}k=ee}	& `the umbrella' \\
			\ve{te\tbr{no}ʔ}&+&\ve{=ee}&{\ra}&\ve{te\tbr{on}ʔ=ee}	& `the egg' \\
			\ve{u\tbr{ku}m}&+&\ve{=ee}&{\ra}&\ve{u\tbr{uk}m=ee}	& `the cuscus' \\
%			\ve{\tbr{}}&+&\ve{=ee}&{\ra}&\ve{=ee}	& `the' \\
	\end{tabular}}
\end{exe}

When a vowel-initial enclitic is attached to a vowel-final stem,
insertion of \ve{\j} or \ve{gw} occurs at the enclitic boundary.
The consonant \ve{\j} is inserted after the front vowels \ve{i} and \ve{e} 
and \ve{gw} is inserted after the back vowels \ve{u} and \ve{o}.
The final consonant and vowel of the stem metathesise,
and the final vowel then assimilates to the quality of the previous vowel.
Examples are given in \qf{ex:VaCVbr+=V->VaVaCb=V -2} below,
with the enclitic \ve{=ee} `{\ee}'.

\begin{exe}
	\ex{V{\sA}CV{\sB}+=V {\ra} V{\sA}V{\sA}CC{\sB}=V}\label{ex:VaCVbr+=V->VaVaCb=V -2}
		\sn{\gw\begin{tabular}{rlllll}
%			\ve{kbit\tbr{i}}&+&\ve{=ee}&{\ra}&\ve{kbiit\tbr{\j}=ee}	& `the scorpion' \\
			\ve{kren\tbr{i}}&+&\ve{=ee}&{\ra}&\ve{kre\tbr{e}n\tbr{\j}=ee}	& `the ring' \\
%			\ve{faf\tbr{i}}	&+&\ve{=ee}&{\ra}&\ve{fa\tbr{a}f\tbr{\j}=ee}	& `the pig' \\
			\ve{on\tbr{i}} 	&+&\ve{=ee}&{\ra}&\ve{o\tbr{o}n\tbr{\j}=ee}		& `the bee; the sugar' \\
			\ve{uk\tbr{i}} 	&+&\ve{=ee}&{\ra}&\ve{u\tbr{u}k\tbr{\j}=ee}		& `the banana' \\
%			\ve{kep\tbr{e}} &+&\ve{=ee}&{\ra}&\ve{keep\tbr{\j}=ee}	& `the tick (parasite)' \\
%			\ve{bar\tbr{e}} &+&\ve{=ee}&{\ra}&\ve{ba\tbr{a}r\tbr{\j}=ee}	& `the stuff' \\
			\ve{nop\tbr{e}}	&+&\ve{=ee}&{\ra}&\ve{no\tbr{o}p\tbr{\j}=ee}	& `the cloud' \\
			\ve{bik\tbr{u}} &+&\ve{=ee}&{\ra}&\ve{bi\tbr{i}k\tbr{gw}=ee}& `the curse' \\
			\ve{tef\tbr{u}} &+&\ve{=ee}&{\ra}&\ve{te\tbr{e}f\tbr{gw}=ee}& `the sugar-cane' \\
%			\ve{fat\tbr{u}} &+&\ve{=ee}&{\ra}&\ve{fa\tbr{a}t\tbr{gw}=ee}& `the stone' \\
			\ve{nop\tbr{u}} &+&\ve{=ee}&{\ra}&\ve{no\tbr{o}p\tbr{gw}=ee}& `the grave' \\
%			\ve{hut\tbr{u}} &+&\ve{=ee}&{\ra}&\ve{huut\tbr{gw}=ee}			& `louse' \\
			\ve{nef\tbr{o}} &+&\ve{=ee}&{\ra}&\ve{ne\tbr{e}f\tbr{gw}=ee}	& `the lake' \\
%			\ve{knaf\tbr{o}}&+&\ve{=ee}&{\ra}&\ve{kna\tbr{a}f\tbr{gw}=ee}	& `the mouse' \\
%			\ve{kor\tbr{o}} &+&\ve{=ee}&{\ra}&\ve{koor\tbr{gw}=ee}				& `the bird' \\
		\end{tabular}}
\end{exe}

All the morphophonemic processes which occur with vowel-initial
enclitics are described and analysed
in full detail in Chapter \ref{ch:PhoMet},
particularly \srf{sec:ConIns} and \srf{sec:VowAss}.

\subsection{Plural enclitic}\label{sec:PluEnc2}
The plural enclitic has two main allomorphs,
\ve{=ein} after consonant-final stems
and \ve{=n} after CV{\#} final stems.
Examples of each are given in \qf{ex:pl->-ein/C -2}
and \qf{ex:pl->-n/CV -2} below.
The form of this enclitic and its enclitic host
is discussed in full detail in \srf{sec:PluEnc}.

\newpage
\begin{exe}
	\ex{\{\tsc{pl}\} {\ra} =ein /C{\#}{\gap}}\label{ex:pl->-ein/C -2}
		\sn{\gw\begin{tabular}{rcll}
			\ve{anah}				&\ra&\ve{aanh=ein}				&`children'\\
			\ve{kaes mutiʔ}&\ra&\ve{kaes muitʔ=ein}	&`Europeans'\\
			\ve{enoʔ}			&\ra&\ve{eonʔ=ein}				&`doors'\\
			\ve{tuaf}			&\ra&\ve{tuaf=ein}				&`people'\\
			\ve{kuan}				&\ra&\ve{kuan=ein}				&`villages'\\
			\ve{n-fesat}		&\ra&\ve{n-feest=ein}		&`(they) throw a party'\\
			\ve{na-barab}		&\ra&\ve{na-baarb=ein}	&`(they) prepare'\\
			\ve{n-ʔonen}		&\ra&\ve{n-ʔoenn=ein}	&`(they) pray'\\
			\ve{na-tuin}		&\ra&\ve{na-tuin=ein}		&`(they) follow'\\
%			\ve{}	&\ra&\ve{=n}	&`'\\
%			\ve{}	&\ra&\ve{=n}	&`'\\
	\end{tabular}}
\end{exe}

\begin{exe}
	\ex{\{\tsc{pl}\} {\ra} =n /CV{\#}{\gap}}\label{ex:pl->-n/CV -2}
		\sn{\gw\begin{tabular}{rcll}
			\ve{kase}		&\ra&\ve{kase=n}	&`foreigners'\\
			\ve{hutu}		&\ra&\ve{hutu=n}	&`head-lice'\\
			\ve{kbiti}	&\ra&\ve{kbiti=n}	&`scorpions'\\
			\ve{koro}		&\ra&\ve{koro=n}	&`birds'\\
			\ve{tuni}		&\ra&\ve{tuni=n}	&`eels'\\
			\ve{n-moʔe}	&\ra&\ve{n-moʔe=n}	&`(they) do/make'\\
			\ve{na-tona}	&\ra&\ve{na-tona=n}	&`(they) tell'\\
			\ve{n-eki}		&\ra&\ve{n-eki=n}		&`(they) bring'\\
			\ve{na-hana}	&\ra&\ve{na-hana=n}	&`(they) cook'\\
%			\ve{}	&\ra&\ve{=n}	&`'\\
%			\ve{}	&\ra&\ve{=n}	&`'\\
	\end{tabular}}
\end{exe}

\section{Segmental inventory}\label{sec:SegInv}
%In this section I discuss the Amarasi segmental phonemes.
%Amarasi has five segmental vowels: /i e a o u/ which can fill V-slots
%and thirteen segmental consonants: /p t k ʔ b {\j} ɡw f s h m n r/ which can fill C-slots.

\subsection{Vowel inventory} \label{sec:Vow}
Amarasi has five contrastive vowels.
All lexical roots contain at least two vowels.
These five vowels are given in \trf{tab:AmaVow} below,
with their usual phonetic realisation given in \trf{tab:AmaVowNarTra}.

\begin{table}[h]
	\caption{Amarasi vowels}\label{tab:AmaVow}
  \begin{subtable}[b]{0.49\textwidth}
		\caption{Broad transcription}\label{tab:AmaVowBroTra}
		\centering
			\stl{0.3em}\begin{tabular}{r|ccc} \lsptoprule
							&Front	&Cent.&Back	\\ \midrule
				High	&\ve{i}	&			&\ve{u} \\
				Mid		&\ve{e}	&			&\ve{o}\\
				Low		&\multicolumn{2}{c}{\ve{a}}&\\ \lspbottomrule
			\end{tabular}
  \end{subtable}
  \begin{subtable}[b]{0.49\textwidth}
		\caption{Narrow transcription}\label{tab:AmaVowNarTra}
		\centering
			\stl{0.3em}\begin{tabular}{r|ccc} \lsptoprule
							&Front&Cent.&Back	\\ \midrule
				High	&i		&			&ʊ	\\
				Mid		&ɛ		&			&ɔ	\\
				Low		&\multicolumn{2}{c}{a}&	\\ \lspbottomrule
			\end{tabular}
  \end{subtable}
\end{table}

The vowel /a/ is low and slightly front.
In post stress position it is usually centralised to [ɐ],
in other word positions it is realised as [a],
though centralised realisations are also sometimes heard in pre-stress position.
Examples of this allophony are given in \qf{ex:a->5} below.

\begin{exe}
	\ex{/a/ {\ra} [ɐ] /ˈ{σ}{\gap} \label{ex:a->5}}
		\sn{\gw\begin{tabular}{llll}
			\ve{nu\tbr{a}} 		& [ˈnʊ\tbr{ɐ}] &{\emb{nua.mp3}{\spk{}}{\apl}}& `two'\\
			\ve{nim\tbr{a}} 	& [ˈnim\tbr{ɐ}] &{\emb{nima.mp3}{\spk{}}{\apl}}& `five'\\
			\ve{am\tbr{a}-f}	& [ˈʔam\tbr{ɐ}f] &{\emb{amaf.mp3}{\spk{}}{\apl}}& `father'\\
			\ve{kof\tbr{a}ʔ}	& [ˈkɔf\tbr{ɐ}ʔ] &{\emb{kofaq.mp3}{\spk{}}{\apl}}& `boat'\\
		\end{tabular}}
\end{exe}

\subsubsection{Mid vowels}\label{sec:MidVow}
The mid vowels /e/ and /o/ have mid-high allophones
[e] and [o] when followed by a high vowel in the same word.\footnote{
		While this is the most common realisation of these phonemes in this environment,
		the mid allophones [ɛ] and [ɔ] are also sometimes heard before high vowels.}
This raising is most pronounced for /o/ before labial phonemes,
and most pronounced for /e/ before /s/ and /k/.
Examples are given in \qf{ex:VRaising} below.
In other environments the mid vowels are usually
realised as [ɛ] and [ɔ] respectively.

\begin{exe}
	\ex{V\tsc{[-high,+mid,+low]} {\ra} V\tsc{[+high,+mid]} / {\gap}(C)V\tsc{[+high,-mid]} \label{ex:VRaising}}
		\sn{\gw\begin{tabular}{llll}
			\ve{a|n-r\tbr{e}ruʔ}	& [ʔanˈdɾ\tbr{e}rʊʔ]	&{\emb{anreruq.mp3}{\spk{}}{\apl}}& `is tired'  \\
			\ve{b\tbr{e}tiʔ} 			& [ˈβ\tbr{e}tiʔ] 			&{\emb{betiq.mp3}{\spk{}}{\apl}}	& `fried' \\
			\ve{k\tbr{o}ʔu} 			& [ˈk\tbr{o}ʔʊ]				&{\emb{koqu.mp3}{\spk{}}{\apl}}		& `big' \\
			\ve{\tbr{o}ri-f} 			& [ˈʔ\tbr{o}ɾɪf]			&{\emb{orif.mp3}{\spk{}}{\apl}}		& `younger sibling' \\
		\end{tabular}}
\end{exe}

In some words a kind of vowel harmony operates in which an initial mid vowel is raised
to mid-high and a final high vowel is also lowered to mid-high.
Such pronunciations are identified by my consultants as specific to Koro{\Q}oto hamlet.
Examples are given in \qf{ex:VRaising2} below.
The conditions under which this vowel harmony operates are not yet fully understood,
though could be partially connected with the quality of the consonants of the word.

\begin{exe}
	\ex{\hspace{-3.8pt}V\tsc{[-high,+mid,+low]CV[+high,-mid]} {\ra} V\tsc{[+high,+mid]CV[+high,+mid]}\label{ex:VRaising2}}
		\sn{\gw\begin{tabular}{llll}
			\ve{kr\tbr{e}n\tbr{i}}		& [ˈkr\tbr{e}n\tbr{e}] 		&{\emb{kreni.mp3}{\spk{}}{\apl}}& `ring'  \\
			\ve{b\tbr{e}s\tbr{i}} 		& [ˈb\tbr{e}s\tbr{e}] 		&{\emb{besi.mp3}{\spk{}}{\apl}}& `knife' \\
			\ve{k\tbr{o}b\tbr{i}} 		& [ˈk\tbr{o}\B\tbr{e}] 		&{\emb{kobi.mp3}{\spk{}}{\apl}}& `cabbage' \\
			\ve{tain\tbr{o}n\tbr{u}s} & [tajˈn\tbr{o}n\tbr{o}s]	&{\emb{tainonus.mp3}{\spk{}}{\apl}}& `earthquake' \\
		\end{tabular}}
\end{exe}

There is also at least one word which has a final mid-high vowel,
\ve{enus} {\ra} [ˈʔɛnos] {\emb{enus.mp3}{\spk{}}{\apl}} `rainbow'.
In the metathesised form of this word the second vowel is high,
\ve{enus}+\ve{=ee} {\ra} \ve{euns=ee} {\ra} [ˈʔɛʊnsɛ] {\emb{euns-ee.mp3}{\spk{}}{\apl}}.
This appears to be a case of high vowel lowering in closed syllables.\footnote{
		In the case of \ve{enus} {\ra} [ˈʔɛnos] `rainbow',
		my main consultant, Heronimus Bani (Roni),
		had independently chosen to write this word orthographically as \it{<enous>}
		in the Amarasi Bible translation.
		When I noticed this and asked him about it,
		he explained that he did this because the vowel
		``has the sound both of \it{o} and \it{u}.''}

When a vowel initial enclitic attaches to a vowel final host,
the final vowel conditions insertion of a consonant.
The consonant /\j/ is inserted after the front vowels /i/ and /e/
and /ɡw/ is inserted after the back vowels /u/ and /o/.
The clitic host then undergoes metathesis and the vowel which
conditioned insertion of the consonant assimilates to the quality of the previous vowel.
This process is discussed in full detail in \srf{sec:ConIns}
Four examples are given in \qf{ex:mid} below.

\begin{exe}
	\ex{V\tsc{[+mid](C)V[-high]} + =V {\ra} V\tsc{[+mid]V[+mid]}(C)C=V \label{ex:mid}}
		\sn{\gw\stl{0.4em}\begin{tabular}{rclllllll}
			\ve{n-fee}&+&\ve{=ee}&{\ra}&\ve{n-fee\j=ee}	&\ra&[ˈn̩f\tbr{ɛ}ː\j ɛ]	&{\emb{nfeej-ee.mp3}{\spk{}}{\apl}}& `gives it' \\
			\ve{oe} 	&+&\ve{=ee}&{\ra}&\ve{oo\j=ee}		&\ra&[\tbr{ˈ}ʔɔː{\j\shiftleft{3.2pt}{̥}}ɛ]	&{\emb{ooj-ee.mp3}{\spk{}}{\apl}}& `the water' \\
			\ve{nefo}	&+&\ve{=ee}&{\ra}&\ve{neefgw=ee}	&\ra&[ˈn\tbr{ɛ}ˑfɣwɛ]	&{\emb{neefgw-ee.mp3}{\spk{}}{\apl}}& `the lake' \\
			\ve{oo}		&+&\ve{=ee}&{\ra}&\ve{oogw=ee}		&\ra&[ˈʔ\tbr{ɔ}ːɡwɛ]		&{\emb{oogw-ee.mp3}{\spk{}}{\apl}}& `the bamboo' \\
		\end{tabular}}
\end{exe}

When the penultimate vowel of the clitic host
is a mid vowel which has been raised to mid-high before a high vowel,
the mid-high allophone is usually preserved after consonant insertion and vowel assimilation.
Examples are given in \qf{ex:mid-high} below.

\begin{exe}
	\ex{V\tsc{[+mid,+hi](C)V[+hi]} + =V {\ra} \tsc{V[+mid,+high]V[+mid,+hi](C)C=V}\label{ex:mid-high}}
		\sn{\gw\stl{0.2em}\begin{tabular}{rclllllll}
			\ve{krei} 	&+&\ve{=ee}&{\ra}&\ve{kree\j=ee}	&\ra&[ˈkr\tbr{e}ː\j ɛ]	&{\emb{kreej-ee.mp3}{\spk{}}{\apl}}& `the church/week' \\
			\ve{n-romi} &+&\ve{=ee}&{\ra}&\ve{n-room\j=ee}&\ra&[ˈndɾ\tbr{o}ˑm\j ɛ]	&{\emb{nroomj-ee.mp3}{\spk{}}{\apl}}& `likes it' \\
			\ve{mepu}		&+&\ve{=ee}&{\ra}&\ve{meepgw=ee}	&\ra&[ˈm\tbr{e}ːpɡwɛ]		&{\emb{meepgw-ee.mp3}{\spk{}}{\apl}}& `the work' \\
			\ve{nopu}		&+&\ve{=ee}&{\ra}&\ve{noopgw=ee}	&\ra&[ˈn\tbr{ɔ\sarc{o}}pɡwɛ]		&{\emb{noopgw-ee.mp3}{\spk{}}{\apl}}& `the grave' \\
		\end{tabular}}
\end{exe}

All these facts indicate that Koro{\Q}oto Amarasi is probably either in the process of acquiring
a seven vowel system, or is in the process of losing an original seven vowel system.\footnote{
		Some varieties of Meto are further along the pathway to a full seven vowel system.
		This is partly due to the complete assimilation after
		metathesis seen in these varieties (discussed in \srf{sec:AssOfA}),
		seen for instance in Naitbelak Amfo{\Q}an in which \ve{na-leko} `is good'
		metathesises to [naˈlɛːk] with open-mid [ɛ]
		while \ve{na-henu} `is full' metathesises to [naˈheːn] with close-mid [e].
		See also the discussion in \citet{st93,st96,st96b,st08}
		who adopts a seven vowel analysis for his Miomafo data.}

\subsubsection{High vowels}
The high front vowel /i/ has a lower allophone [ɪ], in several environments:
before the fricative /f/, before a voiceless alveolar consonant followed by a high vowel,
after a voiceless alveolar consonant which is preceded by a front vowel, and when preceding stress.
It also tends to be slightly lower when it occurs after the alveolar fricative /s/.
This rule is given with examples in \qf{ex:i>Iegs} below.

\begin{exe}
\ex{/i/ {\ra} [ɪ] \label{ex:i>Iegs}}
\sn{\begin{tabular}{l|llll}
		\rcl																									&\ve{b\tbr{i}fee}			& [b\tbr{ɪ}ˈfɛː]		&{\emb{bifee.mp3}{\spk{}}{\apl}}& `woman' \\
		\rcl\multirow{-2}{*}{/ {\gap}f}												&\ve{nu\tbr{i}-f}			& [ˈnʊ\tbr{ɪ}f]			&{\emb{nuif.mp3}{\spk{}}{\apl}}& `bone'	\\
																													&\ve{h\tbr{i}tu}				& [ˈh\tbr{ɪ}t̪ʊ]	&{\emb{hitu.mp3}{\spk{}}{\apl}}& `seven' \\
				\multirow{-2}{*}{/ {\gap}{\{}s,t{\}}V\tsc{[+hi]}}	&\ve{s\tbr{i}si}				& [ˈs\tbr{ɪ}sɪ]			&{\emb{sisi.mp3}{\spk{}}{\apl}}& `flesh' \\
		\rcl																									&\ve{nis\tbr{i}f}				& [ˈnis\tbr{ɪ}f]			&{\emb{nisif.mp3}{\spk{}}{\apl}}& `tooth' \\
		\rcl\multirow{-2}{*}{/{\{}s,t{\}}V\tsc{[+fr]}{\gap}}	&\ve{sis\tbr{i}}				& [ˈsis\tbr{ɪ}] 		&{\emb{sisi.mp3}{\spk{}}{\apl}}& `flesh' \\
																													&\ve{b\tbr{i}kaseʔ}		& [b\tbr{ɪ}ˈkasɛʔ]	&{\emb{bikaseq.mp3}{\spk{}}{\apl}}& `horse' \\
\multirow{-2}{*}/{\gap}ˈ{σ}&\ve{r\tbr{i}ʔanaʔ}& [ɾ\tbr{ɪ}ˈʔa̰nɐʔ]&{\emb{riqanaq.mp3}{\spk{}}{\apl}}& `child' \\
		\rcl																									&\ve{s\tbr{i}ʔu-f} 		& [ˈs\tbr{ɪ}ʔʊf]		&{\emb{siquf.mp3}{\spk{}}{\apl}}& `elbow' \\
		\rcl\multirow{-2}{*}{/s{\gap}}												&\ve{mas\tbr{i}k}			& [ˈmas\tbr{ɪ}k]		&{\emb{masik.mp3}{\spk{}}{\apl}}& `salt' \\
	\end{tabular}}
\end{exe}

While the environments in which /i/ is realised
as [ɪ] are rather miscellaneous,
it does not seem possible to unify them into a more general
environment such as ``in (unstressed) closed syllables''.
Examples of unstressed realisations of /i/ as [i]
in closed syllables include \ve{bet\tbr{i}ʔ} {\ra} [ˈβet\tbr{i}ʔ] {\emb{betiq.mp3}{\spk{}}{\apl}} `fried'
and \ve{a|n-to\tbr{i}t} {\ra} [ʔan̪ˈt̪ɵ\tbr{i}t̪] \emb{antoit.mp3}{\spk{}}{\apl} `asks'.
The high back vowel /u/ is realised as [ʊ] in all environments.
Examples include \ve{\tbr{u}ki} {\ra} [ˈʔ\tbr{ʊ}kʲi]{\emb{uki.mp3}{\spk{}}{\apl}} `banana'
and \ve{\tbr{u}ran} {\ra} [ˈʔ\tbr{ʊ}ɾɐn]{\emb{uran.mp3}{\spk{}}{\apl}} `rain'.

\subsubsection{Vowel type frequencies}\label{sec:VowFre}
A count of the frequency of each vowel was carried out on my current
dictionary of 1,975 unique roots (including bound morphemes).
This yielded a total of 4,306 vowels,
the frequencies of which are given in \trf{tab:VowFre}.

\begin{table}[h]
	\centering\caption{Vowel frequencies}\label{tab:VowFre}
		\begin{tabular}{r|ccccc} \lsptoprule
			V		&\ve{i}	&\ve{e}	&\ve{a}	&\ve{o}	&\ve{u}	\\ \midrule
			no.	&714		&757		&1,352	&682		&782		\\
					&17\%		&18\%		&32\%		&16\%		&18\%		\\	\lspbottomrule
		\end{tabular}
\end{table}

As \trf{tab:VowFre} shows, the vowel /a/
is nearly twice as frequent as each other vowel.
The vowel /a/ is also the vowel inserted epenthetically to break
up clusters of more than two consonants (\srf{sec:Epe}),
and it can be considered the default vowel.

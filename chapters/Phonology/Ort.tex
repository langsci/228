\section{Practical orthography}\label{sec:Ort}
A practical orthography for Amarasi has been developed
by the Kupang based Language and Culture Unit (UBB) for writing Amarasi.
This orthography is used in the Amarasi Bible translation \cite{UBB15}
and several early grade Amarasi readers (e.g. \citealp{or16,or16b,or16c}).
This orthography is based on the standard Indonesian orthography.
The correspondence between orthographic letters and phones is given in \trf{tab:AmaOrt}.
The letters \it{<c d l q v w x y z>} only occur in foreign loanwords and names.

\begin{table}[h]
	\centering\caption{Amarasi practical orthography}\label{tab:AmaOrt}
		\stl{0.45em}\begin{tabular}{r|ccccccccccccccccccc}\lsptoprule
		Letter	&\it{a}	&\it{b}	&\it{e}	&\it{f}	&\it{gu/go}	&\it{h}	&\it{i}	&\it{j}	&\it{k}	&\it{\Q}&\it{m}	&\it{n}	&\it{ng}&\it{o}	&\it{p}	&\it{r}	&\it{s}	&\it{t}	&\it{u}\\
		Phone		&a			&b			&e			&f			&gw					&h			&i			&\j			&k			&ʔ			&m			&n			&[ŋ]		&o			&p			&r			&s			&t			&u \\
	\end{tabular}
\end{table}

The digraph \it{<ng>} is only used for assimilation
of /n/ {\ra} [ŋ] before \it{<g>} when a word break does not intervene:
e.g. /tunɡwuru/ {\ra} 

The digraph \it{<ng>} is only used for morpheme internal assimilations of /n/ {\ra} [ŋ].
Clitics are written with a space between them and the clitic host and
voiced obstruents which appear after consonant insertion
are written with the clitic rather than with the host.
Word final clusters of identical consonants (created via metathesis) are not written,
and while speakers agree that word final clusters of a
consonant followed by a glottal stop should be written,
in practice they do so somewhat inconsistently.

There are also a number of (mostly minor) non-phonemic
orthographic practices in place to facilitate
morpheme and word recognition for readers.
Such practices (among others) include writing certain consonants
deleted word finally after metathesis and not writing
the vowel assimilation which occurs after consonant insertion.

Because the primary audience of this book is linguists
rather than native speakers of Amarasi.
I do not use this orthography and instead
transcribe words phonemically with their standard IPA symbols.
I depart from this phonemic transcription in two instances.

Firstly, as discussed in \srf{sec:VoiObs}, the unrounded
allophone [ɡ] of the phoneme /ɡw/ is transcribed \it{<g>}.
Secondly, when the phonetic sequence [ŋɡ] occurs
morpheme internally, I transcribe it \it{<\ng g>} to avoid
confusion with the (non-native) consonant [N].\footnote{
		Morpheme internal /nk/ {\ra} [Nk] is transcribed \it{<nk>};
		e.g. /bankofaP/ {\ra} \ve{bankofaʔ} `caterpillar'.}
Both these deviations from the strictly phonemic transcription
can be seen in the word for `teacher',
which according to my phonemic analysis has the form /tungwuru/,
but is transcribed as \ve{tu\ng guru}.

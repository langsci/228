\subsection{Glottal stop insertion}\label{sec:GloStoIns}
Amarasi has two processes of glottal stop insertion.
One process occurs before vowel-initial stems
after addition of an initial CV syllable.
A second process occurs word initially before all vowels.
In both cases the glottal stop is inserted to provide
either the foot and/or the prosodic word with an onset consonant.

\subsubsection{Glottal stop insertion foot initially}\label{sec:GloStoInsVocPre}
A glottal stop is inserted foot initially
when a CV prefix attaches to a vowel-initial foot.
This insertion can be analysed as occurring because
feet in Amarasi require an onset consonant.
A requirement for an onset is a common
cross-linguistic constraint \citep[111f]{mccpr93,prsm93}.

\begin{table}[ht]
	\centering\caption{Glottal stop insertion at morpheme boundaries}\label{tab:GloStoInsMor}
	\begin{tabular}{llll}\lsptoprule
												&Intransitive	& Transitive 		& \\\midrule
			`enter, go into'	&\ve{n-tama}	&\ve{na-tama}		&`make enter, put inside'\\
			`go up, ascend'		&\ve{n-sae}		&\ve{na-sae-b}	&`put up, lift up'\\
			`push down'				&\ve{n-ʔai}		&\ve{na-ʔai-b}	&`push down'\\
			`rise, get up'		&\ve{n-fena}	&\ve{na-fena-ʔ}	&`raise, get s.o. up'\\
			`drink'						&\ve{n-inu}		&\ve{na-\tbr{ʔ}inu-ʔ}	&`give a drink to s.o.'\\
			`see'							&\ve{n-ita}		&\ve{na-\tbr{ʔ}ita-b}	&`show, make see'\\
			`eat (hard food)'	&\ve{n-eku}		&\ve{na-\tbr{ʔ}eku-ʔ}	&`feed'\\
			`run, flee'				&\ve{n-aena}	&\ve{na-\tbr{ʔ}aena-b}&`chase away, make run'\\
			`pick up'					&\ve{n-aiti}	&\ve{na-\tbr{ʔ}aiti-ʔ}&`pick up'\\
			%`'&\ve{}&\ve{}&`'\\
		\lspbottomrule
	\end{tabular}
\end{table}

This process is clearly exemplified by roots which
take consonantal agreement prefixes when intransitive
and vocalic agreement prefixes when transitive (\srf{sec:VerAgrPre}).
Examples are given in \trf{tab:GloStoInsMor},
which shows several verb pairs which take the third person
agreement prefix \ve{n-} when intransitive and \ve{na-} when transitive.
With \ve{na-}, a glottal stop occurs after the prefix.\footnote{
		Transitive verbs also usually take either
		of the transitive suffixes
		\ve{-ʔ} or \ve{-b} (\srf{sec:TraSuf}).}

The prosodic and morphological structures of \ve{n-ita} `see' and \ve{na-ʔita-b} `show'
are given in \qf{as:nita} and \qf{as:naqitab} respectively.
For \ve{n-ita} `see', the first C-slot of the foot is filled by the prefix \ve{n-}.
This foot thus has an onset consonant and no further processes are needed.
However, for \ve{na-ʔita-b} `show' the prefix is external to the foot
and the first C-slot of this foot is thus filled by an epenthetic glottal stop.
This glottal stop is not linked to any of the morphemes
of this word, as befits its status as a non-meaningful epenthetic segment.

\begin{multicols}{2}
	\begin{exe}
		\exa{\label{as:nita}\xy
				<3em,5cm>*\as{PrWd}="PrWd",<3em,4cm>*\as{Ft}="ft1",
				<2em,3cm>*\as{σ}="s1",<4em,3cm>*\as{σ}="s2",
				<1em,2cm>*\as{C}="CV1",<2em,2cm>*\as{V}="CV2",<3em,2cm>*\as{C}="CV3",<4em,2cm>*\as{V}="CV4",<5em,2cm>*\as{C}="CV5",
				<1em,1cm>*\as{n}="cv1",<2em,1cm>*\as{i}="cv2",<3em,1cm>*\as{t}="cv3",<4em,1cm>*\as{a}="cv4",<5em,1cm>*\as{}="cv5",
				<1em,0cm>*\as{\hp{\sub{1}}M\sub{1}}="m1",<3em,0cm>*\as{\hp{\sub{2}}M\sub{2}}="m2",
				<2em,0cm>*\as{-}="-",
				"m1"+U;"cv1"+D**\dir{-};"m2"+U;"cv2"+D**\dir{-};"m2"+U;"cv3"+D**\dir{-};"m2"+U;"cv4"+D**\dir{-};
				"cv1"+U;"CV1"+D**\dir{-};"cv2"+U;"CV2"+D**\dir{-};"cv3"+U;"CV3"+D**\dir{-};"cv4"+U;"CV4"+D**\dir{-};
				"CV1"+U;"s1"+D**\dir{-};"CV2"+U;"s1"+D**\dir{-};"CV3"+U;"s1"+D**\dir{-};
				"CV3"+U;"s2"+D**\dir{-};"CV4"+U;"s2"+D**\dir{-};"CV5"+U;"s2"+D**\dir{-};
				"s1"+U;"ft1"+D**\dir{-};"s2"+U;"ft1"+D**\dir{-};
				"ft1"+U;"PrWd"+D**\dir{-};
		\endxy}
		\exa{\label{as:naqitab}\xy
				<3.5em,5cm>*\as{PrWd}="PrWd",<5em,4cm>*\as{Ft}="ft1",
				<2em,3cm>*\as{σ}="s1",<4em,3cm>*\as{σ}="s2",<6em,3cm>*\as{σ}="s3",
				<1em,2cm>*\as{C}="CV1",<2em,2cm>*\as{V}="CV2",<3em,2cm>*\as{C}="CV3",<4em,2cm>*\as{V}="CV4",<5em,2cm>*\as{C}="CV5",<6em,2cm>*\as{V}="CV6",<7em,2cm>*\as{C}="CV7",
				<1em,1cm>*\as{n}="cv1",<2em,1cm>*\as{a}="cv2",<3em,1cm>*\as{ʔ}="cv3",<4em,1cm>*\as{i}="cv4",<5em,1cm>*\as{t}="cv5",<6em,1cm>*\as{a}="cv6",<7em,1cm>*\as{b}="cv7",
				<1.5em,0cm>*\as{\hp{\sub{1}}M\sub{1}}="m1",<5em,0cm>*\as{\hp{\sub{2}}M\sub{2}}="m2",<7em,0cm>*\as{\hp{\sub{3}}M\sub{3}}="m3",
				<3em,0cm>*\as{-}="-",<6.5em,0cm>*\as{-}="-2",
				"m1"+U;"cv1"+D**\dir{-};"m1"+U;"cv2"+D**\dir{-};"m3"+U;"cv7"+D**\dir{-};
				"m2"+U;"cv4"+D**\dir{-};"m2"+U;"cv5"+D**\dir{-};"m2"+U;"cv6"+D**\dir{-};
				"cv1"+U;"CV1"+D**\dir{-};"cv2"+U;"CV2"+D**\dir{-};"cv3"+U;"CV3"+D**\dir{-};
				"cv4"+U;"CV4"+D**\dir{-};"cv5"+U;"CV5"+D**\dir{-};"cv6"+U;"CV6"+D**\dir{-};"cv7"+U;"CV7"+D**\dir{-};
				"CV1"+U;"s1"+D**\dir{-};"CV2"+U;"s1"+D**\dir{-};"CV3"+U;"s1"+D**\dir{-};
				"CV3"+U;"s2"+D**\dir{-};"CV4"+U;"s2"+D**\dir{-};"CV5"+U;"s2"+D**\dir{-};
				"CV5"+U;"s3"+D**\dir{-};"CV6"+U;"s3"+D**\dir{-};"CV7"+U;"s3"+D**\dir{-};
				"s1"+U;"PrWd"+D**\dir{-};"s2"+U;"ft1"+D**\dir{-};"s3"+U;"ft1"+D**\dir{-};
				"ft1"+U;"PrWd"+D**\dir{-};
		\endxy}
	\end{exe}
\end{multicols}

Such foot-initial glottal stop insertion is also seen with the
reciprocal prefix \ve{ma-} (\srf{sec:RecPre}) and when the
property circumfix \ve{ma-{\ldots}-ʔ} attaches to a nominal stem (\srf{sec:PropCir}).
An example with the reciprocal prefix is \ve{ori-tata-ʔ} `siblings' {\ra}
\ve{n-ma-\tbr{ʔ}ori-tata=n} `be siblings with one another'
and an example with the property circumfix is 
\ve{umi} `house' {\ra} \ve{ma-\tbr{ʔ}umi-ʔ} `having a house, housed'.

To summarise, attachment of a CV- prefix
to a vowel-initial foot triggers glottal stop insertion
as feet in Amarasi require an onset consonant.
While it is obligatory for feet to have an onset,
it is not obligatory for syllables to have an onset.
However, the only syllable which occurs without an onset
is the second syllable of a foot.
This is seen in VV(C){\#} final words 
such as \ve{kaut} `papaya' which contain an empty
medial C-slot, or M\=/forms such as \ve{fatu} {\ra} \ve{faut}
in which case the M-foot contains no medial C-slot (\srf{sec:TheFoo}).\footnote{
		Thersia Tamelan (p.c. July 2018) reports that foot-initial glottal stop
		insertion is a distinctive feature of the speech of some Meto speakers
		who have acquired Dela, a language of Rote, as adults.
		Thus, for instance, Dela \it{na-oe} [naˈɔɛ] `watery'
		is pronounced [naˈʔɔɛ] with an erroneous medial glottal stop
		by some native speakers of Meto.}

\subsubsection{Word-initial glottal stop insertion}
Word initially there is probably also a process of pre-vocalic
glottal stop insertion, though the data for this is somewhat ambiguous.
A more detailed, though earlier, discussion
of the issues surrounding initial glottal stops is given in \citet{ed17}.
This second process of glottal stop insertion can be analysed as occurring
to provide the prosodic word with an onset consonant.

The glottal stop is clearly a contrastive consonant.
Near minimal pairs are given in \trf{tab:GloStoCon}
which shows the contrast between \ve{ʔ}, \ve{k}, \ve{h}, and {\0}
medially, finally, and initially before another consonant.

\begin{table}[h]
	\centering\caption{Contrasts between \it{ʔ} : \it{k} : \it{h} : {\0}}\label{tab:GloStoCon}
	\begin{tabular}{lll|ll|ll}\lsptoprule
							&	V{\gap}V					&	Gloss				&	{\gap}{\#}				&	Gloss			&	{\#}{\gap}C			&	Gloss			\\ \midrule
			\ve{ʔ}	& \ve{pa\tbr{ʔ}e} 	&	`fortune' 	& \ve{menu\tbr{ʔ}} 	&	`bitter' 	&	\ve{\tbr{ʔ}bibi}&	`goat'		\\
			\ve{k}	& \ve{na\tbr{k}e}		&	`stocks' 		& \ve{tenu\tbr{k}}	&	`umbrella'&	\ve{\tbr{k}biti}&	`scorpion'\\
			{\0}		& \ve{fae} 					&	`k.o. tree' & \ve{tenu} 				&	`three' 	&	\ve{\hp{k}biki}	&	`scar'		\\
			\ve{h}	& \ve{na\tbr{h}e-n} &	`down' 			& \ve{inu\tbr{h}}		&	`beads' 	&									&						\\
		\lspbottomrule
	\end{tabular}
\end{table}

However, there are no phonetically vowel-initial words in Amarasi
and there are no contrasts between the glottal stop and zero word initially.
Both these facts are true of all words in all phrase positions.
Three analyses of this data are logically possible:

\begin{exe}
	\ex{\begin{xlist}
		\ex{All pre-vocalic initial glottal stops are distinctive.}\label{ex:Underlying}
		\ex{All pre-vocalic initial glottal stops are automatic.}\label{ex:Epenthetic}
		\ex{There is a difference between pre-vocalic initial distinctive and automatic glottal stops.
				(The difference emerges in certain environments.)}\label{ex:Contrast}
	\end{xlist}}\label{ex:AnaIniGloSto}
\end{exe}

In his analysis of the Miomafo variety of Meto, \cite{st93,st96}
takes analysis \qf{ex:Underlying} and treats all pre-vocalic word-initial
glottal stops as distinctive.
In \cite{ed16,ed16b} I adopted analysis \qf{ex:Epenthetic} and posited that all
pre-vocalic word-initial glottal stops were epenthetic.
In \cite{ed17} I took analysis \qf{ex:Contrast}
and provided evidence that some pre-vocalic initial
glottal stops are distinctive and some are automatic.
This is still the analysis I favour,
though since the publication of \cite{ed17}
I have collected additional data which indicates that Amarasi
may be transitioning from a system in which
some pre-vocalic initial glottal stops are automatic and some are distinctive
(analysis \ref{ex:Contrast}) to a system in which all are distinctive (analysis \ref{ex:Underlying}).

What is not ambiguous, is that pre-vocalic glottal
stops contrast with zero \emph{root} initially.
This contrast is revealed by the addition
of prefixes consisting of a single consonant,
such as the consonantal agreement prefixes (\srf{sec:VerAgrPre}).
Examples of pre-vocalic root-initial glottal stops
and vowel-initial roots are given in \qf{ex:GloIniRoo}
and \qf{ex:VowIniRoo} with the third person agreement prefix \ve{n-}.
The examples in \qf{ex:GloIniRoo} show that
any initial glottal stop surfaces after the addition of this prefix.

\begin{exe}
	\ex{\ve{n-} before glottal stop initial roots:}\label{ex:GloIniRoo}
	\sn{\gw\begin{tabular}{llllll}
		\ve{n-} + \ve{{\rt}ʔator}&\ra&\ve{n-ʔator}&[ˈn\tbr{ʔ}at̪ɔr]&{\emb{nqator.mp3}{\spk{}}{\apl}}&`arrange'\\
		\ve{n-} + \ve{{\rt}ʔani}&\ra&\ve{n-ʔain}&[n\tbr{ʔ}ajn]&{\emb{nqain.mp3}{\spk{}}{\apl}}&`head towards'\\
		\ve{n-} + \ve{{\rt}ʔoban}&\ra&\ve{n-ʔoban}&[ˈn\tbr{ʔ}ɔbɐn]&{\emb{nqoban.mp3}{\spk{}}{\apl}}&`dig up (with snout)'\\
		\ve{n-} + \ve{{\rt}ʔonen}&\ra&\ve{n-ʔonen}&[ˈn\tbr{ʔ}ɔnɛn]&{\emb{nqonen.mp3}{\spk{}}{\apl}}&`pray'\\
		\ve{n-} + \ve{{\rt}ʔere}&\ra&\ve{n-ʔeer}&[n̩ˈ\tbr{ʔ}ɛːr]&{\emb{nqeer.mp3}{\spk{}}{\apl}}&`look intently'\\
	\end{tabular}}
%\end{exe}
%\newpage
%\begin{exe}
	\ex{\ve{n-} before vowel-initial roots:}\label{ex:VowIniRoo}
	\sn{\gw\begin{tabular}{llllll}
		\ve{n-} + \ve{{\rt}akan}&\ra&\ve{n-akan}&[ˈnakɐn]&{\emb{nakan.mp3}{\spk{}}{\apl}}&`grumble'\\
		\ve{n-} + \ve{{\rt}ani}&\ra&\ve{n-ain}&[najn]&{\emb{nain.mp3}{\spk{}}{\apl}}&`before'\\
		\ve{n-} + \ve{{\rt}ono}&\ra&\ve{n-oon}&[nɔːn]&{\emb{noon.mp3}{\spk{}}{\apl}}&`harvest'\\
		\ve{n-} + \ve{{\rt}oʔen}&\ra&\ve{n-oʔen}&[ˈnɔʔɛn]&{\emb{noqen.mp3}{\spk{}}{\apl}}&`call'\\
		\ve{n-} + \ve{{\rt}eku}&\ra&\ve{n-euk}&[ˈnɛ̝ʊk]&{\emb{neuk.mp3}{\spk{}}{\apl}}&`eat'\\
	\end{tabular}}
\end{exe}

However, with a single exception, none of the 35 unambiguously
vowel-initial roots in my database have ever been attested without a prefix.
This means that \emph{word}-initial glottal stop
insertion has never been observed with these roots.

The only exception is the root \ve{{\rt}isa} `completely, totally, utterly; win'.
This root has the inflected verbal form \ve{n-isa} {\ra} \ve{n-iis} [nɪːs] {\emb{niis.mp3}{\spk{}}{\apl}},
showing that it is indeed vowel-initial, and the nominalised form
\ve{isa-t} [ʔɪsɐt̪] {\emb{isat.mp3}{\spk{}}{\apl}} with an initial glottal stop analysable as an insertion.
The nominalisation \ve{isa-t} is identified by speakers
as archaic and the form \ve{m-n-isa-t} is more common in my data.

All other instances of pre-vocalic glottal stops in my database
are either ambiguous, as the root has not yet been attested
with mono-consonantal prefixes (112 examples),
or the glottal stop can be shown to be distinctive (76 examples).

Instances of distinctive pre-vocalic glottal stops include
examples in which the initial glottal stop is almost certainly a historic insertion.
Three examples are Proto-Malayo-Polynesian (PMP) *ama > \ve{n-ʔama} {\ra} \ve{n-ʔaam} [n̩ˈʔaːm] {\emb{nqaam.mp3}{\spk{}}{\apl}}
`address as father' (cf. \ve{ama-f} [ˈʔamɐf] `father' {\emb{amaf-Roni.mp3}{\spk{}}{\apl}}),
PMP *anak > \ve{n-ʔana} {\ra} \ve{n-ʔaan} [n̩ˈʔaːn] {\emb{nqaan.mp3}{\spk{}}{\apl}} `address as child, produce a sapling'
(cf. \ve{anah} [ˈʔanɐh] `child' {\emb{anah-Roni.mp3}{\spk{}}{\apl}}),
and PMP *ina > \ve{n-ʔaina} {\ra} \ve{n-ʔain} [n̩ˈʔain] {\emb{nqain-mother.mp3}{\spk{}}{\apl}} `address as mother'
(cf. \ve{aina-f}  [ˈʔajnɐf] `mother'  {\emb{ainaf-Roni.mp3}{\spk{}}{\apl}}).\footnote{
		There is no evidence for identifying the initial glottal stop in such forms as a prefix.}

In addition to the form \ve{isa-t} [ʔɪsɐt̪] {\emb{isat.mp3}{\spk{}}{\apl}},
there is one process which probably does provide evidence that word-initial glottal
stop insertion before vowels remains productive in Amarasi.
This is epenthesis of the vowel [a]
before which glottal stop insertion also occurs.

Phrase initially, or after another consonant,
epenthesis of [a] optionally occurs before a
consonant cluster (see \srf{sec:Epe} for full details).
This epenthetic [a] is usually, though not obligatorily,
preceded by a glottal stop.
Examples are given in \trf{tab:GloStoInsEpe} which contains
the citation form of several consonant-initial verb roots from recorded wordlists.
All verbs were cited with the third person agreement prefix \ve{n-},
with [ʔa] before the initial consonant cluster.

\begin{table}[ht]
	\centering\caption{Glottal stop insertion before epenthetic [a]}\label{tab:GloStoInsEpe}
	\begin{tabular}{lllll}\lsptoprule
		Root						&	Citation						&	Phonetic					&																		&	Gloss \\ \midrule
		\ve{{\rt}\j ari}&	\ve{\tbr{a}|n-ʤair}	&	[\tbr{ʔa}ɲˈʤaer]	&	{\emb{anjair.mp3}{\spk{}}{\apl}}	&	`become'\\
		\ve{{\rt}hake}	&	\ve{\tbr{a}|n-haek}	&	[\tbr{ʔa}nˈhaɛkʲ]	&	{\emb{anhaek.mp3}{\spk{}}{\apl}}	&	`stand'\\
		\ve{{\rt}kisu}	&	\ve{\tbr{a}|n-kius}	&	[\tbr{ʔa}nˈkiʉs]	&	{\emb{ankius.mp3}{\spk{}}{\apl}}	&	`see'\\
		\ve{{\rt}mani}	&	\ve{\tbr{a}|n-main}	&	[\tbr{ʔa}nˈmain]	&	{\emb{anmain.mp3}{\spk{}}{\apl}}	&	`laugh'\\
		\ve{{\rt}reruʔ}	&	\ve{\tbr{a}|n-reruʔ}&	[\tbr{ʔa}nˈdɾeɾʊʔ]&	{\emb{anreruq.mp3}{\spk{}}{\apl}}	&	`sleepy'\\
		\ve{{\rt}roʔa}	&	\ve{\tbr{a}|n-rooʔ}	&	[\tbr{ʔa}nˈdɾɔːʔ]	&	{\emb{anrooq.mp3}{\spk{}}{\apl}}	&	`spews'\\
		\ve{{\rt}sii}		&	\ve{\tbr{a}|n-sii}	&	[\tbr{ʔa}nˈsiː]		&	{\emb{ansii.mp3}{\spk{}}{\apl}}		&	`sing'\\
		\ve{{\rt}topu}	&	\ve{\tbr{a}|n-toup}	&	[\tbr{ʔa}n̪ˈt̪ɘwp]	&	{\emb{antoup.mp3}{\spk{}}{\apl}}	&	`receive'\\
		\ve{{\rt}toti}	&	\ve{\tbr{a}|n-toit}	&	[\tbr{ʔa}n̪ˈt̪ɵit̪]	&	{\emb{antoit.mp3}{\spk{}}{\apl}}	&	`ask'\\
		\ve{{\rt}tupa}	&	\ve{\tbr{a}|n-tuup}	&	[\tbr{ʔa}n̪ˈt̪ʊːp]	&	{\emb{antuup.mp3}{\spk{}}{\apl}}	&	`sleep'\\
		\lspbottomrule
	\end{tabular}
\end{table}

Given that vowel epenthesis is a predictable process,
it would be extremely unusual for this epenthetic vowel
to be accompanied by a distinctive, contrastive consonant.
Instead, the glottal stop that precedes epenthetic
[a] in Amarasi is best analysed as epenthetic.
%Both the initial segments in forms such as \ve{a|n-reruʔ} [ʔanˈdɾeɾʊʔ] `sleepy'
%are automatic insertions; the vowel occurs because of the following consonant
%cluster and the glottal stop occurs because of the following word-initial vowel.

%An alternate analysis of the same data would be to posit that
%epenthesis in Amarasi consists of the sequence \ve{ʔa-}.
%However, this analysis provides no reason why the
%first consonant of this sequence is [ʔ] rather than any other consonant.
%Under the epenthesis analysis, the selection of [ʔ] 
%is consistent with its status as an epenthetic consonant
%foot initially, as discussed in \srf{sec:GloStoInsVocPre} above.

The prosodic and morphological structure of \ve{a|n-reruʔ}
[ʔanˈdɾeɾʊʔ] `sleepy' is shown in \qf{as:anreruq}.
The initial glottal stop [ʔ] and vowel [a] are not linked to any morphemes,
as befits their putative status as non-meaningful insertions.

\begin{exe}
	\exa{\label{as:anreruq}\xy
		<4.5em,5cm>*\as{PrWd}="PrWd",<6em,4cm>*\as{Ft}="ft1",
		<2em,3cm>*\as{σ}="s1",<5em,3cm>*\as{σ}="s2",<7em,3cm>*\as{σ}="s3",
		<1em,2cm>*\as{C}="CV1",<2em,2cm>*\as{V}="CV2",<3em,2cm>*\as{C}="CV3",
		<4em,2cm>*\as{C}="CV4",<5em,2cm>*\as{V}="CV5",<6em,2cm>*\as{C}="CV6",<7em,2cm>*\as{V}="CV7",<8em,2cm>*\as{C}="CV8",
		<1em,1cm>*\as{ʔ}="cv1",<2em,1cm>*\as{a}="cv2",<3em,1cm>*\as{n}="cv3",
		<4em,1cm>*\as{r}="cv4",<5em,1cm>*\as{e}="cv5",<6em,1cm>*\as{r}="cv6",<7em,1cm>*\as{u}="cv7",<8em,1cm>*\as{ʔ}="cv8",
		<3em,0cm>*\as{M}="m1",<6em,0cm>*\as{M}="m2",
		"m1"+U;"cv3"+D**\dir{-};"m2"+U;"cv4"+D**\dir{-};"m2"+U;"cv5"+D**\dir{-};"m2"+U;"cv6"+D**\dir{-};"m2"+U;"cv7"+D**\dir{-};"m2"+U;"cv8"+D**\dir{-};
		"cv1"+U;"CV1"+D**\dir{-};"cv2"+U;"CV2"+D**\dir{-};"cv3"+U;"CV3"+D**\dir{-};
		"cv4"+U;"CV4"+D**\dir{-};"cv5"+U;"CV5"+D**\dir{-};"cv6"+U;"CV6"+D**\dir{-};"cv7"+U;"CV7"+D**\dir{-};"cv8"+U;"CV8"+D**\dir{-};
		"CV1"+U;"s1"+D**\dir{-};"CV2"+U;"s1"+D**\dir{-};"CV3"+U;"s1"+D**\dir{-};
		"CV4"+U;"s2"+D**\dir{-};"CV5"+U;"s2"+D**\dir{-};"CV6"+U;"s2"+D**\dir{-};
		"CV6"+U;"s3"+D**\dir{-};"CV7"+U;"s3"+D**\dir{-};"CV8"+U;"s3"+D**\dir{-};
		"s1"+U;"PrWd"+D**\dir{-};"s2"+U;"ft1"+D**\dir{-};"s3"+U;"ft1"+D**\dir{-};
		"ft1"+U;"PrWd"+D**\dir{-};
	\endxy}
\end{exe}

The presence of a glottal stop before epenthetic [a],
indicates that word-initial pre-vocalic glottal
stop insertion is still productive in Amarasi.
This is consistent with glottal stop insertion foot initially,
as discussed in \srf{sec:GloStoInsVocPre} above.

However, the fact that nearly all (historic) word-initial insertions
of glottal stop have been reanalysed as distinctive,
combined with the productivity of the process
in only one unambiguous environment and a single archaic form,
indicates that Amarasi is transitioning from a system in which
some initial pre-vocalic glottal stops are automatic and some are distinctive
to a system in which all are distinctive.

On a practical level, I only transcribe root-initial
pre-vocalic glottal stops when such roots take a prefix
or when such a glottal stop is itself a prefix.
This is consistent with the orthographic practices
of native speakers of Amarasi.
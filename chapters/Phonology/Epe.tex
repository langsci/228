\section{Epenthesis}\label{sec:Epe}
When a consonant final word occurs before a consonant cluster,
epenthesis of the vowel [a] usually occurs to break up the cluster.
Four examples of epenthesis before a CC initial roots are given in
\qf{ex:130825-6, 8.44}--\qf{ex:130823-9, 0.25} below.
Throughout this book I separate epenthetic /a/ from the
stem with the vertical line |.

\begin{exe}
\let\eachwordone=\textnormal \let\eachwordtwo=\ve
	\ex{\gllll	[ʔʊmɐ ʔt̪ɛ t̪ ʊhakɵb asnʊkʊ niʏskɔrɛ]\\
							\hp{[}uma ʔ-tea =t u-haku-b \tbr{a}|snuku Niuskore\\
							\hp{[}uma ʔ-tea =te u-haku-b {\a}snuku Niuskore\\
							\hp{[}{\uma} \q-arrive ={\te} \qu-force-{\b} {\a}trim Niuskore\\
			\glt		\lh{[}`When I arrived I forced myself to do the mowing at Niuskore'
							\txrf{130825-6, 8.44}{\emb{130825-6-08-44.mp3}{\spk{}}{\apl}}}\label{ex:130825-6, 8.44}
	\ex{\gllll	[nbi nɔmɛr əmsɐʔ rɛkɔ]\\
							\hp{[}n-bi nomer \tbr{a}|msaʔ reko\\
							\hp{[}n-bi nomer {\a}msaʔ reko\\
							\hp{[}\n-{\bi} number {\a}also good\\
			\glt		\lh{[}`(writing) on the number is also fine.'
							\txrf{130905-1, 0.34}{\emb{130905-1-00-34.mp3}{\spk{}}{\apl}}}\label{ex:130905-1, 0.34}
	\ex{\gllll	[kʊɐn ʔaʔpina m faɔf]\\
							\hp{[}kuan \hp{ʔ}\tbr{a}|ʔpina =m faof\\
							\hp{[}kuan {\a}ʔpina =ma fafo\\
							\hp{[}village {\a}below =and above\\
			\glt		\lh{[}`(There was) a village down below and up above'
							\txrf{130823-9, 0.25}{\emb{130823-9-00-25.mp3}{\spk{}}{\apl}}}\label{ex:130823-9, 0.25}
\end{exe}

Epenthesis of [a] also optionally occurs before
a phrase initial consonant cluster.
Examples have been given in \trf{tab:GloStoInsEpe}
on \prf{tab:GloStoInsEpe} during the discussion
of glottal stop insertion.

\subsection{Frequency of epenthesis}
While epenthesis usually occurs to break up
a cluster of three consonants which would be created
across a word boundary, it is not obligatory.
The number of instances of epenthesis after a consonant
final word and before a consonant cluster was counted
in my corpus of 182.49 minutes (three hours and two minutes)
of recorded Kotos Amarasi texts.
The results are summarised in \trf{tab:FreEpe}.

\begin{table}[h]
	\caption[Frequency of epenthesis]{Frequency of epenthesis\su{†}}\label{tab:FreEpe}
	\centering\stl{0.4em}
		\begin{threeparttable}
			\begin{tabular}{r|ccccccccccc|c} \lsptoprule
C{\#}	&	\ve{p}	&	\ve{r}	&	\ve{s}	&	\ve{b}	&	\ve{t}	&	\ve{n}	&	\ve{f}	&	\ve{k}	&	\ve{m}	&	\ve{ʔ}	&	\ve{h}	&	Obs.		\\ \midrule
C{\#}CC	&	--	&	--	&	3	&	1	&	5	&	23	&	2	&	7	&	2	&	44	&	4	&	18		\\
C{\#}a|CC	&	--	&	6	&	60	&	7	&	11	&	42	&	3	&	9	&	2	&	26	&	--	&	96		\\
ep.{\%}	&	--	&	100\%	&	95\%	&	88\%	&	69\%	&	65\%	&	60\%	&	56\%	&	50\%	&	37\%	&	0\%	&	84{\%}		\\
			\lspbottomrule
			\end{tabular}
				\begin{tablenotes}
					\item [†] The second row gives the number of instances in which each word final
										consonant occurs before a consonant cluster without epenthesis
										and the third row the number of times epenthesis occurs between
										that consonant and a following cluster.
				\end{tablenotes}
		\end{threeparttable}
\end{table}

This table shows that epenthesis usually occurs
when the final consonant is an ``obstruent''
(defined here loosely as any of /p t k b f s r/)
with epenthesis occurring before a consonant cluster
and after an obstruent in 96/114 (84{\%}) instances.
Epenthesis also usually occurs when the final consonant is \ve{n},
though at a lower rate than for obstruents with 42/65 (65\%) examples.
Epenthesis is least common when the final consonant is a glottal stop
with 26/70 (37\%) attestations.
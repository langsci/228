\subsubsection{Comparative support}\label{sec:ComSup}
There is also comparative support for empty C-slots in Amarasi.
This comes from comparison with other varieties of Meto amd
Proto-Malyo-Polynesian (PMP) forms with their Meto reflexes.
All reconstructions cited in this section are from \citep{bltr}.

Firstly, there are a handful of words in which another variety of Meto
has a full consonant where Amarasi has a medial empty C-slot.
One example is the word for `two',
for which Amarasi \ve{nua} (< *dua < PMP *duha) can
be compared with Naitbelak Amfo{\Q}an \ve{nuga} `two' 
and Baikeno \ve{nuban} `two'.\footnote{
		The final \ve{n} in Baikeno \ve{nuban} could be a fossilised plural marker.}
Both medial consonants can be analysed as resulting from features of the
previous vowel spreading into an empty C-slot.

An additional Naitbelak Amfo{\Q}an example is \ve{na-guab} `talks',
which can be compared with Amarasi \ve{na-ʔuab} `speaks'.
In this case both varieties have a root-initial consonant,
which is probably originally epenthetic.
Naitbelak Amfo{\Q}an has inserted a consonant conditioned
by the following vowel and Amarasi has inserted default [ʔ].
(See \srf{sec:GloStoIns} for more discussion of epenthetic [ʔ] in Amarasi.)

\paragraph{Word-final consonant insertion}\label{sec:WorFinConIns}
In addition to sporadic examples of medial C-slots which are empty in Amarasi
but filled in other varieties of Meto,
there is also a regular system of word-final consonant insertion
phrase finally in some varieties of Meto.\footnote{
		See Figure \ref{fig:SelIdeVarUabMeto} on \prf{fig:SelIdeVarUabMeto}
		for the locations of the varieties of Meto discussed in this section.}
Such consonants are not historic retentions
and can be predicted based on the quality of the final vowel.
Phrase-final consonant insertion has been described
and analysed for one variety of Meto, Nai{\Q}bais Amfo{\Q}an, by \citet{cu18}.

In Nai{\Q}bais Amfo{\Q}an, consonant insertion affects
all nouns which are final within the noun phrase (including the citation form),
verbs which have an unspecified object, and pronominal possessors
with an unspecified possessum \citep[31ff]{cu18}.
Consonant insertion also occurs in Nai{\Q}bais Amfo{\Q}an
before vowel-initial enclitics as it does in Amarasi
(\srf{sec:EmpCSloConIns}, \srf{sec:ConIns}).

In Amfo{\Q}an /ɡ/ is inserted after the back vowels /o/ and /u/.
Final /ɡ/ is usually unreleased and slightly devoiced thus: [ɡ̊˺].
After the high front vowel /i/ the consonant /\j/ is inserted. 
Word-final /\j/ is usually devoiced, often de-palatalised
and often tends towards a non-sibilant fricative.\footnote{
All of the following phonetic symbols occur as transcriptions of inserted /\j/ 
		in my own Naitbelak Amfo{\Q}an data: [\j], [\tS], [ʦ], [ʒ] and [s].
		Speakers identify this sound with the letter \it{<j>},
		used in Indonesian for /\j/.}
After /e/ the consonant /l/ is inserted.
Examples are given in \qf{ex:AmfExcCon} below
alongside their Amarasi cognates and Proto-Malayo-Polynesian (PMP) etyma.

\begin{exe}
	\ex{Amfo{\Q}an (Naitbelak) consonant insertion:}\label{ex:AmfExcCon}
	\sn{\stl{0.33em}\begin{tabular}{rglglglg}
		PMP					&*taqi						&*punti						&*bahi 						&*wahiR 				&*qaləjaw			& *asu					& *batu			\\
		Amarasi			&\hp{*}\ve{tei}		&\hp{*}\ve{uki}		&\hp{*}\ve{fee}		&\hp{*}\ve{oe}	&\hp{*}\ve{neno}	&\hp{*}\ve{asu}	&\hp{*}\ve{fatu}	\\
		Amfo{\Q}an	&\hp{*}\ve{tei\j}	&\hp{*}\ve{uki\j}	&\hp{*}\ve{feel}	&\hp{*}\ve{oel}	&\hp{*}\ve{nenog}	&\hp{*}\ve{asug}&\hp{*}\ve{fatug}	\\
		gloss				&`faeces'					&`banana'					&`wife'						&`water'				& `day, sky'			&`dog'					& `stone'			\\
	\end{tabular}}
\end{exe}

In Baikeno, consonants are only inserted after vowel sequences:
/b/ is inserted after /o/ and /u/,
/\j/ is inserted after /i/ and /l/ is inserted after /e/.
Baikeno /\j/ is almost always realised as a fricative [ʒ] or [z],
likewise Baikeno /b/ is almost always the fricative [β].
Final /l/ in Baikeno is usually laminal [l̻] in recordings available to me.
Examples of Baikeno consonant insertion are given in \qf{ex:BaiConIns} below.

\begin{exe}
	\ex{Baikeno consonant insertion:}\label{ex:BaiConIns}
	\sn{\stl{0.35em}\begin{tabular}{rglglglg}
		PMP			&*hapuy				&*taqi					&*bahi 				&*wahiR				&*qapuR				& *kahiw			& *qihu				\\
		Amarasi	&\hp{*}\ve{ai}	&\hp{*}\ve{tei}		&\hp{*}\ve{fee}	&\hp{*}\ve{oe}	&\hp{*}\ve{ao}	&\hp{*}\ve{hau}	&\ve{iik iu}	\\
		Baikeno	&\hp{*}\ve{ai\j}&\hp{*}\ve{tei\j}	&\hp{*}\ve{feel}&\hp{*}\ve{oel}	&\hp{*}\ve{aob}	&\hp{*}\ve{haub}&\ve{iik iub}	\\
		gloss		&`fire'				&`faeces'				&`wife'				&`water'			& `lime'			&`wood'				& `shark'			\\
	\end{tabular}}
\end{exe}

Fatule{\Q}u consonant insertion is very similar to that of Baikeno,
though vowel assimilation occurs after insertion of /\j/.
Additionally, there is one probable example of /l/
being inserted after /a/; PMP *quay > Fatule{\Q}u \ve{ual} `rattan'.
Examples of Fatule{\Q}u consonant insertion are given in \qf{ex:FatConIns} below.

\begin{exe}
	\ex{Fatule{\Q}u (Bineon-Koa{\Q} hamlet) consonant insertion:}\label{ex:FatConIns}
	\sn{\stl{0.35em}\begin{tabular}{rglglglg}
		PMP			&*hapuy				&*waRi				&*taqi					&*bahi 				&*wahiR				&*qapuR				& *kahiw			\\
		Amarasi	&\hp{*}\ve{ai}	&\hp{*}\ve{fai}	&\hp{*}\ve{tei}		&\hp{*}\ve{fee}	&\hp{*}\ve{oe}	&\hp{*}\ve{ao}	&\hp{*}\ve{hau}	\\
		Fatule{\Q}u	&\hp{*}\ve{aa\j}&\hp{*}\ve{faa\j}	&\hp{*}\ve{tee\j}	&\hp{*}\ve{feel}&\hp{*}\ve{oel}	&\hp{*}\ve{aob}	&\hp{*}\ve{haub}\\
		gloss		&`fire'				& `night'			&`faeces'				&`wife'				&`water'			& `lime'			&`wood'				\\
	\end{tabular}}
\end{exe}

In Kopas, consonant insertion takes place only after vowel sequences.
Unlike Baikeno and Fatule{\Q}u (but like Amfo{\Q}an),
/ɡ/ is inserted after back vowels.
After insertion of /\j/ or /ɡ/ final vowels assimilate in Kopas.
Inserted /ɡ/ in Kopas is always voiced in my data,
while inserted /\j/ is usually somewhat devoiced and tends towards a fricative.
Examples of Kopas consonant insertion from Tuale{\Q}u hamlet
are given in \qf{ex:KopConIns} below.
Consonant insertion in Kopas as spoken in
Usapisonba{\Q}i hamlet is almost identical,
though [ɡw] is inserted rather than [ɡ].\footnote{
		Insertion of [ɡw] rather than [ɡ] is viewed by inhabitants
		of Usapisonba{\Q}i as a distinguishing feature of their speech
		compared with the speech of inhabitants of Tuale{\Q}u.}

\begin{exe}
	\ex{Kopas (Tuale{\Q}u hamlet) consonant insertion:}\label{ex:KopConIns}\setlength{\tabcolsep}{0.5em}
	\sn{\stl{0.28em}\begin{tabular}{rglglglg}
		PMP			&*hapuy				&*taqi					&*bahi 				&*wahiR				&*qapuR				& *kahiw			&---\\
		Amarasi	&\hp{*}\ve{ai}	&\hp{*}\ve{tei}		&\hp{*}\ve{fee}	&\hp{*}\ve{oe}	&\hp{*}\ve{ao}	&\hp{*}\ve{hau}	&\ve{kiu}\\
		Kopas	&\hp{*}\ve{aa\j}&\hp{*}\ve{tee\j}	&\hp{*}\ve{feel}&\hp{*}\ve{oel}	&\hp{*}\ve{aag}	&\hp{*}\ve{haag}&\ve{kiig}\\
		gloss		&`fire'				&`faeces'				&`wife'				&`water'			& `lime'			&`wood'				& `tamarind'\\
	\end{tabular}}
\end{exe}

The most unusual kind of consonant insertion so far encountered
occurs in Timaus, spoken on the border of the Amarasi area.\footnote{
		Timaus speakers trace their origins to \it{Timau} mountain in southern Amfo{\Q}an.}
Like Amfo{\Q}an, Timaus consonant insertion affects all vowel-final words,
not just words which end in a vowel sequence.
Timaus consonant insertion is also accompanied by a shift in the quality of the final vowel:
root final /i/ is replaced by /ar/,\footnote{
		Timaus /r/ is from original *{\j} which is
		inserted word finally in Amfo{\Q}an: e.g. Amarasi \ve{fafi}
		= Amfo{\Q}an \ve{fafi\j} = Timaus \ve{fafar} `pig'.
		Other instances of Amarasi /\j/ which correspond to Timaus /r/
		include Amarasi \ve{bi{\j}ae} Timaus \ve{birael},
		and Amarasi \ve{nai{\j}eer} Timaus \ve{naireel} `ginger'.}
final /o/ is replaced by /uɡw/,
final /u/ is replaced by /i\j/,
and /l/ is inserted after /e/ which then lowers to /a/.\footnote{
		Lowering of /e/ to /a/ is a general feature of Timaus spoken in Sananu
		hamlet and affects all CV-final words, not just those which have undergone consonant insertion.
		One example is Amarasi \ve{n-pake} = Timaus \ve{n-paka} `use' (both from Malay \it{pakai} [pake]).}
Examples are given in \qf{ex:TimConIns} below.

\begin{exe}
	\ex{Timaus (Sanenu hamlet) CV{\#} consonant insertion:}\label{ex:TimConIns}
	\sn{\stl{0.33em}\begin{tabular}{rglglglg}
		PMP			&*babuy						&*talih						&*Rumaq 				&	---			 	&*qaləjaw			&*asu							& *batu	\\
		Amarasi	&\hp{*}\ve{fafi}	&\hp{*}\ve{tani}	&\hp{*}\ve{ume}	&\ve{koro}	&\hp{*}\ve{neno}	&\hp{*}\ve{asu}		&\hp{*}\ve{fatu}	\\
		Timaus	&\hp{*}\ve{fafar}	&\hp{*}\ve{tanar}	&\hp{*}\ve{umal}&\ve{kolugw}&\hp{*}\ve{nenugw}&\hp{*}\ve{asi\j}	&\hp{*}\ve{fati\j}	\\
		gloss		&`pig'						&`rope'						&`house'				&`bird'			& `day, sky'			&`dog'						& `stone'	\\
	\end{tabular}}
\end{exe}

When a noun ends in a vowel sequence in Timaus,
the same consonants are inserted after a single vowel,
with subsequent assimilation of the final vowel to the quality of the previous vowel.
Vowel assimilation does not occur after insertion of /l/.
Examples of Timaus consonant insertion after vowel sequences
are given in \qf{ex:TimConIns2} below.\footnote{
		The insertion of /\j/ after /u/ in Timaus
		may be explicable in terms of a push-pull chain.
		Word-final front vowels /i/ and /e/ condition insertion of /r/ and /l/
		respectively after which these vowels lower to /a/.
		Word-final /o/ conditions insertion of /ɡw/ after which /o/ is then raised to /u/.
		Word-final /u/ is either pushed or pulled into the empty high front vowel position,
		and then conditions insertion of /\j/.}

\begin{exe}
	\ex{Timaus (Sanenu hamlet) VV{\#} consonant insertion:}\label{ex:TimConIns2}
	\sn{\stl{0.28em}\begin{tabular}{rglglglg}
		PMP			&*hapuy					&*taqi					&*bahi 					&*wahiR					&*qapuR					& *kahiw					&	---		\\
		Amarasi	&\hp{*}\ve{ai}	&\hp{*}\ve{tei}	&\hp{*}\ve{fee}	&\hp{*}\ve{oe}	&\hp{*}\ve{ao}	&\hp{*}\ve{hau}		&\ve{kiu}\\
		Timaus	&\hp{*}\ve{aar}	&\hp{*}\ve{teer}&\hp{*}\ve{feel}&\hp{*}\ve{oel}	&\hp{*}\ve{aagw}&\hp{*}\ve{haa\j}	&\ve{kii\j}\\
		gloss		&`fire'					&`faeces'				&`wife'					&`water'				& `lime'				&`wood'						& `tamarind'\\
	\end{tabular}}
\end{exe}

Word-final consonant insertion in other varieties of
Meto provides evidence for positing final empty C-slots
in Amarasi. That consonant insertion is most common
after vowel sequences may be due to Meto varieties
dis-preferring more than one empty C-slot per foot.

\paragraph{Non-etymological glottal stops}\label{sec:NonEtyGloSto}
Some words in Amarasi occur with a medial glottal stop
which is not expected by regular sound changes.
Cognates of these words in Amanuban occur with a word-final glottal stop.
The Amarasi words which are clear inheritances from
Proto-Malayo-Polynesian (PMP) in which this non-etymological
glottal stop occurs are given in \qf{ex:NonEtyGloSto} below,
along with Amanuban cognates for comparison.

\begin{exe}
	\ex{Non-etymological glottal stops in Amarasi and Amanuban:}\label{ex:NonEtyGloSto}
	\sn{\begin{tabular}{rglglg}
			PMP			&*baqəRu					&*dahun						&*ma-iRaq				&*kakay						&*puqun		\\
			Amanuban&\hp{*}\ve{feuʔ}	&\hp{*}\ve{nooʔ}	&\hp{*}\ve{meeʔ}	&\hp{*}\ve{haeʔ}	&\hp{*}\ve{uuʔ}\\
			Amarasi	&\hp{*}\ve{feʔu}	&\hp{*}\ve{noʔo}	&\hp{*}\ve{meʔe}	&\hp{*}\ve{haʔe}	&\hp{*}\ve{uʔu}\\
				gloss	&`new'						&`leaf						&`red'						&`leg'						&`source'	\\
	\end{tabular}}
\end{exe}

Although the forms in \qf{ex:NonEtyGloSto} are
reconstructed with medial consonants,
each of PMP *q, *R, *h, and *k are otherwise regularly lost word medially in Meto.
An example of each with an Amarasi reflex includes *ma-qitəm > \ve{metan} `black',
*diRus > \ve{na-niu} `bathe', *duha > \ve{nua} `two', and *sakay > \ve{n-sae} `go up'.
Many more examples can be found in \citet{ed16b}.

Additionally, when a genitive suffix is attached to
the Amarasi words in \qf{ex:NonEtyGloSto},
the medial glottal stop does not appear.
Examples are given in \qf{ex:MedGloStoDel} below.
A complete list of the forms (including those not inherited from PMP)
in which a medial glottal stop is deleted after genitive suffixation
is given in \srf{sec:GenSuf} on \prf{ex:MedGloStoDel2}.

\begin{exe}
	\ex{Medial glottal stop deletion:}\label{ex:MedGloStoDel}
	\sn{\gw\begin{tabular}{rlllll}
		\ve{feʔu}	&+&\ve{-f}&{\ra}& \ve{moen feu-f} & `son-in-law' (\it{lit.} `new male')\\
		\ve{noʔo}	&+&\ve{-f}&{\ra}& \ve{noo-f}			& `leaf'\\
		\ve{haʔe}	&+&\ve{-f}&{\ra}& \ve{hae-f} 			& `leg'\\
		\ve{uʔu}	&+&\ve{-f}&{\ra}& \ve{uu-f} 			& `tree trunk, source'\\
	\end{tabular}}
\end{exe}

In addition to the words given in \qf{ex:NonEtyGloSto},
the PMP inheritances *taqi > \ve{tei} `faeces' 
and  *kəmiq > \ve{kmii} `urine' have verbal forms
with an unexpected medial glottal stop: \ve{na-teʔi} `defecates'
and \ve{na-kmiʔi} `urinates'.

Historically these glottal stops are usually a result
of a historic suffix metathesising with the final vowel.
This suffix is attested in the Rote languages,
as seen in for instance in Termanu \it{beu-k}, Dengka \it{feu-ʔ} `new',
Termanu \it{doo-k}, Dengka \it{loo-ʔ} `leaf', and
Termanu \it{huu-k}, Dengka \it{huu-ʔ} `tree trunk, source'.

Synchronically, the presence of medial non-etymological
glottal stops in some forms of certain roots is evidence
for medial empty C-slots between other vowel sequences.

\subsubsection{Summary}
In this section I discussed five language-internal phenomena
and three comparative phenomena which provide evidence for
empty C-slots in Amarasi.
Amarasi is not an isolated example of a language with empty C-slots.
Other languages analysed with empty C-slots include
Turkish and Finnish \citep{clke83},
Seri \citep{mast83}, and Irish \citep{an16}.

One way in which the empty C-slots in Amarasi differ from those of Turkish, Finnish, and Seri
is that in each of these languages there is only a sub-set of words with empty C-slots,
with these words behaving exceptionally due to the loss of a historic consonant. %\footnote{
		%In Turkish there is evidence that this consonant was /ɣ/,
		%and it is represented orthographically as the \it{yumu\c{s}ak ge}: <\u{g}>.}

However, empty C-slots in Amarasi are different in several respects.
Firstly, empty C-slots are not restricted to a lexically specified sub-set of words,
but are found in any word whose final foot does not surface as CVCVC.
Secondly, empty C-slots in Amarasi have not arisen from the loss of a historic consonant.
To take just two examples, the word \ve{asu} `dog' with an empty initial C-slot
and empty final C-slot is a reflex of proto-Austronesian
*asu without any consonants in these positions.
Likewise, Amarasi \ve{fua-f} `fruit', with an empty medial C-slot,
is a reflex of proto-Austronesian *buaq without any medial consonant \citep{bltr}.
Instead, empty C-slots in Amarasi have arisen from the highly
constrained CVCVC foot structure of the language.

It is worth considering what it means at a philosophical
level to say that Amarasi has empty C-slots.
Obviously, native speakers cannot hear these empty segments.
Are they then merely a notational convenience for the analyst?
While this clearly \emph{is} part of the reason for positing empty C-slots,
I do not think it is the only reason.
By positing empty C-slots I am fundamentally saying
that in the in the grammar of Amarasi a word such as \ve{ai} `fire' is 
treated the same way as a word such as \ve{muʔit} `animal',
despite their surface phonotactic differences.

Finally, do these empty C-slots have psychological reality for speakers?
The comparative data from other varieties in which consonants occur where I posit
empty C-slots indicates that they may indeed
have some level of psychological reality for at least some speakers.
Furthermore, discussions with speakers of Amarasi
who have extensive exposure to varieties such as Amfo{\Q}an,
with its regular insertion of word-final consonants,
show that these Amarasi speakers are aware of a rule of the kind:
``All Amarasi words which are vowel final end in a consonant in Amfo{\Q}an.''
This is, perhaps, the way in which empty C-slots are present to Amarasi speakers.
They exist in the social dynamics of interactions between
speakers of different varieties of what is conceptualised
as a single language.
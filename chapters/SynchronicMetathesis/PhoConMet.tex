\subsection{Phonologically conditioned metathesis}\label{sec:PhoMet}
Phonologically conditioned metathesis is any process of metathesis
which is an automatic result of a phonological environment.
Amarasi has a process of phonological metathesis conditioned
by vowel-initial enclitics (see Chapter \ref{ch:PhoMet}).

Processes of phonologically conditioned metathesis
are similar to other more familiar phonological processes
such as final obstruent devoicing in \ili{German}.
In German a voiced obstruent is devoiced word finally,
as can be seen from the data given in \qf{ex:GerFinObsDev} below.

\begin{exe}
	\ex{German final obstruent devoicing \hfill\citep[11f]{br95}}\label{ex:GerFinObsDev}
	\sn{\gw\begin{tabular}{lllll}
			\mc{2}{l}{Singular}					&\mc{2}{l}{Plural} 			& gloss		\\
			\it{Dieb}		&/diː\tbr{p}/		&\it{Diebe}		&/diː\tbr{b}ə/	&`thief'	\\
			\it{halb}		&/hal\tbr{p}/		&\it{halbe}		&/hal\tbr{b}ə/	&`half'	\\
	%		\it{Bord}		&/bɔr\tbr{t}/		&\it{Borde}		&/bɔr\tbr{d}ə/	&`shelf'	\\
			\it{Bund}		&/bʊn\tbr{t}/		&\it{Bunde}		&/bʊn\tbr{d}ə/	&`league'	\\
	%		\it{Sieg}		&/ziː\tbr{k}/		&\it{Siege}		&/ziː\tbr{ɡ}ə/	&`victory'	\\
	%		\it{Tag}		&/taː\tbr{k}/		&\it{Tage}		&/taː\tbr{ɡ}ə/	&`day'	\\
			\it{Zweig}	&/ʦvaɪ\tbr{k}/	&\it{Zweige}	&/ʦvaɪ\tbr{ɡ}ə/	&`twig'	\\
			\it{brav}		&/braː\tbr{f}/	&\it{brave}		&/braː\tbr{v}ə/	&`well-behaved'	\\
			\it{Gas}		&/ɡaː\tbr{s}/		&\it{Gase}		&/ɡaː\tbr{z}ə/		&`gas'	\\
	%		\it{Los}		&/loː\tbr{s}/		&\it{Lose}		&/loː\tbr{z}ə/		&`lottery ticket'	\\
	%		\it{}		&/\tbr{}/		&\it{}				&/\tbr{}ə/		&`'	\\
	%		\it{}		&/\tbr{}/		&\it{}				&/\tbr{}ə/		&`'	\\
	\end{tabular}}
\end{exe}

The standard (and simplest) analysis of this data is to propose
that voiced obstruents are devoiced finally.
A simple formal rule for German obstruent devoicing is given in \qf{ex:GerObs} below.\footnote{
		German obstruent devoicing involves additional complexities.
		See \citep[200ff]{wi96} and \cite{br95} for discussion
		of the way such complexities have been resolved.}

\begin{exe}
	\ex{\tsc{[+obstruent]} {\ra} \tsc{[-voice]} /{\_}]\sub{σ} \hfill\citep[201]{wi96}}\label{ex:GerObs}
\end{exe}

In German a phonological process (devoicing) affects a
segment in a specific phonological environment.
Similarly, in the case of phonologically conditioned
metathesis a phonological process (metathesis)
occurs in a specific phonological environment.

A simple example of phonological metathesis
is provided by \ili{Faroese} (Germanic, Faroe Islands).
In Faroese the neuter form of adjectives is formed by adding the suffix \it{-t}.
When this suffix is added to a stem which ends in /sk/,
this cluster metathesises to /ks/.
Examples are shown in \qf{ex:Farsk->ks} below.
(Such metathesis is not written in Faroese.)

\begin{exe}
\ex{Faroese sk {\ra} ks /{\_}t \hfill\citep[56]{th04}}\label{ex:Farsk->ks}
	\sn{\gw\begin{tabular}{lllllll}
		\tsc{masc} 		&					 				&\tsc{fem}& 					 		&\tsc{neut}	&								& \\
		\it{grøn-ur}	&/kɹøːnʊɹ/				&\it{grøn}& /kɹœn/ 				&\it{grønt}	& /kɹœn̥t/ 			& `green' \\
		\it{fesk-ur}	&/fɛ\tbr{sk}ʊɹ/		&\it{fesk}& /fɛ\tbr{sk}/	&\it{fesk-t}& /fɛ\tbr{ks}t/	& `fresh' \\
		\it{rask-ur}	&/ɹa\tbr{sk}ʊɹ/		&\it{rask}& /ɹa\tbr{sk}/	&\it{rask-t}& /ɹa\tbr{ks}t/	& `good' \\
		\it{týsk-ur}	&/tʰʊi\tbr{sk}ʊɹ/	&\it{týsk}& /tʰʊi\tbr{sk}/&\it{týsk-t}& /tʰʊi\tbr{ks}t/& `German' \\
	\end{tabular}}
\end{exe}

This Faroese metathesis is motivated
by a phonological constraint against having a cluster of a fricative, plosive, and another plosive in that order.
If such a cluster would occur, the fricative and plosive metathesise to prevent it surfacing,
and thereby avoid violating the obligatory contour principle.
Faroese metathesis of \it{fesk} /fɛ\tbr{sk}/ {\ra} \it{feskt} /fɛ\tbr{ks}t/
is illustrated in \qf{as:fesk->fekst} below in which \emph{F} = fricative and \emph{P} = plosive.
A similar metathesis involving fricatives and plosives is also found in Lithuanian.
\cite{huse04} provide a detailed analysis of metathesis in both Faroese and Lithuanian.

\begin{multicols}{3}
	\begin{exe}
		\ex{\begin{xlist}
			\exa{\xy
				<0em,2cm>*\as{C}="x1",<1em,2cm>*\as{V}="x2",<2em,2cm>*\as{C}="x3",<3em,2cm>*\as{C}="x4",<4em,2cm>*\as{C}="x5",
				<0em,1cm>*\as{F}="C1",<1em,1cm>*\as{V}="C2",<2em,1cm>*\as{F}="C3",<3em,1cm>*\as{P}="C4",<4em,1cm>*\as{P}="C5",
				<0em,0cm>*\as{f}="c1",<1em,0cm>*\as{ɛ}="c2",<2em,0cm>*\as{s}="c3",<3em,0cm>*\as{k}="c4",<4em,0cm>*\as{t}="c5",
				"C1"+U;"x1"+D**\dir{-};"C2"+U;"x2"+D**\dir{-};"C3"+U;"x3"+D**\dir{-};"C4"+U;"x4"+D**\dir{-};"C5"+U;"x5"+D**\dir{-};
				"c1"+U;"C1"+D**\dir{-};"c2"+U;"C2"+D**\dir{-};"c3"+U;"C3"+D**\dir{-};"c4"+U;"C4"+D**\dir{-};"c5"+U;"C5"+D**\dir{-};
			\endxy}
			\exa{\xy
				<0em,2cm>*\as{C}="x1",<1em,2cm>*\as{V}="x2",<2em,2cm>*\as{C}="x3",<3em,2cm>*\as{C}="x4",<4em,2cm>*\as{C}="x5",
				<0em,1cm>*\as{F}="C1",<1em,1cm>*\as{V}="C2",<2em,1cm>*\as{F}="C3",<3em,1cm>*\as{P}="C4",<4em,1cm>*\as{P}="C5",
				<0em,0cm>*\as{f}="c1",<1em,0cm>*\as{ɛ}="c2",<2em,0cm>*\as{s}="c3",<3em,0cm>*\as{k}="c4",<4em,0cm>*\as{t}="c5",
				"C1"+U;"x1"+D**\dir{-};"C2"+U;"x2"+D**\dir{-};"C3"+U;"x4"+D**\dir{.};"C4"+U;"x3"+D**\dir{.};"C5"+U;"x5"+D**\dir{-};
				"c1"+U;"C1"+D**\dir{-};"c2"+U;"C2"+D**\dir{-};"c3"+U;"C3"+D**\dir{-};"c4"+U;"C4"+D**\dir{-};"c5"+U;"C5"+D**\dir{-};
			\endxy}
			\exa{\xy
				<0em,2cm>*\as{C}="x1",<1em,2cm>*\as{V}="x2",<3em,2cm>*\as{C}="x3",<2em,2cm>*\as{C}="x4",<4em,2cm>*\as{C}="x5",
				<0em,1cm>*\as{F}="C1",<1em,1cm>*\as{V}="C2",<3em,1cm>*\as{F}="C3",<2em,1cm>*\as{P}="C4",<4em,1cm>*\as{P}="C5",
				<0em,0cm>*\as{f}="c1",<1em,0cm>*\as{ɛ}="c2",<3em,0cm>*\as{s}="c3",<2em,0cm>*\as{k}="c4",<4em,0cm>*\as{t}="c5",
				"C1"+U;"x1"+D**\dir{-};"C2"+U;"x2"+D**\dir{-};"C3"+U;"x3"+D**\dir{-};"C4"+U;"x4"+D**\dir{-};"C5"+U;"x5"+D**\dir{-};
				"c1"+U;"C1"+D**\dir{-};"c2"+U;"C2"+D**\dir{-};"c3"+U;"C3"+D**\dir{-};"c4"+U;"C4"+D**\dir{-};"c5"+U;"C5"+D**\dir{-};
			\endxy}		
		\end{xlist}}\label{as:fesk->fekst}
	\end{exe}
\end{multicols}

\ili{Sidamo} (Cushitic, Ethiopia) also has phonologically conditioned metathesis.
In Sidamo a cluster of an obstruent followed by a nasal is disallowed.
If such a cluster is created by the addition of morphology,
the obstruent-nasal sequence undergoes metathesis.
Examples are given in \qf{ex:SidMet} below,
with the first person plural simple perfect suffix.

\begin{exe}
\ex{Sidamo obstruent+nasal {\ra} nasal-obstruent \hfill\citep[46]{ka07}}\label{ex:SidMet}
	\sn{\gw\begin{tabular}{rlllll}
		 stem 						&& \mc{2}{l}{\tsc{1pl-s.prf1-1pl}} && \\
		\it{laʔ} 					&+& \it{-n-u-mmo}&{\ra} & \it{laʔnummo} & `see' \\
		\it{mee\tbr{d}} 	&+& \it{-\tbr{n}-u-mmo}&{\ra} & \it{mee\tbr{nd}ummo} & `shave' \\
		\it{t'oo\tbr{k'}} &+& \it{-\tbr{n}-u-mmo}&{\ra} & \it{t'oo\tbr{nk'}ummo} & `flee from' \\
		\it{bi\tbr{ʧ'}}		&+& \it{-\tbr{n}-u-mmo}&{\ra} & \it{bi\tbr{nʧ'}ummo} & `scar' \\
		\it{k'aa\tbr{f}} 	&+& \it{-\tbr{n}-u-mmo}&{\ra} & \it{k'aa\tbr{nf}ummo} & `step over/walk' \\
		\it{mi\tbr{ʃ}} 		&+& \it{-\tbr{n}-u-mmo}&{\ra} & \it{mi\tbr{nʃ}ummo} & `despise' \\
	\end{tabular}}
\end{exe}

\ili{Selaru} (Austronesian, Maluku) exhibits glide-consonant metathesis.
In Selaru, glide-consonant sequences are disallowed.
The addition of a consonant-initial suffix thus
triggers metathesis with any stem-final glide.
Examples are shown in \qf{ex:SelGC->CG} below,
with suffixes attached to glide-final stems.
The glide-final stems can be contrasted with
vowel-final stems in which no metathesis occurs.
The glide-final forms, such as \it{hatw} `rock' occur with
a final glide phrase finally or in isolation.

\begin{exe}%ʷʸ
	\ex{Selaru GC {\ra} CG \hfill\citep[22]{coco00}}\label{ex:SelGC->CG}
	\sn{\gw\begin{tabular}{rlll}
		\it{tas\tbr{j}}		+ \it{-\tbr{k}e}&{\ra}&\it{tas\tbr{kj}e}& `the rope'\\
		\it{hat\tbr{w}}		+ \it{-\tbr{k}e}&{\ra}&\it{hat\tbr{kw}e}& `the rock'\\
		\it{r-lu\tbr{j}}	+ \it{-\tbr{b}o}&{\ra}&\it{rlu\tbr{bj}o}& `they are only spinning'\\
		\it{a\tbr{j}}			+ \it{-\tbr{k}e}&{\ra}&\it{a\tbr{kj}e}& `the fire'\\ \hline
		\it{tasi}					+ \it{-ke}&{\ra}&\it{tasike}& `the ocean'\\
		\it{khatu}				+ \it{-ke}&{\ra}&\it{khatuke}& `the seed'\\
		\it{r-ukui}				+ \it{-bo}&{\ra}&\it{rukuibo}& `they only cut'\\
		\it{sai}					+ \it{-de}&{\ra}&\it{saide}& `what?'\\
	\end{tabular}}
\end{exe}

In addition to occurring across affix or clitic boundaries, 
metathesis in Selaru also occurs across word boundaries.
Three examples of glide-consonant metathesis across word boundaries
are given in \qf{ex:ThaIsjou}--\qf{ex:OurFatIs} below,
in which the underlying (unmetathesised) forms
of morphemes are given in the second line.
These underlying forms surface without any
metathesis in isolation or phrase finally.

\begin{exe}
\let\eachwordone=\itshape
	\ex{\glll hinam \tbr{hw}ahkje desj\\
						hina-m\tbr{w} \tbr{h}ahj-ke desj\\
						have-\tsc{2sg.gen} pig-\tsc{def} that \\
			\glt `That is your pork (food).'}\label{ex:ThaIsjou}
\end{exe}
\newpage
\begin{exe}
\let\eachwordone=\itshape
	\ex{\glll arawasim \tbr{sj}ekje desj \\
						ara-wasi-m\tbr{j} \tbr{s}ej-ke desj\\
						\tsc{1px.gen}-have-\tsc{1px.gen} house-\tsc{def} that \\
				\glt `That is our (exclusive) house.'}
	\ex{\glll	itjamatke mjat \tbr{dj}e\\
						itj-ama-t-ke j-mat\tbr{j} \tbr{d}e\\
						\tsc{1pi.gen}-father-\tsc{1pi.gen-def} \tsc{3sg}-die already\\
			\glt	`Our father is already dead.' \hfill\citep[43]{coco00}}\label{ex:OurFatIs}
\end{exe}

\cite{coco00} analyse this metathesis as a result of automatic glide spreading.
They analyse glides as unassociated elements which spread rightwards to an adjacent C-slot.
If there is no following C-slot, they attach to the C-slot to the left.
Their analysis is shown in \qf{as:askwe} below.

\begin{multicols}{3}
	\begin{exe}
		\ex{\begin{xlist}
			\ex\raisebox{\dimexpr-\totalheight+2ex\relax}{\xy
				<0em,2cm>*\as{V}="c1",<1em,2cm>*\as{C}="c2",<2em,2cm>*\as{}="c3",<3em,2cm>*\as{C}="c4",<4em,2cm>*\as{V}="c5",
				<0em,1cm>*\as{a}="p1",<1em,1cm>*\as{s}="p2",<2em,1cm>*\as{w}="p3",<3em,1cm>*\as{k}="p4",<4em,1cm>*\as{e}="p5",
				"p1"+U;"c1"+D**\dir{-};"p2"+U;"c2"+D**\dir{-};"p4"+U;"c4"+D**\dir{-};"p5"+U;"c5"+D**\dir{-};
			\endxy}
			\ex\raisebox{\dimexpr-\totalheight+2ex\relax}{\xy
				<0em,2cm>*\as{V}="c1",<1em,2cm>*\as{C}="c2",<2em,2cm>*\as{}="c3",<3em,2cm>*\as{C}="c4",<4em,2cm>*\as{V}="c5",
				<0em,1cm>*\as{a}="p1",<1em,1cm>*\as{s}="p2",<2em,1cm>*\as{w}="p3",<3em,1cm>*\as{k}="p4",<4em,1cm>*\as{e}="p5",
				"p1"+U;"c1"+D**\dir{-};"p2"+U;"c2"+D**\dir{-};"p3"+U;"c4"+D**\dir{.};"p4"+U;"c4"+D**\dir{-};"p5"+U;"c5"+D**\dir{-};
			\endxy}
			\ex\raisebox{\dimexpr-\totalheight+2ex\relax}{\xy
				<0em,2cm>*\as{V}="c1",<1em,2cm>*\as{C}="c2",<2em,2cm>*\as{}="c3",<3em,2cm>*\as{C}="c4",<4em,2cm>*\as{V}="c5",
				<0em,1cm>*\as{a}="p1",<1em,1cm>*\as{s}="p2",<2em,1cm>*\as{w}="p3",<3em,1cm>*\as{k}="p4",<4em,1cm>*\as{e}="p5",
				"p1"+U;"c1"+D**\dir{-};"p2"+U;"c2"+D**\dir{-};"p3"+U;"c4"+D**\dir{-};"p4"+U;"c4"+D**\dir{-};"p5"+U;"c5"+D**\dir{-};
			\endxy}
		\end{xlist}\label{as:askwe}}
	\end{exe}
\end{multicols}

Similar examples of glide consonant metathesis are found in a number of languages
of the south-eastern Maluku area.
Such metathesis has been described for \ili{Fordata} and \ili{Yamdena} \citep[250]{mi91},
Roma (\srf{sec:Rom}), Luang (\srf{sec:Lua}) and Leti (\srf{sec:Let}).
See \frf{fig:CVMetTimReg} on \prf{fig:CVMetTimReg} for the locations of these languages.

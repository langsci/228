\subsection{Roma}\label{sec:Rom}
\il{Roma|(}Roma, or Romang, is an Austronesian language spoken on an
island of the same name to the
north-east of Timor (see \frf{fig:CVMetTimReg}).
It is closely related to both Leti and Luang.
Roma has been described by \cite{st91},
who focusses on the phonology of the language.

\subsubsection{Forms}
Roma has three different processes of metathesis.
Two of these metathesis processes are phonologically conditioned and one is morphological.
Firstly, Roma has a process of phonologically conditioned metathesis
in which a high vowel or glide metathesises with a following consonant.
This process is similar to the processes in Selaru (\srf{sec:PhoMet}) and Luang (\srf{sec:PhoMet}).
Examples are given in \qf{ex:RomPhoMet} below.

\begin{exe}
	\ex{Phonological glide metathesis: \hfill\citep[63f]{st91}}\label{ex:RomPhoMet}
	\sn{\gw\stl{0.3em}
	\begin{tabular}{rclclll}	
		\it{a\tbr{w}-}		&+&\it{\tbr{k}arar}	&\ra&\it{a\tbr{kw}arar}		&[ʔaˈkʷaʔar]&`I cry'\\
		\it{m\tbr{w}-}		&+&\it{\tbr{k}arar}	&\ra&\it{m\tbr{kw}arar}		&[m̩ˈkʷaɾar]&`you (sg.) cry'\\
		\it{m\tbr{j}-} 		&+&\it{\tbr{k}arar}	&\ra&\it{m\tbr{kj}arar}		&[m̩ˈkʲaɾar]&`you (pl.) cry'\\
		\it{n-ma\tbr{i}}	&+&\it{\tbr{m}e}		&\ra&\it{nama\tbr{mj}e}		&[n̩ˈmamɲe]&`he came in order to'\\
		\it{anik\tbr{u}}	&+&\it{\tbr{k}aka}	&\ra&\it{anik\tbr{kw}aka}	&[ˌʔanikˈkʷaka]&`my older brother'\\
	\end{tabular}}
\end{exe}

Secondly, /h/ obligatorily metathesises with a preceding consonant in Roma.
One example is \it{am-} \tsc{`pl.excl} + \it{hapa} `plant'
{\ra} \it{ahmapa} `we (excl.) plant' \citep[69]{st91}.

Thirdly, Roma has a process of final VC {\ra} CV morphological metathesis.
This process is similar to the same process described for Leti.
This process only affects consonant-final nouns in Roma.
Examples are given in \qf{ex:RomVC->CVMet} below.

\begin{exe}
	\ex{VC {\ra} CV metathesis \hfill\cite[64ff]{st91}}\label{ex:RomVC->CVMet}
		\sn{\gw\stl{0.5em}
		\begin{tabular}{lcll|lcll}
			U\=/form						&		&M\=/form							& gloss		&U\=/form							&		&M\=/form							& gloss\\
			\it{hiw\tbr{it}}	&\ra&\it{hiw\tbr{ti}}		&`machete'&\it{snjin\tbr{in}}	&\ra&\it{snjin\tbr{ni}}	&`song'\\
			\it{ul\tbr{it}}		&\ra&\it{ul\tbr{ti}}		&`skin'		&\it{ja\tbr{ir}}		&\ra&\it{ja\tbr{ri}}		&`wave'\\
			\it{ih\tbr{an}}		&\ra&\it{ih\tbr{na}}		&`fish'		&\it{o\tbr{ir}}			&\ra&\it{o\tbr{ri}}			&`water'\\
			\it{hur\tbr{at}}	&\ra&\it{hur\tbr{ta}}		&`letter'	&\it{hlja\tbr{ut}}	&\ra&\it{hlja\tbr{tu}}	&`story'\\
		\end{tabular}}
\end{exe}

Evidence that the consonant-final forms are underlying
comes from processes of consonant assimilation which occur after metathesis.
These processes include devoicing of medial /d/ and
assimilation of final /l/ and /r/.
(These processes of consonant assimilation 
are similar to those described for Leti on page \prf{ex:LetConAss}.)
Examples of Roma consonant assimilation are given in \qf{ex:RomConAss} below.

\begin{exe}
	\ex{Consonant assimilation \hfill\cite[31,65]{st91}}\label{ex:RomConAss}
		\sn{\gw\begin{tabular}{llll}
			U\=/form						&		&M\=/form						& gloss\\
			\it{ma\tbr{dar}}	&\ra&\it{ma\tbr{tta}}	&`cuscus'\\
			\it{o\tbr{dan}}		&\ra&\it{o\tbr{tna}}	&`drying rack'\\
			\it{wu\tbr{lan}}	&\ra&\it{wu\tbr{lla}}	&`moon'\\
			\it{me\tbr{lan}}	&\ra&\it{me\tbr{lla}}	&`mouse'\\
			\it{tja\tbr{lan}}	&\ra&\it{tja\tbr{lla}}&`road'\\
			%\it{}	&\ra&\it{}&`'\\
			%\it{}	&\ra&\it{}&`'\\
		\end{tabular}}%
\end{exe}

\subsubsection{Functions}\label{sec:RomFun}
Only nouns undergo metathesis in Roma.
Verbs occur with a single consonant-final form.
For nouns, metathesis has two main functions.
Firstly, subjects undergo metathesis while objects occur unmetathesised.
Metathesis is thus a subject marker or marker of nominative case.
Compare examples \qf{ex:RomMet1} and \qf{ex:RomMet2} below.

\begin{exe}\let\eachwordone=\itshape
	\ex{\gll	n-la n-dahal hiw\tbr{it}-a.\\
						\tsc{3sg}-go \tsc{3sg}-search machete{\textbackslash}\tbr{\tsc{u}}-\tsc{epenth}\\
			\glt	`He searched for a machete'}\label{ex:RomMet1}
	\ex{\gll	hiw\tbr{ti} ta-walli.\\
						machete{\textbackslash}\tbr{\tsc{m}} \tsc{neg}-exist\\
			\glt	`There wasn't any machetes.' \hfill\citep[67]{st91}}\label{ex:RomMet2}
\end{exe}

In \qf{ex:RomMet1} the noun \it{hiwit} is an object
and thus occurs unmetathesised.
The final vowel found after this object is an epenthetic
vowel which occurs after all phrase-final consonants \citep[69f]{st91}.
In \qf{ex:RomMet2} the same noun is the subject and thus occurs metathesised.

Secondly, nouns occur metathesised in isolation
(including the citation form) but
unmetathesised when an attributive modifier follows.
Metathesis thus signals that the noun is unmodified;
a kind of anti-construct form.
%This is the same function as the prefix \it{o-} in Tolaki (discussed in \srf{sec:ConFor}).
\cite{st91} gives the examples in \qf{ex:RomMetNouPhr} below.

\newpage
\begin{exe}
	\ex{Unmetathesised forms in the noun phrase: \hfill\citep[67]{st91}}\label{ex:RomMetNouPhr}
	\sn{\gw\begin{tabular}{rclcll}
		\it{horar\tbr{na}}	& &\it{}	&&\it{}&`clothes' (citation)\\
		\it{horaran}	&+&\it{ehi}	&\ra&\it{horar\tbr{an} ehi}&`these clothes'\\
		\it{krah\tbr{na}}	& &\it{}	&&\it{}&`house' (citation)\\
		\it{krahan}		&+&\it{popotna}	&\ra&\it{krah\tbr{an} popotna}&`large house'\\
		%\it{} &+&\it{}	&\ra&\it{}&`'\\
	\end{tabular}}
\end{exe}

However, if a genitive pronoun or the locative marker \it{la}
precedes the noun it obligatorily occurs in the unmetathesised form
even if a modifier follows.
\cite{st91} gives the examples in \qf{ex:RomMetLoc} below.

\begin{exe}
	\ex{Metathesis after locative or possessive pronouns: \hfill\citep[67]{st91}}\label{ex:RomMetLoc}
	\sn{\gw\begin{tabular}{l}
		\it{aniku} + \it{horar\tbr{an}} + \it{ehi} {\ra} \it{aniku horar\tbr{na} ehi}	`these clothes of mine'\\
		\it{la} + \it{krah\tbr{an}} + \it{popotna} {\ra} \it{la krah\tbr{na} popotna}	`at the large house'\\
	\end{tabular}}
\end{exe}

Similarly, before the enclitics \it{=ei} \tsc{def} and \it{=ida} \tsc{indef}
nouns obligatorily occur metathesised.
Final high vowels then become glides and final /a/ is deleted.
Glide formation and deletion of /a/
are both regular process in Roma which occur
whenever a vowel-initial enclitic or suffix attaches to a
host which ends in a vowel \citep[78f]{st91}.
Examples are given in \qf{ex:RomMetVowIni} below.

\begin{exe}
	\ex{Metathesis before vowel-initial enclitics: \hfill\citep[67]{st91}}\label{ex:RomMetVowIni}
	\sn{\gw\stl{0.5em}
	\begin{tabular}{rcllclll}
		\it{hiw\tbr{it}}	&+&\it{=ei}	&\ra&\it{hiw\tbr{ti}ei}		&\ra&\it{hiw\tbr{tj}ei}&`the machete'\\
	\it{horar\tbr{an}}	&+&\it{=ei}	&\ra&\it{horar\tbr{na}ei}	&\ra&\it{horar\tbr{n}ei}&`the clothes'\\
		\it{hlja\tbr{ut}}	&+&\it{=ida}&\ra&\it{hlja\tbr{tu}ida}	&\ra&\it{hlja\tbr{tw}ida}&`a story'\\
	\end{tabular}}
\end{exe}

In Roma metathesis marks the subject of a verb phrase
as well as signalling that a noun is unmodified.
Metathesis is also obligatory when a noun occurs after possessive pronouns, locative \it{la}
or before vowel-initial enclitics.

There are two similarities between metathesis in Roma and Amarasi.
Firstly, metathesis interacts with attributive modifiers.
In Roma metathesis signals lack of an attributive modifier
while in Amarasi metathesis signals the presence of an attributive modifier.
Secondly, in both Roma and Amarasi metathesis is obligatory
before vowel-initial enclitics.\il{Roma|)}
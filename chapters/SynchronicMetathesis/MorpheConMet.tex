\subsection{Morphemically conditioned metathesis}\label{sec:MorpheConMet}
Morphemically conditioned metathesis refers to instances of metathesis
which are triggered by the combination of morphemes,
but not any new phonological environment created by this combination.
%A number of languages with synchronic metathesis have both morphemically
%conditioned metathesis and morphological metathesis.
%Such languages include Rotuman (\srf{sec:Rot}),
%Tunisian Arabic, Mutsun Ohlone, Sierra Miwok, and Alsea.
%(Metathesis in these last five languages is discussed in Appendix \ref{app:MorMet}.)

Morphemically conditioned metathesis can be compared to more familiar
examples of morphemically conditioned processes,
such as \ili{German} umlaut in the formation of plural nouns.
In German, umlaut involves the fronting of a back vowel.
One environment which (often) triggers umlaut in German
is addition of either of the plural suffixes \it{-e} /-ə/ or \it{-er} /-ər/.
Examples of German nouns in which umlaut occurs
before plural \it{-e} /-ə/ are given in \qf{ex:GerUml} below.

\newpage
\begin{exe}
	\ex{German umlaut \hfill}\label{ex:GerUml}
	\sn{\gw\begin{tabular}{lllll}
			\mc{2}{l}{Singular} 					&\mc{2}{l}{Plural} & gloss		\\
			\it{Fuchs}	&/f\tbr{ʊ}ks/	&\it{Füchse}	&/f\tbr{ʏ}ksə/	&`fox'	\\
			\it{Fuß}		&/f\tbr{uː}s/	&\it{Füße}		&/f\tbr{yː}sə/	&`foot'	\\
			\it{Kopf}		&/k\tbr{ɔ}pf/	&\it{Köpfe}		&/k\tbr{œ}pfə/&`head'	\\
			\it{Sohn}		&/z\tbr{oː}n/	&\it{Söhne}		&/z\tbr{øː}nə/	&`son'	\\
			\it{Hand}		&/h\tbr{a}nt/	&\it{Hände}		&/h\tbr{ɛ}ndə/	&`hand'	\\
			\it{Zahn}		&/ʦ\tbr{aː}n/	&\it{Zähne}		&/ʦ\tbr{ɛː}nə/	&`tooth'	\\
			\it{Maus}		&/m\tbr{aʊ}s/	&\it{Mäuse}		&/m\tbr{ɔʏ}zə/	&`mouse'	\\
%			\it{}		&/\tbr{/		&\it{}	&/\tbr{ə/	&`'	\\
	\end{tabular}}
\end{exe}

It is not a universal feature of German phonology that back vowels are fronted before schwa.
This can be seen with other suffixes, such as the plural \it{-en} /-ən/ which does not trigger umlaut.
Two examples are \it{Dorn} /dɔrn/ `thorn' {\ra} \it{Dornen} /dɔrnən/
and \it{Frau} /fraʊ/ `woman' {\ra} \it{Frauen} /fraʊən/.
Similarly, not all words undergo umlaut before plural \it{-e} /-ə/.
Two examples are \it{Brot} /broːt/ `bread' {\ra} \it{Brote} /broːtə/ `breads'
and \it{Tag} /taːk/ `day' {\ra} \it{Tag} /taːɡə/ `days'.
Such data shows that the vowel of the suffix in examples such as \qf{ex:GerUml}
is not a plausible conditioning environment for triggering umlaut.

Such facts lead most analysts to view synchronic umlaut in German
as a process separate from that of suffixation.
This, for instance, is the approach taken by \citet[181ff]{wi96},
who posits that certain lexical entries in German have a floating
\tsc{[+front]} feature, the linking of which is triggered partly by morphological features.
\citet{wi96} analyses German umlaut as a lexical phonological rule
which is triggered in certain morphologically derived environments.

Under such an analysis, German umlaut is a phonological process
just like final obstruent devoicing (\srf{sec:PhoMet} \prf{sec:PhoMet}).
The difference between the two processes is that final 
obstruent devoicing is triggered by a phonological environment (word finally)
while umlaut is triggered by a morphological environment (e.g. plural).

One case of morphemically conditioned metathesis is
described by \citet{bu07} for Alsea (Penutian, Oregan).
In Alsea certain suffixes trigger sonorant-vowel metathesis while other suffixes do not.
One suffix which triggers metathesis is the intransitive imperative suffix \it{-χ},
while the phonologically identical realis completive suffix \it{-χ}
does not trigger metathesis.
Examples are given in \qf{ex:AlsMorphemicMet3} below,
in which (unmetathesised) stems with realis completive \it{-χ}
are given on the left and metathesised stems with
intransitive imperative \it{-χ} are given on the right.

\newpage
\begin{exe}
	\ex{Alsea morphemically conditioned metathesis \hfill\citep[8f]{bu07}}\label{ex:AlsMorphemicMet3}
	\sn{\gw\begin{tabular}{lrll}
											&\tsc{cmpl.rl}			&\tsc{intr.imp}		& \\
		`dances with them'&\it{k\tbr{ná}χ-χ}	&\it{k\tbr{án}χ-χ}&`dance with them!' \\
		`are lying in bed'&\it{ʦ\tbr{nú}s-χ}	&\it{ʦ\tbr{ún}s-χ}&`lie down!' \\
		`is hiding'				&\it{p\tbr{já}χ-χ}	&\it{p\tbr{áj}χ-χ}&`hide!' \\
		`is floating'			&\it{ʦp\tbr{jú}t-χ}	&\it{ʦp\tbr{új}t-χ}&`float!' \\
	\end{tabular}}		
\end{exe}

Unlike metathesis in Faroese, Sidamo, or Selaru discussed
in \srf{sec:PhoMet}, metathesis in Alsea cannot be derived
from any new phonological environment created by the concatenation of morphemes
-- after all, the phonological properties of the
realis completive \it{-χ} suffix and intransitive imperative \it{-χ} are identical.
Instead, like German umlaut which is triggered by certain suffixes
but not by the phonological properties of those suffixes,
metathesis in Alsea is morphemically conditioned.
Alsea metathesis is discussed in more detail in \srf{sec:Als}.
\subsection{Rotuman}\label{sec:Rot}
\il{Rotuman|(}Rotuman has perhaps the most famous case of morphological metathesis.
Rotuman is an Austronesian Oceanic language spoken on Rotuma island
in the Pacific Ocean located about 480 kilometres north of the main islands of Fiji.
Metathesis occurs in multiple environments in Rotuman.
In some cases metathesis is phonologically conditioned (\srf{sec:PhoMet}),
in some cases it is morphemically conditioned (\srf{sec:MorpheConMet}),
and in some cases it is morphological (\srf{sec:MorMet}).

Rotuman was first described by \citet{ch40}
who provides a grammar and dictionary of the language.
\citeauthor{ch39} also published several Rotuman texts
between 1937--39 in the journal \it{Oceania}
which were reprinted in one volume as \citet{ch39}.
Both \citet{be87} and \citet{va02} also present descriptions of
Rotuman based on their own fieldwork.
Each of these descriptions differs in details.
This may be partly because the authors worked with different
speakers at different times and may also be partly because they use
different terminology to describe the same phenomena.

\subsubsection{Forms}\label{sec:RotFor}
Each word in Rotuman has two forms, which I call the U\=/form and M\=/form.
The traditional names coined by \cite{ch40} are the \it{complete phase} for the U\=/form
and the \it{incomplete phase} for the M\=/form.
The U\=/form is historically more conservative than the M\=/form.

\cite{ch40} identifies four phonological processes
which derive the M\=/form from the U\=/form.
These processes are vowel deletion (a.k.a apocope, truncation, or subtraction),
umlaut, metathesis, and vowel shortening.
There are also words which do not have two distinct forms.
Which process applies depends on the phonological shape of the U\=/form.
%Examples are given in \qf{ex:RotLonShoChu} below,
%and discussed in more detail in \srf{sec:RotDip}--\srf{sec:RotSum}.

%\begin{exe}
%	\ex{Rotuman U\=/forms and M\=/forms \hfill\citep[85]{ch40}}\label{ex:RotLonShoChu}
%		\sn{\begin{tabular}{llll}
%			U\=/form&M\=/form	&						&Process \\
%			\it{mata}	&\it{mat}			&`wet'			&apocope \\
%			\it{mose}	&\it{mœs}			&`to sleep'	&umlaut	\\
%			\it{toka}	&\it{toak}		&`to cease'	&metathesis\\
%			\it{ʧao}	&\it{ʧ\u{a}o}	&`spear'		&V shortening \\
%			\it{rii}	&\it{rii}			&`house'		&n./a. \\
%		\end{tabular}}
%\end{exe}

\paragraph{Vowel shortening/diphthongisation}\label{sec:RotDip}
%For words which end in a sequence of non-identical vowels,
For words which end in a vowel sequence in which the vowels are non-identical,
\cite[85]{ch40} describes the M\=/form as being formed
by shortening the initial vowel of the sequence.
Examples are given in \qf{VV->VV-Chu} below.

\begin{exe}
	\ex{Rotuman V{\sub{α}}V{\sub{β}} {\ra} \u{V}{\sub{α}}V{\sub{β}} \hfill\citep[85]{ch40}}\label{VV->VV-Chu}
	\sn{\gw\begin{tabular}{lcll}
			U\=/form		&		&M\=/form			&\\
		\it{pupui}	&\ra& \it{pup\u{u}i}	& `floor' \\
		\it{ʔesʔao} &\ra& \it{ʔesʔ\u{a}o} & `useful' \\
		\it{lelei}	&\ra& \it{lel\u{e}i}	& `good' \\
		\it{foʔou}	&\ra& \it{foʔ\u{o}u}	& `new' \\
	\end{tabular}}
\end{exe}

\cite{va02} describes a process of diphthongisation
in which the less sonorous vowel becomes a glide.
This glide formation may be either a further development 
of \citeauthor{ch40}'s shortened vowels, or it may that
a single phenomenon was perceived and described differently
by each of these authors.

\begin{exe}
	\ex{Rotuman V{\sub{α}}V{\sub{β}} {\ra} V̯{\sub{α}}V{\sub{β}} {\tl} V{\sub{α}}V̯{\sub{β}}
	\hfill\citet[4,7--9]{va02}}\label{VV->VV-Vam}
		\sn{\gw\begin{tabular}{lcll}
			U\=/form			&		&	M\=/form		&\\
			\it{lio}		&\ra& \it{ljo}	& `voice' \\
			\it{fau}		&\ra& \it{faw}	& `year' \\
			\it{fui}		&\ra& \it{fuj}	& `piece of garland' \\
			\it{fɒi}		&\ra& \it{fɒj}	& `chop down' \\
			\it{momoe}	&\ra& \it{momoe̯} & `k.o. tree' \\
		\end{tabular}}
\end{exe}

According to \citet[210]{be87} the vowel sequences which
diphthongise are those in which the second vowel is /a/
as well as sequences of a high vowel followed by /o/.
\citeauthor{be87} also reports that /a/
is realised as [ɔ] after a glide derived
from one of the high-front vowels.

\begin{exe}
	\ex{Rotuman V{\sub{α}}V{\sub{β}} {\ra} V̯{\sub{α}}V{\sub{β}} \hfill\citep[210]{be87}}\label{RotDip-be}
	\sn{\gw\begin{tabular}{lcll}
			U\=/form	&		&	M\=/form		&\\
		\it{ʔea}	&\ra& \it{ʔja}	& `to say' \\
		\it{foa}	&\ra& \it{fwa}	& `coconut scraper' \\
		\it{kia}	&\ra& \it{kjɔ}	& `neck' \\
		\it{sua}	&\ra& \it{swɔ}	& `shoot (of a plant)' \\
	\end{tabular}}
\end{exe}

\paragraph{Metathesis}\label{sec:RotMet}
When the U\=/form ends in VCV and the penultimate vowel is higher than the final vowel,
the M\=/form is derived by final consonant-vowel metathesis.
Examples are given in \qf{ex:VCV->VVC} below.

\newpage
\begin{exe}
	\ex{Rotuman V\sub{1}CV\sub{2} {\ra} V\sub{1}V\sub{2}C \hfill\citep[14]{ch40}}\label{ex:VCV->VVC}
		\sn{\gw\begin{tabular}{lcll}
			U\=/form					&		&M\=/form&\\
			\it{pu\tbr{re}} &\ra& \it{pu\tbr{er}} & `to rule, decide' \\
			\it{ho\tbr{sa}} &\ra& \it{ho\tbr{as}} & `flower' \\
			\it{ti\tbr{ko}} &\ra& \it{ti\tbr{ok}} & `flesh' \\
			\it{pe\tbr{pa}} &\ra& \it{pe\tbr{ap}} & `paper' \\
		\end{tabular}}
\end{exe}

Both \citet{va02} and \citet{be87} report that after metathesis
the penultimate vowel becomes a glide;
/u/ and /o/ become [w] while /i/ and /e/ become [j].
Examples are given in \qf{ex:VCV->VVC-Vam} below.

\begin{exe}
	\ex{Rotuman V\sub{1}CV\sub{2} {\ra} V̯\sub{1}V\sub{2}C \hfill\citep[3]{va02}}\label{ex:VCV->VVC-Vam}
	\sn{\gw\begin{tabular}{lcll}
			U\=/form				&		&	M\=/form&\\
		\it{pu\tbr{re}} &\ra& \it{pw\tbr{ɛr}} & `rule' \\
		\it{fu\tbr{pa}} &\ra& \it{fw\tbr{ap}} & `to distribute' \\
		\it{ʔi\tbr{ko}} &\ra& \it{ʔj\tbr{ɔk}} & `thrust' \\
	\end{tabular}}
\end{exe}

\citet[208]{be87} reports that when the penultimate vowel is a high vowel,
the final vowel becomes [ɔ] after metathesis.
Otherwise, the final vowel retains its original quality.
Examples are given in \qf{ex:VCV->VVC-Bes} below.

\begin{exe}
	\ex{Rotuman V\sub{1}CV\sub{2} {\ra} V̯\sub{1}V\sub{2}C \hfill\citep[208]{be87}}\label{ex:VCV->VVC-Bes}
	\sn{\gw\begin{tabular}{lcll}
			U\=/form				&		&	M\=/form&\\
		\it{ti\tbr{fe}}	&\ra& \it{tj\tbr{ɔf}} & `pearl shell' \\
		\it{pi\tbr{ʧa}} &\ra& \it{pj\tbr{ɔʧ}} & `rat' \\
		\it{hu\tbr{ŋe}}	&\ra& \it{hw\tbr{ɔŋ}} & `to breathe' \\
		\it{pu\tbr{ka}}	&\ra& \it{pw\tbr{ɔk}} & `k.o. creeper' \\
		\it{he\tbr{pa}}	&\ra& \it{hj\tbr{ap}} & `broad' \\
		\it{lo\tbr{ŋa}}	&\ra& \it{lw\tbr{aŋ}} & `towards the interior of the island' \\
	\end{tabular}}
\end{exe}

It is not entirely clear whether the diphthongisation
after metathesis reported by \cite{be87} and \cite{va02}
is a recent development or whether it was also present
while \citeauthor{ch40} worked on Rotuman.

On the one hand, it is clear from the detailed account of
Rotuman phonetics given by \citet[64--84]{ch40}
that he was an excellent phonetician.
Given his identification of shortened vowels
in the derivation of M\=/forms (\srf{sec:RotDip}),
it seems likely that if diphthongisation (or shortened vowels)
were present after metathesis he would have reported it.

On the other hand, \citet[86]{ch40} states
``the stress seems to be levelled out, so
to speak, in the inc[omplete] phase.
Thus: \it{fo}ra becomes \it{foar}, which is pronounced almost,
though perhaps not quite, as one syllable,
the stress being evenly distributed [\ldots]''
This statement perhaps indicates that diphthongisation
was an optional feature of Rotuman metathesised
forms in \citeauthor{ch40}'s day.

\paragraph{Umlaut}\label{sec:RotUml}
When the penultimate vowel is a back vowel and the final vowel a front vowel,
the M\=/form is derived via umlaut of the penultimate vowel
so long as this vowel is not higher than the final vowel.

\citet[79]{ch40} reports that /u/ becomes [y],
/o/ becomes [œ] when the final vowel is /e/,
and that /o/ becomes [ø] when the final vowel is /i/.\footnote{
	\cite{ch40} describes the vowel in the M\=/form of oCe{\#} final words
	(e.g. \it{mose} {\ra} \it{mœs} `sleep')
	as ``[\ldots] similar to the wider German \it{ö}, as in \it{gespött},
	and to the sound of \it{eu} in the French \it{jeune}.''
	He contrasts ``normal \it{ö}''  (which ``[\ldots]
	arises in place of normal \it{o} when a following \it{e} is elided'')
	with so-called ``narrow \it{ö}'' (arising ``[\ldots]
	in place of narrow \it{o} when a following \it{i} is elided'')
	which is described as ``[\ldots] similar to the narrower German \it{ö},
	as in \it{schön},
	and to the sound of \it{eu} in the French \it{peu}.''
	I interpret ``normal \it{ö}'' as a mid-low front-rounded vowel [œ]
	and ``narrow \it{ö}'' as a mid-high front-rounded vowel [ø].}
He also transcribes the outcome of umlauted /ɒ/ as \it{<\.a>},
describing it as ``[\ldots] a little wider [lower] than \it{a} in `cat' [\ldots]
but differs from it in containing just a suggestion of the sound of \it{u} in `cut' or `but.'{''}
I interpret \citeauthor{ch40}'s \it{<\.a>} as a low front rounded vowel [ɶ].

Examples of Rotuman umlaut are given in \qf{RotUml-ch} below,
which also gives hypothetical intermediate forms
showing the way such umlaut probably developed from metathesis.
In Kwara'ae words containing some of the vowel combinations
shown in \qf{RotUml-ch} have M\=/forms which
vary between displaying metathesis and umlaut (\srf{sec:KwaVoShi}).

\begin{exe}
\ex{V\tsc{[+rnd]}CV\tsc{[+fr]} {\ra} V\tsc{[+rnd,+fr]}C \hfill\citep[79-80]{ch40}}\label{RotUml-ch}
	\sn{\gw\begin{tabular}{lclcll}
			U\=/form	& &						& &	M\=/form&\\
		\it{ʔuli} &>&{*ʔuil}	&>& \it{ʔyl} & `skin' \\
		\it{mori} &>&{*moir}	&>& \it{mør} & `orange (fruit)'	 \\
		\it{mose} &>&{*moes}	&>& \it{mœs} & `to sleep' \\
		\it{ʔɒfi} &>&{*ʔɒif}	&>& \it{ʔɶf} & `to bite' \\
	\end{tabular}}
\end{exe}

\citet{va02} reports that /o/ becomes [ø] under umlaut,
/u/ becomes [y] and /ɒ/ becomes the lower mid-front-rounded [œ].
Examples are given in \qf{RotUml-va}

\newpage
\begin{exe}
\ex{Rotuman V\tsc{[+ba]}CV\tsc{[+fr]} {\ra} V\tsc{[+fr]}C \hfill\citep[3]{va02}}\label{RotUml-va}
	\sn{\gw\begin{tabular}{lcll}
			U\=/form	&		&	M\=/form&\\
		\it{futi} &\ra& \it{fyt} & `to pull' \\
		\it{mose} &\ra& \it{møs} & `to sleep' \\
		\it{pɒri} &\ra& \it{pœri} & `banana' \\
	\end{tabular}}
\end{exe}

\citeauthor{be87}'s data agrees with \citeauthor{va02} on the outcome of /o/ and /u/,
though he reports that /ɔ/ (equivalent to \citeauthor{ch40}'s and \citeauthor{va02}'s /ɒ/)
becomes either [ɛ] or [æ] in free variation in certain words.
Examples are given in \qf{RotUml-be} below.

\begin{exe}
\ex{Rotuman V\tsc{[+ba]}CV\tsc{[+fr]} {\ra} V\tsc{[+fr]}C \hfill\citep[209]{be87}}\label{RotUml-be}
	\sn{\gw\begin{tabular}{lcll}
			U\=/form	&		&	M\=/form&\\
		\it{pɔti} &\ra& \it{pɛt} & `scar' \\
		\it{hɔʔi} &\ra& \it{hɛʔ} & `to pull' \\
		\it{pɔni} &\ra& \it{pɛn} & `paint' \\
	\end{tabular}}
\end{exe}

All authors agree that umlaut of /u/ or /o/ spreads leftwards to identical vowels.
Examples are given in \qf{RotUmlSpr} below

\begin{exe}
\ex{Rotuman umlaut spreading: \hfill\citep[79f]{ch40}}\label{RotUmlSpr}
	\sn{\gw\begin{tabular}{lcll}
			U\=/form			&\ra&	M\=/form&\\
		\it{furfuruki}&\ra& \it{fyrfyryk} & `pimple' \\
		\it{roromi} 	&\ra& \it{rørøm} & `unexpectedly' \\
		\it{popore} 	&\ra& \it{pœpœr} & `to dash, dart' \\
	\end{tabular}}%
\end{exe}

\paragraph{Apocope}\label{sec:RotApo}
In all situations not covered by diphthongisation, metathesis, or umlaut,
the M\=/form is derived by deleting the final vowel of the U\=/form.
This includes when each vowel is identical and
when the penultimate vowel is lower than a final back vowel.
Examples are shown in \qf{ex:VCV->VC} below.

\begin{exe}
	\ex{Rotuman VCV {\ra} VC \hfill\citep[13]{ch40}}\label{ex:VCV->VC}
	\sn{\gw\begin{tabular}{lcll}
			U\=/form		&		&	M\=/form			&\\
		\it{haŋa}		&\ra& \it{haŋ}		& `to feed' \\
		\it{hɒŋu}		&\ra& \it{hɒŋ}		& `to awaken' \\
		\it{læʧe}	&\ra& \it{læʧ}		& `coral' \\
		\it{tokiri} &\ra& \it{tokir}	& `to roll' \\
		\it{hoto} 	&\ra& \it{hot}		& `to jump' \\
		\it{heleʔu} &\ra& \it{heleʔ}	& `to arrive' \\
	\end{tabular}}
\end{exe}

The lack of overt metathesis in such examples is comparable
to the Amarasi data in which words with a certain phonotactic
shape form their M\=/form by surface vowel deletion and/or consonant deletion (Chapter \ref{ch:StrMetAma}).

\paragraph{No change}
Words ending in two identical vowels
do not usually have distinct U\=/forms and M\=/forms according to \citet[85]{ch40},
except before certain suffixes in which case the final vowel of U\=/form is lengthened.
Examples are given in \qf{ex:VV->VV} below.

\begin{exe}
	\ex{Rotuman V{\sub{α}}V{\sub{α}} {\ra} V{\sub{α}}V{\sub{α}} \hfill\citep[85]{ch40}}\label{ex:VV->VV}
	\sn{\gw\begin{tabular}{lcll}
			U\=/form	&		&	M\=/form&\\
		\it{rii}	&\ra& \it{rii}	& `house' \\
		\it{ree}	&\ra& \it{ree}	& `to do' \\
%		\it{reeː-} & \it{ree-}	& `to do' \\
	\end{tabular}}
\end{exe}

\citet{be87} reports that when the sequence of two identical vowels is /aa/,
the M\=/form is formed by deleting the final vowel.
In other situations \citeauthor{be87} reports no difference in the two forms.
Examples are given in \qf{ex:aa->a} below.

\begin{exe}
	\ex{Rotuman /aa/ {\ra} /a/ \hfill\citep[212]{be87}}\label{ex:aa->a}
	\sn{\gw\begin{tabular}{lcll}
		 U\=/form		&		&	M\=/form&\\
		\it{ʔaa}	&\ra& \it{ʔa} & `bite' \\
		\it{ree}	&\ra& \it{ree} & `do' \\
		\it{luu}	&\ra& \it{luu} & `rope' \\
	\end{tabular}}%
\end{exe}

%This is consistent with \citeauthor{be87}'s account of diphthongisation in Rotuman.
%in which he reports that the only
%combinations of non-high vowels which diphthongise
%are those involving the vowel /a/
%(see example \qf{RotDip-be} \prf{RotDip-be}).

\paragraph{Summary of forms}\label{sec:RotSum}
The ways in which the Rotuman M\=/form is derived from the U\=/form
for CV{\#} final words are shown in \trf{tab:RotLonShoFor}.
In most cases the M\=/form is one syllable shorter than the U\=/form,
the main exceptions being word-final sequences of identical vowels
and \citeauthor{ch40}'s metathesised forms.

\begin{table}[ht]\stl{0.4em}
	\caption{Medial vowels of Rotuman U-forms and M-forms} \label{tab:RotLonShoFor}
		\begin{tabular}{r|ccccc|ccccc|ccccc|l}
		\lsptoprule
					&\mc{5}{c|}{\citet{ch40}}&\mc{5}{c|}{\citet{va02}}&\mc{4}{c}{\citet{be87}}\\
	V\sub{1}{\da}	&i	&e	&a	&o	&u	&i	&e	&a	&o	&u	&i	&e	&a	&o	&u	&{\la}V\sub{2}\\\midrule
							i	&i	&ie	&ia	&io	&i	&i	&jɛ	&ja	&jɔ	&i	&i	&jɔ	&jɔ	&jo	&i	&i\\
							e	&e	&e	&ea	&e	&e	&ɛ	&ɛ	&ja	&ɛ	&ɛ	&e	&e	&ja	&e	&e	&e\\
							a &ɶ	&æ	&a	&a	&ɒ	&œ	&æ	&a	&a	&ɒ	&ɛ	&ɛ	&a	&a	&ɔ	&a\\
							o	&ø	&œ	&oa	&o	&o	&ø	&ø	&wa	&ɔ	&ɔ	&ø	&ø	&wa	&o	&o	&o\\
							u	&y	&ue	&ua	&uo	&u	&y	&wɛ	&wa	&wɔ	&u	&y	&wɔ	&wɔ	&wo	&u	&u\\
		\lspbottomrule
	\end{tabular}
\end{table}

\subsubsection{Distribution of metathesis}\label{sec:RotFun}
Three uses of M\=/forms can be identified in Rotuman:
phonologically conditioned, morphemically conditioned, and morphological.
Each is discussed in turn.

\paragraph{Phonologically conditioned M\=/forms}\label{sec:RotPhoConMfo}
\citet{haki98} show that, with two exceptions,
the U\=/form is used before suffixes and enclitics
which are monosyllabic or non-syllabic,
while the M\=/form is used before polysyllabic suffixes and enclitics.

An example of the U\=/form before a monosyllabic suffix is given in \qf{RotUse1}
and an example before a non-syllabic suffix is given in \qf{RotUse2}.
An example of the M\=/form before a disyllabic affix is given in \qf{RotUse3}
and an example before a trisyllabic enclitic is given in \qf{RotUse4}.
These examples are taken from \cite[120f]{haki98}

\begin{exe}\let\eachwordone=\itshape
	\ex{\gll	puʔa + ŋa {\ra} puʔa-ŋa \\
						{be greedy} {} \tsc{nmlz} {} greedy{\U}-\tsc{nmlz} \\
			\glt	`greed'}\label{RotUse1}
	\ex{\gll	vaka + t {\ra} vaka-t \\
						canoe {} \tsc{sg} {} canoe{\U}-\tsc{sg} \\
			\glt	`a canoe'}\label{RotUse2}
	\ex{\gll	furi + ʔian {\ra} fyr-ʔian \\
						turn {} \tsc{ingressive} {} turn{\M}-\tsc{ingressive} \\
			\glt	`start turning' }\label{RotUse3}
	\ex{\gll	vaka + teʔisi {\ra} vak=teʔisi \\
						canoe {} this {} canoe{\M}=this \\
			\glt	`this canoe' }\label{RotUse4}
\end{exe}

Similarly, each non-final word in the noun phrase occurs in the M\=/form.
That is, the M\=/form is used when a noun is modified;
it is used to mark the presence of a dependent modifier.
This is also a function of metathesis in Leti (\srf{sec:Let})
and Amarasi (Chapter \ref{ch:SynMet}).

Compare the phrases in \qf{ex:ThePeoAre} and \qf{ex:TheZeaPeo} below, from \citet[14]{ch40}.
Each phrase consists of the noun \it{famori} `people'
followed by the adjective \it{feʔeni} `zealous'.
In \qf{ex:ThePeoAre} the noun \it{famori} `people' is in the U\=/form
and the adjective has a predicative reading,
as illustrated in \qf{tr:ThePeoAre}.
In \qf{ex:TheZeaPeo} the noun \it{famør} `people' is in the M\=/form,
and the adjective has an attributive meaning,
as illustrated in \qf{tr:TheZeaPeo}.
(The use of the M\=/form of the adjective in \qf{ex:ThePeoAre}
and \qf{tr:ThePeoAre} is discussed in \srf{sec:RotMorMfo} below.)

\begin{multicols}{2}
	\begin{exe}\let\eachwordone=\itshape
		\ex{\gll fam\tbr{ori} feʔen\\
						people{\tbrU} zealous{\M}\\
				\glt `The people are zealous.'}\label{ex:ThePeoAre}
		\ex{\gll fam\tbr{ør} feʔeni\\
						people{\tbrM} zealous{\U}\\
				\glt `(The) zealous people.'}\label{ex:TheZeaPeo}
	\end{exe}
\end{multicols}
\begin{multicols}{2}
	\begin{exe}
		\ex{\begin{forest} where n children=0{tier=word}{}
			[S,[NP,[N,[\it{fam\tbr{ori}}\\people{\tbrU}]]][PRED,[\it{feʔen}\\zealous{\M}]]]
		\end{forest}}\label{tr:ThePeoAre}
		\ex{\begin{forest}
			[S,[NP,[N,[\it{fam\tbr{ør}}\\people{\tbrM}]][ADJ,[\it{feʔeni}\\zealous{\U}]]][{\ldots}]]
		\end{forest}}\label{tr:TheZeaPeo}
	\end{exe}
\end{multicols}

The generalisation identified by \citet{haki98}
is that the M\=/form is (mostly) used before polysyllabic modifiers,
while the U\=/form is used elsewhere.
This generalisation is the basis for the analysis
of \cite{mcc00} under the frameworks of prosodic
morphology and Optimality Theory.
This analysis is discussed in more detail in \srf{sec:ProMorRot}.

\paragraph{Morphemically conditioned M\=/forms}
As acknowledged by \citet{haki98},
there are two exceptions to their generalisation
that the M\=/form occurs before polysyllabic suffixes,
enclitics, and modifiers.

The first exception is the monosyllabic singular marker \emph{-ta}.
Before this article M\=/forms occur, despite the fact that this suffix is monosyllabic.
An example of is given in \qf{RotUse6} below.

\begin{exe}
\let\eachwordone=\itshape
	\ex{\gll mori + ta {\ra} m{\o}r-ta *mori-ta\\
		{orange} {} \tsc{sg} {} orange{\textbackslash}\tsc{m}-\tsc{sg} \\
		\glt `the orange' \hfill\citep[14]{va02}}\label{RotUse6}
\end{exe}

The second exception is that M\=/forms of nouns are used without any
affix or enclitic for plural indefinite,
while the U\=/form is used for plural definite.
Examples are given in \qf{ex:RotDef} and \qf{ex:RotInd}
from \citet[15]{ch40}

\begin{multicols}{2}
	\begin{exe}\let\eachwordone=\itshape
		\ex{\gll fam\tbr{ori} ʔea\\
						people{\tbrU} say\\
				\glt `The people say.'}\label{ex:RotDef}
		\ex{\gll fam\tbr{ør} ʔea\\
						people{\tbrM} say\\
				\glt `Some people say.' }\label{ex:RotInd}
	\end{exe}
\end{multicols}

\citet[121f]{haki98} analyse these exceptions by positing zero affixes with moraic weight.
Their analysis of the exceptional forms of \it{vaka/vak} `canoe'
is shown in \qf{ex:RotExcShoFor} below.\footnote{
	I cannot find a clear explanation in \cite{haki98} for why the noun \it{vaka}
	surfaces in the U\=/form when followed by the two suffixes
	{\0}\sub{\tsc{pl}} and {\0}\sub{\tsc{def}}.
	If I understand the analysis correctly,
	each null suffix should have moraic weight,
	with this combination of two suffixes being poly-moraic (polysyllabic)
	and thus triggering the M\=/form.}

\begin{exe}
\ex{Rotuman exceptional M\=/forms: \hfill\citep[122]{haki98}}\label{ex:RotExcShoFor}
	\sn{\gw\begin{tabular}{lll}
		\it{vaka} 	& /vaka + {\0}\sub{\tsc{pl}} + {\0}\sub{\tsc{def}}/ & `the canoes' \\
		\it{vak ta} & /vaka + ta + {\0}\sub{\tsc{def}}/ & `the one canoe' (i.e. `the canoe') \\
		\it{vaka-t} & /vaka + ta/ & `a/one canoe'\\
	\end{tabular}}
\end{exe}

An analysis involving multiple null suffixes with moraic weight
is not particularly convincing as an appropriate synchronic analysis of the Rotuman data.
Instead, given that U\=/forms are normally used before monosyllabic suffixes,
uses of the M\=/form before the singular suffix \it{-ta}
is better analysed as morphemically conditioned
and use of the M\=/forms to mark an indefinite plural is better
analysed as a morphological use of M\=/forms.

\paragraph{Morphological M\=/forms}\label{sec:RotMorMfo}
In addition to phonologically conditioned M\=/forms before polysyllabic modifiers
and morphemically conditioned M\=/forms before the singular suffix \it{-ta},
morphological M\=/forms also occur
as the only phonological realisation of a semantic difference.
A number of different morphological uses of M\=/forms can be identified in Rotuman.

Firstly, as mentioned above,
U\=/forms and M\=/forms are used in noun phrases to mark definiteness.
When the final word of the noun phrase is in the U\=/form it is definite plural,
when the final word is in the M\=/form it is indefinite.
Examples\qf{ex:RotDef} and \qf{ex:RotInd} above
are repeated as \qf{ex2:RotDef} and \qf{ex2:RotInd} below to illustrate.

\begin{multicols}{2}
	\begin{exe}\let\eachwordone=\itshape
		\ex{\gll fam\tbr{ori} ʔea\\
						people{\tbrU} say\\
				\glt `The people say.'}\label{ex2:RotDef}
		\ex{\gll fam\tbr{ør} ʔea\\
						people{\tbrM} say\\
				\glt `Some people say.' }\label{ex2:RotInd}
	\end{exe}
\end{multicols}

Secondly, verbs and predicative adjectives normally occur in the M\=/form.
This has already been seen in \qf{ex:ThePeoAre} above, repeated as \qf{ex2:ThePeoAre} below.
This is due to ``[\ldots] the general rule that, except in certain circumstances, a verb
-- or an adjective used as a verb -- is used in its incomplete phase [M\=/form]'' \citep[15]{ch40}.
This is similar to Amarasi in which the default form of verbs is the M\=/form
(see \srf{sec:DefFor1}).

\begin{exe}\let\eachwordone=\itshape
	\ex{\gll famori feʔ\tbr{en}\\
						people{\U} zealous{\tbrM}\\
			\glt `The people are zealous.'}\label{ex2:ThePeoAre}
\end{exe}

One environment in which verbs and adjectives occur in the U\=/form is to mark
``positiveness, finality or (in questions) the desire to be positive or certain'' \citep[88]{ch40}.
This function also occurs with a number of other word classes including
locative pronouns, some temporal nouns, demonstratives, and interrogative pronouns.
Two examples of Rotuman U\=/form questions with corresponding answers
are given in \qf{ex:RotQA1} and \qf{ex:RotQA2} below.

\begin{multicols}{2}
	\begin{exe}\let\eachwordone=\itshape
			\sn{Rotuman U\=/form questions:}
		\ex{\begin{xlist}
				\ex{\gll ʔe u\tbr{na}\\
								\tsc{loc} middle{\tbrU}\\
						\glt `In the middle, did you say?'}\label{ex:QeUan}
			\sn{\hfill \citep[95]{ch40}}
				\ex{\gll ʔe u\tbr{an}\\
								\tsc{loc} middle{\tbrM}\\
						\glt `In the middle.'}\label{ex:QeUna}
		\end{xlist}}\label{ex:RotQA1}
	\end{exe}
\end{multicols}
\begin{multicols}{2}
	\begin{exe}\let\eachwordone=\itshape
		\ex{\begin{xlist}
				\ex{\gll ʔe fapʔa{\ng}\tbr{a}\\
								\tsc{loc} three.days{\tbrU}\\
						\glt `In three days time, did you say?'}\label{ex:QeFapqanga}
				\ex{\gll ʔe fapʔa{\ng}\\
								\tsc{loc} three.days{\tbrM}\\
						\glt `In three days time.'}\label{ex:QeFapqang}
		\end{xlist}}\label{ex:RotQA2}
	\end{exe}
\end{multicols}

\citet[95]{ch40} also gives the imperative \it{leume!} `come{\U}' which is
``freq[uently] used when one or more calls of \it{leum!} [`come{\M}']
fail to move the person summoned'' as another example of this ``positiveness'' use.

The use of U\=/forms in Rotuman with verbs (and some other word classes) to mark ``positiveness''
is comparable the use of Amarasi U\=/forms on verbs
(and some other word classes) to mark discourse structures.
In Amarasi, such U\=/forms mark an unresolved state/event
which requires another clause for resolution (Chapter \ref{ch:DisMet}).
In particular, in both Rotuman and Amarasi, verbal U\=/forms are used in questions (\srf{sec:IntUnm}).

Finally, \citet[88]{ch40} states that for verbs ending in a pronominal suffix,
the U\=/form is used to mark the completive tense, though he does not give examples.
This use of verbal U\=/forms is similar to Helong (\srf{sec:VerMet})
in which U\=/forms mark the perfective aspect.\il{Rotuman|)}
\section{Kinds of synchronic metathesis}\label{sec:KinSynMet}
In this section I present a categorisation of
processes of synchronic metathesis.
I identify three kinds of metathesis: 
phonologically conditioned metathesis (\srf{sec:PhoMet}),
morphemically conditioned metathesis (\srf{sec:MorpheConMet}),
and morphological metathesis (\srf{sec:MorMet}).
The categorisation into these three types of metathesis
is intended to facilitate an understanding of different
metathesis patterns and their systematicity.
I discuss each type of synchronic metathesis
and relate them to other, more familiar, phonological processes.

It is frequently the case that a unitary analysis of 
a single process of synchronic
metathesis is not always possible.
Such a process of metathesis may be phonologically
conditioned in some environments, morphemically conditioned in others,
and morphological in yet other situations.
This, for instance, is the situation with Rotuman metathesis (\srf{sec:Rot}).
It is also the situation in Amarasi which has
phonologically conditioned metathesis
before vowel-initial enclitics (Chapter \ref{ch:PhoMet})
and two process of morphological metathesis
(Chapter \ref{ch:SynMet} and \ref{ch:DisMet}).

One kind of synchronic metathesis which does not fit
into any of these three categories is when metathesised
and unmetathesised forms are in free variation.
This situation is found in Kui (Trans-New-Guinea, Alor),
in which the perfective affix \it{-i}
optionally metathesises with a previous /n/ or /l/.
Examples are given in \qf{ex:KuiPer} below.
As currently described, this alternation is a case of free variation.

\begin{exe}
	\ex{Kui metathesis of perfective \it{-i} \citep[124f]{wish17}}\label{ex:KuiPer}
	\sn{\gw\begin{tabular}{rclllllll}
		\it{alon} 	&+&\it{i}&{\ra}& \it{alon\tbr{i}}		&{\tl}& \it{alo\tbr{i}n} &`write'\\
		\it{gaman} 	&+&\it{i}&{\ra}& \it{gaman\tbr{i}}	&{\tl}& \it{gama\tbr{i}n} &`do'\\
		\it{aka:l} 	&+&\it{i}&{\ra}& \it{akaːl\tbr{i}}	&{\tl}& \it{akaː\tbr{i}l} &`eat'\\
		\it{taŋgan} &+&\it{i}&{\ra}& \it{taŋgan\tbr{i}}	&{\tl}& \it{taŋga\tbr{i}n} &`ask'\\
		\it{uban} 	&+&\it{i}&{\ra}& \it{uban\tbr{i}}		&{\tl}& \it{uba\tbr{i}n} &`talk'\\
		\it{gatan} 	&+&\it{i}&{\ra}& \it{gatan\tbr{i}}	&{\tl}& \it{gata\tbr{i}n} &`free'\\
		\end{tabular}}%vsp
\end{exe}

While this data bears some similarities to the Ulwa data
discussed above the existence of alternations
such as \it{aloni} and \it{aloin} `write-\tsc{perf}'
indicates that this is indeed a case of metathesis.
That perfective \it{-i} is a suffix
after stems without final /n/ or /l/ indicates
that the infixal allomorph in examples such as those in \qf{ex:KuiPer}
is a result of CV {\ra} VC metathesis.

\section{Origins of synchronic metathesis}\label{sec:OriSynMet}
In this section I discuss the origins of synchronic
processes of metathesis, focussing on morphological metathesis.
I begin in \srf{sec:OriUml} with a discussion of the development
of umlaut in the Germanic languages.
In \srf{sec:OriMorMet} I then discuss the ways in which morphological metathesis
develops and show that its development closely parallels that of Germanic umlaut in many ways.

\subsection{Origins of umlaut}\label{sec:OriUml}
In this section I provide an overview of the development of Germanic umlaut;
a process with which readers are likely familiar.
The way in which umlaut developed and became a morphological
process has many similarities to the ways in which
synchronic processes of metathesis develop and become morphological.

Umlaut is the term given to a vowel shift which happened in
many of the Germanic languages and resulted in pairs such as
English \it{foot} /fʊt/ {\tl} \it{feet} /fiːt/
and \it{mouse} /maʊs/ {\tl} \it{mice} /maɪs/.
In these English examples the vowel of the plural forms
is descended from an original rounded vowel which was
fronted before a suffix with the front vowel /i/.
This suffix was then lost but the front rounded vowel remained.
The process is illustrated in \qf{ex:mu:s-2} below.
See \citet[58ff]{ha07} for an overview.

\begin{exe}\let\eachwordone=\textnormal
	\ex{\gll *muːs {} {} {} {} {} {} {} {} {} {} > /maʊs/ \it{mouse} \\
					 *muːs-iz > *muːs-i > *myːs-i > *myːs-ə > *myːs > *miːs > /maɪs/ \it{mice} \\}\label{ex:mu:s-2}
\end{exe}

In modern English umlaut is a purely morphological process
with all trace of its original conditioning environment lost.
However, this is not always the case.
One language in which a phonologically
conditioned process of umlaut developed into a morphological process
in some environments but not in others is Old Norse.
This is similar to metathesis in languages such as Rotuman, Helong,
and Amarasi in which metathesis is a phonologically conditioned process
in some environments and a morphological process in others.

In Old Norse there is a process of vowel shift known as \it{u}-mutation or \it{u}-umlaut.
Under this process stressed /a/ {\ra} [ɔ] (transcribed \it{<ǫ>}) before /u/ and
unstressed /a/ {\ra} /u/ before /u/.
This process is formalised in \qf{ex:OldNorUMut} below.
(Primary stress fails on the initial syllable in Old Norse.)

\begin{exe}
	\ex{Old Norse \it{u}-umlaut:}\label{ex:OldNorUMut}
	\sn{\begin{tabular}{llll}
		a	&\ra&ǫ&/ˈ{\_}(C)u \\
			&\ra&u&/\hp{ˈ}{\_}(C)u \\
			&\ra&a&elsewhere \\
	\end{tabular}}
\end{exe}

When a suffix containing /u/ is attached to a stem with /a/, \it{u}-umlaut occurs.
Examples include \it{stað-} `place' + \it{-um} \tsc{dat.pl} {\ra} \it{st\tbr{ǫ}ð\tbr{u}m} and
\it{harm-} `sorrow, grief' + \it{-um} \tsc{dat.pl} {\ra} \it{h\tbr{ǫ}rm\tbr{u}m} \citep[283,286]{go57}.
The declension of two weak feminine nouns is given in \qf{ex:OldNorWeaFemDec}
below to further illustrate the productivity of the process.

\begin{exe}
	\ex{Old Norse weak feminine declension \hfill\citep[289]{go57}}\label{ex:OldNorWeaFemDec}
		\sn{\gw\begin{tabular}{rllll}
										&\tsc{sg}							&\tsc{pl}								&\tsc{sg}									&\tsc{pl}								\\
					\tsc{nom}	&\it{saga}						&\it{s\tbr{ǫ}g\tbr{u}r}	&\it{stjarna}							&\it{stj\tbr{ǫ}rn\tbr{u}r}	\\
					\tsc{acc}	&\it{s\tbr{ǫ}g\tbr{u}}&\it{s\tbr{ǫ}g\tbr{u}r}	&\it{stj\tbr{ǫ}rn\tbr{u}}	&\it{stj\tbr{ǫ}rn\tbr{u}r}	\\
					\tsc{gen}	&\it{s\tbr{ǫ}g\tbr{u}}&\it{sagna}							&\it{stj\tbr{ǫ}rn\tbr{u}}	&\it{stjarna}		\\
					\tsc{dat}	&\it{s\tbr{ǫ}g\tbr{u}}&\it{s\tbr{ǫ}g\tbr{u}m}	&\it{stj\tbr{ǫ}rn\tbr{u}}	&\it{stj\tbr{ǫ}rn\tbr{u}m}	\\
										&\mc{2}{l}{`story'}														&\mc{2}{l}{`star'}				\\
		\end{tabular}}
\end{exe}

This phonological process also affects verbs.
The conjugation of the verb \it{kalla} `to call'
is given in \qf{ex:OldNorDecKal} to illustrate.
This paradigm also shows examples of unstressed /a/ {\ra} /u/.

\begin{exe}
	\ex{Old Norse conjugation of \it{kalla} `to call' \hfill\citep[305]{go57}}\label{ex:OldNorDecKal}
		\sn{\gw\begin{tabular}{lllll}
								&\mc{2}{c}{\tsc{present}}		&\mc{2}{c}{\tsc{past}}  \\
								&\tsc{active}		&\tsc{middle} 		&\tsc{active}			&\tsc{middle} \\
			\tsc{1sg}	&\it{kalla}		&\it{k\tbr{ǫ}ll\tbr{u}mk}	&\it{kallaða}		&\it{k\tbr{ǫ}ll\tbr{u}ð\tbr{u}mk}	\\
			\tsc{2sg}	&\it{kallar}	&\it{kallask}	&\it{kallaðir}	&\it{kallaðisk}	\\
			\tsc{3sg}	&\it{kallar}	&\it{kallask}	&\it{kallaði}		&\it{kallaðisk}	\\
			\tsc{1pl}	&\it{k\tbr{ǫ}ll\tbr{u}m}	&\it{k\tbr{ǫ}ll\tbr{u}mk}	&\it{k\tbr{ǫ}ll\tbr{u}ð\tbr{u}m}	&\it{k\tbr{ǫ}ll\tbr{u}ð\tbr{u}mk}	\\
			\tsc{2pl}	&\it{kallið}	&\it{kallisk}	&\it{k\tbr{ǫ}ll\tbr{u}ð\tbr{u}ð}	&\it{k\tbr{ǫ}ll\tbr{u}ð\tbr{u}sk}	\\
			\tsc{3pl}	&\it{kalla}		&\it{kallask}	&\it{k\tbr{ǫ}ll\tbr{u}ð\tbr{u}}		&\it{k\tbr{ǫ}ll\tbr{u}ð\tbr{u}sk}	\\
		\end{tabular}}
\end{exe}

With this data alone we would conclude that Old Norse \it{u}-umlaut
is a purely phonologically conditioned process.
However, there are also instances in which \it{u}-umlaut
occurs where there is no following /u/.
One example is in the nominative and accusative plurals of neuter nouns,
two of which are given in \qf{ex:OldNorStrNeuDec} below.\footnote{
		Historically such forms had a suffix \it{-u}.
		This suffix had been lost by the time of Old Norse.}
In fact, this single paradigm attests both phonologically conditioned
and morphological instances of \it{u}-umlaut.

\begin{exe}
	\ex{Old Norse strong neuter declension \hfill\citep[283]{go57}}\label{ex:OldNorStrNeuDec}
		\sn{\gw\begin{tabular}{rllll}
										&\tsc{sg}		&\tsc{pl}								&\tsc{sg}		&\tsc{pl}					\\
					\tsc{nom}	&\it{barn}	&\it{b\tbr{ǫ}rn}				&\it{land}	&\it{l\tbr{ǫ}nd}	\\
					\tsc{acc}	&\it{barn}	&\it{b\tbr{ǫ}rn}				&\it{land}	&\it{l\tbr{ǫ}nd}	\\
					\tsc{gen}	&\it{barns}	&\it{barna}							&\it{lands}	&\it{landa}				\\
					\tsc{dat}	&\it{barni}	&\it{b\tbr{ǫ}rn\tbr{u}m}&\it{landi}	&\it{l\tbr{ǫ}nd\tbr{u}m}	\\
										&\mc{2}{l}{`child'}									&\mc{2}{l}{`land'}	\\
		\end{tabular}}
\end{exe}

The best analysis of this Old Norse data is probably to posit a morphological process
of \it{u}-umlaut to account for the neuter plural forms
and posit a phonologically conditioned process
of \it{u}-umlaut before suffixes with the vowel /u/.

In modern Icelandic the process of \it{u}-umlaut still occurs,
as illustrated in the paradigm of \it{barn} /partn/ `child'
given in \qf{ex:IceStrNeuDec} below,
and also seen in the verb \it{kalla} /kʰatla/ `call'
with the \tsc{1pl.pres} form \it{k\tbr{ö}ll\tbr{u}m} /kʰ\tbr{œ}tl\tbr{ʏ}m/
and the \tsc{1pl.past} form \it{k\tbr{ö}ll\tbr{u}ð\tbr{u}m}
/kʰ\tbr{œ}tl\tbr{ʏ}ð\tbr{ʏ}m/ \citep[43]{sv89}.

\newpage
\begin{exe}
	\ex{Icelandic declension of \it{barn} /partn/ `child' \hfill\citep[36]{sv89}}\label{ex:IceStrNeuDec}
		\sn{\gw\begin{tabular}{rllll}
										&\tsc{sg}		&							&\tsc{pl}								&\\
					\tsc{nom}	&\it{barn}	&/partn/	&\it{b\tbr{ö}rn}				&/p\tbr{œ}rtn/\\
					\tsc{acc}	&\it{barn}	&/partn/	&\it{b\tbr{ö}rn}				&/p\tbr{œ}rtn/\\
					\tsc{gen}	&\it{barns}	&/partns/	&\it{barna}							&/partna/\\
					\tsc{dat}	&\it{barni}	&/partnɪ/	&\it{b\tbr{ö}rn\tbr{u}m}&/p\tbr{œ}rtn\tbr{ʏ}m/\\
		\end{tabular}}
\end{exe}

%\begin{exe}
%	\ex{Icelandic conjugation of \it{kalla} /kʰatla/ `call' \hfill\citep[43]{sv89}}\label{ex:IceConKal}
%		\sn{\begin{tabular}{lllll}
%								&\mc{2}{l}{\tsc{present}}		&\mc{2}{l}{\tsc{past}}  \\
%			\tsc{1sg}	&\it{kalla}		&/kʰatla/		&\it{kallaða}		&/kʰatlaða/	\\
%			\tsc{2sg}	&\it{kallar}	&/kʰatlar/	&\it{kallaðir}	&/kʰatlaðɪr/	\\
%			\tsc{3sg}	&\it{kallar}	&/kʰatlar/	&\it{kallaði}		&/kʰatlaðɪ/	\\
%			\tsc{1pl}	&\it{k\tbr{ö}ll\tbr{u}m}	
%														&/kʰ\tbr{œtl\tbr{ʏ}m/}	
%																						&\it{k\tbr{ö}ll\tbr{u}ð\tbr{u}m}	&/kʰ\tbr{œtl\tbr{ʏ}ð\tbr{ʏ}m/}	\\
%		\tsc{2pl}	&\it{kallið}	&/kʰatlɪð/	&\it{k\tbr{ö}ll\tbr{u}ð\tbr{u}ð}	&/kʰ\tbr{œtl\tbr{ʏ}ð\tbr{u}ð/}	\\
%			\tsc{3pl}	&\it{kalla}		&/kʰatla/		&\it{k\tbr{ö}ll\tbr{u}ð\tbr{u}}		&/kʰ\tbr{œtl\tbr{ʏ}ð\tbr{ʏ}/}	\\
%		\end{tabular}}
%\end{exe}

In Icelandic the phonological conditioning
environment has become so opaque due to later
processes including epenthesis of /u/ -- e.g.
\it{harmur} /harmʏr/ `grief, sorrow' (from Old Norse \it{harmr}) --
that it is best to analyse \it{u}-umlaut as a
morphological process in environments such as the neuter plural
and as a morphemically conditioned process in other environments.

The development of Old Norse \it{u}-umlaut shows how a
process which originally occurred only in certain phonological environments
can develop into a morphological process.
Old Norse has a single phonological process which is morphological
in some environments and phonologically conditioned in other environments.
Similarly, Icelandic has a single phonological process which is morphological
in some environments and morphemically conditioned in other environments.

In \srf{sec:OriMorMet} below I discuss some phonologically natural processes
by which morphological metathesis can develop.
Such pathways can result in some languages synchronically having a single process of metathesis
which is phonologically conditioned in some environments
and morphemically conditioned or morphological in other environments.
%Such languages include Rotuman (\srf{sec:Rot}) and Helong (\srf{sec:Hel}),
%as well as Amarasi. %which has phonologically conditioned metathesis before
%vowel-initial enclitics (Chapter \ref{ch:PhoMet})
%and morphological metathesis in other environments (Chapters \ref{ch:SynMet} and \ref{ch:DisMet}).

\subsection{Bunak}\label{sec:Bun}
Bunak (Trans-New Guinea, Timor) has both morphemically
conditioned metathesis (\srf{sec:MorpheConMet})
and morphological metathesis (\srf{sec:MorMet}).
In Bunak the initial CV sequence of a CVVC stem metathesises
when a prefix is added and the first vowel of the root is high, /i/ or /u/,
and the second vowel is non-high, /e/, /a/ or /o/.
While stress is normally penultimate in Bunak,
CV\tsc{[+high]}V\tsc{[-high]}C words have final stress
(Antoinette Schapper p.c. September 2015).
Such final stress remains after metathesis.
(I indicate stress in this section with an acute accent.)

Examples of Bunak metathesis are given in \qf{ex:BunMet} with the prefix \it{gV-}
which marks third person animate possessors on nouns
and third person animate objects or undergoers with verbs.
\cite{sc09} notes that the eight stems in \qf{ex:BunMet} are the only ones in her corpus which
are both eligible to take prefixes and of the appropriate phonological structure to undergo metathesis.
Before other consonant initial stems, the unspecified vowel of the prefix \it{gV-} is a copy vowel.

Before vowel initial stems the unspecified vowel of a prefix is deleted:
e.g. \it{gV- + íwal} `pick' {\ra} \it{gíwal}
and \it{gV- + úbe} `block' {\ra} \it{gúbe}.
Such vowel deletion also takes place before the metathesising stems.

\begin{exe}
	\ex{Bunak metathesis \hfill\citep[67]{sc09}}\label{ex:BunMet}
	\sn{\gw\begin{tabular}{rcllll}
		\it{gV-} &+&\it{tékeʔ} 			&\ra& \it{ge-tékeʔ}			& `watch' \\
		\it{gV-} &+&\it{íwal} 			&\ra& \it{g-íwal}				& `pick' \\
		\it{gV-} &+&\it{\tbr{lu}él}	&\ra& \it{g-\tbr{ul}él}	& `skin, peel' \\
		\it{gV-} &+&\it{\tbr{mi}én}	&\ra& \it{g-\tbr{im}én}	& `immediately' \\
		\it{gV-} &+&\it{\tbr{ni}át}	&\ra& \it{g-\tbr{in}át}	& `first (one)' \\
		\it{gV-} &+&\it{\tbr{nu}ás}	&\ra& \it{g-\tbr{un}ás}	& `stink' \\
		\it{gV-} &+&\it{\tbr{nu}ék}	&\ra& \it{g-\tbr{un}ék}	& `be smelly' \\
		\it{gV-} &+&\it{\tbr{si}éʔ}	&\ra& \it{g-\tbr{is}éʔ}	& `rip' \\
		\it{gV-} &+&\it{\tbr{tu}ék}	&\ra& \it{g-\tbr{ut}ék}	& `be heavy' \\
		\it{gV-} &+&\it{\tbr{zi}ék}	&\ra& \it{g-\tbr{iz}ék}	& `fry' \\
	\end{tabular}}
\end{exe}

It does not seem possible to motivate
the metathesis in Bunak on the basis of
the new phonological context created by the addition of the prefix.
Thus, I identify this as a case of morphemically metathesis (\srf{sec:MorpheConMet}).

An alternate analysis of the Bunak data would be to posit
that the shape VCVC for these stems is underlying,
with metathesis of initial VC {\ra} CV
when such stems are used in isolation.
\cite{sc09} does discuss this possibility.

The \tsc{1incl/2} prefix consists only of an unspecified vowel \it{V-}.
Given the rule whereby the final vowel of a prefix is deleted
before vowel initial (and metathesising stems),
this means that metathesis is the only phonological signal of \tsc{1incl/2}
agreement for metathesising stems.
Thus, metathesis in Bunak can be identified as a morphological device
to mark \tsc{1incl/2} agreement.
The paradigms of two consonant initial stems, two vowel initial stems
and two metathesising stems are given in \trf{tab:BunPre} below
to show the different allomorphs of the agreement prefixes.
I follow \citet{sc09} in representing the deleted \tsc{1incl/2}
affix as a zero prefix in \trf{tab:BunPre}

\begin{table}[h]
	\caption[Bunak prefixation]{Bunak prefixation \citep[66,340]{sc09}}\label{tab:BunPre}
		\begin{tabular}{rrr|rr|rr}
		\lsptoprule
							&\mc{2}{c|}{C-initial}		&\mc{2}{c|}{V-initial}			& \mc{2}{c}{metathesising} 							\\
							&`watch'			&`fetch'		&`pick'				&`hang'				& `peel'						& `rip' 						\\ \midrule
Stem					&\it{tékeʔ}		&\it{wit}		&\it{íwal}		&\it{óbon}		&	\it{\tbr{lu}él} 	& \it{\tbr{si}éʔ}		\\
\tsc{1excl}		&\it{ne-tékeʔ}&\it{ni-wít}&\it{n-íwal}	&\it{n-óbon}	& \it{n-\tbr{ul}él}	& \it{n-\tbr{is}éʔ}	\\
\tsc{1incl/2}	&\it{e-tékeʔ}	&\it{i-wít}	&\it{\0-íwal}	&\it{\0-óbon}	& \it{\0-\tbr{ul}él}& \it{\0-\tbr{is}éʔ}\\
\tsc{3anim}		&\it{ge-tékeʔ}&\it{gi-wít}&\it{g-íwal}	&\it{g-óbon}	& \it{g-\tbr{ul}él}	& \it{g-\tbr{is}éʔ}	\\
		\lspbottomrule
	\end{tabular}
\end{table}

With the loss of the vowel of the \tsc{1incl/2} prefix,
the morphemically conditioned metathesis in Bunak has developed a morphological function.
In this respect its development is similar to the development of Germanic umlaut (\srf{sec:OriUml})
in which an original conditioning environment was lost.
The Bunak data shows one pathway in which morphological metathesis can develop.
Other pathways are discussed in \srf{sec:OriMorMet} below.
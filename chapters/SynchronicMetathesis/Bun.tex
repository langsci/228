\subsection{Bunak}\label{sec:Bun}
Bunak (Trans-New Guinea, Timor) has morphemically
conditioned metathesis (\srf{sec:MorpheConMet})
and morphological metathesis (\srf{sec:MorMet}).
In Bunak the initial CV sequence of a CVVC stem metathesises
when a prefix is added and the first vowel of the root is high, /i/ or /u/,
and the second vowel is non-high, /e/, /a/ or /o/.
While stress is normally penultimate in Bunak,
CV\tsc{[+high]}V\tsc{[-high]}C words have final stress
(Antoinette Schapper p.c. September 2015).
Such final stress remains after metathesis.

Examples of Bunak metathesis are given in \qf{ex:BunMet} with the prefix \it{gV-}
which marks third person animate possessors on nouns
and third person animate objects or undergoers with verbs.
\cite{sc09} notes that the eight stems in \qf{ex:BunMet}
are the only ones in her corpus which
are both eligible to take prefixes and of the appropriate
phonological structure to undergo metathesis.
Before other consonant-initial stems,
the unspecified vowel of the prefix \it{gV-} is a copy vowel.

Before vowel-initial stems the unspecified vowel of a prefix is deleted:
e.g. \it{gV- + ˈiwal} `pick' {\ra} \it{ˈgiwal}
and \it{gV- + ˈube} `block' {\ra} \it{ˈgube}.
Such vowel deletion also takes place before the metathesising stems.

\begin{exe}
	\ex{Bunak metathesis \hfill\citep[67]{sc09}}\label{ex:BunMet}
	\sn{\gw\begin{tabular}{rcllll}
		\it{gV-} &+&\it{ˈtekeʔ} 			&\ra& \it{ge-ˈtekeʔ}			& `watch' \\
		\it{gV-} &+&\it{ˈiwal} 			&\ra& \it{ˈg-iwal}				& `pick' \\
		\it{gV-} &+&\it{\tbr{lu}ˈel}	&\ra& \it{g-\tbr{ul}ˈel}	& `skin, peel' \\
		\it{gV-} &+&\it{\tbr{mi}ˈen}	&\ra& \it{g-\tbr{im}ˈen}	& `immediately' \\
		\it{gV-} &+&\it{\tbr{ni}ˈat}	&\ra& \it{g-\tbr{in}ˈat}	& `first (one)' \\
		\it{gV-} &+&\it{\tbr{nu}ˈas}	&\ra& \it{g-\tbr{un}ˈas}	& `stink' \\
		\it{gV-} &+&\it{\tbr{nu}ˈek}	&\ra& \it{g-\tbr{un}ˈek}	& `be smelly' \\
		\it{gV-} &+&\it{\tbr{si}ˈeʔ}	&\ra& \it{g-\tbr{is}ˈeʔ}	& `rip' \\
		\it{gV-} &+&\it{\tbr{tu}ˈek}	&\ra& \it{g-\tbr{ut}ˈek}	& `be heavy' \\
		\it{gV-} &+&\it{\tbr{zi}ˈek}	&\ra& \it{g-\tbr{iz}ˈek}	& `fry' \\
	\end{tabular}}
\end{exe}

It does not seem possible to motivate
the metathesis in Bunak on the basis of
the new phonological context created by the addition of the prefix.
Thus, I identify this as a case of morphemically metathesis (\srf{sec:MorpheConMet}).

An alternate analysis of the Bunak data would be to posit
that the shape VCVC for these stems is underlying,
with metathesis of initial VC {\ra} CV
when such stems are used in isolation.
\cite{sc09} does discuss this possibility.

The \tsc{1incl/2} prefix consists only of an unspecified vowel \it{V-}.
Given the rule whereby the final vowel of a prefix is deleted
before vowel-initial (and metathesising stems),
this means that metathesis is the only phonological signal of \tsc{1incl/2}
agreement for metathesising stems.
Thus, metathesis in Bunak can be identified as a morphological device
to mark \tsc{1incl/2} agreement.
The paradigms of two consonant-initial stems, two vowel-initial stems
and two metathesising stems are given in \trf{tab:BunPre} below
to show the different allomorphs of the agreement prefixes.
I follow \citet{sc09} in representing the deleted \tsc{1incl/2}
affix as a zero prefix in \trf{tab:BunPre}

\begin{table}[ht]
	\caption[Bunak prefixation]{Bunak prefixation \citep[66,340]{sc09}}\label{tab:BunPre}
		\begin{tabular}{rrr|rr|rr}
		\lsptoprule
							&\mc{2}{c|}{C-initial}				&\mc{2}{c|}{V-initial}			& \mc{2}{c}{metathesising} 									\\
							&`watch'				&`fetch'			&`pick'				&`hang'				& `peel'							& `rip' 							\\ \midrule
Stem					&\it{ˈtekeʔ}		&\it{wit}			&\it{ˈiwal}		&\it{ˈobon}		&	\it{\tbr{lu}ˈel} 		& \it{\tbr{si}ˈeʔ}		\\
\tsc{1excl}		&\it{ne-ˈtekeʔ}	&\it{ni-ˈwit}	&\it{ˈn-iwal}	&\it{ˈn-obon}	& \it{n-\tbr{ul}ˈel}	& \it{n-\tbr{is}ˈeʔ}	\\
\tsc{1incl/2}	&\it{e-ˈtekeʔ}	&\it{i-ˈwit}	&\it{ˈ\0-iwal}&\it{ˈ\0-obon}& \it{\0-\tbr{ul}ˈel}	& \it{\0-\tbr{is}ˈeʔ}	\\
\tsc{3anim}		&\it{ge-ˈtekeʔ}	&\it{gi-ˈwit}	&\it{ˈg-iwal}	&\it{ˈg-obon}	& \it{g-\tbr{ul}ˈel}	& \it{g-\tbr{is}ˈeʔ}	\\
		\lspbottomrule
	\end{tabular}
\end{table}

With the loss of the vowel of the \tsc{1incl/2} prefix,
the morphemically conditioned metathesis in Bunak has developed a morphological function.
In this respect its development is similar to that of Germanic umlaut (\srf{sec:OriUml})
in which an original conditioning environment was lost.
The Bunak data shows one pathway in which morphological metathesis can develop.
Other pathways are discussed in \srf{sec:OriMorMet} below.
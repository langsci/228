\subsection{Historic origins of morphological metathesis}\label{sec:OriMorMet}
The most comprehensive account of the historic origins of metathesis
is that of \cite{blga98} with an updated, but shorter, account given in \cite{blga04}.
According to this account there are three sources of metathesis:
perceptual metathesis, compensatory metathesis, and metathesis which arises
out of epenthesis and apocope (``pseudometathesis'').
The examples of morphological metathesis discussed
in this chapter are instances of epenthesis and apocope (\srf{sec:EpeApo})
and/or compensatory metathesis (\srf{sec:ComMet}).

\subsubsection{Epenthesis and apocope}\label{sec:EpeApo}
One pathway by morphological metathesis can develop
is through epenthesis and apocope.
Languages which appear to have acquired metathesis in this way include Leti (\srf{sec:Let})
and probably the north American Salishan languages
(discussed in more detail in Appendix \ref{app:MorMet}).

Under this process, epenthesis of a vowel occurs in one part of a word
and deletion of an original non-epenthetic vowel in another part of the word.
One version of this process, which took place for Leti internal metathesis,
is shown in \qf{ex:EpeApo} below.
At stage 2 an epenthetic vowel is added word finally.
The previous vowel is then deleted at stage 3 and at stage 4
the final epenthetic vowel is reinterpreted as non-epenthetic.

\begin{exe}\let\eachwordone=\textnormal
	%\ex{\gll	VCVC > VCVC\u{V} > VCC\u{V} > VCCV \\
	\ex{\gll	V\sub{1}C\sub{2}V\sub{3}C\sub{4} > V\sub{1}C\sub{2}V\sub{3}C\sub{4}\u{V}\sub{3} > V\sub{1}C\sub{2}C\sub{4}\u{V}\sub{3} > V\sub{1}C\sub{2}C\sub{4}V\sub{3} \\
						{stage 1} {} {stage 2} {} {stage 3} {} {stage 4}\\}\label{ex:EpeApo}
\end{exe}

Each stage of this process is illustrated
for Leti in \trf{tab:DevLetMet} below.
%according to the account given by \citet[542--547]{blga98}.
%according to the account given by \citet[542ff]{blga98}.
At stage 1 a word-final schwa is inserted,
at stage 2 this schwa then either assimilates
to the quality of the previous vowel or is lowered to /a/,
finally at stage 3 the unstressed penultimate vowel is deleted,
giving rise to the metathesised forms.
Other developments such as consonant assimilation and glottal stop
deletion with compensatory lengthening of the previous vowel
then occurred at stage 3′.
Proto-Malayo-Polynesian (PMP) reconstructions in \trf{tab:DevLetMet} are from \cite{bltr}.
Stress is marked by an acute accent.

\begin{table}[h]
	\caption[Development of Leti metathesis]
					{Development of Leti metathesis \citep{blga98}}\label{tab:DevLetMet}
	\centering\stl{0.3em}
		\begin{tabular}{rclclclclcll} \lsptoprule
			PMP			& &pre-Leti & &stage 1		& &stage 2		& &stage 3	& &\mc{2}{l}{stage 3′}\\ \midrule
			*haŋin 	&>&*ánin		&>&*áninə			&>&*ánini	 		&>&\it{anni}& &							&`wind'\\
			*kulit 	&>&*úlit		&>&*úlitə			&>&*úliti	 		&>&\it{ulti}& &							&`skin'\\
			*kambu	&>&*ápun		&>&*ápunə			&>&*ápunu	 		&>&\it{apnu}& &							&`belly'\\
			*likud 	&>&*líʔur		&>&*líʔurə		&>&*líʔuru	 	&>&*líʔru		&>&\it{liiru}		&`last'\\
			*maqitəm&>&*m\'ɛtam&>&*m\'ɛtamə	&>&*m\'ɛtama	&>&\it{mɛtma}&&							&`red'\\
			*bulan 	&>&*\B úlan	&>&*\B úlanə	&>&*\B úlana	&>&*\B úlna &>&\it{\B ulla}	&`moon'\\
			*ŋajan 	&>&*náʔan		&>&*náʔanə		&>&*náʔana		&>&*náʔna 	&>&\it{naana}		&`name'\\
			*hikan 	&>&*íʔan		&>&*íʔanə			&>&*íʔana			&>&*íʔna 		&>&\it{iina}		&`fish'\\ \lspbottomrule
		\end{tabular}
\end{table}

According to this account, epenthesis of final schwa
only occurred in certain phonological environments, such as phrase finally,
while no epenthesis occurred in other positions.
Unmetathesised Leti forms are developments
of the pre-Leti forms in \trf{tab:DevLetMet}
without the subsequent processes of epenthesis, assimilation, and deletion
which yielded the metathesised forms.

This analysis can account for instances of Leti internal metathesis.
However, recall from \srf{sec:LetExtMet} that Leti also has external metathesis,
seen, for instance, in \it{as\tbr{u}} `dog' + \it{\tbr{l}ala\B na} `big' {\ra} \it{as\tbr{lu}ala\B ne}.
Such metathesis can be accounted for by compensatory metathesis,
discussed in \srf{sec:ComMet} below.
Thus, \it{as\tbr{lu}ala{\B}na} `dog + big' is hypothesised to have gone through the pathway
*as\tbr{ul}ala{\B}na > *as\tbr{ulu}ala{\B}na > *as\tbr{\u{u}lu}ala{\B}na > \it{as\tbr{lu}ala{\B}na}.

Another probable case of metathesis developing by epenthesis and apocope
occurred in the Salishan languages (\srf{sec:Sal}),
though in this case apocope was apparently motivated by stress shift.
The various processes have been discussed by \cite{de74} who cites data from Lummi,
a straits Salish variety closely related to both Saanich (\srf{sec:Saa}) and Klallam (\srf{sec:Kla}).
Examples of Lummi metathesis are given in \qf{ex:LumMet} below.

\begin{exe}
	\ex{Lummi metathesis \hfill\citep[15]{de74}}\label{ex:LumMet}
		\sn{\stl{0.4em}\gw\begin{tabular}{lrcll}
											\mc{2}{r}{\tsc{perfective}}	&		&\mc{2}{l}{\tsc{imperfective}}\\
			`Someone hit him' &\it{ʦ'\tbr{s\'ə}-tŋs}		&\ra&\it{ʦ'\tbr{\'əs}-tŋs}	&`He's getting hit'	\\
			`I smashed it' 		&\it{t'\tbr{s\'ə}-tsən}		&\ra&\it{t'\tbr{\'əs}-t}		&`He's breaking it'	\\
			`They gather it' 	&\it{q'\tbr{p\'ə}-ts}			&\ra&\it{q'\tbr{\'əp}-ŋ}		&`gathering'					\\
			`I'm stuck' 			&\it{t͜ɬ'\tbr{qʷ\'ə}-tsən}	&\ra&\it{t͜ɬ'\tbr{\'əqʷ}-sən}&`I'm getting stuck'	\\
		\end{tabular}}
\end{exe}

\cite{de74} proposes that the imperfective is formed in
all instances by infixation of a glottal stop,
which is associated with a number of other rules.
These processes are summarised in \trf{tab:ForLumImp} below,
for the metathesis of \it{ʦ'\tbr{s\'ə}-} {\ra} \it{ʦ'\tbr{\'əs}-} `hit'.

\begin{table}[h]
	\caption{Formation of Lummi (im)perfectives}\label{tab:ForLumImp}
	\centering
		\begin{tabular}{llll}\lsptoprule
				&	process								& \tsc{perfective}		& \tsc{imperfective} \\ \midrule
			1.&	base									& \it{ʦ'\'əsə-t-ŋ-s}	& \it{ʦ'\'əsə-t-ŋ-s}	\\
			2.&	infixation						&											& \it{ʦ'\'əʔsə-t-ŋ-s} \\
			3.&	stress protraction		& \it{ʦ'əs\'ə-t-ŋ-s}	& 										\\
			4.&	schwa deletion				& \it{ʦ's\'ə-t-ŋ-s}		& \it{ʦ'\'əʔs-t-ŋ-s} 	\\
			5.&	glottal stop deletion	&											& \it{ʦ'\'əs-t-ŋ-s} 	\\\lspbottomrule
		\end{tabular}
\end{table}

The first row gives the proposed underlying base forms.
Each form has two vowels, with stress on the first vowel.
The second row shows infixation of the glottal stop in the imperfective.
In the third row so-called ``stress protraction'' occurs in the perfective,
whereby stress moves over an obstruent to the adjacent closed syllable.
Stress protraction does not occur in the imperfective as the glottal stop is treated as a sonorant,
and syllables closed by a sonorant maintain stress.
In row four unstressed schwas are deleted and in row five
any glottal stop before an obstruent is deleted,
thus deleting the original marker of the imperfective.

In summary, \cite{de74} analyses (surface) metathesis in Lummi
as resulting from glottal stop infixation followed by stress shift
followed by unstressed schwa deletion followed by glottal stop deletion.

While such a combination of processes may be the
historic source of metathesis in the Salishan languages,
it does not seem possible to apply this analysis to the synchronic data in every language.
In particular \citet[540]{blga98} note that the Klallam data,
in which roots containing vowels other than schwa also undergo metathesis,
resists such a synchronic analysis (\srf{sec:Kla}).\footnote{
		\citet[540]{blga98} do, however, compare Klallam
		\it{χ\tbr{ʧ}\,\tbr{'í}-t} {\ra}\it{χ\tbr{íʧ}\,\tbr{'}-t} `scratch'
		to Lushootseed \it{χʷiʧ\,'i-d} `mark it, plough land' and  \it{χʷiʧ\,'-dup} `I'm ploughing now',
		citing data from \cite{bahehi94}.}
		
Vowel deletion in different environments also appears
to be a likely source of metathesis in Tunisian Arabic, Ohlone, and Sierra Miwok.
The synchronic data for these languages is discussed in Appendix \ref{app:MorMet}.
%Thus, for instance, \citep[66]{kidr86} compare Classical Arabic \it{malak-a}
%to Tunisian Arabic \it{m{\0}lək{\0}} `he owned'.

\subsubsection{Compensatory metathesis}\label{sec:ComMet}
Compensatory metathesis is a process %\citet[527--539]{blga98}
of metathesis which arises through anticipatory co-articulation
of an unstressed vowel with the stressed vowel, followed by reduction
and eventual loss of the unstressed vowel.
In \srf{sec:OriMetAma} I present evidence showing that Amarasi
metathesis probably developed via this route.
The progression of this process is shown in \qf{ex:ComMet} below,
illustrated with Rotuman \it{pure} {\ra} \it{puer} `rule, decide'.\footnote{
		While I only discuss examples of this process operating from the right edge of a word,
		it can also operate from the left edge of a word.
		\citet[537]{blga98} discuss the case of Ngkot̪ (Pama-Nyungan, Australia)
		in which left edge metathesis has occurred historically.}

\begin{exe}\let\eachwordone=\textnormal
	\ex{\glll {púre} {} {púere} {} {púer\u{e}} {} {púer}\\
						\'V\sub{1}CV\sub{2} > \'V\sub{1}V\sub{2}CV\sub{2} > \'V\sub{1}V\sub{2}C\u{V}\sub{2} > \'V\sub{1}V\sub{2}C\\
						{stage 1} {} {stage 2} {} {stage 3} {} {stage 4}\\}\label{ex:ComMet}
\end{exe}

There is direct evidence that this process has occurred in Kwara'ae (\srf{sec:Kwa})
as intermediate stage 3 forms are still attested in certain environments (\srf{sec:KwaVoiVow}).
While there is no direct evidence that this is the process which occurred in Rotuman,
\cite{blga98} argue that the distribution of metathesised forms in Rotuman
is consistent with their account.

This distribution is the observation provided by \cite{haki98} (\srf{sec:RotFun})
that M\=/forms mostly occur before polysyllabic suffixes
while U\=/forms occur before monosyllabic suffixes.
This is combined with the fact that stress regularly falls
on the penultimate syllable of a word in Rotuman and that
some affixes count as part of the word for stress placement,
while other affixes do not \citep[75]{ch40}.

Due to penultimate stress, stems with a monosyllabic suffix 
were stressed on the stem-final vowel,
and such vowels were ``protected'' from the co-articulation
and weakening which affected final unstressed vowels elsewhere.
This resulted in the U\=/form surviving before monosyllabic suffixes,
with M\=/forms occurring elsewhere.
The different development of isolated stems,
stems with a monosyllabic suffix, and stems with a
polysyllabic suffix in Rotuman are given in \qf{ex:DevRotShoFor} below.

\begin{exe}
	\ex{Development of metathesis in Rotuman \hfill\citep[532]{blga98}}\label{ex:DevRotShoFor}
	\sn{\stl{0.45em}\gw\begin{tabular}{llclclcl}
					&stage 1				& &stage 2						& &stage 3				& &stage 4\\
		{\0}	&\'V\sub{1}CV\sub{2}			&>&\'V\sub{1}V\sub{2}CV\sub{2}			&>&\'V\sub{1}V\sub{2}C\u{V}\sub{2}	&>&\'V\sub{1}V\sub{2}C \\
		{-σ}	&V\sub{1}C\'V\sub{2}-σ		&>&V\sub{1}C\'V\sub{2}-σ						&>&V\sub{1}C\'V\sub{2}-σ						&>&V\sub{1}C\'V\sub{2}-σ \\
		{-σσ}	&\`V\sub{1}CV\sub{2}-\'σσ	&>&\`V\sub{1}V\sub{2}CV\sub{2}-\'σσ	&>&\`V\sub{1}V\sub{2}C\u{V}\sub{2}-\'σσ	&>&\`V\sub{1}V\sub{2}C-\'σσ
					\\
	\end{tabular}}
\end{exe}

In modern-day Rotuman metathesis not only occurs before certain
suffixes, but is also a morphological process marking 
a definite/indefinite contrast.
The final step for this development was for the suffix marking definiteness to be lost.
This suffix was probably originally a monosyllabic copy vowel \citep{gr59,blga98}.
The presumed development for the word \it{pure {\ra} puer} `rule, decide'
is shown in \qf{ex:pure>puer} below.

\begin{exe}\let\eachwordone=\textnormal
	\ex{\gll	-{\0}	{}	*púre		> *púere	> *púer\u{e}	> *púer > \it{púer}\\
							-V	{}	*puré-e	> *puré-e	> *puré-\u{e} >	*puré > \it{púre}\\
			}\label{ex:pure>puer}
\end{exe}

Compensatory metathesis with subsequent loss of the conditioning environment is
one way in which a language can develop a morphological process of metathesis.
The distribution of M\=/forms and U\=/forms in Rotuman, Helong, and Mambae
appears to be consistent with such a process.
In Amarasi there is comparative evidence attesting the posited
intermediate forms (\srf{sec:OriMorMet}).

Metathesis can thus arise in a language in a specific phonological environment
through a series of phonetically natural changes,
in a similar manner to the development of umlaut in the Germanic languages.
As with Germanic umlaut, when the original conditioning environment is lost,
metathesis can become the only expression of a morphological process.

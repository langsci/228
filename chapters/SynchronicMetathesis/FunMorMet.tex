\section{Functions of morphological metathesis}\label{sec:FunMorMet}
In this section I summarise the functions of metathesis
in the languages discussed in this chapter.
When it has developed into a morphological process,
metathesis has uses which are completely unsurprising for
a morphological process and which are found with other
more familiar types of morphological processes such as affixation.

The different functions associated with morphological metathesis
in languages discussed in this chapter are summarised in \trf{tab:FunMorMet}.
Kwara'ae (\srf{sec:Kwa}), Wersing (\srf{sec:Wer}), and Luang (\srf{sec:Lua})
are not listed in \trf{tab:FunMorMet}
as metathesis in these languages is analysed as phonologically conditioned.

\begin{table}[h]
	\caption{Functions of morphological metathesis}\label{tab:FunMorMet}
		\centering
			\begin{tabular}{llll}\lsptoprule
				Language	&Verbs					&Nouns 			& 					\\ \midrule
				Bunak			&agreement			&n./a.			&					 	\\
				Rotuman		&imperfective		&indefinite	&modified 	\\
				Leti			&imperfective		&indefinite	&modified 	\\
				Roma			&								&subject		&unmodified \\
				Mambae		&nominalisation	&						&modified, possessed\\
				Helong		&imperfective		&definite		&modified 	\\
				Amarasi		&unresolved			&						&modified 	\\\lspbottomrule
			\end{tabular}
\end{table}

In addition to the morphological functions of metathesis
listed in \trf{tab:FunMorMet}, metathesis in a number
of these languages is also phonologically conditioned in some environments.
This is the case for Rotuman, Helong, and Amarasi.
In this respect, these languages are similar to Old Norse \it{u}-umlaut (\srf{sec:OriUml})
in which a single phonological process is phonologically conditioned in
some environments and morphological in other environments.

\trf{tab:FunMorMet} shows that Rotuman
and the languages of the greater Timor region
use morphological metathesis to mark
fairly typical morphological categories.
Two common functions are to mark aspect on verbs
and definiteness in the noun phrase.
Additionally, Rotuman and every language of the greater Timor region
with morphological metathesis uses
it to express the presence or absence of an attributive modifier in the noun phrase.
This is the only function of morphological metathesis in the noun phrase in Amarasi.
The morphological form used to mark the presence of an attributive modifier
is known as the construct form and I include a brief overview
of this function in \srf{sec:ConFor} below.

\subsection{Metathesis as a construct form}\label{sec:ConFor}
Morphological metathesis is frequently used as a construct form.
The construct form (also ``construct state'' or ``annexed state/form'')
is a morphological form best known in the Semitic languages.
It is a form used to mark the head-dependent relationship
between two members of a syntactic phrase,
usually by a special morphological form taken by the head of that phrase.

One language with a construct form is Syrian Arabic,
in which two nouns can stand in juxtaposition
with the head noun in the construct form.
Most such Syrian Arabic noun phrases can be compared
to English compound nouns or genitive constructions.
In Syrian Arabic the construct form is marked by the suffix \it{-(e)t}.
Examples of the Syrian Arabic construct form are given in \qf{ex:SyrAraConFor}.

\begin{exe}
	\ex{Syrian Arabic construct form \hfill\citep[163]{co64}}\label{ex:SyrAraConFor}
		\sn{\stl{0.24em}\gw\begin{tabular}{llll}
			\mc{2}{l}{Absolute}					&Construct &\\
			\it{ħafle}		&`show'				&\it{ħafle-\tbr{t} muːsiːqa}			&`concert (\emph{lit.} music show)' \\
			\it{χzaːne}		&`closet'			&\it{χzaːne-\tbr{t} ʔuːdˠtˠi}			&`the closet of my room' \\
			\it{masʔale}	&`matter'			&\it{masʔale-\tbr{t} ʒadd}				&`a matter of concern' \\
			\it{ħaːle}		&`condition'	&\it{ħaːle-\tbr{t} əʃ-ʃərke} 			&`condition of the company' \\
			\it{zjaːra}		&`visit'			&\it{zjaːr-\tbr{et} ʔaχi} 				&`my brother's visit' \\
			\it{ʔəsˠːa}		&`story'			&\it{ʔəsˠː-\tbr{etˠ} haz-zalame}	&`that fellow's story' \\
			\it{ʔuːdˠa}		&`room'				&\it{ʔuːdˠ-\tbr{etˠ} əl-ʔaʕde} 		&`sitting room' \\ %ₔ
			\it{wazˠiːfe}	&`assignment'	&\it{wazˠiːf-\tbr{t} əl-fiːzja} 	&`physics assignment' \\
			\it{doːχa}		&`nausea'			&\it{doːχ-\tbr{t} ətˠ-tˠajːaːra} 	&`airsickness' \\
			%\it{}&`'&\it{-\tbr{t} } &`' \\
		\end{tabular}}
\end{exe}

In Iraqw (Cushitic, Tanzania) the construct form occurs with a wider
variety of nominal modifiers including nouns, adjectives, numerals and relative clauses.
The construct form in Iraqw is signalled by a suffix which agrees with
the gender of the noun to which it attaches.
Examples of the construct form in Iraqw are given in \qf{ex:IraConFor} below.
All construct suffixes have a high tone in Iraqw.

\begin{exe}
	\ex{Iraqw construct form \hfill\citep[94]{mo93}}\label{ex:IraConFor}
		\sn{\stl{0.3em}\gw\begin{tabular}{lllll}
			Stem				&					&Gender			&Construct &\\
			\it{ʦ'axwel}&`trap'		&\tsc{masc}	&\it{ʦ'axwel-\tbr{ú} daŋʷ}		&`elephant trap' \\
			\it{kuru}		&`year'		&\tsc{masc1}&\it{kur-\tbr{kú} ʕisáʔ}			&`last year' \\
			\it{waahla}	&`python'	&\tsc{fem}	&\it{waahl\tbr{á}-\tbr{r} ur}				&`a big python' \\
			\it{ga}			&`thing'	&\tsc{fem}	&\it{g\tbr{á}-\tbr{r} ni hláaʔ}		&`the thing that I want' \\
			\it{diʕi}		&`fat'		&\tsc{fem1}	&\it{diʕi-\tbr{tá} ʕáwak}				&`cream (\emph{lit.} white fat)' \\
			\it{ħar}		&`stick'	&\tsc{fem1}	&\it{ħar-\tbr{tá} baabúʕéeʔ}		&`the stick of my father' \\
			\it{giʔi}		&`ghost'	&\tsc{neut}	&\it{giʔ-\tbr{á} heedáʔ}				&`the ghost of that man' \\
		%	\it{}		&\tsc{}&`'				&\it{-\tbr{t}&`' \\
		\end{tabular}}
\end{exe}

In Tolaki (Austronesian, Sulawesi) it is unmodified nouns
which are morphologically marked, while the construct form is unmarked.
This is similar to the function of metathesis in Roma (\srf{sec:Rom}).
In Tolaki all two syllable nouns obligatorily occur with the prefix \it{o-},
except when another adjective or noun occurs within the noun phrase.

Compare example \qf{ex:TolPred} and \qf{ex:TolAtt} below.
Each of these examples consists of a demonstrative, noun, and adjective.
In \qf{ex:TolPred} the prefix \it{o-} occurs,
and the following adjective is interpreted as predicative,
while in \qf{ex:TolAtt} this prefix does not occur
and the following adjective is interpreted as attributive.

\begin{exe}\let\eachwordone=\textit
	\ex{\gll ŋgituʔo \brac{NP} {\tbr{o}-tina \bracr{}} momahe \\
					\tsc{dem} {} woman beautiful \\
					\glt `That woman is beautiful.'\label{ex:TolPred}}
	\ex{\gll ŋgituʔo \brac{NP} tina momahe \bracr{}\\
					\tsc{dem} {} woman beautiful \\
					\glt `that beautiful woman'\label{ex:TolAtt}}
\end{exe}

A number of other Tolaki nominal phrases are given in \trf{tab:TolAttAdj}.
For all such phrases the citation (unmodified) form of each element is also given.
When this word is a disyllabic noun it occurs with the prefix \it{o-}.\footnote{
		The Tolaki prefix \it{o-} has a restricted phonological distribution,
		only occurring on two syllable nouns.
		See \cite{be12} for a discussion and analysis of this phonological restriction
		based on an earlier interpretation of the Tolaki data.}

\begin{table}[h]
	\caption[Tolaki nominal phrases]{Tolaki nominal phrases (own fieldnotes)}\label{tab:TolAttAdj}
	\centering
		\stl{0.4em}\begin{tabular}{rlcllcll}\lsptoprule
							&Noun						& &Mod.					&					&		&Phrase							&\\ \midrule
			`dog'		&\it{o-ɗahu}		&+&\it{o{\B}ose}&`big'		&\ra&\it{ɗahu o{\B}ose}	&`(a) big dog'\\
			`table'	&\it{o-meɗa}		&+&\it{momea}		&`red'		&\ra&\it{meɗa momea}		&`(a) red table'\\
			`hair'	&\it{o-{\B}uu}	&+&\it{mokuni}	&`yellow'	&\ra&\it{{\B}uu mokuni}	&`blond hair'\\
			`eye'		&\it{o-mata}		&+&\it{meʔeto}	&`black'	&\ra&\it{mata meʔeto}		&`pupil'\\
			`tooth'	&\it{o-ŋisi}		&+&\it{o-haɗa}	&`monkey'	&\ra&\it{ŋisi haɗa}			&`canine tooth'\\
			`hair'	&\it{o-{\B}ulu}	&+&\it{o-mata}	&`eye'		&\ra&\it{{\B}ulu mata}	&`eyelashes'\\ \lspbottomrule
		\end{tabular}
\end{table}

The construct form is a morphological form used to
mark a head-dependent relationship between
two members of a syntactic phrase.
Such a morphological category is not at all rare in languages of the world.
There is no fundamental functional difference between metathesis marking a
construct form in a language like Leti
and the suffix \it{-(e)t} marking the construct form in Syrian Arabic.
The only difference is in the form of the morphology;
metathesis in one case and suffixation in the other.
Use of metathesis to mark a head-dependent relationship requires
no special model of interaction between morphology and syntax
beyond that which is required for any model of attributive marking.
See \citet[15--22]{ri16} for a discussion of the syntax-morphology interface
within the context of attributive marking.

What \emph{is} surprising is that every language of the greater Timor
region (as well as Rotuman in the Pacific)
with morphological metathesis uses it to mark a construct form.
The reason for this is at least partly connected with the historical development
of metathesis in these languages.
As discussed by \cite{blga98} and summarised in \srf{sec:ComMet},
metathesis in these languages originally
arose only in certain phonological environments,
and only affected unstressed syllables.

In Amarasi, for instance, the final member of a syntactic
phrase bears the main phrasal stress
while non-final members bear secondary stress  (\srf{sec:Str}).
This then creates an environment in
which the processes which ultimately give rise to metathesis are most likely to occur.
Such phonologically conditioned metathesis has then developed into a morphological process.

%\section{Discourse driven metathesis}
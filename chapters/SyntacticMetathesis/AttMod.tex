\section{Attributive modification}\label{sec:AttMod}
Having established the formal criteria by which we can identify a word class of nominals,
I now discuss the structure of the Amarasi nominal phrase and the use of syntactic M\=/forms.
The structure of the Amarasi nominal phrase is given in \qf{tr:NouPhr} below,
following the conventions of a version of X-bar theory \citep{br16}.
The specifier of the nominal phrase can be filled by a possessive phrase (\srf{sec:SynPoss})
and the adjunct position can be filled by another nominal.
Non-final nominals below the level of \br{N} obligatorily occur in the M\=/form.

\begin{exe}
	\ex{\begin{forest}
		[NP,
			[PossP,
			[\br{N},[N]][N]]]
	\end{forest}}\label{tr:NouPhr}
\end{exe}

After a discussion of the basic facts of attributive nominal phrases
I discuss a number of specific cases.
Most phrases involving loans (\srf{sec:LoaNou})
and proper nouns (\srf{sec:ProNam})
behave identically to other nominal phrases
and provide additional evidence that the use of M\=/forms
is a productive process in Amarasi.
Nominal phrases with a conventionalised meaning
are discussed in \srf{sec:LexAtt}.

In \srf{sec:MulMod} I discuss phrases with
multiple attributive modifiers in which every
nominal except the final one occurs in the M\=/form.
I conclude my discussion with the use of M\=/forms before ordinal numbers
and the use of U\=/forms before cardinal numbers (\srf{sec:OrdNum}).
This provides strong evidence that M\=/forms
in attributive phrases cannot be analysed as phonologically conditioned.

A number of attributive nominal phrases extracted from my corpus
are given in \trf{tab:AttAdj}.
The syntactic structure of one of these,
\ve{faut mutiʔ} `white stone', is given in \qf{tr:FautMutiq} below.

\begin{table}[ht]
	\caption{Attributive nominal phrases}\label{tab:AttAdj}
	\centering\stl{0.3em}
		\begin{tabular}{rcllrcll} \lsptoprule
			N\sub{1}			&+&N\sub{2}			&Phrase							&N\sub{1}	&+&N\sub{2}		&Phrase\\\midrule
			\ve{afu}			&+&\ve{meʔe}		&\ve{auf meʔe}			&`earth'	&+&`red'			&`red earth'\\
			\ve{anah}			&+&\ve{mone}		&\ve{aan mone}			&`child'	&+&`male'			&`son'\\
			\ve{atoniʔ}		&+&\ve{reko}		&\ve{atoin reko}		&`man'		&+&`good'			&`good man'\\
			\ve{bare}			&+&\ve{koʔu}		&\ve{baer koʔu}			&`place'	&+&`big'			&`big place'\\
			\ve{baba-f}		&+&\ve{mone}		&\ve{baab mone}			&`FZ/MB'	&+&`male'			&`MB'\\
%			\ve{bruuk}		&+&\ve{oe{\gap}metan}	&\ve{bruu oe{\gap}metan}	&`pants'	&+&`dirty'		&`dirty pants'\\
			\ve{fatu}			&+&\ve{mutiʔ}		&\ve{faut mutiʔ}		&`stone'	&+&`white'		&`white stone'\\
			\ve{kase}			&+&\ve{mutiʔ}		&\ve{kaes mutiʔ}		&`foreign'&+&`white'		&`European'\\
			\ve{kaut} 		&+&\ve{sufaʔ}		&\ve{kau sufaʔ}			&`papaya'	&+&`blossom'	&`papaya blossom'\\
			\ve{manus}		&+&\ve{fua-f}		&\ve{maun fua-f}		&`betel'	&+&`fruit'		&`betel pepper'\\
			\ve{mata-f}		&+&\ve{tei}			&\ve{maat tei}			&`eye'		&+&`faeces'		&`rheum'\\
			\ve{muʔit}		&+&\ve{fui}			&\ve{muiʔ fui}			&`animal'	&+&`wild'			&`wild animal'\\
			\ve{rasi} 		&+&\ve{reʔuf}		&\ve{rais reʔuf}		&`matter'	&+&`bad'			&`evil matter'\\
			\ve{riʔanaʔ}	&+&\ve{munif}		&\ve{riʔaan munif}	&`child'	&+&`young'		&`young child'\\
			\ve{utan}			&+&\ve{kaut}		&\ve{uut kaut}			&`vegetable'&+&`papaya'	&`papaya leaves'\\
%			\ve{} &+&\ve{}	&\ve{}	&`+'&`'\\
		\lspbottomrule
		\end{tabular}
\end{table}

\begin{exe}
	\ex{\begin{forest} where n children=0{tier=word}{}
		[NP,[\br{N},[\br{N},[N,[\ve{faut}\\stlone{\M}]]][N,[\ve{mutiʔ}\\white{\U}]]]]
	\end{forest}}\label{tr:FautMutiq}
\end{exe}

The use of attributive nominal phrases is highly productive in Amarasi
and speakers freely innovate new ones
in a similar way to the use of adjective and noun phrases in English.
Such examples show that the use of M\=/forms in attributive nominal phrases
in Amarasi is a productive morphological process.
One example is given in \qf{120923-2, 1.37-1.39} below.
In \qf{120923-2, 1.37} the speaker introduces the nominal \ve{tani} `rope',
what kind of rope is then specified in \qf{120923-2, 1.39}
with the complex nominal \ve{tain tuni};
it is a rope made from a gewang palm.

\begin{exe}
	\ex{Making a magical sign to protect one's garden from theft:
			\txrf{120923-2} {\emb{120923-2-01-37-01-39.mp3}{\spk{}}{\apl}}}\label{120923-2, 1.37-1.39}
		\begin{xlist}
		\ex{\gll	\sf{ja,} n-pake ʔsokoʔ. n-heer \tbr{tani}.\\
							yes {\n}-use sign {\n}-pull rope{\U}\\
				\glt	`Yes, (he) uses a sign. Ties a rope.'
							\txrf{1.37}}\label{120923-2, 1.37}
		\ex{\glll	na-tuuʔ \tbr{tain} \tbr{tuni}, tua, =ma\\
							na-tuʔu tani tuni tua =ma\\
							{\na}-make.knot rope{\tbrM} gewang.palm{\U} {\tua} =and\\
				\glt	`(He) ties up a rope made from gewang palm (leaves) and {\ldots}' \txrf{1.39}}\label{120923-2, 1.39}
	\end{xlist}\label{ex:120923-2, 5.29}
\end{exe}

Another two examples are given in \qf{ex:120923-1, 0.50-0.55} below
which is part of a story about a kind of curse: the \ve{biku} curse.
In \qf{ex:120923-1, 0.53} we find the nominal phrase \ve{rais biku} `the matter of \it{biku}'.
This nominal is elaborated on in \qf{ex:120923-1, 0.55}
by the noun phrase \ve{moaʔ biku}, `the doing/practice of \it{biku}'.

\begin{exe}
	\ex{Casting the \it{biku} curse: \txrf{120923-1} {\emb{120923-1-00-53-00-55.mp3}{\spk{}}{\apl}}}
		\begin{xlist}
		\ex{\glll	iin n-nao n-ok reʔ \tbr{rais} \tbr{biku} reʔ ia,\\
							ini n-nao n-oka reʔ rasi biku reʔ ia\\
							{\iin} {\n}-go {\n-\ok} {\reqt} matter{\M} curse{\U} {\req} {\ia}\\
				\glt	`He went along with this matter of cursing (people),' \txrf{0.53}}\label{ex:120923-1, 0.53}
		\ex{\glll	\tbr{moaʔ} \tbr{biikgw}=ii\\
							moʔe biku=ii\\
							deed{\M} curse{\Mv}={\ii}\\
				\glt	`the practice of cursing.' \txrf{0.55}}\label{ex:120923-1, 0.55}
	\end{xlist}\label{ex:120923-1, 0.50-0.55}
\end{exe}
\section{Modifiers which are not nominals}\label{sec:OthNomMod}
In this section I discuss other nominal modifiers
which are not themselves nominals.
These include numerals (\srf{sec:NumPhr}),
demonstratives, and determiners (\srf{sec:DetPhr}),
as well as quantifiers (\srf{sec:QuaPhr}).
Nearly all of these modifiers occur after the (attributive) nominal phrase,
and as a result syntactically conditioned M\=/forms do not usually occur before any of these modifiers.
The position of these phrases within the extended nominal is shown in \qf{tr:ExtNom3} below.

\begin{exe}
	\ex{\begin{forest} %where n children=0{tier=word}{}
		[QP,
			[DP,
				[NumP, tikz={\node [draw,fit=(!1)(!2)(!lr)(!rl)] {};}
					[NP,
						[PossP,[NP][poss]]
						[\br{N},[\br{N},[N]][N,]]]
					[Num,[,phantom[,phantom]]]]
				[D,[,phantom]]]
			[Q,[,phantom]]]
	\end{forest}}\label{tr:ExtNom3}
\end{exe}

%\begin{exe}
%	\ex{NP NumP DP QP}\label{ex:ExtNom}
%\end{exe}

%\begin{exe}
%	\ex{\begin{forest} %\tikz[label distance=5mm] where n children=0{tier=word}{}
%		[QP,[DP,[NumP,[NP][Num]][D,[,phantom]]][Q,[,phantom]]]
%	\end{forest}}\label{tr:ExtNom3}
%\end{exe}

\subsection{Number phrase}\label{sec:NumPhr}
The number phrase occurs immediately after the nominal phrase
and before the determiner phrase.
The number phrase takes either a cardinal number or number enclitic
as its head, and a nominal phrase as its specifier.
%Number phrases with a number enclitic as their head are discussed in \srf{sec:NumEnc sec:NumPhr}
Cardinal numbers follow the nominal they quantify and this nominal occurs in the U\=/form.
This is straightforwardly explained by positing that numerals
are the head of a numeral phrase which is outside the nominal phrase.
The basic Amarasi cardinal numbers are given in \trf{tab:AmaCarNum}.
%(The default form of vowel-final numerals is the M\=/form.)

\begin{table}[h]
	\caption{Amarasi cardinal numerals}\label{tab:AmaCarNum}
	\centering
		\begin{tabular}{rll|rll} \lsptoprule
				no.	&Amarasi				&gloss	&no.	&Amarasi				&gloss			\\ \midrule
				1 	&\ve{meseʔ}			&`one'	&7		&\ve{hitu/hiut}	&`seven'		\\
				2 	&\ve{nua}				&`two'	&8		&\ve{fanu/faun}	&`eight'		\\
				3		&\ve{tenu/teun}	&`three'&9		&\ve{seo}				&`nine'			\\
				4		&\ve{haa}				&`four'	&10		&\ve{boʔ}				&`ten'			\\
				5		&\ve{nima/niim}	&`five'	&100	&\ve{natun}			&`hundred'	\\
				6		&\ve{nee}				&`six'	&1,000&\ve{nifun}			&`thousand'	\\ \lspbottomrule
				\end{tabular}
\end{table}

Examples of a cardinal numeral following a nominal are given in 
\qf{ex:130907-3, 3.41}--\qf{ex:130906-1, 3.23}.
Nominals followed by a cardinal numeral take the U\=/form.
The syntactic structure of the number phrase in \qf{ex:130906-1, 3.23}
is given in \qf{tr:130906-1, 3.23}.
%Cardinal numbers occur in the U\=/form or M\=/form
%according to discourse structures (Chapter \ref{ch:DisMet}).

\begin{exe}
	\ex{\glll ʔ-meup ʔ-aan, nehh, ume, u\tbr{me} \tbr{boʔ}=\tbr{ees}, \sf{termasuk} hiit ume.\\
						ʔ-mepu ʔ-ana {} ume ume boʔ=esa \sf{termasuk} hiti ume\\
						{\q}-work {\q}-{\ana} {} house house{\tbrU} ten=one including {\hiit} house\\
			\glt	I worked on houses, ten houses, including our house.'
						\txrf{130907-3, 3.41} {\emb{130907-3-03-41.mp3}{\spk{}}{\apl}}}\label{ex:130907-3, 3.41}
	\ex{\glll	n-fee naan too\tbr{n} \tbr{teun}.\\
						n-fee naan toon tenu\\
						\n-give {\naan} year{\tbrU} three\\
			\glt	`That one has been given three years.'
						\txrf{160326, 15.08} {\emb{160326-15-08.mp3}{\spk{}}{\apl}}}\label{ex:160326, 15.08}
	\ex{\glll	saap n-ak no\tbr{noʔ} \tbr{seo}, a|n-poi na-raar\j=oo-k =te, oakʔ=een.\\
						saap n-ak nonoʔ seo, {\a}n-poi na-rari=oo-k =te okeʔ=ena\\
						since {\n}-say rod{\tbrU} nine {\a\n}-exit {\na}-finish={\oo}-{\k} ={\te} all={\een}\\
			\glt	`Since s/he said there were nine rods, when they'd come out, it was done.'
						\txrf{130906-1, 3.23} {\emb{130906-1-03-23.mp3}{\spk{}}{\apl}}}\label{ex:130906-1, 3.23}
\end{exe}

\begin{exe}
	\ex{\begin{forest} where n children=0{tier=word}{}
		%[NumP,[NP,[\br{N},[N,[\ve{nonoʔ}\\rod{\U}]]]][Num,[\ve{seo}\\nine]]] %no roof
		[NumP,[NP,[\ve{nonoʔ}\\rod{\U}, roof]][Num,[\ve{seo}\\nine]]] %roof
	\end{forest}}\label{tr:130906-1, 3.23}
\end{exe}

A combination of a nominal and numeral
can also occur as the head of a number phrase
to indicate measurements.
Examples are given in \qf{ex:130825-6, 11.21}--\qf{ex:130823-5, 0.26 3} below.
The structure of the number phrase in \qf{ex:130823-5, 0.26 3}
is shown in \qf{tr:130823-5, 0.26 3} below.

\begin{exe}
	\ex{\glll	uma ʔ-tee =ma ʔ-\sf{istarika} bruu\tbr{k} \tbr{pasan} \tbr{nima} =m,
						\hspace{5mm} u-paan ba\tbr{ru} \tbr{pasan} \tbr{nima} =m.\\
						uma ʔ-tea =ma ʔ-\sf{istarika} bruuk pasan nima =ma,
						{} u-pana baru pasan nima =ma.\\
						{\uma} {\q}-arrive =and {\q}-iron pants{\tbrU} set{\U} five =and
						{} {\qu}-fill shirt{\tbrU} set{\U} five =and\\
			\glt	`I came and ironed five pairs of pants and packed five sets of shirts.'
						\newline \txrf{130825-6, 11.21} {\emb{130825-6-11-21.mp3}{\spk{}}{\apl}}}\label{ex:130825-6, 11.21}
	\ex{\glll	ʔ-ak ``ehh, au rookgw=ii a\tbr{ra} \tbr{nonoʔ} \tbr{meseʔ}  =t,\\
						ʔ-ak {} au roko=ii ara nonoʔ meseʔ =te\\
						{\q}-say \hp{``}hey {\au} cigarette={\ii} rest{\tbrU} rod{\U} one ={\te}\\
			\glt	`I said: I've got one cigarette left.' (\emph{lit} `one rod rest')
						\txrf{130825-6, 12.54} {\emb{130825-6-12-54.mp3}{\spk{}}{\apl}}}\label{ex:130825-6, 12.54}
	\ex{\glll	au u-hana mi\tbr{naʔ} \tbr{\sf{taus}} \tbr{meseʔ} mes\\
						au u-hana minaʔ \sf{taus} meseʔ, mes\\
						{\au} {\qu}-cook oil{\U} wok one but\\
			\glt	`I cooked a single wok of oil but {\ldots}' \txrf{130825-6, 0.44} {\emb{130825-6-00-44.mp3}{\spk{}}{\apl}}}\label{ex:130825-6, 0.44}
	\ex{\glll	esa n-teniʔ, mnee\tbr{s} \tbr{kiro} \tbr{niim} \sf{deŋan}, faaf\j=ee iin eku-n.\\
						esa n-teniʔ mneas kiro nima \sf{deŋan} fafi=ee ini eku-n\\
						one {\n}-again rice{\tbrU} kilo{\U} five with pig={\ee} {\iin} neck-{\N}\\
			\glt	`The next one, is five kilos of rice with a pig's neck.'
						\txrf{130823-5, 0.26} {\emb{130823-5-00-26.mp3}{\spk{}}{\apl}}}\label{ex:130823-5, 0.26 3}
\end{exe}

\begin{multicols}{2}
	\begin{exe}
		\ex{\begin{forest} where n children=0{tier=word}{}
			[NumP,
			[NP,[\br{N},[N,[\ve{mnees}\\rice{\U}]]]]
			[NumP,[NP,[\br{N},[N,[\ve{kiro}\\kilo{\U}]]]][Num,[\ve{niim}\\five]]]] %no roofs
			%[NumP,
			%[NP,[\ve{mnees}\\rice{\U}, roof]]
			%[NumP,[NP,[\ve{kiro}\\kilo{\U}, roof]][Num,[\ve{niim}\\five]]]] %roofs
		\end{forest}}\label{tr:130823-5, 0.26 3}
		\ex{\begin{forest} where n children=0{tier=word}{}
			[S,[NP,[\br{N},[N,[\ve{iin}\\{\iin}]]]][VP,[\br{V},[V,[\ve{n-roi}\\carried]]][NumP,[Num,[\ve{haa}\\four]]]]]
		\end{forest}}\label{tr:130925-1, 3.21}
	\end{exe}
\end{multicols}

Cardinal numerals can occur independently without any preceding nominal phrase.
This provides evidence (apart from the use of U\=/forms) that numerals
are the head of their own phrase.
Two numerals as objects of verbs
are given in \qf{ex:130925-1, 3.21} and \qf{ex:130920-1, 0.51} below.
The structure of example \qf{ex:130925-1, 3.21} is given in \qf{tr:130925-1, 3.21} above.

\begin{exe}
	\ex{\gll	iin n-roi \tbr{haa}.\\
						{\iin} {\n}-carry four\\
			\glt	`He carried off four.'
						\txrf{130925-1, 3.21} {\emb{130925-1-03-21.mp3}{\spk{}}{\apl}}}\label{ex:130925-1, 3.21}
	\ex{\gll	n-reuk \tbr{fanu} =te, \sf{paʔ} Charles, \sf{paʔ} Graims, \hspace{40mm} a|n-koen=oo-n neem.\\
					%	n-reku fanu =te, paʔ Charles, paʔ Graims {} {\a}n-koen=oo-n nema\\
						\n-hit eight	={\te} Mr. Charels Mr. Grimes {} \a\n-depart={\oo-\N} {\nema}\\
			\glt	`As it struck 8:00 Mr. Charles, Mr. Grimes came.'
						\txrf{130920-1, 0.51} {\emb{130920-1-00-51.mp3}{\spk{}}{\apl}}}\label{ex:130920-1, 0.51}
\end{exe}

When a subject pronoun is enumerated,
a nominative form of the pronoun occurs before the numeral
and an accusative form usually occurs after the numeral.
This is the same structure found in pronominal equative clauses (\srf{sec:ProEquCla}).
Two examples are given in \qf{ex:130909-6, 3.39} and \qf{ex:Mark 16:3--4} below.

\begin{exe}
	\ex{\gll	hai nua =kai m-mees.\\
						{\hai} two ={\kai} \m-alone\\
			\glt	`The two of us are alone.' \txrf{130909-6, 3.39} {\emb{130909-6-03-39.mp3}{\spk{}}{\apl}}}\label{ex:130909-6, 3.39}
	\ex{\gll	hiit teun =kiit ka= n-eu ta-beeʔ\j=ee =fa.\\
						{\hiit} three ={\kiit} {\ka}= {\n-\eu} \ta-capable={\eeV} ={\fa}\\
			\glt	`The three of us are not going to be able to.' \txrf{Mark 16:3-4}}\label{ex:Mark 16:3--4}
\end{exe}

\subsubsection{Number enclitics}\label{sec:NumEnc sec:OthNomMod}
The head of a number phrase can also be filled by one of the number enclitics
\ve{=ein/=n} `{\ein}' or \ve{=esa/=ees} `one'.
Evidence that the number enclitics form a separate word class to nominal
determiners comes from the fact that they can co-occur with nominal determiners.

Examples of each co-occurring with an enclitic
are given in \qf{ex:130909-6, 3.16} and \qf{ex:130825-8, 1.41 ch:SynMet} below.
This distribution is straightforwardly explained by positing that the number phrase
occurs before the determiner phrase and that the number enclitics
are the head of the former.

\begin{exe}
	%\ex{\gll	iin n-fee mainuan henatiʔ iin aanh=\tbr{ein}=\tbr{aa}, iin aet a-meup-t=eni\\
						%{\iin} \n-give opportunity {\he} {\iin} child=\tbr{\ein}=\tbr{\aa} {\iin} servant {\at}-work-{\at}={\ein}\\
			%\glt	`He has given an opportunity to his children, his servants and workers,'
						%\txrf{130920-1, 4.45} {\emb{130920-1-04-45.mp3}{\spk{}}{\apl}}}\label{ex:130920-1, 4.45}
	\ex{\gll	siin uupʔ=\tbr{ein}=\tbr{ee}, hoo m-ok fauk et umi?\\
						{\siin} CC=\tbr{\ein}=\tbr{\ee} {\hoo} \m-{\ok} how.many {\et} house\\
			\glt	`Those grandkids, how many are at home with you?'
						\txrf{130909-6, 3.16} {\emb{130909-6-03-16.mp3}{\spk{}}{\apl}}}\label{ex:130909-6, 3.16}
	\ex{\gll	ʔ-fei kraan=\tbr{ees}=\tbr{ii} =ma ʔ-toroʔ, ohh.\\
						\q-open tap=\tbr{\es}=\tbr{\ii} =and \q-catch.liquid \\
			\glt	`I turned on one of the taps and tested, ohh.'
						\txrf{130825-8, 1.41} {\emb{130825-8-01-41.mp3}{\spk{}}{\apl}}}\label{ex:130825-8, 1.41 ch:SynMet}
	%\ex{\gll	taaʔ\j=\tbr{ees}=\tbr{ii} mutiʔ.\\
						%branch=\tbr{\es}=\tbr{\ii} white \\
			%\glt	`One of the branches was white.'
						%\txrf{130823-2, 0.49} {\emb{130823-2-00-49.mp3}{\spk{}}{\apl}}}\label{ex:130823-2, 0.49}
\end{exe}

It is possible for the enclitic \ve{=ein} to co-occur with a numeral.
When it does, the numeral must take the ordinal form, despite having no ordinal meaning.
It thus occurs as an attributive modifier within the noun phrase (\srf{sec:OrdNum}).
Two examples are given in \qf{ex:130821-1, 7.38} and \qf{ex:el. 16/03/16 p.47, 41.50} below,
with the structure of \qf{ex:el. 16/03/16 p.47, 41.50} given in \qf{tr:KuaHiutEinNaan}.

\begin{exe}
	\ex{\glll	\sf{lantas,} na-ʔko reʔ \tbr{tua} \tbr{haaʔ}=\tbr{ein} reʔ ia, \hspace{25mm} siin na-honi n-teinʔ=ein =ama,\\
						 \sf{lantas} na-ʔko reʔ tua-\tbr{f} haa-ʔ=eni reʔ ia {} sini na-honi n-teniʔ=eni =ma\\
						forthwith \na-{\qko} {\reqt} person{\tbrM} four-{\qnum}={\ein} {\req} {\ia} {} {\siin} \na-birth \n-again={\ein} =and\\
			\glt	`Then from these four people, they gave birth again.'
						\txrf{130821-1, 7.38} {\emb{130821-1-07-38.mp3}{\spk{}}{\apl}}}\label{ex:130821-1, 7.38}
	\ex{\glll	ahh, \tbr{kua} \tbr{hiut-ʔ}=\tbr{ein} naan, au ʔ-tea ʔ-rair.\\
						{} kua\tbr{n} hitu-ʔ=eni naan au ʔ-tea ʔ-rari\\
						{} village{\tbrM} seven-{\qnum}={\ein} {\naan} {\au} \q-up.to \q-finish\\
			\glt	`Ah yes, those seven villages, I've already been to.'
						\txrf{elicit. 15/03/16 41.50} {\emb{elicit-160315-41-50.mp3}{\spk{}}{\apl}}}\label{ex:el. 16/03/16 p.47, 41.50}
	\ex{\begin{forest} where n children=0{tier=word}{}
		[DP,
		[NumP,[NP,[\br{N},[\br{N},[N,[\ve{kua}\\village{\M}]]][N,[\ve{hiut-ʔ}\\seven]]]]
					[Num,[\ve{=ein}\\{\ein}]]][D,[\ve{naan}\\{\naan}]]]
	\end{forest}}\label{tr:KuaHiutEinNaan}
\end{exe}

To summarise, nouns take the U\=/form before cardinal numerals as numerals
are the head of their own phrase which is outside the nominal phrase.
When the head of the numeral phrase is occupied by a number enclitic,
any numeral is forced to take its ordinal form and occurs within the nominal phrase,
thus inducing metathesis on the head nominal.

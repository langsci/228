\section{Equative clauses}\label{sec:EquCla}
An equative clause involves two adjacent nominal phrases which have the same referent.
One nominal functions as the subject and the other as a non-verbal predicate.
Given examples such that in \qf{ex:FatuKoqu} below,
which has been cited several times in this book,
we do not expect M\=/forms to occur on either member of an equative clause.
This is indeed the case.

%\begin{multicols}{2}
	\begin{exe}
		\ex{\gll \brac{NP} fa\tbr{tu} \bracr{} \brac{NP} koʔu \bracr{}\\
							%\hp{\brac{NP}} fatu {} {} koʔu {}\\
							{} stone{\tbrU} {} {} big{\U} {}\\
				\glt \lh{\brac{NP}}`Stones are big.'}\label{ex:FatuKoqu}
%		\ex{\gll \brac{NP} fa\tbr{ut} koʔu \bracr{}\\
%							\hp{\brac{NP}} fatu koʔu {}\\
%							{} stone{\tbrM} big{\U} {}\\
%				\glt \lh{\brac{NP}}`(a) big stone'}\label{ex:BigSto2}
	\end{exe}
%\end{multicols}

While sentence \qf{ex:FatuKoqu}
is judged acceptable by native speakers,
equative clauses in which both halves consist
of only a single nominal phrase are extremely rare in natural data.
It is much more usual for one half of the equative clause to be a determiner phrase.
Two textual examples of an equative clause
are given in \qf{ex:130907-3, 0.33} and \qf{ex:120923-1, 12.49} below.
In each of these examples the first part of the equative clause
is a determiner phrase (\srf{sec:DetPhr}) and the second part is a nominal phrase.

\begin{exe}\let\eachwordone=\textnormal
	\ex{\gll	[	\ve{ʔnaka} {\ve{skoor=ii} ]\sub{\it{i}}} [ {\ve{bifee}. ]\sub{\it{i}}} \\
						{}	head school={\ii} {}  woman \\
			\glt \lh{[ }`The headmaster was a woman.' \txrf{130907-3, 0.33} {\emb{130907-3-00-33.mp3}{\spk{}}{\apl}}}\label{ex:130907-3, 0.33}
	\ex{\gll	[	{\ve{meens=ii} ]\sub{\it{i}}} [ \ve{humaʔ} {\ve{mes{\tl}meseʔ.} ]\sub{\it{i}}} \ve{ka=} \ve{n-\sf{beda}} \ve{=fa}\\
						{}	sickness={\ii} {} kind {\prd}one {\ka}= \n-different ={\fa}\\
			\glt	\lh{[ }`The sickness was exactly the same (\emph{lit.} one kind). It wasn't different.'
						\txrf{120923-1, 12.49} {\emb{120923-1-12-49.mp3}{\spk{}}{\apl}}}\label{ex:120923-1, 12.49}
\end{exe}

Two similar examples are given in
\qf{ex:130823-2, 0.49 ch:SynMet} and \qf{ex:130914-3, 1.03} below.
In each of these examples the second part of the equative
phrase consists of a property nominal.

\begin{exe}\let\eachwordone=\textnormal
	\ex{\gll	[ 	{\ve{taaʔ\j=ees=ii} ]\sub{\it{i}}} [ {\ve{mutiʔ.} ]\sub{\it{i}}} \\
					%	{}	taʔe=esa=ii {} mutiʔ {}\\
						{}	branch={\es}={\ii} {} white \\
			\glt	\lh{[ }`One of these branches was white.'
						\txrf{130823-2, 0.49} {\emb{130823-2-00-49.mp3}{\spk{}}{\apl}}}\label{ex:130823-2, 0.49 ch:SynMet}
	\ex{\gll	\ve{mama,} [ \ve{au} {\ve{huutgw=ii} ]\sub{\it{i}}} [ {\ve{maʔtaneʔ,} ]\sub{\it{i}} \,} \ve{aa?} \\
					%			mama {} au hutu=ii {} maʔtaneʔ aa\\
								mum {} {\au} louse={\ii} {} strong {\aaQ} \\
			\glt	`I've got lots of lice, haven't I?' (\emph{lit.} my lice are strong)
			\txrf{130914-3, 1.03} {\emb{130914-3-01-03.mp3}{\spk{}}{\apl}}}\label{ex:130914-3, 1.03}
\end{exe}

%Equative clauses illustrate clearly the morphological nature of Amarasi nominal M\=/forms.
%Compare, for instance, example \qf{ex:FatuKoqu} with \qf{ex:FautKoqu}.
%Each example is phonologically identical with the exception of the contrast
%between the U\=/form in \qf{ex:FatuKoqu} and the M\=/form in \qf{ex:FautKoqu}.
%This difference in U\=/form and M\=/form results in each phrase having
%a different syntactic structure. %as shown in \qf{tr:Fatu/FautKoqu}.
%Example \qf{ex:FatuKoqu} has two nominal phrases
%while example \qf{ex:FautKoqu} consists of a single nominal phrase.

%\begin{multicols}{2}
%	\begin{exe}
%%		\ex{
%%			\begin{xlist}
%				\ex{\gll \brac{NP} fa\tbr{tu} \bracr{} \brac{NP} koʔu \bracr{}\\
%									%\hp{\brac{NP}} fatu {} {} koʔu {}\\
%									{} stone{\tbrU} {} {} big{\U} {}\\
%						\glt \lh{\brac{NP}}`Stones are big.'}\label{ex:FatuKoqu}
%				\ex{\gll \brac{NP} fa\tbr{ut} koʔu \bracr{}\\
%								%	\hp{\brac{NP}} fatu koʔu {}\\
%									{} stone{\tbrM} big{\U} {}\\
%						\glt \lh{\brac{NP}}`(a) big stone'}\label{ex:FautKoqu}
%		}\label{ex2:Met}
%	\end{exe}
%\end{multicols}
%
%\begin{multicols}{2}
%	\begin{exe}\ex{
%		\begin{xlist}
%			\ex{\begin{forest} %where n children=0{tier=word}{}
%				[S,[NP,[\br{N},[N,[\ve{fatu}\\stlone{\U}]]]][NP,[\br{N},[N,[\ve{koʔu}\\big{\U}]]]]]
%			\end{forest}}
%			\ex{\begin{forest} where n children=0{tier=word}{}
%				[NP,[\br{N},[\br{N},[N,[\ve{faut}\\stlone{\M}]]][N,[\ve{koʔu}\\big{\U}]]]]
%			\end{forest}}
%		\end{xlist}}\label{tr:Fatu/FautKoqu}
%	\end{exe}
%\end{multicols}

\subsection{Pronominal equative clauses}\label{sec:ProEquCla}
When the first part of an equative clause is a third person pronoun,
the nominal phrase simply follows the pronoun.
Two examples are given in \qf{ex:120715-4, 1.16} and \qf{ex:Genesis 29:6} below
with a thing nominal and property nominal respectively.
Pronominal equative clauses do not induce M\=/forms
on either member of the equative clause.
Example \qf{ex:120715-4, 1.16} is a left-dislocated topic,
with the referential information outside the clause proper,
and the trace pronoun being the syntactic subject of the equative clause.

\begin{exe}
	\ex{\gll	Mooʔhituʔ reʔ naan, \brac{} {\ve{iin} \bracr{\it{i}}} \ve{ahh} \brac{} {kaunaʔ. \bracr{\it{i}}}\\
						Moo{\Q}hitu{\Q} {\req} {\naan} {} {\iin} {} {} snake{\U}\\
			\glt	`Now as for that Moo{\Q}-Hitu, he was a snake.'
						\txrf{120715-4, 1.16} {\emb{120715-4-01-16.mp3}{\spk{}}{\apl}}\footnote{
								In the accompanying audio file for \qf{ex:120715-4, 1.16} another speaker first
								completes the equative clause for the main narrator with the word \ve{kaunaʔ} `snake',
								before the narrator completes it himself.
								}}\label{ex:120715-4, 1.16}
	\ex{\gll		a|ʔnaef=ee et reʔ nee. \brac{} {iin \bracr{\it{i}} } \brac{} {reko. \bracr{\it{i}}}\\
							{\a}old.man={\ee} {\et} {\reqt} {\nee} {} {\iin} {} good{\U}\\
			\glt		\lh{a|}`The old man is there. He is well.' \txrf{Genesis 29:6}}\label{ex:Genesis 29:6}
\end{exe}

When the pronoun in an equative clause is not third person singular,
the nominal phrase is preceded by a nominative pronoun
and usually followed by an accusative pronoun.
Four examples are given in \qf{ex:120923-2, 1.51}--\qf{ex:1 Peter 1:16} below.

\begin{exe}
	\ex{\gll	\brac{} {{hoo} \bracr{\it{i}} } \brac{} {{a-baka-t} \bracr{\it{i}}} \brac{} {{=koo.} \bracr{\it{i}}}
						{m-ak} {mu-baka} {=ma} {m-tama} {=te}\\
						{} {\hoo} {} {\at}-steal{\U}-{\at} {} ={\koo} {\m}-say \muu-steal =and \m-go.in ={\te} \\
				\glt	`You are a thief, meaning when you steal and enter,'
							\txrf{120923-2, 1.51} {\emb{120923-2-01-51.mp3}{\spk{}}{\apl}}}\label{ex:120923-2, 1.51}
	\ex{\gll	{au} {u-toon,} {au} {au} {au} \brac{} {{au} \bracr{\it{i}} } \brac{} {{a-mon{\tl}mono-t} \bracr{\it{i}}} \brac{} {{=kau!} \bracr{\it{i}}}\\
							{\au} {\qu}-tell {} {} {} {} {\au} {} {\at}-{\prd}stupid{\U}-{\at} {} {\kau}\\
				\glt	`I tell (you), I was a real idiot!' \txrf{130825-6, 3.12} {\emb{130825-6-03-12.mp3}{\spk{}}{\apl}}}\label{ex:130825-6, 3.12}
	\ex{\gll	{n-ak} {hei} {=te,} \brac{} {{hoo} \bracr{\it{i}} } \brac{} {{mun{\tl}munif} \bracr{\it{i}}} \brac{} {{=koo} \bracr{\it{i}}} {=t,}
						{mu-snaas} {mu-ʔko{\ldots}}\\
						\n-say hey ={\te} {} {\hoo} {} {\prd}young{\U} {} ={\koo} ={\te} {\muu}-stop {\muu}-{\qko}\\
			\glt	`Saying: Now, while you're young, you should stop.'
						\txrf{130907-3, 4.52} {\emb{130907-3-04-52.mp3}{\spk{}}{\apl}}}\label{ex:130907-3, 4.52}
	%\ex{\glll	hii ro he kninu{\Q}, natuin {} Au {} kninu{\Q} Kau. {}\\
	\ex{\gll	{hii} {ro} {he} {kninuʔ,} {na-tuin} \brac{} {{au} \bracr{\it{i}} } \brac{} {{kninuʔ} \bracr{\it{i}}} \brac{} {{=kau.} \bracr{\it{i}}}\\
						{\hii} must {\he} clean {\na}-because {} {\au} {} clean {} ={\kau} {}\\
			\glt	`You must be holy because I am holy.' \txrf{1 Peter 1:16}}\label{ex:1 Peter 1:16}
\end{exe}

Equative clauses do not trigger M\=/forms on either nominal because
neither nominal is syntactically modifying the other within the nominal phrase.
The only phonological difference between an equative clause
and an attributive phrase is that the first nominal
in an equative clause is in the U\=/form while the first
nominal in an attributive phrase is in the M\=/form.
The comparison of equative clauses with attributive phrases
provides strong evidence of the morphological nature of Amarasi metathesis.

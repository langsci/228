
\section{Introduction}
In this chapter I describe and analyse the function of Amarasi metathesis in the syntax.
In the syntax, metathesis is a morphological device
marking the presence of an attributive modifier.\footnote{
		There is no competition between metathesis in the syntax
		and phonologically conditioned metathesis (Chapter \ref{ch:PhoMet}).
		Syntactically triggered metathesis affects word-medial members of
		a phrase while vowel-initial enclitics attach to the final members of phrases.
		Similarly, there is no competition between metathesis marking syntactic
		structures and metathesis marking discourse structures.
		This is discussed in full detail in \srf{sec:SynDisDriMet}.}
A metathesised word is a construct form (\srf{sec:ConFor})
signalling the presence of a dependent modifier.
A syntactic M\=/form (metathesised form) canonically
occurs in a complementary relationship
with a \mbox{U\=/form} (unmetathesised form),
the latter of which syntactically completes the former.

An example of the syntactic function of metathesis
can be seen by comparing examples
\qf{ex:NenoMeseq} and \qf{ex:NeonMeseq} below.
Each consists of the noun \ve{neno} `day'
followed by the numeral \ve{meseʔ} `one'.
When the head nominal occurs in the U\=/form,
the numeral is the head of a number phrase and has a cardinal meaning.
However, when the head nominal occurs in the M\=/form,
the numeral occurs within the noun phrase and has an ordinal meaning.

\begin{multicols}{2}
	\begin{exe}
	\let\eachwordone=\textnormal
	\let\eachwordtwo=\itshape
		\ex{\glll {}				ˌnɛnɔ {} {} ˈmɛsɛʔ	{} \\
							\brac{NP} ne\tbr{no} \bracr{} \brac{Num} meseʔ \bracr{}\\
							{} day{\tbrU} {} {} one{\U} {}\\
				\glt \lh{\brac{NP}} `one day' {\emb{neno-meseq.mp3}{\spk{}}{\apl}}}\label{ex:NenoMeseq}
		\ex{\glll	{} ˌnɛ.ɔn ˈmɛsɛʔ {} \\
							\brac{NP} ne\tbr{on} meseʔ \bracr{}\\
							{} day{\tbrM} one{\U} {}\\
				\glt \lh{\brac{NP}}`first day (i.e. Monday)' {\emb{neon-meseq.mp3}{\spk{}}{\apl}}}\label{ex:NeonMeseq}
	\end{exe}
\end{multicols}

Each of the phrases in \qf{ex:NenoMeseq} and \qf{ex:NeonMeseq}
has identical intonation and stress,
as can be heard with the accompanying audio files.
Neither do the vowels of the M\=/form collapse into a single phonetic syllable.
The \emph{only} phonetic difference between each of these phrases
is the order of the final consonant and vowel of the head nominal; metathesis.
(See \srf{sec:OrdNum} for more discussion of this, and similar examples.)
Trees showing the structure of each of \qf{ex:NenoMeseq} and \qf{ex:NeonMeseq}
are given in \qf{tr:NenoMeseq} and \qf{tr:NeonMeseq} respectively.

\begin{multicols}{2}
	\begin{exe}
		\ex{\begin{forest} where n children=0{tier=word}{}
			[NumP,[NP,[\br{N},[N,[\ve{ne\tbr{no}}\\day{\tbrU},label={below:{\hspace{15mm}`one day'}},]]]]
			[Num,[\ve{meseʔ}\\one{\U}]]]
		\end{forest}}\label{tr:NenoMeseq}
		\ex{\begin{forest} where n children=0{tier=word}{}
			[NP,[\br{N},[\br{N},[N,[\ve{ne\tbr{on}}\\day{\tbrM},label={below:{\hspace{17mm}`first day'}},]]]
			[N,[\ve{meseʔ}\\one{\U}]]]]
		\end{forest}}\label{tr:NeonMeseq}
	\end{exe}
\end{multicols}

Another example of metathesis in the syntax
can be seen by comparing examples \qf{ex:BigSto1} and \qf{ex:BigSto2} below.
Example \qf{ex:BigSto1} with an initial U\=/form is an equative clause
(\srf{sec:EquCla}) with two nominals as subject and predicate,
while example \qf{ex:BigSto2} with an initial M\=/form consists of a single nominal phrase
with the second nominal functioning attributively as a dependent modifier.
Each of these phrases also has identical stress and intonation,
with the different syntax signalled only by metathesis.
%Trees showing the structure of each of these phrases
%are given in \qf{tr:BigSto1} and \qf{tr:BigSto2} respectively.

\begin{multicols}{2}
	\begin{exe}
		\ex{\gll \brac{NP} fa\tbr{tu} \bracr{} \brac{NP} koʔu \bracr{}\\
							%\hp{\brac{NP}} fatu {} {} koʔu {}\\
							{} stone{\tbrU} {} {} big{\U} {}\\
				\glt \lh{\brac{NP}}`Stones are big.'}\label{ex:BigSto1}
		\ex{\gll \brac{NP} fa\tbr{ut} koʔu \bracr{}\\
						%	\hp{\brac{NP}} fatu koʔu {}\\
							{} stone{\tbrM} big{\U} {}\\
				\glt \lh{\brac{NP}}`(a) big stone'}\label{ex:BigSto2}
	\end{exe}
\end{multicols}

%\begin{multicols}{2}
%	\begin{exe}
%		\ex{\begin{forest} where n children=0{tier=word}{}
%			[S,[NP,[\br{N},[N,[\ve{fatu}\\stlone{\U}]]]][NP,[\br{N},[N,[\ve{koʔu}\\big{\U}]]]]]
%		\end{forest}}\label{tr:BigSto1}
%		\ex{\begin{forest} where n children=0{tier=word}{}
%			[NP,[\br{N},[\br{N},[N,[\ve{faut}\\stlone{\M}]]][N,[\ve{koʔu}\\big{\U}]]]]
%		\end{forest}}\label{tr:BigSto2}
%	\end{exe}
%\end{multicols}

Similarly, within the verb phrase metathesis marks the presence
of a modifying verb and thus marks a serial verb construction.
Compare examples \qf{ex:130825-6, 0.36 3} and \qf{ex:130925-1, 3.32} below.
Example \qf{ex:130825-6, 0.36 3} contains two adjacent verbs with the first in the M\=/form.
Thus, both verbs belong to a single verb phrase and
are a serial verb construction describing a single event.
%(The final verb in \qf{ex:130825-6, 0.36 3} occurs in the
%M\=/form as determined by discourse, see Chapter \ref{ch:DisMet}.)
Example \qf{ex:130925-1, 3.32}, on the other hand,
has two adjacent verbs with the first in the U\=/form,
and each verb is the head of its own verb phrase
and describes two separate events.
Syntactic trees of each of these examples are
given in \qf{tr:130925-1, 3.32} and \qf{tr:130825-6, 0.36 3}
below respectively.

\begin{exe}
	\ex{\glll	saap au \brac{VP} ʔ-so\tbr{iʔ} {u-rair. \bracr{}}\\
						saap au {} ʔ-soʔi u-rari\\
						because {\au} {} {\q}-count{\tbrM} {\qu}-finish{\M}\\
			\glt	`Because I'd finished counting.'
						\txrf{130825-6, 0.36} {\emb{130825-6-00-36.mp3}{\spk{}}{\apl}}}\label{ex:130825-6, 0.36 3}
	\ex{\glll	Maksen \brac{VP} {n-a\tbr{mi} \bracr{}} \brac{VP} {n-aim \bracr{}} n-ak suuk na-hine =t,\\
						Maksen {} n-ami {} n-aim n-ak suuk na-hine =te\\
						Maksen {} \n-search{\tbrU} {} \n-search{\M} \n-say rather \na-know{\U} ={\te}\\
			\glt	`Maksen searched and searched, he said that when he knew {\ldots}'
						\txrf{130925-1, 3.32} {\emb{130925-1-03-32.mp3}{\spk{}}{\apl}}}\label{ex:130925-1, 3.32}
\end{exe}
\begin{multicols}{2}
	\begin{exe}
		\ex{\begin{forest} where n children=0{tier=word}{}
			[S,[NP,[\br{N},[N,[\ve{au}\\{\au}]]]]
			[VP,[\br{V},[\br{V},[V,[\ve{ʔso\tbr{iʔ}}\\count{\tbrM}]]][V,[\ve{urair}\\finish{\M}]]]]]
		\end{forest}}\label{tr:130925-1, 3.32}
		\ex{\begin{forest} for tree={s sep=0.8mm, where n children=0{tier=word}{},}
			[S,[NP,[\br{N},[N,[\ve{Maksen}\\{Maksen}]]]]
			[VP,[\br{V},[V,[\ve{na\tbr{mi}}\\search{\tbrU}]]]]
			[VP,[\br{V},[V,[\ve{naim}\\search{\M}]]]]]
		\end{forest}}\label{tr:130825-6, 0.36 3}
	\end{exe}
\end{multicols}

Under the syntactic analysis I propose, metathesis is restricted
to the domain of \br{X} (X-bar); \br{N} within the nominal phrase
and \br{V} within a phonotactically restricted subset of verb phrases.
Whenever a word of the same word class as the head occurs within \br{X},
the head occurs in the M\=/form.
Each non final word in \br{X} is in the M\=/form
with the final word of \br{X} in the U\=/form.
The maximal structure of the extended nominal in Amarasi is given in \qf{tr:ExtNom}
and the structure of the verb phrase in \qf{tr:VerPhr}
with the domain of metathesis indicated.

Attributive modification is a phenomenon which typically
occurs in syntax but it can also occur in morphology.
In this chapter I analyse attribution within the syntax.
(The possibility of analysing attribution within the morphology is discussed in \srf{sec:ProsMet}.)
In Amarasi, the marking of modification is a functional requirement
which impacts on the surface realisation.
An M\=/form is the morphological marking of
a syntactic relationship between two nominals or two verbs.

\begin{multicols}{2}
	\begin{exe}
		\ex{\begin{forest} for tree={s sep=0.6mm,}
			[QP,
				[DP,
					[NumP,
						[NP,
							[PossP,[NP][poss]]
							[\br{N},[\br{N}, tikz={\node [draw,dashed,inner sep=0.5,fit to=tree,]{};}[N]]
							[N,label={[label distance=4mm]below:\it{\footnotesize \hspace{18mm}metathesis domain}},]]]
						[Num,[,phantom[,phantom]]]]
					[D,[,phantom]]]
				[Q,[,phantom]]]
		\end{forest}}\label{tr:ExtNom}
		\ex{\begin{forest} %where n children=0{tier=word}{}
				[VP,
					[\br{V},[\br{V}, tikz={\node [draw,dashed,inner sep=0.5,fit to=tree,]{};}[V]][V]]
					[NP,label={[label distance=17mm]below:\it{\footnotesize \hspace{4mm}metathesis domain}},[,phantom]]]
		\end{forest}}\label{tr:VerPhr}
	\end{exe}
\end{multicols}

Most of this chapter is devoted to a discussion of the extended
nominal phrase in which M\=/forms are more obviously and thoroughly constrained by syntax.
I begin in \srf{sec:NomWorCla} by discussing the
syntactic and morphological criteria which
define a word class of nominals in Amarasi.
There is no morpho-syntactic basis for distinguishing
between a class of adjectives and nouns in Amarasi.

In \srf{sec:AttMod} I discuss the structure of attributive phrases
which trigger metathesis on the head nominal.
In \srf{sec:Poss} I show that possession
does not trigger metathesis on the head nominal.
In \srf{sec:OthNomMod} I show that modifiers which are not nominals
do not induce metathesis on the head nominal.
Such modifiers include numerals, demonstratives, and quantifiers.
In \srf{sec:EquCla} I discuss the structure
of equative clauses which involve two nominal phrases but do not trigger M\=/forms.
In \srf{sec:SVC} I discuss the structure
of the verb phrase and serial verb constructions
in which non-final verbs usually occur in the M\=/form.
I conclude in \srf{sec:ProsMet} by discussing an analysis
of metathesis in the syntax as being conditioned by prosodic structures.
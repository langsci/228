\section{Serial verb constructions}\label{sec:SVC}
Syntactically conditioned M\=/forms also occur in the verb phrase to
mark a serial verb construction (SVC).
Formal properties which allow us to identify a word class
of verbs were given in \trf{tab:AmaWorCla} above,
repeated as \ref{tab:AmaWorCla 2} below.
The clearest of these formal properties is that verbs obligatorily
agree with the person and number of the subject (except in some imperatives)
by taking a verbal agreement prefix.\footnote{
		There are only four verbs in my corpus which do
		not agree with the subject of the sentence.
		These are the auxiliaries \ve{he} {\he} `irrealis',
		and \ve{bisa} `can' (from Malay \it{bisa}),
		as well as the locational verbs \ve{on} {\on} `irrealis locative'
		and \ve{et} {\et} `imperfective locative'.
		Other locational verbs including \ve{n-bi} {\bi} `realis locative'
		and \ve{na-ʔko} `{\qko}, ablative' all take agreement prefixes as expected.}
The form and distribution of these prefixes is discussed in \srf{sec:VerAgrPre}.\footnote{
		The verb \ve{{\rt}Vma} `come' has an irregular conjugation.
		See \trf{tab:ConCome} on \prf{tab:ConCome} for details.}

\begin{table}[ht]
	\caption[Amarasi word classes]{Amarasi word classes\su{†}}\label{tab:AmaWorCla 2}
	\centering
		\begin{threeparttable}\stl{0.4em}
			\begin{tabular}{rcccccccc}\lsptoprule
											&agr-&{\at}&{\b}&{\mak}&C{\#}{\ra}{\0}&\tsc{subj/obj}&=Det&=Num\\ \midrule
				Nominal 			&--&--&--&--&{✔}&{✔}&{✔}&{✔}\\
				Precategorial	&{✔}&{✔}&{✔}&{✔}&{✔}&{✔}&{✔}&{✔}\\
				Verb	 				&{✔}&{✔}&{✔}&{✔}&--&--&--&--\\ \lspbottomrule
			\end{tabular}
		\begin{tablenotes}
			\item[†]	agr-: take verbal agreement prefixes (\srf{sec:VerAgrPre}),
								{\at} can be nominalised with the circumfix \ve{a-{\ldots}-t} (\srf{sec:NomA--t}),
								{\b} can take the transitive suffix \ve{-b} (\srf{sec:TraSuf}), 
								{\mak} can take the reciprocal prefix \ve{ma(k)-} (\srf{sec:RecPre}),
								C{\#}{\ra}{\0} final consonant can be deleted to derive verbs (\srf{sec:BasVerDer}),
								\tsc{subj/obj} can be the subject or object of a verb,
								=Det can take definiteness marking determiners (\srf{sec:Det}),
								=Num can take number enclitics (\srf{sec:NumEnc sec:OthNomMod}).
		\end{tablenotes}
		\end{threeparttable}
\end{table}

``A serial verb construction (SVC) is a sequence of verbs which
act together as a single predicate'' \citep[1]{ai06}.
Non-initial verbs in an Amarasi SVC occur in the M\=/form.
The analysis I adopt is one in which members of an SVC
are adjuncts below the level of \br{V}, in the same way
as attributive modifiers are adjuncts below the level of \br{N}.
The proposed structure of the Amarasi verb phrase is given in \qf{tr:VerPhr 2} below.
The object nominal phrase fills the specifier position.

\begin{exe}
		\ex{\begin{forest}
				[VP,
					[\br{V},[\br{V}, tikz={\node [draw,dashed,inner sep=2,fit to=tree,]{};}[V]][V]]
					[NP,label={[label distance=17mm]below:\it{\footnotesize \hspace{4mm}metathesis domain}},[,phantom]]]
		\end{forest}}\label{tr:VerPhr 2}
\end{exe}

Having the object nominal appear in the specifier position of the VP
in \qf{tr:VerPhr 2} is cross-linguistically unusual.
The reason it occurs in this position rather than a complement position,
close to the head verb, as is commonly the case in other languages,
results from its competition with the attributive adjunct in relation
to the structural domain of metathesis marking dependency in Amarasi.
Unlike attributive verbs, object nominals do not induce M\=/forms on the verb,
and verbs with an object freely occur in the U\=/form or M\=/form as determined by the discourse
structures of the entire phrase, as discussed in Chapter \ref{ch:DisMet}.

Three examples of SVCs in Amarasi are given in \qf{ex:130902-1, 2.09}--\qf{ex:130821-1, 5.00 1} below.
The final verb of an SVC occurs in the U\=/form or M\=/form depending on the discourse
structures of the clause (Chapter \ref{ch:DisMet}).
The structure of the verbal clause of example \qf{ex:130821-1, 5.00 1}
is given in \qf{tr:130821-1, 5.00 1} below.

\begin{exe}
	\ex{\glll	ahh kaah, neno kree\j=ii =te, hai m-ta\tbr{am} mi-krei.\\
						{} kaah neno krei=ii =te hai m-tama mi-krei\\
						{} {\kaah} day church={\ii} ={\te} {\hai} {\m}-enter{\tbrM} {\mi}-church\\
			\glt	\lh{ahh}`Ah no, when it was Sunday, we went to church.'
						\txrf{130902-1, 2.09} {\emb{130902-1-02-09.mp3}{\spk{}}{\apl}}}\label{ex:130902-1, 2.09}
	\ex{\glll	saap au ʔ-so\tbr{iʔ} u-rair.\\
						saap au ʔ-soʔi u-rari\\
						because {\au} {\q}-count{\tbrM} {\qu}-finish{\M}\\
			\glt	`Because I'd finished counting.' \txrf{130825-6, 0.36} {\emb{130825-6-00-36.mp3}{\spk{}}{\apl}}}\label{ex:130825-6, 0.36 1}
	\ex{\glll	tuaf=esa n-fa\tbr{in} neem, kaan-n=ee naiʔ Tuʔas.\\
						tuaf=esa n-fani nema kana-n=ee naiʔ Tuʔas\\
						person={\es} {\n}-return{\tbrM} {\nema\M} name-{\N}={\ee} {\naiq} Tu{\Q}as\\
			\glt	`One person came back (here), his name was Tu{\Q}as.'
						\txrf{130821-1, 5.00} {\emb{130821-1-05-00.mp3}{\spk{}}{\apl}}}\label{ex:130821-1, 5.00 1}
	\ex{\begin{forest} where n children=0{tier=word}{}
		[S,[NumP,[NP,[\br{N},[N,[\ve{tuaf}\\person]]]][Num,[\ve{=esa}\\{\es}]]][VP,[\br{V},[\br{V},[V,[\ve{n-fa\tbr{in}}\\return{\tbrM}]]][V,[\ve{neem}\\come]]]]]
	\end{forest}}\label{tr:130821-1, 5.00 1}
\end{exe}


In a cross-linguistic survey of SVCs \cite{ai06}
gives five properties of canonical SVCs.
Of these, Amarasi SVCs clearly conform to at least four,
listed in \qf{ex:ProSVCAma} below.

\begin{exe}
	\ex{Properties of serial verb constructions in Amarasi: }\label{ex:ProSVCAma}
		\begin{xlist}
			\ex{Single predicate (SVCs function on par with monoverbal clauses)}
			%\ex{Single clause (SVCs are monoclausal and allow no markers of syntactic dependency on their components.)}
			\ex{Single intonation (intonation is the same as monoverbal clauses)}
			\ex{Single tense/aspect/mood/polarity}
			\ex{encode a single event}
		\end{xlist}
\end{exe}

The only property of an SVC given by \cite{ai06} to which Amarasi
SVCs arguably do not conform is that SVCs should be
``monoclausal and allow no markers of syntactic dependency on their components'' \cite[6]{ai06}.
In Amarasi non-final verbs of an SVC occur in the M\=/form,
which I analyse as a marker of syntactic dependency;
M\=/forms are a construct form (\srf{sec:ConFor}) which mark the presence
of a dependent modifier.

\cite{ai06} includes this criterion in her definition to distinguish
SVCs from other structures including ``coordination, consecutivization,
complement clauses [and] subordinate clauses''.
In Amarasi each of these kinds of clauses have different structures.
The differences between an SVC, coordination, and complement clauses
are illustrated with an example each in \qf{ex:130825-6, 0.52}--\qf{ex:130914-1, 1.01} below.

Example \qf{ex:130825-6, 0.52} is an instance of an SVC.
The two verbs are immediately adjacent and the first is in the M\=/form.
Example \qf{ex:130825-6, 21.14} is an instance of coordination
or consecutivization and the connector \ve{=ma} occurs between the two verbs.
(See \srf{sec:DepCoo} for more discussion of the structure of coordination.)
Example \qf{ex:130914-1, 1.01} is an instance of complementization or subordination,
and the subordinate clause is introduced with the irrealis verb \ve{he}.

\begin{exe}
	\ex{Serial verb construction}\label{ex:SVC}
	\sn{\glll	au ka= ʔ-a\tbr{im} u-hiin =fa roit.\\
						au ka= ʔ-ami u-hini =fa roit\\
						{\au} {\ka}= {\q}-look.for{\tbrM} {\qu}-know{\M} ={\fa} money\\
			\glt	`I don't know how to look for money.' \txrf{130825-6, 0.52} {\emb{130825-6-00-52.mp3}{\spk{}}{\apl}}}\label{ex:130825-6, 0.52}
	\ex{coordination/consecutivization}
	\sn{\glll	ʔ-aiti \tbr{=ma} ʔ-rees=een.\\
						ʔ-aiti =ma ʔ-resa=ena\\
						\q-pick.up{\U} =and \q-read{\Mv}={\een} \\
			\glt	`(I) picked (them) up and started to read.' \txrf{130825-6, 21.14} {\emb{130825-6-21-14.mp3}{\spk{}}{\apl}}}\label{ex:130825-6, 21.14}
	\ex{complement/subordinate clause}
	\sn{\glll	ʔ-aim \tbr{he} ʔ-soos \sf{bantal}.\\
						ʔ-ami he ʔ-sosa \sf{bantal}\\
						{ʔ-look.for\M} {\he} \q-buy{\M} cushion\\
			\glt	`I'm looking to buy a cushion.'
						\txrf{130914-1, 1.01} {\emb{130914-1-01-01.mp3}{\spk{}}{\apl}}}\label{ex:130914-1, 1.01}
\end{exe}

SVCs with more than two verbs also occur.
All non-final verbs in such SVCs occur in the M\=/form,
and the final verb in the U\=/form or M\=/form as determined by the discourse (Chapter \ref{ch:DisMet}).
Two examples are given in \qf{ex:130825-6, 16.55} and \qf{ex:130925-1, 1.30} below,
though only in \qf{ex:130925-1, 1.30} does each of the non-final verbs have a distinct M\=/form.
The structure of \qf{ex:130925-1, 1.30} is given in \qf{tr:130925-1, 1.30} below.

\begin{exe}
	\ex{\glll	ʔ-ak: ``au ʔ-nao ʔ-meo ʔ-aan bi Uli n-ok bi Rahel''.\\
						ʔ-ak \hp{``}au ʔ-nao ʔ-meo ʔ-ana bi Uli n-oka bi Rahel\\
						\q-say \hp{``}{\au} \q-go \q-see \q-{\ana\M} {\BI} Uli \n-{\ok\M} {\BI} Rachel\\
			\glt	`I said: ``I'll go and see Uli and Rachel.{''}.' \txrf{130825-6, 16.55} {\emb{130825-6-16-55.mp3}{\spk{}}{\apl}}}\label{ex:130825-6, 16.55}
	\ex{\glll	n-ak ``au ʔ-o\tbr{at} ʔ-i\tbr{is} ʔ-aan''.\\
						n-ak \hp{``}au ʔ-ote ʔ-isa ʔ-ana\\
						\n-say \hp{``}{\au} \q-cut{\tbrM} \q-completely{\tbrM} \q-{\ana\M}\\
			\glt	he said: ``I'll cut (him) dead.''.'
						\txrf{130925-1, 1.30} {\emb{130925-1-01-30.mp3}{\spk{}}{\apl}}}\label{ex:130925-1, 1.30}
	\ex{\begin{forest} where n children=0{tier=word}{}
			[S
				[NP,[\br{N},[N,[\ve{au}\\{\au}]]]]
				[VP,[\br{V},[\br{V},[\br{V},[V,[\ve{ʔ-oat}\\cut{\M}]]]
				[V,[\ve{ʔ-iis}\\complete{\M}]]][V,[\ve{ʔ-aan}\\resultative{\M}]]]]
			]
	\end{forest}}\label{tr:130925-1, 1.30}
\end{exe}

There are a number of verbs which occur frequently or exclusively in SVCs.
The discussion in this section draws upon that of \cite{jagr11}, who analyse SVCs in Kupang Malay.
The similarities between Kupang Malay and Amarasi SVCs are a result of 
Kupang Malay calquing on structures found in the local languages of western Timor.

The root \ve{\rt ani} occurs almost exclusively as the final verb of an SVC.
It carries a temporal meaning, indicating that the event encoded by
the SVC occurs before some other event.
Occasionally it also means `directly, straight-away'.
An example with the meaning `before' is given in
\qf{ex:130913-1, 2.43 ch:SynMet} below,
and an example with the meaning `directly, straight-away' in \qf{ex:130921-1, 0.59}.

\begin{exe}
	\ex{\glll	iin n-toko =t, iin ofa n-reis \tbr{n}-\tbr{ain} areʔ haef=ein a|msaʔ.	\\
						ini n-toko =te ini ofa n-resi n-ani areʔ haef=eni {\a}msaʔ \\
						{\iin} \n-sit ={\te} {\iin} sure {\n}-plan{\M} \n-before each messenger={\ein} {\a}even \\
			\glt	`While sitting he'll surely even plan the messengers beforehand.'
						\txrf{130913-1, 2.43} {\emb{130913-1-02-43.mp3}{\spk{}}{\apl}}}\label{ex:130913-1, 2.43 ch:SynMet}
	\ex{\glll	ma hai mi-soup \tbr{m}-\tbr{ain} =siin ees reʔ neno nana msaʔ.\\
						ma hai mi-sopu m-ani =sini esa reʔ neno nana msaʔ\\
						and {\hai} \mi-finish{\M} \m-direct{\M} ={\siin} {\esc} {\req} day {\naan} even\\
			\glt	`We even finished them straight-away (it was) on that day.'
						\txrf{130921-1, 0.59} {\emb{130921-1-00-59.mp3}{\spk{}}{\apl}}}\label{ex:130921-1, 0.59}
\end{exe}

Another verb which occurs almost exclusively
as the final member of an SVC in my corpus (72 attestations) is \ve{\rt ana}
which converts activities into accomplishments
with a focus on the resulting state of the accomplishment.
(This insight comes from the analysis of the equivalent Kupang Malay
verb \it{ame} `take' discussed in \citealt[349f]{jagr11}.)
In addition to the aspectual function, it sometimes indicates
the object of the SVC has on-going discourse relevance.
It is glossed {\ana} `resultative'.
Two examples are given in \qf{ex:130823-2, 0.24} and \qf{ex:130825-6, 18.34} below.

\begin{exe}
	\ex{\glll	siin n-seen \tbr{n}-\tbr{ana} ʔreanʔ=ees.\\
						sini n-sena n-ana ʔrenoʔ=esa\\
						{\siin} {\n}-plant{\M} {\n}-{\ana\Uc} lemon={\es}\\
			\glt	`They (had) planted a lemon tree.'
						\txrf{130823-2, 0.24} {\emb{130823-2-00-24.mp3}{\spk{}}{\apl}}}\label{ex:130823-2, 0.24}
	\ex{\glll	n-bain he au u-taan \tbr{ʔ}-\tbr{aan} =koo raas\j=ees.\\
						n-bani he au u-tana ʔ-ana =koo rasi=esa.\\
						{\n}-let{\M} {\he} {\au} {\qu}-ask{\M} {\q}-{\ana\M} ={\koo} matter={\es}\\
			\glt	`Let me ask you about something.' \txrf{130825-6, 18.34} {\emb{130825-6-18-34.mp3}{\spk{}}{\apl}}}\label{ex:130825-6, 18.34}
\end{exe}

Example \qf{ex:130823-2, 0.24} is the first event of its story,
with the rest of the story revolving around what
happens because of this particular lemon tree.
(The full version of this text is given in \srf{sec:DisStrAma}.)
Similarly, in example \qf{ex:130825-6, 18.34} the speaker
interrupts the main storyteller to have him change topic.
The act of asking is irrelevant, the speaker being interested
in its desired result: the contents of the new topic.

The verb \ve{\rt rari} `finish' can occur as an independent verb.
It also frequently occurs as the second member of an SVC with a completive meaning.
The difference between \ve{{\rt}ana} and \ve{{\rt}rari} in SVCs
lies in the part of the event which each verb emphasises.
With \ve{\rt ana}, the focus is on the resulting state of the event,
while with \ve{\rt rari} the focus is on the event itself.
Two examples of \ve{\rt rari} as the second member of an SVC
are given in \qf{ex:130825-6, 0.36} and \qf{ex:130825-7, 2.29} below.

\begin{exe}
	\ex{\glll	saap au ʔ-soiʔ \tbr{u}-\tbr{rair}.\\
						saap au ʔ-soʔi u-rari\\
						because {\au} {\q}-count{\M} {\qu}-finish{\M}\\
			\glt	`Because I'd finished counting.' \txrf{130825-6, 0.36} {\emb{130825-6-00-36.mp3}{\spk{}}{\apl}}}\label{ex:130825-6, 0.36}
	\ex{\glll	haeʔ, a-skau-t=aan, a|m-bukae \tbr{m}-\tbr{raar\j}=een?\\
						haeʔ a-skau-t=ana, {\a}m-bukae m-rari=ena\\
						hey {\at}-invite-{\at}={\aan} {\a\m}-eat {\m}-finish{\Mv}={\een}\\
			\glt	`Hey inviter/host, have you eaten?'
						\txrf{130825-7, 2.29} {\emb{130825-7-02-29.mp3}{\spk{}}{\apl}}}\label{ex:130825-7, 2.29}
\end{exe}

The verb \ve{{\rt}Vma} `come' occurs as an independent verb
as well as the second member of an SVC
indicating action oriented toward the speaker.\footnote{
		The verb \ve{{\rt}Vma} `come' has an irregular conjugation,
		discussed in \srf{sec:VerAgrPre} on \prf{tab:ConCome}.)}
Two examples of this verb as the second member of an SVC
indicating speaker oriented action are given in
\qf{ex:130907-3, 7.36} and \qf{ex:130821-1, 5.00} below.


\begin{exe}
	\ex{\glll	{oka =t} m-ʔeer \tbr{uma} m-bi ia.\\
						{okeʔ =te} m-ʔere uma m-bi ia\\
						{after.that} {\m}-look.intently{\M} {\uma\Uc} {\m}-{\bi} {\ia}\\
			\glt	`After that, keep looking this way.'
						\txrf{130907-3, 7.36} {\emb{130907-3-07-36.mp3}{\spk{}}{\apl}}}\label{ex:130907-3, 7.36}
	\ex{\glll	tuaf=esa n-fain \tbr{neem}, kaan-n=ee naiʔ Tuʔas.\\
						tuaf=esa n-fani nema kana-n=ee naiʔ Tuʔas\\
						person={\es} {\n}-return{\M} {\nema\M} name-{\N}={\ee} {\naiq} Tu{\Q}as\\
			\glt	`One person came back (here), his name was Tu{\Q}as.'
						\txrf{130821-1, 5.00} {\emb{130821-1-05-00.mp3}{\spk{}}{\apl}}}\label{ex:130821-1, 5.00}
\end{exe}

\subsection{Phonological restrictions on M\=/forms in SVCs}\label{sec:PhoResMfrSVC}
In both the nominal phrase and the verb phrase
M\=/forms mark the presence of a dependent modifier.
However, in the nominal phrase all heads occur in the M\=/form
while in the verb phrase only vowel-final verbs occur in the M\=/form,
and then only when the following verb begins with a single consonant.

When a consonant-final verb occurs as the first member
of an SVC it occurs in the U\=/form.
Such phonologically predictable U\=/forms are glossed {\Uc},
and are discussed in more detail in \srf{sec:ConFinVer}.
Two examples of consonant-final verbs as non-final within an SVC
are given in \qf{ex:130825-7, 3.10} and \qf{ex:130821-1, 7.18} below.

\begin{exe}
	\ex{\glll	au msaʔ =at au ʔ-poi ʔ-po\tbr{riʔ} ʔ-aan oa.\\
						au msaʔ =te au ʔ-poi ʔ-poriʔ ʔ-ana oa \\
						{\au} also ={\te} {\he} \q-exit \q-throw{\tbrUc} \q-{\ana\M} water\\
			\glt	`Me too, I'll go out to relieve myself (throw water).'
						\txrf{130825-7, 3.10} {\emb{130825-7-03-10.mp3}{\spk{}}{\apl}}}\label{ex:130825-7, 3.10}
	\ex{\glll	iin na-ho\tbr{niʔ} n-ain riʔanaʔ nua\\
						ini na-honiʔ n-ani riʔanaʔ nua \\
						{\iin} \na-birth{\tbrUc} \n-before{\M} child two\\
			\glt	`She first gave birth to two children.'
						\txrf{130821-1, 7.18} {\emb{130821-1-07-18.mp3}{\spk{}}{\apl}}}\label{ex:130821-1, 7.18}
\end{exe}

Similarly, when a vowel-final verb occurs before a consonant cluster
it usually occurs in the U\=/form.
Such verbal U\=/forms are mostly phonologically predictable
and are also glossed {\Uc} (\srf{sec:VerBefCC}).
Two examples of an SVC in which the second member begins with a cluster
are given in \qf{ex:130921-1, 0.43} and \qf{ex:130825-7, 2.24}.

\begin{exe}
	\ex{\glll	hai mi-so\tbr{pu} m-rair Roma, ees nean haa-ʔ=ii.\\
						hai mi-sopu m-rari Roma esa neno haa-ʔ=ii\\
						{\hai} \mi-complete{\tbrUc} \m-finish{\M} Roman {\esc} day four-{\qnum}={\ii}\\
			\glt	`We'd completed (reading) Romans on Thursday.'
						\txrf{130921-1, 0.43} {\emb{130921-1-00-43.mp3}{\spk{}}{\apl}}}\label{ex:130921-1, 0.43}
	\ex{\glll	sa-- n-ak, he m-sa\tbr{nu} m-fain he mi-ah\\
						{} \n-ak he m-sanu m-fani he mi-ah\\
						{} \n-say {\he} \m-descend{\tbrUc} \m-back{\M} {\he} \mi-eat\\
			\glt	\lh{sa{\textendash}}`he thought we would go back down to eat.'
						\txrf{130825-7, 2.24} {\emb{130825-7-02-24.mp3}{\spk{}}{\apl}}}\label{ex:130825-7, 2.24}
\end{exe}

A vowel-final verb can occur in the M\=/form
before a verb with an initial consonant cluster.
This is the minority pattern in my corpus with 13 attestations 
compared with 198 attestations of a vowel-final U\=/form in the same environment.
One example is given in \qf{ex:130913-1, 0.57 ch:SynMet} below,
which is immediately followed by another
speaker who repeats the same SVC, though with an initial U\=/form.

\begin{exe}
	\ex{A man who's already made preparations for his funeral:
			\txrf{130913-1} {\emb{130913-1-00-57-00-59.mp3}{\spk{}}{\apl}}}\label{ex:130913-1, 0.57-0.59 ch:SynMet}
		\begin{xlist}
			\ex[α:]{\glll	m-ak iin n-ha\tbr{in} \tbr{n}-\tbr{m}ees?\\
										m-ak ini n-hani n-mese\\
										\m-say {\iin} \n-dig{\tbrM} \n-alone{\M}\\
							\glt	`Do you think he dug it alone?' \txrf{0.57}}\label{ex:130913-1, 0.57 ch:SynMet}
			\ex[β:]{\glll	iin ofa n-ha\tbr{ni} \tbr{n}-\tbr{m}ees.\\
										ini ofa n-hani n-mese\\
										{\iin} sure \n-dig{\tbrUc} \n-alone{\M}\\
							\glt	`He must've dug it alone.' \txrf{0.59}}\label{ex:130913-1, 0.59 ch:SynMet}
		\end{xlist}
\end{exe}

When a consonant-final verb occurs before a consonant cluster
either epenthesis takes place, as in \qf{ex2:130913-1, 2.30} below,
or the cluster of three consonants is not resolved,
as in \qf{ex:120923-1, 6.59} below.

\begin{exe}
	\ex{\glll	t-pe{\tl}pea mes \sf{baptua} Banus iin na-ba\tbr{rab} \tbr{a}|\tbr{n}-\tbr{r}air\\
						t-pe{\tl}peo mes \sf{baptua} Banus ini na-barab {\a}n-rari\\
						{\t}-{\prd}talk but old.father Banus {\iin} {\na}-prepare{\tbrUc} {\a\n}-finish{\M}\\
			\glt	`We're talking about it, but father Banus is prepared.'
						\txrf{130913-1, 2.30} {\emb{130913-1-02-30.mp3}{\spk{}}{\apl}}}\label{ex2:130913-1, 2.30}
	\ex{\glll	neem he t-ʔo\tbr{nen} \tbr{t}-\tbr{p}asat t-aan=ee.\\
						nema he t-ʔonen t-pasat t-ana=ee\\
						{\nema\M} {\he} {\tg}-pray{\tbrUc} {\tg}-whack.away{\Uc} {\t}-{\ana\Mv}={\eeV}\\
			\glt	`He comes to have it prayed away.'
						\txrf{120923-1, 6.59} {\emb{120923-1-06-59.mp3}{\spk{}}{\apl}}}\label{ex:120923-1, 6.59}
\end{exe}

Consonant-final verbs always occur in the U\=/form
when they are a member of an SVC.
This phonotactic restriction is also found with
unassimilated consonant-final loan nominals
in attributive phrases (\srf{sec:LoaNou}).
Similarly, when a non-final verb in an SVC is followed by a consonant cluster,
it usually occurs in the U\=/form.
This behaviour is different from (native) nominals followed
by an attributive modifier in which M\=/forms are obligatory
no matter the phonotactic shape of the nominal and modifier.

To partly account for this fact we can
posit that different word classes are sensitive to different phonotactic constraints.
Within the nominal phrase preservation of a final consonant
is less important than marking the presence of an attributive modifier.
Within the verb phrase preservation of a final consonant is more
important than marking a following modifier.

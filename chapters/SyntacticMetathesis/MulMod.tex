\subsection{Multiple modifiers}\label{sec:MulMod}
It is possible for a nominal phrase to contain multiple attributive modifiers.
This can occur in two ways.
Firstly, the head nominal can be modified by two modifiers, as shown in \qf{tr:MulMod} below,
or the attributive modifier can itself consist of a modified nominal, as shown in \qf{tr:ModModN} below.
The syntactic head(s) which occur in the M-from are indicated by a box.
For each kind of structure both the first and second nominals occur in the M\=/form as expected.

\begin{multicols}{2}
	\begin{exe}
		\ex{\begin{forest} %where n children=0{tier=word}{}
			[NP,[\br{N},[\br{N},label={[label distance=8mm]305:\sub{\it{mod.}}},
			[\br{N},[N, tikz={\node [draw,dashed,inner sep=2,fit to=tree,]{};}]]
			[N,tikz={\node [draw,dashed,inner sep=2,fit to=tree,]{};}]]
			[N,label={[label distance=-2mm]358:\sub{\it{mod.}}},]]]
		\end{forest}}\label{tr:MulMod}
		\ex{\begin{forest} %where n children=0{tier=word}{}
			[NP,[\br{N},[\br{N},[N, tikz={\node [draw,dashed,inner sep=2,fit to=tree,]{};}]]
			[\br{N},label={[label distance=-2mm]358:\sub{\it{mod.}}},
			[\br{N},[N, tikz={\node [draw,dashed,inner sep=2,fit to=tree,]{};}]]
			[N,label={[label distance=-2mm]358:\sub{\it{mod.}}},]]]]
		\end{forest}}\label{tr:ModModN}
	\end{exe}
\end{multicols}

Examples of nominals followed by multiple modifiers
are given in \trf{tab:NouMulAttMod}.
Each of these nominals has the structure [[[N\sub{1}]N\sub{2}]N\sub{3}]
with an attributive phrase modified by a third nominal.
This structure corresponds to the tree given in \qf{tr:MulMod} above.
Of these, the first two have at least partially compositional meanings
while the third has a lexicalised meaning.

\begin{table}[h]
	\caption{Nominals with multiple attributive modifiers: [[[N\sub{1}]N\sub{2}]N\sub{3}]}\label{tab:NouMulAttMod}
	\centering
		\begin{tabular}{l@{ }cl@{ }cl@{ }cll}\lsptoprule
			{[[[}N\sub{1}{]}	&	&N\sub{2}{]}			&	&N\sub{3}{]}&		&Phrase											& \\ \midrule
			\ve{u\tbr{tan}}		&+&\ve{kau\tbr{t}}	&+&\ve{sufaʔ}	&\ra&\ve{u\tbr{ut} kau sufaʔ}		&`papaya blossom \\
			vegetable					&	&papaya						&	&blossom		&		&														&\hp{`}as a vegetable'\\
			\ve{ʔbi\tbr{bi}}	&+&\ve{ka\tbr{se}}	&+&\ve{anaʔ}	&\ra&\ve{ʔbi\tbr{ib}{\gap}ka\tbr{es} anaʔ}	&`lamb' \\
			goat							&	&foreign					&	&baby				&		&														&\\
			\ve{ko\tbr{ro}}		&+&\ve{kae}					&+&\ve{mutiʔ}	&\ra&\ve{ko\tbr{or}{\gap}kae mutiʔ}	&`Yellow-crested \\
			bird							&	&cry							&	&white			&		&														&\hp{`}Cockatoo'\\
			\lspbottomrule
		\end{tabular}
\end{table}

A number of nominal phrases with the structure [N\sub{1}[[N\sub{2}]N\sub{3}]]
are given in \trf{tab:NouMulAttMod2} below.
In such phrases the second two nominals form a phrase which modifies the first nominal,
thus corresponding to tree \qf{tr:ModModN} above.
All of the nominal phrases in \trf{tab:NouMulAttMod2} have a lexicalised meaning.

\begin{table}[ht]
	\caption{Nominals with multiple attributive modifiers: [N\sub{1}[[N\sub{2}]N\sub{3}]]}\label{tab:NouMulAttMod2}
	\centering
		\begin{tabular}{l@{ }cl@{ }clcll}\lsptoprule
			{[}N\sub{1}				&	&{[[}N\sub{2}{]}	&	&N\sub{3}{]]}	&&& \\ \midrule
			\ve{o\tbr{teʔ}}		&+&\ve{bi\j ae}			&+&\ve{suna}	&\ra&\ve{o\tbr{et}{\gap}bi\j ae suna}&`pickaxe' \\
			hoe								&	&cow							&	&horn				&&&\\
			\ve{u\tbr{nus}}		&+&\ve{fua\tbr{ʔ}}	&+&\ve{koʔu}	&\ra&\ve{u\tbr{un} fua koʔu}		&`Holland chilli' \\
			chilli						&	&fruit						&	&big				&&&\\
			\ve{si\tbr{mah}}	&+&\ve{tai\tbr{ʔ}}	&+&\ve{boko}	&\ra&\ve{si\tbr{im}{\gap}tai boko}		&`k.o. large green\\
			katydid						&	&belly						&	&curved			&&&\hp{`}katydid'\\
			\ve{u\tbr{nus}}		&+&\ve{fua\tbr{ʔ}}	&+&\ve{mnutuʔ}&\ra&\ve{u\tbr{un} fua mnutuʔ}	&`bird's eye chilli' \\
			chilli						&	&fruit						&	&fine				&&&\\
			\ve{kaun\tbr{aʔ}}	&+&\ve{fee}					&+&\ve{mnasiʔ}&\ra&\ve{kaun{\gap}fee{\gap}mnasiʔ}	&`woodlouse' \\
			creature					&	&wife							&	&old				&&&\\
			\lspbottomrule
		\end{tabular}
\end{table}

The structure of two nominal phrases with multiple modifiers
are given in \qf{tr:QbiibKaesAnaq} and \qf{tr:OetBijaeSuna} below
to illustrate their differing structures.
The structure of \ve{ʔbiib kaes anaʔ} `lamb'
is given in \qf{tr:QbiibKaesAnaq} and that
of \ve{oet biʤae suna} `pickaxe' in \qf{tr:OetBijaeSuna}.

\begin{multicols}{2}
	\begin{exe}
		\ex{\begin{forest} where n children=0{tier=word}{}
			[NP,[\br{N},[\br{N},[\br{N},[N,[\ve{ʔbiib}\\goat{\M}]]][N,[\ve{kaes}\\foreign{\M}]]][N,[\ve{anaʔ}\\baby]]]]
		\end{forest}}\label{tr:QbiibKaesAnaq}
		\ex{\begin{forest} where n children=0{tier=word}{}
			[NP,[\br{N},[\br{N},[N,[\ve{oet}\\hoe{\M}]]][\br{N},[\br{N},[N,[\ve{bi\j ae}\\cow]]][N,[\ve{suna}\\horn]]]]]
		\end{forest}}\label{tr:OetBijaeSuna}
	\end{exe}
\end{multicols}

The largest attributive nominal phrase in my dictionary is
\ve{anah} `child' + \ve{mone} `male' + \mbox{\ve{a-heti-t}} `{\at}-stop-{\at}
+ \ve{susu} `milk' {\ra} \ve{aan moen aheit susu} `youngest son',
literally `male child (who) stopped the milk'.
This nominal phrase has the structure [[[\ve{aan}] [\ve{moen}]] [[\ve{aheit}] [\ve{susu}]]],
with the second nominal modifying the first,
the fourth modifying the third, and the final attributive
phrase modifying the first attributive phrase.

\newpage
As with attributive phrases consisting of two nominals,
the use of multiple modifiers is highly productive in Amarasi.
Two textual examples of the structure [[[N\sub{1}]N\sub{2}]N\sub{3}],
with a single nominal modified by multiple modifiers
are given in \qf{ex:130907-3, 12.15} and \qf{ex:120715-2, 0.25} below.

\begin{exe}
	\ex{\glll	au ʔ-sao neʔ riʔaanʔ=ee, \tbr{aan} \tbr{feat} \tbr{koʔu}.\\
						au ʔ-sao neʔ riʔanʔa=ee anah feto koʔu\\
						{\au} {\q}-marry {\reqt} child{\Mv}={\ee} child{\tbrM} female{\tbrM} big{\U}\\
			\glt	`I married the daughter, the eldest daughter.'
						\txrf{130907-3, 12.15} {\emb{130907-3-12-15.mp3}{\spk{}}{\apl}}}\label{ex:130907-3, 12.15}
	\ex{\glll	au he u-toon n-ok \tbr{meup} \tbr{reen} \tbr{abit}, n-bi Nekmeseʔ.\\
						au he u-tona n-oka mepu rene abit n-bi Nekmeseʔ\\
						{\au} {\he} {\qu}-tell {\n}-{\ok} work{\tbrM} field{\tbrM} inhabitant  {\n}-{\bi} Nekmese{\Q}\\
			\glt	`I want to talk about how an inhabitant of Nekmese{\Q} farms.' (\emph{lit.} `inhabitant field work')
						\txrf{120715-2, 0.25} {\emb{120715-2-00-25.mp3}{\spk{}}{\apl}}}\label{ex:120715-2, 0.25}
\end{exe}

Two textual examples of the structure [N\sub{1}[[N\sub{2}]N\sub{3}]],
where a nominal modified by another nominal in turn modifies another nominal,
are given in \qf{ex:120923-1, 0.47} and \qf{ex:120923-2, 6.47}.

\begin{exe}
	\ex{\glll	n-nakaʔfatu=n n-bi reʔ \tbr{rais} \tbr{moaʔ} \tbr{reuʔf}=ii.\\
						n-nakaʔfatu=n n-bi reʔ rasi moʔe reʔuf=ii\\
						n-stubborn={\einV} {\n-\bi} {\reqt} matter{\M} deed{\M} bad{\Mv}={\ii}\\
			\glt	`They're stubborn in the matter of this evil practice.'
						\txrf{120923-1, 0.47} {\emb{120923-1-00-47.mp3}{\spk{}}{\apl}}}\label{ex:120923-1, 0.47}
		\ex{\glll	ta-tenab on reʔ hiit \tbr{atoin} \tbr{a}-\tbr{moaʔ} \tbr{reuʔf}=ii =te, \\
							ta-tenab on reʔ hiti atoni a-moʔe-t reʔuf=ii =te \\
							{\tg}-think like {\reqt} {\hiit} man{\tbrM} {\at}-do{\tbrM} bad{\Mv}={\ii} ={\te} \\
				\glt	`When you think like (this) you're a person who is an evildoer.'
							\txrf{120923-2, 6.47} {\emb{120923-2-06-47.mp3}{\spk{}}{\apl}}}\label{ex:120923-2, 6.47}
\end{exe}

The structure of the nominal phrase %\ve{mepu} `work' + \ve{rene} `field' + \ve{a-bi-t} `inhabitant' {\ra}
\ve{meup reen abit} in \qf{ex:120715-2, 0.25} above is given in \qf{tr:MeupReenAbit}.
Similarly, the structure of the phrase %containing \ve{atoni} `man' combined with \ve{a-moʔe-t} `doer' + \ve{reʔuf} `bad' {\ra}
\ve{atoin amoaʔ reuʔf=ii} in \qf{ex:120923-2, 6.47} is given in \qf{tr:AtoinAmoeqReuqf}.
(Metathesis of the final nominal in this phrase is induced by
the following enclitic, see Chapter \ref{ch:PhoMet})

\begin{multicols}{2}
	\begin{exe}
		\ex{\begin{forest} for tree={s sep=1mm, where n children=0{tier=word}{},}
			[NP,[\br{N},[\br{N},[\br{N},[N,[\ve{meup}\\work{\M}]]][N,[\ve{reen}\\field{\M}]]][N,[\ve{abit}\\inhabitant]]]]
		\end{forest}}\label{tr:MeupReenAbit}
		\ex{\begin{forest} where n children=0{tier=word}{}
			[NP,[\br{N},[\br{N},[N,[\ve{atoin}\\man{\M}]]][\br{N},[\br{N},[N,[\ve{amoaʔ}\\doer{\M}]]][N,[\ve{reuʔf}\\bad{\Mv}]]]]]
		\end{forest}}\label{tr:AtoinAmoeqReuqf}
	\end{exe}
\end{multicols}

The longest string of attributive nominals I have so far encountered 
occurs in the Amarasi Bible translation.
This is in Genesis 11:5 in the description of the tower of Babel.
This passage is given in \qf{ex:Genesis 11:5} below,
with the structure of the nominal phrase given in \qf{tr:Genesis 11:5}.
In this example six nominals occur in a single nominal phrase.
%\footnote{
%		The nominal \ve{kota} means `city' or `fort'.
%		This form is related to Malay \it{kota} `city, fort' ultimately,
%		from Sanskrit \it{ko\textsubdot{t}a} /koːta/ `fortress, stronghold'.}

\begin{exe}
	\ex{\glll	{onai =m} Uisneno n-saun neem ma n-noon kota =ma 
						\hspace{5mm} \tbr{koot} \tbr{ma}-\tbr{ʔuim} \tbr{faut} \tbr{ma}-\tbr{ktuta}
						\tbr{mnaun} \tbr{a}-\tbr{ʔrata}-\tbr{s} reʔ mansian=ein naan na-feen-ʔ=ee.\\
						{onai =ma} Uisneno n-sanu nema ma n-noon kota =ma
						{} kota ma-ʔumi-ʔ fatu ma-ktutaʔ mnanuʔ a-ʔrata-s reʔ mansian=eni naan na-fena-ʔ=ee.\\
						and.so God \n-descend {\nema} and \n-walk.around city{\U} =and
						{} city{\tbrM} {\ma}-house{\tbrM} stone{\tbrM}
						{\ma}-stack{\tbrMc} long{\tbrM} {\at}-elevate{\U}-{\at} {\req} human={\ein} {\naan} {\nat}-rise-\qV={\eeV}\\
			\glt	`And so God came down and walked around (in) the city and the high, tall residential city (made
							of) stacked stones which those humans were building.' \txrf{Genesis 11:5}}\label{ex:Genesis 11:5}
		\ex{\begin{forest} where n children=0{tier=word}{}
			[NP,[\br{N},[\br{N},[\br{N},[\br{N}[\br{N},[N,[\ve{koot}\\city{\M}]]][N,[\ve{maʔuim}\\residential{\M}]]]
			[\br{N},[\br{N},[N,[\ve{faut}\\stlone{\M}]]][N,[\ve{maktuta}\\stlacked{\Mc}]]]]
			[N,[\ve{mnaun}\\long{\M}]]][N,[\ve{aʔratas}\\elevated{\U}]]]]
		\end{forest}}\label{tr:Genesis 11:5}
\end{exe}

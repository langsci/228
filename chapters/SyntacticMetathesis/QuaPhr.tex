\subsection{Quantifier phrase}\label{sec:QuaPhr}
Amarasi has two kinds of quantifiers: those which occur before the nominal phrase
and those which occur after the nominal phrase.
The post-nominal quantifiers are \ve{okeʔ} `all' and \ve{fauk} `several, how many?'.
The pre-nominal quantifiers are \ve{baʔuk} `many, how many?' and \ve{areʔ} `every'.
None of these quantifiers trigger M\=/forms.\footnote{
		Despit its semantics \ve{mfaun} `many, much' is a nominal
		and triggers M\=/forms as expected when used attributively.}

\subsubsection{Post-nominal quantifiers}
Examples of post nominal \ve{fauk} `several' and \ve{okeʔ} `all'
or their reduplicated variants are given in
\qf{ex:130920-1, 4.53-4.56}--\qf{ex:Genesis 8:17} below.
The reduplicated variants of these quantifiers are extremely common.

\begin{exe}
	\ex{\begin{xlist}
		\ex{\gll	{onai =m} hai mi-rair su\tbr{rat} \tbr{fak}{\tl}\tbr{fauk}=een aiʔ nai--\\
							and.so {\hai} {\mi}-finish paper{\tbrU} {\prd}several={\een} or\\
				\glt	`So we've now finished several books, or'
							\txrf{130920-1, 4.53} {\emb{130920-1-04-53-04-56.mp3}{\spk{}}{\apl}}}\label{ex:130920-1, 4.53}
		\ex{\gll	ʔna\tbr{kaʔ} \tbr{fak}{\tl}\tbr{fauk}=een.\\
							chapter{\tbrU} {\prd}several={\een}\\
				\glt	`several chapters' \txrf{4.56}}
	\end{xlist}}\label{ex:130920-1, 4.53-4.56}
	\ex{\gll	oras hai m-took m-ok siin of ne\tbr{no} \tbr{fauk} =ate,\\
						time {\hai} {\m}-sit{\M} {\m}-{\ok} {\siin} later day{\tbrU} several ={\te}\\
			\glt	`When we had stayed with them for several days,' \txrf{Acts 21:10}}\label{ex:Acts 21:10}
	\ex{\gll	bi Ripka na-honiʔ riʔaan koen. mo\tbr{ne}=n okeʔ.\\
						{\BI} Rebecca \na-birth child twin male{\tbrU}={\ein} all\\
			\glt	`Rebecca gave birth to twins. (They were) both male.' \txrf{Genesis 25:24}}\label{ex:Genesis 25:24}
	\ex{\gll	mi-sanut muiʔt=ein naan \tbr{ok}{\tl}\tbr{okeʔ}! \\
						\mi-go.down animal={\ein} {\naan} {\prd}all \\
			\glt	`Put all those animals down.' \txrf{Genesis 8:17}}\label{ex:Genesis 8:17}
\end{exe}

Most instances of \ve{okeʔ} in my corpus are of the phrase \ve{okeʔ =te} `all ={\te}'
which has the conventionalised meaning `after that'.
The form \ve{okeʔ} also frequently occurs as an adverbial
with the meaning `completely, finished'.

Neither of these quantifiers can float,
instead they must follow the nominal phrase they modify.
This is shown in \qf{ex:elicitation 15/03/2016 p.47} below,
in which the phrase-final quantifier in \qf{ex:elicitation 15/03/2016 p.47 b}
and pre-nominal quantifier in \qf{ex:elicitation 15/03/2016 p.47 c}
are both ungrammatical.

\begin{exe}
	\ex{\begin{xlist}
		\ex[]{\glll	au ʔ-tea ʔ-rair kuan \tbr{fak}{\tl}\tbr{fauk}=een.\\
								au ʔ-tea ʔ-rari kuan fak{\tl}fauk=ena\\
							{\au} \q-arrive \q-finish village{\U} {\prd}several={\een}\\}\label{ex:elicitation 15/03/2016 p.47 a}
		\ex[*]{\gll	au ʔ-tea ʔ-rair \tbr{fak}{\tl}\tbr{fauk} kuan=een\\
								{\au} \q-arrive \q-finish {\prd}several village={\een}\\}\label{ex:elicitation 15/03/2016 p.47 b}
		\ex[*]{\gll	au ʔ-tea ʔ-rair kuan=een \tbr{fak}{\tl}\tbr{fauk}\\
								{\au} \q-arrive \q-finish village={\een} {\prd}several\\
				\glt	`I've already been to several villages.'
							\txrf{elicitation 15/03/16 p.47}}\label{ex:elicitation 15/03/2016 p.47 c}
	\end{xlist}}\label{ex:elicitation 15/03/2016 p.47}
\end{exe}

These post nominal modifiers rarely modify a nominal phrase
already modified by a demonstrative or determiner.
When they do, they occur after the demonstrative or determiner.
Two examples are given in \qf{ex:Acts 16:15} and \qf{ex:Genesis 43:23} below.

\begin{exe}
	\ex{\gll	rari =t hai mi-srain ain Lidia n-ok iin uum\j=ee naan-n=\tbr{ee} \tbr{okeʔ}.\\
						finish ={\te} {\hai} \mi-baptize mother Lidia {\n-\ok} {\iin} house={\ee} inside-{\N-\ee} all\\
			\glt	`Then we baptised Lidia with all her household.' \txrf{Acts 16:15}}\label{ex:Acts 16:15}
	\ex{\gll	neeŋgw=ees=ii, au ʔ-toup u-rair hii roit=ein \tbr{naan} \tbr{okeʔ}.\\
						day={\es=\ii} {\au} \q-receive \qu-finish {\hii} money={\ein} {\naan} all\\
			\glt	`That day I received all that money of yours.' \txrf{Genesis 43:23}}\label{ex:Genesis 43:23}
\end{exe}

The extended nominal in example \qf{ex:Genesis 43:23}
above attests every possible nominal modifier
with the exception of an attributive nominal.
Its structure is given in \qf{tr:Genesis 43:23} below.

\begin{exe}
	\ex{\begin{forest} where n children=0{tier=word}{}
		[QP,
			[DP,
				[NumP,
					[NP,
						%[PossP[NP,[\br{N},[N,[\ve{hii}\\{\hii}]]]]]
						[PossP[NP,[\ve{hii}\\{\hii},roof]]]
					[\br{N},[N,[\ve{roit}\\money]]]]
				[Num,[\ve{=ein}\\{\ein}]]]
			[D,[\ve{naan}\\{\naan}]]]
		[Q,[\ve{okeʔ}\\all]]]
	\end{forest}}\label{tr:Genesis 43:23}
\end{exe}

As with numerals and demonstratives, quantifiers can occur independent of a nominal phrase.
Two examples of \ve{fauk} are given in \qf{ex:130925-1, 3.47} and \qf{ex:130909-6, 0.52}.
Similarly, two examples of independent \ve{okeʔ} are given in \qf{ex:130909-6, 0.39} and \qf{ex:130825-6, 0.08},
though in these examples \ve{okeʔ} could be being used adverbially to mean `completely'.	

\begin{exe}
	\ex{\gll	of{\gap}oniʔ n-poirʔ=ee n-ak =am, ``hoo m-eik \tbr{fauk} ia.''\\
						maybe {\n}-throw={\eeV} {\n}-say =and \hp{``}{\hoo} {\m}-bring several {\ia}\\
			\glt	`Maybe he got rid of it saying: Take some of these.'
						\txrf{130925-1, 3.47} {\emb{130925-1-03-47.mp3}{\spk{}}{\apl}}}\label{ex:130925-1, 3.47}
	\ex{\gll toon=ees=ii, hoo m-seik \tbr{fauk}? \\
						year={\es}={\ii} {\hoo} {\m}-harvest.corn how.many \\
			\glt `How much corn did you harvest last year?'
						\txrf{130909-6, 0.52} {\emb{130909-6-00-52.mp3}{\spk{}}{\apl}}}\label{ex:130909-6, 0.52}
	\ex{\gll	m-ak, hai nua =kai m-taikobi =m hai m-maet \tbr{okeʔ}.\\
						{\m}-say {\hai} two ={\kai} {\m}-fall =and {\hai} {\m}-die all\\
			\glt	`So the two of us fell and we both/completely died.'
						\txrf{130909-6, 0.39} {\emb{130909-6-00-39.mp3}{\spk{}}{\apl}}}\label{ex:130909-6, 0.39}
	\ex{\gll	areʔ paah=ii n-heʔe=n =kau \tbr{okeʔ} =m, hii ka= mi-hine =f.\\
						every country={\ii} {\n}-deride={\einV} =and ={\kau} all {\hii} {\ka}= {\mi}-know ={\fa}\\
			\glt	`The whole world derided me. / completely derided me. You don't know (how it was).'
						\txrf{130825-6, 0.08} {\emb{130825-6-00-08.mp3}{\spk{}}{\apl}}}\label{ex:130825-6, 0.08}
\end{exe}

In example \qf{ex:130925-1, 3.47} the quantifier \ve{fauk} occurs
before the demonstrative \ve{ia}.
In this instance the quantifier is the head of a nominal phrase.
There is one other example of a quantifier within a nominal phrase in my data,
this example is given in \qf{ex:130909-6, 1.26} below,
in which the quantifier occurs as an attributive modifier
with the head nominal taking the M\=/form as expected.

\begin{exe}
	\ex{\glll	ne\tbr{an} \tbr{fauk}=ii na-ʔuur ?\\
						neno fauk=ii na-ʔura\\
						day{\tbrM} how.many={\ii} \na-rain\\
			\glt	`Which day did it rain?'
						\txrf{130909-6, 1.26} {\emb{130909-6-01-26.mp3}{\spk{}}{\apl}}}\label{ex:130909-6, 1.26}
\end{exe}

In \qf{ex:130909-6, 1.26} the quantifier is an attributive modifier
``replacing'' the ordinal numeral which would occur here as the name of a day (\srf{sec:OrdNum}).
The phrase \ve{nean fauk} `which day' in \qf{ex:130909-6, 1.26}
can be compared with the phrase \ve{neno fauk} `several/how many days'
in examples such as \qf{ex:Acts 21:10}.
Syntactic tress showing the structure of each of these phrases
are given in \qf{tr:NeanFauk} and \qf{tr:NenoFauk} respectively.

\begin{multicols}{2}
	\begin{exe}
		\ex{\begin{forest} where n children=0{tier=word}{}
			[NP,[\br{N},[\br{N},[N,[\ve{nean}\\day{\M},label={below:{\hspace{17mm}`which day'}},]]]
			[N,[\ve{fauk}\\several]]]]
		\end{forest}}\label{tr:NeanFauk}
	\end{exe}
	\begin{exe}
		\ex{\begin{forest} where n children=0{tier=word}{}
			[QP,[NP,[\br{N},[N,[\ve{neno}\\day{\U},label={below:{\hspace{17mm}`several days'}},]]]]
			[Q,[\ve{fauk}\\several]]]
		\end{forest}}\label{tr:NenoFauk}
	\end{exe}
\end{multicols}

%[NP,label={[label distance=17mm]below:\it{\footnotesize \hspace{4mm}metathesis domain}},[,phantom]]]
%[NP,label={below:{`several days'}},]

\subsubsection{Pre-nominal quantifiers}
The quantifiers \ve{baʔuk} `many, how many?'
and \ve{areʔ} `all, every' occur before the nominal phrase.
%The semantic difference, if any, between \ve{fauk} and \ve{baʔuk}
%is currently unclear as I have only two instances of \ve{baʔuk} in my corpus.
Post-nominal \ve{okeʔ} `all' focusses on the quantified unit as a complete whole,
while \ve{areʔ} focusses on the quantified unit as a collection of individuals.
Examples of \ve{baʔuk} `many, how many?' and \ve{areʔ} `all, every'
are given in \qf{ex:130911-2, 0.59}--\qf{ex:130902-1, 0.51}.
The structure of the quantified nominal in \qf{ex:130902-1, 0.51} is given in \qf{tr:AreqSaksiiMahonit}.

\begin{exe}
	\ex{\gll	aina! iin na-sae-b \tbr{baʔ}{\tl}\tbr{baʔuk} atoinʔ=ein?\\
						mother {\iin} \na-rise-{\b} {\prd}how.many man={\ein}\\
			\glt	`Oh my! How many people was it carrying?'
						\txrf{130911-2, 0.59} {\emb{130911-2-00-59.mp3}{\spk{}}{\apl}}}\label{ex:130911-2, 0.59}
	\ex{\gll	nema =t, na-ha n-rair \tbr{areʔ} mnaaht=ii =m ka= na-ʔoi\\
						{\nema} ={\te} {\na}-eat \n-finish every food={\ii} =and {\ka}= {\na}-leave\\
			\glt	`(They) came and ate all the food and didn't leave.'
						\txrf{130906-1, 5.19} {\emb{130906-1-05-19.mp3}{\spk{}}{\apl}}}\label{ex:130906-1, 5.19}
	\ex{\gll	hai mi-rari =te, hai m--, m-fee mainuan n-eu a-naʔapreent =ama, \tbr{areʔ} saksii, mahonit he n-fee, ahh, faineka-t\\
						{\hai} \mi-finish ={\te} {\hai} {} \m-give opportunity \n-{\eu} {\at}-official =and every witness clan.elder {\he} \n-give {} advise-{\at}\\
			\glt	`We gave an opportunity to the government officials and all the witnesses, the clan elders to give advice.'
						\txrf{130902-1, 0.51} {\emb{130902-1-00-51.mp3}{\spk{}}{\apl}}}\label{ex:130902-1, 0.51}
\end{exe}

\begin{exe}
	\ex{\begin{forest} where n children =0{tier=word}{}
		[QP,[Q,[\ve{areʔ}\\all]][NP,[\ve{saksii}\\{witness},roof]]]
	%	[QP,[Q,[\ve{areʔ}\\all]][NP,[\br{N},[N,[\ve{saksii}\\{witness},roof]]]]]
	\end{forest}}\label{tr:AreqSaksiiMahonit}
\end{exe}

The quantifier \ve{areʔ} `every' can co-occur with \ve{okeʔ} `all'.
Two examples from the Amarasi Bible translation
are given in \qf{ex:Hebrews 12:6} and \qf{ex:Revelation 20:13} below.

\begin{exe}
	\ex{\gll	batuur, \tbr{areʔ} tuaf=ein \tbr{ok}{\tl}\tbr{okeʔ} reʔ ina n-toup =sina n-{\j}ari=n iin aanh=ein,\\
						true every person={\ein} {\prd}all {\req} {\iin} {\n}-receive ={\siin} {\n}-become={\einV} {\iin} child={\ein}\\
			\glt	`Every person whom He accepts becomes His children.' \txrf{Hebrews 12:6}}\label{ex:Hebrews 12:6}
	\ex{\gll	rari =t \tbr{areʔ} a-maet-s=ein \tbr{ok}{\tl}\tbr{okeʔ}, siin ar=siin nema=n =ma n-baiseun ʔ-toko prenat naan.\\
						finish ={\te} every {\at}-die-{\at}={\ein} {\prd}all {\siin} all={\siin} \nema={\einV} =and \n-look.up {\qq}-sit govern {\naan} \\
			\glt	`Then all the dead people came and stood before that governing seat (i.e. throne).' \txrf{Revelation 20:13}}\label{ex:Revelation 20:13}
\end{exe}

When a pronoun is quantified, the usual strategy is for \ve{ar=} `all'
to precede the accusative form of the pronoun.
This \ve{ar=} is a phonologically reduced form of \ve{areʔ}.
Examples are given in \qf{ex:130821-1, 1.40}--\qf{ex:130821-1, 9.56} below.
Such quantified pronouns are also usually preceded by an agreeing pronoun,
as in \qf{ex:130821-1, 1.40}--\qf{ex:130821-1, 2.46},
but this preceding pronoun is optional, as seen in \qf{ex:130821-1, 9.56}.

\begin{exe}
	\ex{\gll	karu \tbr{hii} \tbr{ar}=\tbr{kii}, m-naaʔ \sf{liturgi} =te\\
						if {\hii} all={\kii} \m-hold liturgy ={\te}\\
			\glt	`If you're all holding a liturgy,'
								\txrf{130821-1, 1.40} {\emb{130821-1-01-40.mp3}{\spk{}}{\apl}}}\label{ex:130821-1, 1.40}
	\ex{\gll	\tbr{hiit} \tbr{ar}=\tbr{kiit} ta-hini t-toom.\\
						{\hiit} all={\kiit} {\ta}-know {\t}-clear\\
				\glt	`We all know (that) clearly.'
								\txrf{130821-1, 7.13}{\emb{130821-1-07-13.mp3}{\spk{}}{\apl}}}\label{ex:130821-1, 7.13}
	\ex{\gll	\sf{itu} \sf{jaŋ} \sf{kemudian} he-- au he u-reetʔ=ee \hspace{35mm} n-eu =\tbr{kiit} \tbr{ar}=\tbr{kiit}.\\
						that \tsc{rel} then {}  {\au} {\he} {\qu}-story={\eeV} {} \n-{\eu} ={\kiit} all={\kiit}\\
			\glt	`That is what I want to tell us all,'
							\txrf{130821-1, 2.46} {\emb{130821-1-02-46.mp3}{\spk{}}{\apl}}}\label{ex:130821-1, 2.46}
	\ex{\gll	\tbr{ar}=\tbr{kita} t-tae \sf{liturgi}, \tbr{ar}=\tbr{kita} t-sii, au u-skau \tbr{ar}=\tbr{kiit} t-fena t-haek.\\
						all{\kiit} {\t}-look.down liturgy all={\kiit} {\t}-sing {\au} {\qu}-invite all={\kiit} {\t}-rise \t-stand\\
			\glt	`We'll all look at the liturgy, we'll all sing, I invite us all to stand.'
						\txrf{130821-1, 9.56} {\emb{130821-1-09-56.mp3}{\spk{}}{\apl}}}\label{ex:130821-1, 9.56}
\end{exe}

I also have one example of this \ve{ar=} attached to a relativiser
and one example of it attached to a numeral.
For the sake of completeness these two examples
are given in \qf{ex:130825-8, 0.23} and \qf{ex:120715-4, 2.59} below.

\newpage
\begin{exe}
	\ex{\gll	ʔ-ait neʔ nehh \sf{persiapan} \sf{leŋkap} ehh \tbr{ar}=\tbr{neʔ} \sf{tampat} \sf{duduk}.\\
						\q-take {\req} {} preparations complete {} all={\reqt} place sit\\
			\glt	`I took that thing, umm, the preparations were complete, all those places to seat.'
						\txrf{130825-8, 0.23} {\emb{130825-8-00-23.mp3}{\spk{}}{\apl}}}\label{ex:130825-8, 0.23}
	\ex{\gll	\sf{oke,} of \tbr{ar}=\tbr{nua} saaʔ, aiʔ kaah?\\
						OK maybe all=two thing or {\kaah}\\
			\glt	`OK, maybe both those things (stories), right?'
						\txrf{120715-4, 2.59} {\emb{120715-4-02-59.mp3}{\spk{}}{\apl}}}\label{ex:120715-4, 2.59}
\end{exe}

In summary, quantifiers do not induce M\=/forms on the nominal
as they form the head of their own quantifier phrase.
This quantifier phrase is outside of the nominal phrase,
and can occur either before or after the nominal phrase.

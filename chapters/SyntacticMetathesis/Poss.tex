\section{Possession}\label{sec:Poss}
In Amarasi the possessor precedes the thing possessed,
with an optional possessive pronoun occurring between the two.
Possessive phrases do not induce M\=/forms on either the possessor or the possessed nominal.
I analyse the possessive phrase as occurring as the specifier of the nominal phrase,
as indicated in \qf{tr:PossP2} below.

\begin{exe}
	\ex{\begin{forest} %where n children=0{tier=word}{}
		[NP,
			[PossP, tikz={\node [draw,fit to=tree,]{};}[NP,][poss]]
			[\br{N},[N]]]
	\end{forest}}\label{tr:PossP2}
\end{exe}

A simple case of possession is given in \qf{ex:130921-1, 0.50} below,
with the syntactic structure of the nominal phrase given in \qf{tr:130921-1, 0.50}.

\begin{exe}
	\ex{\gll	\tbr{naiʔ} \tbr{Yohanis} \tbr{iin} \tbr{surat} reʔ, a-hunu-t\\
						{\naiq} John {\iin} paper{\U} {\req} {\at}-first{\at}\\
			\glt	`John's book which is the first.'
						\txrf{130921-1, 0.50} {\emb{130921-1-00-50.mp3}{\spk{}}{\apl}}}\label{ex:130921-1, 0.50}
	\ex{\begin{forest} where n children=0{tier=word}{} 
		[NP,
			[PossP,[NP,[\ve{naiʔ Yohanis}\\{Mr. John}, roof]][poss,[\ve{iin}\\{\iin}]]]
			[\br{N},[N,[\ve{surat}\\paper{\U}]]]]
	\end{forest}}\label{tr:130921-1, 0.50}
\end{exe}

After a brief discussion of the details of possession in Amarasi, including
determiner enclitics to mark the thing possessed (\srf{sec:PosDet})
and genitive suffixes (\srf{sec:GenSuf ch:SynMet}), I discuss the syntactic structure
of possession in more detail in \srf{sec:SynPoss}.

\subsection{Possessum determiners}\label{sec:PosDet}
When the thing possessed is not indicated by a full nominal phrase,
it is referenced by one of the enclitic determiners
\ve{=ii}, \ve{=ana/=aan}, \ve{=ee} or \ve{=aa} which
attach directly to the pronoun indexing the possessor.
(See \srf{sec:DetPhr} for the syntactic position of these determiners.)
When the possessum is plural, it is referenced by the enclitic
\ve{=n} on CV{\#} final pronouns and either \ve{=nu} or \ve{=ŋgwein} on VV{\#} final pronouns.\footnote{
		As discussed in \srf{sec:PluEnc} \ve{=ŋgwein} is analysable at least historically
		as a combination of \ve{=nu} and \ve{=ein} with regular insertion of /ɡw/.}
A plural possessum can further be referenced by an enclitic.

The ways in which a possessum which is not expressed by
a full noun phrase is encoded in Amarasi are summarised in \trf{tab:EncUnePos} below
with the determiner \ve{=aa} `{\aa}' where appropriate.
These constructions are one of the few in which the double vowel
sequence of the pronouns \ve{iin} `{\iin}', \ve{siin} `{\siin}', and \ve{hiit} `\tsc{1pl.incl}'
are stressed and thus realised as a full long vowel.
Two examples of singular possessums encoded with a determiner
are given in \qf{ex:130909-6, 2.54} and \qf{ex:130909-6, 3.23} below.

\begin{table}[h]
	\caption{Encoding of unexpressed possessums}\label{tab:EncUnePos}
	\begin{tabular}{llllll}\lsptoprule
				\mc{2}{l}{Possessor}			&\tsc{sg} Possessum 	&\mc{3}{l}{\tsc{pl} Possessum}	\\ \midrule
				\tsc{1sg}				&\ve{au}	&\ve{aagw=aa}		&\ve{au=nu}		&\ve{au=ŋgwein}		&\ve{au=ŋgw=aa}		\\
				\tsc{2sg}				&\ve{hoo}	&\ve{hoogw=aa}	&\ve{hoo=nu}	&\ve{hoo=ŋgwein}	&\ve{hoo=ŋgw=aa}	\\
				\tsc{3sg}				&\ve{iin}	&\ve{iin\j=aa}	&\ve{ini=n}		&									&\ve{iin=n=aa}		\\
				\tsc{1pl.excl}	&\ve{hai}	&\ve{haa\j=aa}	&\ve{hai=nu}	&\ve{hai=ŋgwein}	&\ve{hai=ŋgw=aa}	\\
				\tsc{1pl.incl}	&\ve{hiit}&\ve{hiit\j=aa}	&\ve{hiti=n}	&									&\ve{hiit=n=aa}		\\
				\tsc{2pl}				&\ve{hii}	&\ve{hii\j=aa}	&\ve{hii=nu}	&\ve{hii=ŋgwein}	&\ve{hii=ŋgw=aa}	\\
				\tsc{3pl}				&\ve{siin}&\ve{siin\j=aa}	&\ve{sini=n}	&									&\ve{siin=n=aa}		\\
		\lspbottomrule
	\end{tabular}
\end{table}

\begin{exe}
	\ex{\glll	bait hoogw=\tbr{ii} n-moni =t, bait hoo on neʔ au.\\
						baiti hoo=ii n-moni =te baiti hoo on neʔ au \\
						actual {\hoo\Mv}=\tbr{\ii} \n-live ={\te} actual {\hoo} like {\reqt} {\au} \\
			\glt	`Actually, while yours is alive it's like me.'
						\txrf{130909-6, 2.54} {\emb{130909-6-02-54.mp3}{\spk{}}{\apl}}}\label{ex:130909-6, 2.54}
	\ex{\glll ehh, n-fain=n=ena? naiʔ Rius iin\j=\tbr{aan} n-fani? \\
						{} n-fani=n=ena naiʔ Rius ini=ana n-fani\\
						{} \n-return={\einV=\een} {\naiq} Lius {\iin\Mv}=\tbr{\aan} {\n}-return\\
			\glt	`They've returned now? Has Lius's (child) returned?'
						\txrf{130909-6, 3.23} {\emb{130909-6-03-23.mp3}{\spk{}}{\apl}}}\label{ex:130909-6, 3.23}
\end{exe}

I have so far only encountered one non-elicited example
of a plural possessum indexed with \ve{=nu} and no other determiner.
This comes from the Amarasi Bible translation
and is given in \qf{ex:Genesis 2:23} below,
followed by an elicited example in \qf{ex:el. 23/09/18}.

\begin{exe}
	\ex{\gll	iin nui-f=ein humaʔ meseʔ n-ok au=\tbr{nu}. \hspace{35mm}
						ma iin sisi-n humaʔ meseʔ n-ok au sisi-k.\\
						{\iin} bone-{\f}={\ein} kind one {\n-\ok} {\au}={\ein} {}
						and {\iin} flesh-{\N} kind one {\n}-{\ok} {\au} flesh-{\k}\\
			\glt	`Her bones are the same as mine. And her flesh is the same as my flesh.'\\
						`This is now bone of my bones and flesh of my flesh.'
						\txrf{Genesis 2:23}}\label{ex:Genesis 2:23}
	\ex{\begin{xlist}
		\ex{\gll	sekau iin=n=aa esa=n reʔ nee? \\
							who {\iin=\ein=\aa} {\esc}={\ein} {\req} {\nee} \\
				\glt `Whose (things) are those?'}
		\ex{\gll	hai=\tbr{nu}. \\
							{\hai=\ein} \\
				\glt	`(They're) ours'.
							\txrf{elicit. 23/09/18, 1.16 {\emb{elicit-180923-01-16.mp3}{\spk{}}{\apl}}}}
		\end{xlist}}\label{ex:el. 23/09/18}
\end{exe}

Two examples of \ve{=ŋgwein} indexing a plural possessum
are given in \qf{ex:130914-3, 1.51} and \qf{ex:130825-6, 7.34}.
This strategy only exists for pronouns which end in a vowel sequence.
Two examples of a plural possessum indexed by \ve{=nu/=n}
as well as a determiner are given in 
\qf{ex:130914-1, 2.41} and \qf{ex:Genesis 31:43}.

\begin{exe}
	\ex{\glll	hoo=\tbr{ŋgwein} na-tuinaʔ =kau, \\
						hoo=nu=eni na-tuinaʔ =kau \\
						{\hoo}=\tbr{\ein}=\tbr{\ein} \na-follow ={\kau} \\
			\glt	`Yours followed me, {\ldots}'
						\txrf{130914-3, 1.51} {\emb{130914-3-01-51.mp3}{\spk{}}{\apl}}}\label{ex:130914-3, 1.51}
	\ex{\glll	ʔ-ak ``hei, hoo \sf{kartu} =sina n-mate=n, baiʔ Kus, \hspace{15mm} au=\tbr{ŋgwein} n-moni=n.\\
						ʔ-ak hei hoo \sf{kartu} =sina n-mate=n, baʔi Kus, {} au=nu=eni n-moni=n\\
						\q-say hey {\hoo} card ={\siinN} {\n}-die={\einV} grandfather Kus {} {\au}=\tbr{\ein}=\tbr{\ein} \n-live={\einV}\\
			\glt	`I said: ``Your cards have expired, Kus. Mine are valid.'
						\txrf{130825-6, 7.34} {\emb{130825-6-07-34.mp3}{\spk{}}{\apl}}}\label{ex:130825-6, 7.34}
	\ex{\glll	Abaʔ iin=\tbr{n}=\tbr{ee} =m ees uum ʔ-ait \sf{wanteks}. \\
						Abaʔ ini=n=ee =ma ees uma ʔ-aiti \sf{wanteks} \\
						Aba{\Q} \iin\Mv=\tbr{\ein}=\tbr{\ee} =and {\esc} {\uma} \q-take dye\\
			\glt	`Aba{\Q}'s (children), and I were the ones who came and picked up these textile dyes.'
						\txrf{130914-1, 2.41} {\emb{130914-1-02-41.mp3}{\spk{}}{\apl}}}\label{ex:130914-1, 2.41}
	\ex{\glll	muiʔt=ein reʔ ia batuur au=\tbr{ŋgw}=\tbr{aa}. \\
						muʔit=eni reʔ ia true au=nu=aa \\
						animal={\ein} {\req} {\ia} true {\au}=\tbr{\ein}=\tbr{\aa} \\
			\glt	`These animals are really mine.' \txrf{Genesis 31:43}}\label{ex:Genesis 31:43}
\end{exe}

\subsection{Genitive suffixes}\label{sec:GenSuf ch:SynMet}
Certain nominals in Amarasi take a genitive suffix when they are possessed.
These genitive suffixes agree with the person
and number of the possessor.
They are given in \trf{tab:GenSuf2}.
A fuller discussion of genitive suffixes is given in \srf{sec:GenSuf}.

\begin{table}[ht]
		\caption{Amarasi genitive suffixes}
		\centering %\setlength{\tabcolsep}{0.7em}
		%\begin{threeparttable}[b]
			\begin{tabular}{rll}\lsptoprule
					& \tsc{sg}&	\tsc{pl}	\\ \midrule
			1		& \ve{-k}	& \ve{-m}		\\
			1,2	& 				& \ve{-k}		\\
			2		& \ve{-m}	& \ve{-m}		\\
			3		& \ve{-n}	& \ve{-k}		\\
			0		& \ve{-f}	&						\\ \lspbottomrule
		\end{tabular}
		\label{tab:GenSuf2}
\end{table}

Most nominals which take genitive suffixes are
in a part-whole relationship with the possessor.
Such nominals are typically body parts.
Three examples of possessed ``parts'' with a genitive suffix
are given in \qf{ex:130823-5, 0.26}--\qf{ex:130825-6, 19.04} below.

\begin{exe}
	\ex{\gll	esa n-teniʔ, mnees kiro niim \sf{deŋan}, faaf\j=ee \tbr{iin} eku-\tbr{n}.\\
						one {\n}-again rice kilo five with pig={\ee} \tbr{\iin} neck-\tbr{\N}\\
			\glt	`The next one, is five kilos of rice with a pig's neck.'
						\txrf{130823-5, 0.26} {\emb{130823-5-00-26.mp3}{\spk{}}{\apl}}}\label{ex:130823-5, 0.26}
	\ex{\gll	papa, \tbr{hoo} kaan-\tbr{m}=ii sekau, papa?\\
						dad \tbr{\hoo} name-\tbr{\mg}={\ii} who dad\\
			\glt	`What is your name, dad?' \txrf{120923-1, 0.01} {\emb{120923-1-00-01.mp3}{\spk{}}{\apl}}}\label{ex:120923-1, 0.01}
	\ex{\gll	hoo mu-ʔtutaʔ au \sf{taas}=ii n-bi \tbr{au} fuuf-\tbr{k}=ii.\\
					%	hoo mu-ʔtuta? au \sf{tas}=ii n-bi au fufu-k=ii\\
						{\hoo} {\muu}-put {\au} bag={\ii} {\n}-{\bi} \tbr{\au} fontanelle-\tbr{\k}={\ii}\\
			\glt	`You put my bag above my head.' \txrf{130825-6, 19.04} {\emb{130825-6-19-04.mp3}{\spk{}}{\apl}}}\label{ex:130825-6, 19.04}
\end{exe}

In example \qf{ex:130825-6, 19.04} above there
are two possessive constructions; \ve{au tas=ii} `my bag' and \ve{au fuuf-k=ii} `my fontanelle'.
In the first instance the possessum is not in a part-whole relationship
with the possessor and thus does not take any genitive suffix,
while in the second instance the possessum a part of
the possessor and thus does take a suffix.

Genitive suffixes also occur on possessed property nominals.
Three examples are given in \qf{ex:130825-6, 11.35}--\qf{ex:Genesis 37:10} below.

\begin{exe}
	\ex{\gll	nait, iin \tbr{ma}-\tbr{hiin}-\tbr{n}=ii hai m-nao m-ʔurus naiʔ Robe n-bi nehh, mee?\\
						like.this {\iin} {\ma}-know-{\N}={\ii} {\hai} {\m}-go {\m}-arrange {\naiq} Robe {\n}-{\bi} um where\\
			\glt	`Like that, he knew (\it{lit.} had knowledge) that we were going to arrange Robe at, err, where?'
						\txrf{130825-6, 11.35} {\emb{130825-6-11-35.mp3}{\spk{}}{\apl}}}\label{ex:130825-6, 11.35}
	\ex{\gll	iin \tbr{mapuut}-\tbr{n}=ii kaah=een n-eu hiit beʔi naʔi =siin.\\
						{\iin} hot-{\N}={\ii} {\kaah=\een} {\n}-{\eu} {\hiit} PM PF ={\siinN}\\
			\glt	`He was very cruel (\it{lit.} hot) to our ancestors.' \txrf{Acts 7:19}}\label{ex:Acts 7:19}
	\ex{\gll	hoo \tbr{reok}-\tbr{m}=ii! \\
						{\hoo} good-{\mg}={\ii} \\
			\glt `You're too much!' (cynical) \txrf{Genesis 37:10}}\label{ex:Genesis 37:10}
\end{exe}

Another kind of nominal on which genitive suffixes occur are nominalised verbs.
Four examples are given in \qf{ex:130907-4, 0.36} and \qf{ex:130825-6, 1.27}
below which show a number of verbs nominalised with the circumfix \ve{ʔ-{\ldots}-ʔ} (\srf{sec:NomQ--q}).
The final element of this circumfix is replaced by the genitive suffix.

\begin{exe}
	\ex{\gll	he mu-skoor m-ain =siin, mu-skoor m-ain siin okeʔ, hoo \tbr{ʔ}-\tbr{muiʔ}-\tbr{m}=ii saaʔ? \\
						{\he} {\muu}-school \m-before ={\siin} \muu-school \m-before siin all {\hoo} {\qq}-have-{\mg=\ii} what \\
			\glt	`If you want to send them to school, send them all to school, (well) what (money) do you have?
						\txrf{130907-4, 0.36} {\emb{130907-4-00-36.mp3}{\spk{}}{\apl}}}\label{ex:130907-4, 0.36}
	\ex{\gll	\sf{taŋguŋ {\j}awab} saap i{\j}a, hita \tbr{ʔ}-\tbr{moni}-\tbr{k}, hita \tbr{ʔ}-\tbr{hake}-\tbr{k}, hita \tbr{ʔ}-\tbr{nao}-\tbr{k} eta krei =ma prenat.\\
						responsibility because {\ia} {\hiit} {\qq}-live-{\k} {\hiit} {\qq}-stand-{\k} {\hiit} {\qq}-go-{\k} {\et} church =and government\\
			\glt	`Because of this responsibility, it is our life, our standing, our way in the Church and government.'
						\txrf{130825-6, 1.27} {\emb{130825-6-01-27.mp3}{\spk{}}{\apl}}}\label{ex:130825-6, 1.27}
\end{exe}

Possessed kin relations also usually take genitive suffixes.
Genitive suffixes for kin relations are drawn from a different
paradigm than those for other nominals (\srf{sec:KinGenSuf}).
In the village of Koro{\Q}oto, where the bulk of my data was collected,
possessed kin relations mostly occur with the suffix \ve{-f}.

\subsection{Syntax of possession}\label{sec:SynPoss}
The possessive phrase forms the specifier of a nominal phrase.
The possessive phrase consists of a nominal phrase and an optional possessive pronoun \ve{iin}.
Possession does not induce M\=/forms on either the possessor or the thing possessed.
The structure of the possessive phrase is given in \qf{tr:PossP} below.

\begin{exe}
	\ex{\begin{forest}
		[NP,
			[PossP, tikz={\node [draw,fit to=tree,]{};}[NP,][poss]]
			[\br{N},[N]]]
	\end{forest}}\label{tr:PossP}
\end{exe}

Example \qf{ex:130902-1, 1.34} below contains a possessive phrase;
\ve{Uisneno iin kana-n} `God's name'.
This is an example of a full nominal possessor
with a possessive pronoun as well a genitive suffix on the thing possessed.
The structure of this possessive phrase is given in \qf{tr:130902-1, 1.34} below.

\begin{exe}
	\ex{\gll 	n-boʔis Uisneno iin kana-n.\\
						\n-praise God{\U} {\iin} name-{\N} \\
			\glt	`(They) praised God's name.'
						\txrf{130902-1, 1.34} {\emb{130902-1-01-34.mp3}{\spk{}}{\apl}}}\label{ex:130902-1, 1.34}
		\ex{\begin{forest} where n children=0{tier=word}{}
			[NP,
				[PossP,[NP,[\br{N},[N,[\ve{Uisneno}\\God{\U}]]]][poss,[\ve{iin}\\{\iin}]]]
				[\br{N},[N,[\ve{kana-n}\\name{\U}-{\N}]]]]
		\end{forest}}\label{tr:130902-1, 1.34}
	\end{exe}

Examples \qf{ex2:120715-4, 0.57}--\qf{ex:130825-7, 1.02}
below show three examples of NP possessors which are not followed by the pronoun \ve{iin}.
Nonetheless, the fact that these are possessive phrases is shown clearly by the fact
that in each instance the \tsc{3sg.gen} suffix \ve{-n} occurs on the thing possessed.

\begin{exe}
	\ex{\gll	na-maikaʔ n-bi \tbr{Smaraʔ} tuna-\tbr{n}.\\
						{\na}-settle  {\n}-{\bi} Smara{\Q}{\U} top{\U}-{\N}\\
			\glt	`He settled on top of Smara{\Q} (name of a headland).'
						\txrf{120715-4, 0.57} {\emb{120715-4-00-57.mp3}{\spk{}}{\apl}}}\label{ex2:120715-4, 0.57}
	\ex{\gll	n-nao n-bi \tbr{taas\j}=\tbr{ee} noon-\tbr{n}=ee =ma n-tee Oeneet.\\
						{\n}-go {\n}-{\bi} sea\Mv={\ee} area\Mv-{\N}={\ee} =and {\n}-arrive Oeneet\\
			\glt	`He went to the coast as far as Oeneet.'
						\txrf{130821-1, 4.42} {\emb{130821-1-04-42.mp3}{\spk{}}{\apl}}}\label{ex:130821-1, 4.42}
	\ex{\gll	\tbr{ainaʔ} tina-\tbr{n}!\\
						mother{\U}  vagina{\U}-{\N}\\
			\glt	`F**k!' (\emph{lit.} `Mother's vagina!') \txrf{130825-7, 1.02} {\emb{130825-7-01-02.mp3}{\spk{}}{\apl}}}\label{ex:130825-7, 1.02}
\end{exe}

On the surface these phrases consist of two nominals;
a structure identical to that of attributive modification.
However, due to the different syntactic structure of possession,
the first nominal does not occur in the M\=/form.
The structure of the possessive phrase \ve{Smaraʔ tuna-n} `top of Smara{\Q}'
is given in \qf{tr:Smara' tunan} and can be contrasted with that of an attributive
nominal phrase, as given in \qf{tr:FautMutiq 2} below.

\begin{multicols}{2}
	\begin{exe}
		\ex{\begin{forest} where n children=0{tier=word}{}
			[NP,
				[PossP,[NP,[\br{N},[N,[\ve{Sma\tbr{raʔ}}\\{Smara{\Q}{\tbrU}}]]]]]
				[\br{N},[N,[\ve{tuna-n}\\top{\U}-{\N}]]]]
		\end{forest}}\label{tr:Smara' tunan}
		\ex{\begin{forest} where n children=0{tier=word}{}
			[NP,[\br{N},[\br{N},[N,[\ve{fa\tbr{ut}}\\stlone{\tbrM}]]][N,[\ve{mutiʔ}\\white{\U}]]]]
		\end{forest}}\label{tr:FautMutiq 2}
	\end{exe}
\end{multicols}


As discussed in \srf{sec:GenSuf ch:SynMet}, only certain nominals
in a part-whole relationship with the possessor take genitive suffixes.
It is also possible for nominals which do not take
genitive suffixes to be possessed by a nominal without
an intervening possessive pronoun.
Two examples are given in \qf{ex:120923-2, 6.21}
and \qf{ex:130825-6, 8.37} below,
each of which contains two possessive phrases.

\newpage
\begin{exe}
	\ex{\gll	hoo m-reaʔ \tbr{atoniʔ} fu-- \tbr{roa}-\tbr{t}=\tbr{aa} \tbr{fua}-\tbr{n} =at \hspace{10mm} \sf{berarti} naan\\
						{\hoo} {\m}-grab man{\U} {} plant-{\at}={\aa} fruit-{\N} ={\te} {} mean {\naan}\\
			\glt	`You grabbed the fruit of a person's plant, that means'
						\xrf{120923-2, 6.21} {\emb{120923-2-06-21.mp3}{\spk{}}{\apl}}}\label{ex:120923-2, 6.21}
	\ex{\gll	Debri, Ornaa\j=ii, \tbr{au} \tbr{aanh}=\tbr{ii} \tbr{kabin}.\\
						Debri Ornai={\ii} {\au} child={\ii} wedding \\
			\glt	`Debri Ornai, my child's wedding.' \txrf{130825-6, 8.37} {\emb{130825-6-08-37.mp3}{\spk{}}{\apl}}}\label{ex:130825-6, 8.37}
\end{exe}

These examples also show that the possessor
can be filled by a determiner phrase (\srf{sec:DetPhr}).
Two additional examples of a determiner phrase as the possessor
are given in \qf{ex:130823-5, 0.26 2} and \qf{ex:130825-6, 15.28} below.
Each of these examples also have the possessive pronoun \ve{iin}.
The structure of the possessive phrase of \qf{ex:130825-6, 15.28}
is given in \qf{tr:130825-6, 15.28}.

\begin{exe}
	\ex{\gll	esa n-teniʔ, mnees kiro niim \sf{deŋan}, \tbr{faaf\j}=\tbr{ee} \tbr{iin} \tbr{eku}-\tbr{n}.\\
						one {\n}-again rice kilo five with pig={\ee} {\iin} neck-{\N}\\
		\glt	`The next one, is five kilos of rice with a pig's neck.'
					\txrf{130823-5, 0.26} {\emb{130823-5-00-26.mp3}{\spk{}}{\apl}}}\label{ex:130823-5, 0.26 2}
	\ex{\gll	\tbr{ʔnaef}=\tbr{ii} \tbr{iin} \tbr{uab}-\tbr{n}=\tbr{ii}, au ka= ʔ-nikan =fa!\\
						old.man={\ii} {\iin} speech-{\N}={\ii} {\au} {\ka}= \q-forget ={\fa}\\
			\glt	`What the old man said, I haven't forgotten it!' \txrf{130825-6, 15.28} {\emb{130825-6-15-28.mp3}{\spk{}}{\apl}}}\label{ex:130825-6, 15.28}
\end{exe}

\begin{exe}
	\ex{\begin{forest} where n children=0{tier=word}{}
		[DP,[NP,[PossP,[DP,[NP,[\ve{ʔnaef}\\old.man, roof]][D,[\ve{=ii}\\{\ii}]]][poss,[\ve{iin}\\{\iin}]]]
				[NP,[\br{N},[N,[\ve{uab-n}\\speech]]]]][D,[\ve{=ii}\\{\ii}]]] %with roof over ʔnaef
		%[DP,[NP,[PossP,[DP,[NP,[\br{N}[N[\ve{ʔnaef}\\old.man]]]][D,[\ve{=ii}\\{\ii}]]][poss,[\ve{iin}\\{\iin}]]]
		%		[NP,[\br{N},[N,[\ve{uab-n}\\speech]]]]][D,[\ve{=ii}\\{\ii}]]] %without roof
	\end{forest}}\label{tr:130825-6, 15.28}
\end{exe}

When determiners are the only instantiation of the thing possessed (\srf{sec:PosDet}),
the head of the nominal phrase can be analysed as empty.
This is consistent with the semantics in which the possessum
of such phrases is non-referential and can only be deduced from context.
One example is given in \qf{ex:130909-6, 3.26 -2} below,
with the structure of the extended nominal given in \qf{tr:130909-6, 3.26 -2}.
(The possessive pronoun is in the M\=/form due to the following vowel-initial enclitic, see Chapter \ref{ch:PhoMet}.)

\begin{exe}
	\ex{\glll	\tbr{bi} ehh \tbr{Esi} \tbr{iin\j}=\tbr{ee} msaʔ iin n-tee et Malesia.\\
						bi {} Esi ini=ee msaʔ ini n-tea et Malesia\\
						{\BI} {} Esi {\iin\Mv=\ee} also {\iin} \n-arrive {\et} Malaysia\\
			\glt	`Esi's (daughter) has also gone to Malaysia.'
						\txrf{130909-6, 3.26} {\emb{130909-6-03-26.mp3}{\spk{}}{\apl}}}\label{ex:130909-6, 3.26 -2}
\end{exe}

\begin{exe}
	\ex{\begin{forest} where n children=0{tier=word}{}
		[DP,[NP,
			[PossP,[NP,[\ve{bi Esi}\\{Ms. Esi}, roof]][poss,[\ve{iin\j}\\{\iin\Mv}]]]
			[\br{N},[N,[{\0}]]]][D,[\ve{=ee}\\{\ee}]]] %with roof
%		[DP,[NP,
%			[PossP,[NP,[\br{N}[N[\ve{bi Esi}\\{Ms. Esi}]]]][poss,[\ve{iin\j}\\{\iin\Mv}]]]
%			[\br{N},[N,[{\0}]]]][D,[\ve{=ee}\\{\ee}]]] %without roof
	\end{forest}}\label{tr:130909-6, 3.26 -2}
\end{exe}

Possession does not induce M\=/forms on either the possessor or the thing possessed.
The evidence from possession shows that only the head of a nominal phrase
undergoes metathesis, and that nominal metathesis is only sensitive to the presence
of adjuncts and not specifiers.
%In the previous two sections I have described the structure
%of the NP node within the nominal phrase.
%I have shown that metathesis occurs below the level of \br{N}
%to mark the presence of an attributive nominal.

%\begin{exe}
	%\ex{\begin{forest}
		%[QP,
			%[DP,
				%[NumP,
					%[NP, tikz={\node [draw,fit to=tree,]{};}
						%[PossP,[NP][poss]]
						%[\br{N},[\br{N}, tikz={\node [draw,dashed,inner sep=1,fit to=tree,]{};}[N]][N,]]]
					%[Num,[,phantom[,phantom]]]]
				%[D,[,phantom]]]
			%[Q,[,phantom]]]
	%\end{forest}}\label{tr:ExtNom2}
%\end{exe}
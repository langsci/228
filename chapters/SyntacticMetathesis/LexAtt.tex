\subsection{Lexicalised attribution}\label{sec:LexAtt}
A nominal phrase can have a conventionalised, lexicalised meaning.
A sample of such nominal phrases is given in \trf{tab:NonComAdjPhr} below.
In all such examples the first nominal takes the expected M\=/form
in the same way as other nominal phrases.
In this book the elements of a conventionalised phrase
are separated by an underscore rather than a space.

\begin{table}[ht]
	\caption{Lexicalised nominal phrases}\label{tab:NonComAdjPhr}
	\centering
		\begin{threeparttable}[b]
			\begin{tabular}{r@{ }c@{ }ll@{ }r@{ }c@{ }ll}\lsptoprule
				N\sub{1}		&+&N\sub{2}				&N\sub{1}		&+&N\sub{2}		&Phrase										&Phrase\\ \midrule
				%\ve{anin} 	&+&\ve{nautus}		&`wind'			&+&`beetle'		&\ve{ain{\gap}nautus}			&`cyclone'\\
				\ve{fafi} 	&+&\ve{tai-f}			&`pig'			&+&`guts'			&\ve{faif{\gap}taif}			&`sea anemone'\\
				\ve{knaaʔ}	&+&\ve{kase}			&`bean'			&+&`foreign'	&\ve{knaa{\gap}kase}			&`peanuts'\\
				\ve{knafo} 	&+&\ve{oe}				&`mouse'		&+&`water'		&\ve{knaof{\gap}oe}				&`mole cricket'\\
				%\ve{kbiti} &+&\ve{oe}				&`scorpion'	&+&`water'		&\ve{kbiit{\gap}oe}				&`pseudo-scorpion'\\
				\ve{koro}		&+&\ve{makaʔ}			&`bird'			&+&`rice'			&\ve{koor{\gap}makaʔ}			&`sparrow'\\
				\ve{ʔbibi}	&+&\ve{kase}			&`goat'			&+&`foreign'	&\ve{ʔbiib{\gap}kase}			&`sheep'\\
				\ve{meʔe} 	&+&\ve{mainukiʔ}	&`red'			&+&`unripe'		&\ve{meeʔ{\gap}mainukiʔ}	&`pink'\\
				\ve{mone} 	&+&\ve{feʔu}			&`male'			&+&`new'			&\ve{moen{\gap}feʔu}			&`son-in-law'\su{†}\\
				\ve{okam} 	&+&\ve{asu}				&`gourd'		&+&`dog'			&\ve{ook{\gap}asu}				&`choko, chayote'\\
				\ve{paha}		&+&\ve{metoʔ}			&`country'	&+&`dry'			&\ve{paah{\gap}metoʔ}			&`Timor'\\
				%\ve{tais} 	&+&\ve{mutiʔ}			&`sarong'		&+&`white'		&\ve{tai{\gap}mutiʔ}			&`sarong for man'\su{‡}\\
				\ve{uabaʔ} 	&+&\ve{metoʔ}			&`speech'		&+&`dry'			&\ve{uab{\gap}metoʔ}			&`Meto'\\
				%\ve{utan} 	&+&\ve{mutiʔ}			&`vegetable'&+&`white'		&\ve{uut mutiʔ}						&`bok choy'\tnote{\su{§}}\\
			\lspbottomrule
			\end{tabular}
				\begin{tablenotes}
					\item [†] \ve{moen{\gap}feʔu} means both `son-in-law' (DH)
										and `opposite sex sibling's son' (ZS [m.s.], BS [w.s.]).
					%\item [‡] Specifically, a traditional Amarasi sarong for men.
					%			While the middle part of \ve{tai mutiʔ} is indeed white,
					%			the dominant colour is maroon.
					%\item [§] A calque from Malay \it{sayur putih}.
					%			An older (now archaic) Amarasi term for bok choy is \ve{uut rariis}.
					%			Speakers cannot identify a meaning for \ve{rariis} by itself.
				\end{tablenotes}
		\end{threeparttable}
\end{table}

One possible analysis of such phrases would be to propose
that they are instances of compounding, with the entire
phrase consisting of only a single nominal.
This analysis is shown in \qf{ex:KoorMakaq1} below,
for \ve{koor{\gap}makaʔ} `sparrow'.
Alternately, such phrases can be analysed as consisting
of two independent nominals, as shown in \qf{ex:KoorMakaq2}.

\begin{multicols}{2}
	\begin{exe}
		\ex{\begin{xlist}
			\ex{\gll	{\brac{NP} \brac{N}} koor {} {makaʔ \bracr{} \bracr{}}\\
								{} bird{\M} {} rice{\U} {}\\
					\glt	\lh{\brac{NP} \brac{N} }`sparrow'}\label{ex:KoorMakaq1}
			\ex{\gll	{\brac{NP} \brac{N}} {koor \bracr{}} \brac{N} {makaʔ \bracr{} \bracr{}} \\
								{} bird{\M} {} rice{\U} {}\\
					\glt	\lh{\brac{NP} \brac{N} }`sparrow'}\label{ex:KoorMakaq2}
		\end{xlist}}
	\end{exe}
\end{multicols}

Apart from the conventionalised meaning of such phrases,
there is no evidence that they have a different
syntactic status to nominal phrases with a compositional meaning.
I return to the possibility of analysing all attributive
phrases in Amarasi as compounds in \srf{sec:ProsMet}.